\section{Proof: The expected highest and second highest draw from a uniform distribution}

\begin{frame}{Proof: The expected highest and second highest draw from a uniform distribution}
    To find \textit{seller's expected revenue} from a sealed bid auction (e.g. bidders simultaneously submit their bids in sealed envelopes without knowing the bids of others) with symmetric bidders with valuation drawn from a uniform distribution, there are two different approaches:
    \begin{enumerate}
      \item One approach is to derive each \textit{bidder's expected payment} as a function of her valuation and then integrate this expression using the PDF to get the \textit{ex-ante expected payment} of each bidder which can then be added together to find seller's expected revenue.
      \item However, a more simple approach is to for $N$ number of bidders to calculate the expected value of the highest (first-price sealed bid auction) or second-highest (second-price sealed bid auction). Plugging the value into the bid-function gives the seller's expected revenue.
    \end{enumerate}
    Deriving the bid-function (best-response function) is a prerequisite for both approaches.\\\medskip
    \textit{[Try to write up the PDF, CDF, and Mean for the uniform distribution $x\sim u(a, b)$, before going to the next slide.]}
\end{frame}
\begin{frame}{Proof: The expected highest and second highest draw from a uniform distribution}
    To find \textit{seller's expected revenue} from a sealed bid auction (e.g. bidders simultaneously submit their bids in sealed envelopes without knowing the bids of others) with symmetric bidders with valuation drawn from a uniform distribution, there are two different approaches:
    \begin{enumerate}
      \item One approach is to derive each \textit{bidder's expected payment} as a function of her valuation and then integrate this expression using the PDF to get the \textit{ex-ante expected payment} of each bidder which can then be added together to find seller's expected revenue.
      \item However, a more simple approach is to for $N$ number of bidders to calculate the expected value of the highest (first-price sealed bid auction) or second-highest (second-price sealed bid auction). Plugging the value into the bid-function gives the seller's expected revenue.
    \end{enumerate}
    Deriving the bid-function (best-response function) is a prerequisite for both approaches.\\\medskip
    For both approaches, we also utilize the standard results for a uniform distribution $x\sim u(a, b):$
    \begin{enumerate}
      \item[PDF:] Probability density function: $f(x)=\frac{1}{b-a}$
      \item[CDF:] Cumulative distribution function: $F(x)=\frac{x-a}{b-a}\Rightarrow\mathbb{P}(c>x)=\frac{c-a}{b-a}$
      \item[Mean:] $\mu=\frac{a+b}{2}\Rightarrow\mathbb{E}(c<x)=\frac{a+x}{2}$
    \end{enumerate}
\end{frame}




\begin{frame}{Proof: The expected highest and second highest draw from a uniform distribution}
    First, consider the uniform distribution of $x$ from 0 to 1: $x\sim u(0, 1)$:
    % \begin{multicols}{2}
    %   \vfill\null\columnbreak
    %   Standard results for $x\sim u(a, b):$
    %   \vspace{-6pt}
    %   \begin{enumerate}
    %     \item[PDF:] $f(x)=\frac{1}{b-a}$
    %     \item[CDF:] $F(x)=\frac{x-a}{b-a}\Rightarrow\mathbb{P}(c>x)=\frac{c-a}{b-a}$
    %     \item[Mean:] $\mu=\frac{a+b}{2}\Rightarrow\mathbb{E}(c<x)=\frac{a+x}{2}$
    %   \end{enumerate}
    %   \vfill\null
    % \end{multicols}
\end{frame}



\begin{frame}{Proof: The expected highest and second highest draw from a uniform distribution}
    \textbf{Rule:}
    For a uniform distribution of $x$ from $a$ to $b$: $x\sim u(a,b)$:
\end{frame}



\begin{frame}{Proof: The expected highest and second highest draw from a uniform distribution}
    Thus, for $N$ number of bidders where each bidder $i$ have her value $v_i$ drawn from a uniform distribution $v_i\sim u(a,b)$:\\\medskip
    The expected highest value $Y$ for all bidders is
    \begin{align*}
      Y=max(v_1,v_2,...,v_N)=a+N\frac{b-a}{N+1}
    \end{align*}
    And the expected second-highest value for all bidders is
    \begin{align*}
      a+(N-1)\frac{b-a}{N+1}
    \end{align*}
    \vfill\null
\end{frame}
