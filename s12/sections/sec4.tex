\section{PS12, Ex. 4: Three-type job-applicant cheap talk game}

\begin{frame}{PS12, Ex. 4: Three-type job-applicant cheap talk game}
    Consider the following job-applicant cheap talk game based on \href{https://www.aeaweb.org/articles?id=10.1257/jep.10.3.103}{Farrell-Rabin (1996)}. Suppose that there are three types of potential applicants (high ability, medium ability, and low ability) and the firm can place the applicant in one of three possible positions (highly qualified, medium qualified, low qualified). The applicant is equally likely to be each of the three types (probability 1/3). Payoff are represented below, where for each cell, the left entry gives the payoff of the applicant, and the right entry gives the payoff of the firm, conditional on the firm’s action and the applicant’s type. Notice: this matrix does not show the normal form game! It merely gives you the payoffs for each type-job combination, but does not incorporate the cheap talk message.\\
    The game is as follows: first, the applicant’s type is realized: $t\in\{L,M,H\}$, where $t=L$ corresponds to low ability etc. The applicant observes his type and sends a cheap talk message $m\in\{L,M,H\}$. The firm observes the message and chooses a job for the applicant: $a\in\{L,M,H\}$, where $a=L$ corresponds to giving the applicant the low qualified job etc.\vspace{-14pt}
    \begin{table}
      \begin{tabular}{l|c|c|c|}
          \multicolumn{1}{c}{} & \multicolumn{1}{c}{Highly qualified} & \multicolumn{1}{c}{Medium qualified} & \multicolumn{1}{c}{Low qualified} \\\cline{2-4}
          High ability   & 3, 3 & 0, 0 & 0, 0 \\\cline{2-4}
          Medium ability & 1, 0 & 2, 2 & 0, 0 \\\cline{2-4}
          Low ability    & 1, 0 & 2, 0 & 1, 1 \\\cline{2-4}
      \end{tabular}
    \end{table}\vspace{-8pt}
    \begin{itemize}
      \item[(a)] Show that no fully separating PBE exist, where each type of applicant sends a different message. What is the intuition behind this result?
      \item[(b)] Show that a partial pooling PBE does exist, where $m(H)=H$ and $m(M)=m(L)=M$. What are the firm's beliefs? Solve for each case.
      \item[(c)] (If time permits) Does a fully pooling PBE exist, $m(H)=m(M)=m(L)=M$?
    \end{itemize}\vspace{-6pt}
    \vfill\null
\end{frame}



\subsection{PS12, Ex. 4.a: Fully separating PBE}

\begin{frame}{PS12, Ex. 4.a: Three-type: Fully separating PBE}
    \begin{table}
      \begin{tabular}{l|c|c|c|}
          \multicolumn{1}{c}{} & \multicolumn{1}{c}{Highly qualified} & \multicolumn{1}{c}{Medium qualified} & \multicolumn{1}{c}{Low qualified} \\\cline{2-4}
          High ability   & 3, 3 & 0, 0 & 0, 0 \\\cline{2-4}
          Medium ability & 1, 0 & 2, 2 & 0, 0 \\\cline{2-4}
          Low ability    & 1, 0 & 2, 0 & 1, 1 \\\cline{2-4}
      \end{tabular}
    \end{table}\vspace{-8pt}
    \begin{itemize}
      \item[(a)] Show that no fully separating PBE exist, where each type of applicant sends a different message. What is the intuition behind this result?
    \end{itemize}\vspace{-6pt}
    \begin{multicols}{2}
      \vfill\null\columnbreak
      \vfill\null
    \end{multicols}
\end{frame}



\subsection{PS12, Ex. 4.b: Partial pooling PBE}

\begin{frame}{PS12, Ex. 4.b: Three-type: Partial pooling PBE}
    \begin{table}
      \begin{tabular}{l|c|c|c|}
          \multicolumn{1}{c}{} & \multicolumn{1}{c}{Highly qualified} & \multicolumn{1}{c}{Medium qualified} & \multicolumn{1}{c}{Low qualified} \\\cline{2-4}
          High ability   & 3, 3 & 0, 0 & 0, 0 \\\cline{2-4}
          Medium ability & 1, 0 & 2, 2 & 0, 0 \\\cline{2-4}
          Low ability    & 1, 0 & 2, 0 & 1, 1 \\\cline{2-4}
      \end{tabular}
    \end{table}\vspace{-8pt}
    \begin{itemize}
      \item[(b)] Show that a partial pooling PBE does exist, where the high-ability applicant sends the message $m = H$, and the other two types send the message $m = M$. What are the firm’s beliefs about the applicant if he receives the message $m = H$ or $m = M$ (on the equilibrium path), or if he receives the message $m = L$ (off the equilibrium path)? In each case, solve for the firm’s optimal action given its beliefs.
    \end{itemize}\vspace{-6pt}
    \begin{multicols}{2}
      \vfill\null\columnbreak
      \vfill\null
    \end{multicols}
\end{frame}



\subsection{PS12, Ex. 4.c: Fully pooling PBE}

\begin{frame}{PS12, Ex. 4.c: Three-type: Fully pooling PBE}
    \begin{table}
      \begin{tabular}{l|c|c|c|}
          \multicolumn{1}{c}{} & \multicolumn{1}{c}{Highly qualified} & \multicolumn{1}{c}{Medium qualified} & \multicolumn{1}{c}{Low qualified} \\\cline{2-4}
          High ability   & 3, 3 & 0, 0 & 0, 0 \\\cline{2-4}
          Medium ability & 1, 0 & 2, 2 & 0, 0 \\\cline{2-4}
          Low ability    & 1, 0 & 2, 0 & 1, 1 \\\cline{2-4}
      \end{tabular}
    \end{table}\vspace{-8pt}
    \begin{itemize}
      \item[(c)] (If time permits) Does a fully pooling PBE exist where all types send the message $m = H$? If so, describe the players’ equilibrium strategies and beliefs, and discuss whether this pooling PBE looks more or less reasonable than the partial pooling PBE from (b).
    \end{itemize}\vspace{-6pt}
    \begin{multicols}{2}
      \vfill\null\columnbreak
      \vfill\null
    \end{multicols}
\end{frame}
