\section{PS12, Ex. 2 (A): Farrell \& Rabin (1996): "Cheap talk"}

\begin{frame}{PS12, Ex. 2 (A): Farrell \& Rabin (1996): "Cheap talk"}
    \textit{(A)} In their paper “Cheap Talk” published in the \href{https://www.aeaweb.org/articles?id=10.1257/jep.10.3.103}{\textit{Journal of Economic Perspectives} (1996)}, Joseph Farrell and Matthew Rabin describe the following situation: “Sally knows which one of two tasks is efficient to perform. Rayco [the firm] could hire Sally specifically to perform Job 1, specifically to perform Job 2, or as a highly paid manager who will choose which job to perform. If Rayco knew which task is efficient, it would still hire her to perform the task, but at a lower salary, because she has lost her informational advantage. Sally wants to be hired as manager, but prefers to be hired to do the right task and be more productive rather than to do the wrong task and be less productive.” Payoffs in this situation are\vspace{-6pt}
    \begin{table}
      \begin{tabular}{ll|c|c|c|}
          & \multicolumn{1}{c}{} & \multicolumn{1}{c}{Job 1} & \multicolumn{1}{c}{Job 2} & \multicolumn{1}{c}{Manager} \\\cline{3-5}
          \parbox[t]{20mm}{\multirow{2}{*}{Sally's knowledge}}
           & Task 1 efficient & 2, 5 & 1, -2 & 3, 3 \\\cline{3-5}
           & Task 2 efficient & 1, -2 & 2, 5 & 3, 3 \\\cline{3-5}
      \end{tabular}
    \end{table}\vspace{-2pt}
    where the left number in each cell is Sally’s payoff and the right number is the firm’s payoff (note that this matrix does not describe the normal form of the game!)\vspace{-4pt}
    \begin{itemize}
      \item[(a)] Formulate this strategic situation as a cheap talk game, assuming that the type space is equal to the message space $(T = M)$.
      \item[(b)] Show that a separating equilibrium exists where Sally truthfully reveals which job is efficient, and the firm then places Sally in that specific job.
      \item[(c)] Discuss whether or not this separating equilibrium seems reasonable. Why isn’t Sally able to convince the firm to give her the manager position?
    \end{itemize}\vspace{-6pt}
    \vfill\null
\end{frame}



\subsection{PS12, Ex. 2.a (A): Formulate a cheap talk game}

\begin{frame}{PS12, Ex. 2.a (A): Farrell \& Rabin (1996). Formulate a cheap talk game}
    \begin{table}
      \begin{tabular}{ll|c|c|c|}
          & \multicolumn{1}{c}{} & \multicolumn{1}{c}{Job 1} & \multicolumn{1}{c}{Job 2} & \multicolumn{1}{c}{Manager} \\\cline{3-5}
          \parbox[t]{20mm}{\multirow{2}{*}{Sally's knowledge}}
           & Task 1 efficient & 2, 5 & 1, -2 & 3, 3 \\\cline{3-5}
           & Task 2 efficient & 1, -2 & 2, 5 & 3, 3 \\\cline{3-5}
      \end{tabular}
    \end{table}\vspace{-12pt}
    \begin{itemize}
      \item[(a)] Formulate this strategic situation as a cheap talk game, assuming that the type space is equal to the message space $(T = M)$.
    \end{itemize}\vspace{-6pt}
    \begin{multicols}{2}
      \vfill\null\columnbreak
      \vfill\null
    \end{multicols}
\end{frame}



\subsection{PS12, Ex. 2.b (A): Find a separating PBE}

\begin{frame}{PS12, Ex. 2.b (A): Farrell \& Rabin (1996). Find a separating PBE}
    \begin{table}
      \begin{tabular}{ll|c|c|c|}
          & \multicolumn{1}{c}{} & \multicolumn{1}{c}{Job 1} & \multicolumn{1}{c}{Job 2} & \multicolumn{1}{c}{Manager} \\\cline{3-5}
          \parbox[t]{20mm}{\multirow{2}{*}{Sally's knowledge}}
           & Task 1 efficient & 2, 5 & 1, -2 & 3, 3 \\\cline{3-5}
           & Task 2 efficient & 1, -2 & 2, 5 & 3, 3 \\\cline{3-5}
      \end{tabular}
    \end{table}\vspace{-12pt}
    \begin{itemize}
      \item[(b)] Show that a separating equilibrium exists where Sally truthfully reveals which job is efficient, and the firm then places Sally in that specific job.
    \end{itemize}\vspace{-6pt}
    \begin{multicols}{2}
      \vfill\null\columnbreak
      \vfill\null
    \end{multicols}
\end{frame}



\subsection{PS12, Ex. 2.c (A): Discuss if PBE is reasonable?}

\begin{frame}{PS12, Ex. 2.c (A): Farrell \& Rabin (1996). Discuss if PBE is reasonable?}
    \begin{table}
      \begin{tabular}{ll|c|c|c|}
          & \multicolumn{1}{c}{} & \multicolumn{1}{c}{Job 1} & \multicolumn{1}{c}{Job 2} & \multicolumn{1}{c}{Manager} \\\cline{3-5}
          \parbox[t]{20mm}{\multirow{2}{*}{Sally's knowledge}}
           & Task 1 efficient & 2, 5 & 1, -2 & 3, 3 \\\cline{3-5}
           & Task 2 efficient & 1, -2 & 2, 5 & 3, 3 \\\cline{3-5}
      \end{tabular}
    \end{table}\vspace{-12pt}
    \begin{itemize}
      \item[(c)] Discuss whether or not this separating equilibrium seems reasonable. Why isn’t Sally able to convince the firm to give her the manager position?
    \end{itemize}\vspace{-6pt}
    \begin{multicols}{2}
      \vfill\null\columnbreak
      \vfill\null
    \end{multicols}
\end{frame}
