\section{PS11, Ex. 1 (A): Signaling effect of the GED education program}

\begin{frame}{PS11, Ex. 1 (A): Signaling effect of the GED education program}
    Does signaling work? Read the article by Tyler, Murnane and Willett and think about
    their results. What is their hypothesis for why they do not find an effect for minority groups? Come up with an example of an education program that has mostly signaling value in your country.\\\medskip
    \textit{(This is a reflection question, no answer will be provided).}
    \vfill\null
\end{frame}



\section{PS11, Ex. 2 (A): Asymmetric/incomplete information (PBE)}

\begin{frame}{PS11, Ex. 2 (A): Asymmetric/incomplete information (PBE)}
    Exercise 4.11 in Gibbons (p. 250). Difficult. A buyer and a seller have valuations $v_b$ and $v_s$. It is common knowledge that there are gains from trade (i.e., that $v_b > v_s$), but the size of the gains is private information, as follows: the seller’s valuation is uniformly distributed on [0,1]; the buyer’s valuation $v_b = kv_s$, where $k > 1$ is common knowledge; the seller knows $v_s$ (and hence $v_b$) but the buyer does not know $v_b$ (and hence $v_s$). Suppose the buyer makes a single offer, $p$, which the seller either accepts or rejects. What is the perfect Bayesian equilibrium when $k < 2$? When $k > 2$? (\href{https://www.jstor.org/stable/1911195}{See Samuelson 1984}.)
    \vfill\null
\end{frame}

\begin{frame}{PS11, Ex. 2 (A): Asymmetric/incomplete information (PBE)}
    Exercise 4.11 in Gibbons (p. 250). Difficult. A buyer and a seller have valuations $v_b$ and $v_s$. It is common knowledge that there are gains from trade (i.e., that $v_b > v_s$), but the size of the gains is private information, as follows: the seller’s valuation is uniformly distributed on [0,1]; the buyer’s valuation $v_b = kv_s$, where $k > 1$ is common knowledge; the seller knows $v_s$ (and hence $v_b$) but the buyer does not know $v_b$ (and hence $v_s$). Suppose the buyer makes a single offer, $p$, which the seller either accepts or rejects. What is the perfect Bayesian equilibrium when $k < 2$? When $k > 2$? (\href{https://www.jstor.org/stable/1911195}{See Samuelson 1984}.) \vspace{-8pt}
    \begin{multicols}{2}
      \begin{itemize}
        \item[Step 1:] \textbf{Consider the uniform distribution $\bm{x\sim U(a, b)}$. Use the cumulative distribution function (CDF) to write up the probability that a random draw of \textit{x} is lower than a constant \textit{c}. Use the mean to write up the expected value of a random draw of \textit{x} where \textit{x} is lower than a constant $\bm{c\in[a,b]}$.}
      \end{itemize}
      \vfill\null\columnbreak
      \vfill\null
    \end{multicols}
\end{frame}
\begin{frame}{PS11, Ex. 2 (A): Asymmetric/incomplete information (PBE)}
    Exercise 4.11 in Gibbons (p. 250). Difficult. A buyer and a seller have valuations $v_b$ and $v_s$. It is common knowledge that there are gains from trade (i.e., that $v_b > v_s$), but the size of the gains is private information, as follows: the seller’s valuation is uniformly distributed on [0,1]; the buyer’s valuation $v_b = kv_s$, where $k > 1$ is common knowledge; the seller knows $v_s$ (and hence $v_b$) but the buyer does not know $v_b$ (and hence $v_s$). Suppose the buyer makes a single offer, $p$, which the seller either accepts or rejects. What is the perfect Bayesian equilibrium when $k < 2$? When $k > 2$? (\href{https://www.jstor.org/stable/1911195}{See Samuelson 1984}.) \vspace{-8pt}
    \begin{multicols}{2}
      \begin{itemize}
        \item[Step 1:] Consider the uniform distribution $x\sim U(a, b)$. Use the cumulative distribution function (CDF) to write up the probability that a random draw of \textit{x} is lower than a constant \textit{c}. Use the mean to write up the expected value of a random draw of \textit{x} where \textit{x} is lower than a constant $c\in[a,b]$.
      \end{itemize}
      \vfill\null\columnbreak
      \begin{enumerate}
        \item Standard results for $x\sim U(a, b):$
        \item[CDF:] $F(x)=\frac{x-a}{b-a}\Rightarrow\mathbb{P}(x<c)=\frac{c-a}{b-a}\ \ (\dagger)$
        \item[Mean:] $\mu=\frac{a+b}{2}\Rightarrow\mathbb{E}(x<c)=\frac{a+c}{2}\quad\quad\ (\ddagger)$
      \end{enumerate}
      \vfill\null
    \end{multicols}
\end{frame}
\begin{frame}{PS11, Ex. 2 (A): Asymmetric/incomplete information (PBE)}
    Exercise 4.11 in Gibbons (p. 250). Difficult. A buyer and a seller have valuations $v_b$ and $v_s$. It is common knowledge that there are gains from trade (i.e., that $v_b > v_s$), but the size of the gains is private information, as follows: the seller’s valuation is uniformly distributed on [0,1]; the buyer’s valuation $v_b = kv_s$, where $k > 1$ is common knowledge; the seller knows $v_s$ (and hence $v_b$) but the buyer does not know $v_b$ (and hence $v_s$). Suppose the buyer makes a single offer, $p$, which the seller either accepts or rejects. What is the perfect Bayesian equilibrium when $k < 2$? When $k > 2$? (\href{https://www.jstor.org/stable/1911195}{See Samuelson 1984}.) \vspace{-8pt}
    \begin{multicols}{2}
      \begin{itemize}
        \item[Step 1:] Consider the uniform distribution $x\sim U(a, b)$. Use the cumulative distribution function (CDF) to write up the probability that a random draw of \textit{x} is lower than a constant \textit{c}. Use the mean to write up the expected value of a random draw of \textit{x} where \textit{x} is lower than a constant $c\in[a,b]$.
        \item[Step 2:] \textbf{The buyer offers a price \textit{p}. Write up the seller's strategy (best response).}
      \end{itemize}
      \vfill\null\columnbreak
      \begin{enumerate}
        \item Standard results for $x\sim U(a, b):$
        \item[CDF:] $F(x)=\frac{x-a}{b-a}\Rightarrow\mathbb{P}(x<c)=\frac{c-a}{b-a}\ \ (\dagger)$
        \item[Mean:] $\mu=\frac{a+b}{2}\Rightarrow\mathbb{E}(x<c)=\frac{a+c}{2}\quad\quad\ (\ddagger)$
      \end{enumerate}
      \vfill\null
    \end{multicols}
\end{frame}
\begin{frame}{PS11, Ex. 2 (A): Asymmetric/incomplete information (PBE)}
    Exercise 4.11 in Gibbons (p. 250). Difficult. A buyer and a seller have valuations $v_b$ and $v_s$. It is common knowledge that there are gains from trade (i.e., that $v_b > v_s$), but the size of the gains is private information, as follows: the seller’s valuation is uniformly distributed on [0,1]; the buyer’s valuation $v_b = kv_s$, where $k > 1$ is common knowledge; the seller knows $v_s$ (and hence $v_b$) but the buyer does not know $v_b$ (and hence $v_s$). Suppose the buyer makes a single offer, $p$, which the seller either accepts or rejects. What is the perfect Bayesian equilibrium when $k < 2$? When $k > 2$? (\href{https://www.jstor.org/stable/1911195}{See Samuelson 1984}.) \vspace{-8pt}
    \begin{multicols}{2}
      \begin{itemize}
        \item[Step 1:] Consider the uniform distribution $x\sim U(a, b)$. Use the cumulative distribution function (CDF) to write up the probability that a random draw of \textit{x} is lower than a constant \textit{c}. Use the mean to write up the expected value of a random draw of \textit{x} where \textit{x} is lower than a constant $c\in[a,b]$.
        \item[Step 2:] The buyer offers a price \textit{p}. Write up the seller's strategy (best response).
      \end{itemize}
      \vfill\null\columnbreak
      \begin{enumerate}
        \item Standard results for $x\sim U(a, b):$
        \item[CDF:] $F(x)=\frac{x-a}{b-a}\Rightarrow\mathbb{P}(x<c)=\frac{c-a}{b-a}\ \ (\dagger)$
        \item[Mean:] $\mu=\frac{a+b}{2}\Rightarrow\mathbb{E}(x<c)=\frac{a+c}{2}\quad\quad\ (\ddagger)$
        \item $S_s(p,v_s)=\left\{\begin{array}{ll}
          Sell  & \text{if }p\geq v_s \\
          Don't & \text{if }p < v_s
        \end{array}\right.$
      \end{enumerate}
      \vfill\null
    \end{multicols}
\end{frame}
\begin{frame}{PS11, Ex. 2 (A): Asymmetric/incomplete information (PBE)}
    Exercise 4.11 in Gibbons (p. 250). Difficult. A buyer and a seller have valuations $v_b$ and $v_s$. It is common knowledge that there are gains from trade (i.e., that $v_b > v_s$), but the size of the gains is private information, as follows: the seller’s valuation is uniformly distributed on [0,1]; the buyer’s valuation $v_b = kv_s$, where $k > 1$ is common knowledge; the seller knows $v_s$ (and hence $v_b$) but the buyer does not know $v_b$ (and hence $v_s$). Suppose the buyer makes a single offer, $p$, which the seller either accepts or rejects. What is the perfect Bayesian equilibrium when $k < 2$? When $k > 2$? (\href{https://www.jstor.org/stable/1911195}{See Samuelson 1984}.) \vspace{-8pt}
    \begin{multicols}{2}
      \begin{itemize}
        \item[Step 1:] Consider the uniform distribution $x\sim U(a, b)$. Use the cumulative distribution function (CDF) to write up the probability that a random draw of \textit{x} is lower than a constant \textit{c}. Use the mean to write up the expected value of a random draw of \textit{x} where \textit{x} is lower than a constant $c\in[a,b]$.
        \item[Step 2:] The buyer offers a price \textit{p}. Write up the seller's strategy (best response).
        \item[Step 3:] \textbf{Write up the buyer's problem.}
      \end{itemize}
      \vfill\null\columnbreak
      \begin{enumerate}
        \item Standard results for $x\sim U(a, b):$
        \item[CDF:] $F(x)=\frac{x-a}{b-a}\Rightarrow\mathbb{P}(x<c)=\frac{c-a}{b-a}\ \ (\dagger)$
        \item[Mean:] $\mu=\frac{a+b}{2}\Rightarrow\mathbb{E}(x<c)=\frac{a+c}{2}\quad\quad\ (\ddagger)$
        \item $S_s(p,v_s)=\left\{\begin{array}{ll}
          Sell  & \text{if }p\geq v_s \\
          Don't & \text{if }p < v_s
        \end{array}\right.$
      \end{enumerate}
      \vfill\null
    \end{multicols}
\end{frame}
\begin{frame}{PS11, Ex. 2 (A): Asymmetric/incomplete information (PBE)}
    Exercise 4.11 in Gibbons (p. 250). Difficult. A buyer and a seller have valuations $v_b$ and $v_s$. It is common knowledge that there are gains from trade (i.e., that $v_b > v_s$), but the size of the gains is private information, as follows: the seller’s valuation is uniformly distributed on [0,1]; the buyer’s valuation $v_b = kv_s$, where $k > 1$ is common knowledge; the seller knows $v_s$ (and hence $v_b$) but the buyer does not know $v_b$ (and hence $v_s$). Suppose the buyer makes a single offer, $p$, which the seller either accepts or rejects. What is the perfect Bayesian equilibrium when $k < 2$? When $k > 2$? (\href{https://www.jstor.org/stable/1911195}{See Samuelson 1984}.) \vspace{-8pt}
    \begin{multicols}{2}
      \begin{itemize}
        \item[Step 1:] Consider the uniform distribution $x\sim U(a, b)$. Use the cumulative distribution function (CDF) to write up the probability that a random draw of \textit{x} is lower than a constant \textit{c}. Use the mean to write up the expected value of a random draw of \textit{x} where \textit{x} is lower than a constant $c\in[a,b]$.
        \item[Step 2:] The buyer offers a price \textit{p}. Write up the seller's strategy (best response).
        \item[Step 3:] Write up the buyer's problem:
      \end{itemize} \vspace{-8pt}
      \begin{align*}
        \displaystyle{\max_p}&\ \mathbb{P}[v_s<p]\mathbb{E}[v_b-p|v_s<p]
      \end{align*}
      \vfill\null\columnbreak
      \begin{enumerate}
        \item Standard results for $x\sim U(a, b):$
        \item[CDF:] $F(x)=\frac{x-a}{b-a}\Rightarrow\mathbb{P}(x<c)=\frac{c-a}{b-a}\ \ (\dagger)$
        \item[Mean:] $\mu=\frac{a+b}{2}\Rightarrow\mathbb{E}(x<c)=\frac{a+c}{2}\quad\quad\ (\ddagger)$
        \item $S_s(p,v_s)=\left\{\begin{array}{ll}
          Sell  & \text{if }p\geq v_s \\
          Don't & \text{if }p < v_s
        \end{array}\right.$
      \end{enumerate}
      \vfill\null
    \end{multicols}
\end{frame}
\begin{frame}{PS11, Ex. 2 (A): Asymmetric/incomplete information (PBE)}
    Exercise 4.11 in Gibbons (p. 250). Difficult. A buyer and a seller have valuations $v_b$ and $v_s$. It is common knowledge that there are gains from trade (i.e., that $v_b > v_s$), but the size of the gains is private information, as follows: the seller’s valuation is uniformly distributed on [0,1]; the buyer’s valuation $v_b = kv_s$, where $k > 1$ is common knowledge; the seller knows $v_s$ (and hence $v_b$) but the buyer does not know $v_b$ (and hence $v_s$). Suppose the buyer makes a single offer, $p$, which the seller either accepts or rejects. What is the perfect Bayesian equilibrium when $k < 2$? When $k > 2$? (\href{https://www.jstor.org/stable/1911195}{See Samuelson 1984}.) \vspace{-8pt}
    \begin{multicols}{2}
      \begin{itemize}
        \item[Step 1:] Use the CDF to write up $\mathbb{P}(x<c)$. Use the mean to write up $\mathbb{E}(x<c)$.
        \item[Step 2:] The buyer offers a price \textit{p}. Write up the seller's strategy (best response).
        \item[Step 3:] Write up the buyer's problem:
      \end{itemize} \vspace{-8pt}
      \begin{align*}
        \displaystyle{\max_p}&\ \mathbb{P}[v_s<p]\mathbb{E}[v_b-p|v_s<p]\\
       =\displaystyle{\max_p}&\ \frac{p-0}{1-0}\mathbb{E}[kv_s-p|v_s<p]&&\text{using }(\dagger)\\
       =\displaystyle{\max_p}&\ p\left(k\mathbb{E}[v_s<p]-p\right)\\
       =\displaystyle{\max_p}&\ p\left(k\frac{0+p}{2}-p\right)&&\text{using }(\ddagger)
      \end{align*}
      \vfill\null\columnbreak
      \begin{enumerate}
        \item Standard results for $x\sim U(a, b):$
        \item[CDF:] $F(x)=\frac{x-a}{b-a}\Rightarrow\mathbb{P}(x<c)=\frac{c-a}{b-a}\ \ (\dagger)$
        \item[Mean:] $\mu=\frac{a+b}{2}\Rightarrow\mathbb{E}(x<c)=\frac{a+c}{2}\quad\quad\ (\ddagger)$
        \item $S_s(p,v_s)=\left\{\begin{array}{ll}
          Sell  & \text{if }p\geq v_s \\
          Don't & \text{if }p < v_s
        \end{array}\right.$
        \item $\displaystyle{\max_p}\ u_b(p,k)=\displaystyle{\max_p}\ p^2\left(\frac{k}{2}-1\right)$
      \end{enumerate}
      \vfill\null
    \end{multicols}
\end{frame}
\begin{frame}{PS11, Ex. 2 (A): Asymmetric/incomplete information (PBE)}
    Exercise 4.11 in Gibbons (p. 250). Difficult. A buyer and a seller have valuations $v_b$ and $v_s$. It is common knowledge that there are gains from trade (i.e., that $v_b > v_s$), but the size of the gains is private information, as follows: the seller’s valuation is uniformly distributed on [0,1]; the buyer’s valuation $v_b = kv_s$, where $k > 1$ is common knowledge; the seller knows $v_s$ (and hence $v_b$) but the buyer does not know $v_b$ (and hence $v_s$). Suppose the buyer makes a single offer, $p$, which the seller either accepts or rejects. What is the perfect Bayesian equilibrium when $k < 2$? When $k > 2$? (\href{https://www.jstor.org/stable/1911195}{See Samuelson 1984}.) \vspace{-8pt}
    \begin{multicols}{2}
      \begin{itemize}
        \item[Step 1:] Use the CDF to write up $\mathbb{P}(x<c)$. Use the mean to write up $\mathbb{E}(x<c)$.
        \item[Step 2:] The buyer offers a price \textit{p}. Write up the seller's strategy (best response).
        \item[Step 3:] Write up the buyer's problem:
      \end{itemize} \vspace{-8pt}
      \begin{align*}
         \displaystyle{\max_p}&\ \mathbb{P}[v_s<p]\mathbb{E}[v_b-p|v_s<p]\\
        =\displaystyle{\max_p}&\ \frac{p-0}{1-0}\mathbb{E}[kv_s-p|v_s<p]&&\text{using }(\dagger)\\
        =\displaystyle{\max_p}&\ p\left(k\mathbb{E}[v_s<p]-p\right)\\
        =\displaystyle{\max_p}&\ p\left(k\frac{0+p}{2}-p\right)&&\text{using }(\ddagger)
      \end{align*} \vspace{-8pt}
      \begin{itemize}
        \item[Step 4:] \textbf{Take the first-order condition.}
      \end{itemize}
      \vfill\null\columnbreak
      \begin{enumerate}
        \item Standard results for $x\sim U(a, b):$
        \item[CDF:] $F(x)=\frac{x-a}{b-a}\Rightarrow\mathbb{P}(x<c)=\frac{c-a}{b-a}\ \ (\dagger)$
        \item[Mean:] $\mu=\frac{a+b}{2}\Rightarrow\mathbb{E}(x<c)=\frac{a+c}{2}\quad\quad\ (\ddagger)$
        \item $S_s(p,v_s)=\left\{\begin{array}{ll}
          Sell  & \text{if }p\geq v_s \\
          Don't & \text{if }p < v_s
        \end{array}\right.$
        \item $\displaystyle{\max_p}\ u_b(p,k)=\displaystyle{\max_p}\ p^2\left(\frac{k}{2}-1\right)$
      \end{enumerate}
      \vfill\null
    \end{multicols}
\end{frame}
\begin{frame}{PS11, Ex. 2 (A): Asymmetric/incomplete information (PBE)}
    Exercise 4.11 in Gibbons (p. 250). Difficult. A buyer and a seller have valuations $v_b$ and $v_s$. It is common knowledge that there are gains from trade (i.e., that $v_b > v_s$), but the size of the gains is private information, as follows: the seller’s valuation is uniformly distributed on [0,1]; the buyer’s valuation $v_b = kv_s$, where $k > 1$ is common knowledge; the seller knows $v_s$ (and hence $v_b$) but the buyer does not know $v_b$ (and hence $v_s$). Suppose the buyer makes a single offer, $p$, which the seller either accepts or rejects. What is the perfect Bayesian equilibrium when $k < 2$? When $k > 2$? (\href{https://www.jstor.org/stable/1911195}{See Samuelson 1984}.) \vspace{-8pt}
    \begin{multicols}{2}
      \begin{itemize}
        \item[Step 1:] Use the CDF to write up $\mathbb{P}(x<c)$. Use the mean to write up $\mathbb{E}(x<c)$.
        \item[Step 2:] The buyer offers a price \textit{p}. Write up the seller's strategy (best response).
        \item[Step 3:] Write up the buyer's problem.
        \item[Step 4:] Take the first-order condition:
      \end{itemize} \vspace{-8pt}
      \begin{align*}
        \frac{\delta u_b(p,k)}{\delta p}&=0\\
        2p\left(\frac{k}{2}-1\right)&=0&&\text{(take the SOC)}\\
        2p\frac{k}{2}&=2p\\
        p\frac{k}{2}&=p
      \end{align*}
      \vfill\null\columnbreak
      \begin{enumerate}
        \item Standard results for $x\sim U(a, b):$
        \item[CDF:] $F(x)=\frac{x-a}{b-a}\Rightarrow\mathbb{P}(x<c)=\frac{c-a}{b-a}\ \ (\dagger)$
        \item[Mean:] $\mu=\frac{a+b}{2}\Rightarrow\mathbb{E}(x<c)=\frac{a+c}{2}\quad\quad\ (\ddagger)$
        \item $S_s(p,v_s)=\left\{\begin{array}{ll}
          Sell  & \text{if }p\geq v_s \\
          Don't & \text{if }p < v_s
        \end{array}\right.$
        \item $\displaystyle{\max_p}\ u_b(p,k)=\displaystyle{\max_p}\ p^2\left(\frac{k}{2}-1\right)$
        \item FOC: $p\frac{k}{2}=p$\vspace{-6pt}
        \begin{align*}
          \text{SOC: }k-2\left\{\begin{array}{ll}
              <0,\ k\in(1,2)&\Rightarrow\text{concave}\\
              =0,\ k=2&\Rightarrow\text{flat}\\
              >0,\ k>2&\Rightarrow\text{convex}
          \end{array}\right.
        \end{align*}
      \end{enumerate}
      \vfill\null
    \end{multicols}
\end{frame}
\begin{frame}{PS11, Ex. 2 (A): Asymmetric/incomplete information (PBE)}
    Exercise 4.11 in Gibbons (p. 250). Difficult. A buyer and a seller have valuations $v_b$ and $v_s$. It is common knowledge that there are gains from trade (i.e., that $v_b > v_s$), but the size of the gains is private information, as follows: the seller’s valuation is uniformly distributed on [0,1]; the buyer’s valuation $v_b = kv_s$, where $k > 1$ is common knowledge; the seller knows $v_s$ (and hence $v_b$) but the buyer does not know $v_b$ (and hence $v_s$). Suppose the buyer makes a single offer, $p$, which the seller either accepts or rejects. What is the perfect Bayesian equilibrium when $k < 2$? When $k > 2$? (\href{https://www.jstor.org/stable/1911195}{See Samuelson 1984}.) \vspace{-8pt}
    \begin{multicols}{2}
      \begin{itemize}
        \item[Step 1:] Use the CDF to write up $\mathbb{P}(x<c)$. Use the mean to write up $\mathbb{E}(x<c)$.
        \item[Step 2:] The buyer offers a price \textit{p}. Write up the seller's strategy (best response).
        \item[Step 3:] Write up the buyer's problem.
        \item[Step 4:] Take the first-order and second-order condition wrt. \textit{p}.
        \item[Step 5:] \textbf{Maximize buyer's utility for $\bm{k<2}$.}
      \end{itemize}
      \vfill\null\columnbreak
      \begin{enumerate}
        \item Standard results for $x\sim U(a, b):$
        \item[CDF:] $F(x)=\frac{x-a}{b-a}\Rightarrow\mathbb{P}(x<c)=\frac{c-a}{b-a}\ \ (\dagger)$
        \item[Mean:] $\mu=\frac{a+b}{2}\Rightarrow\mathbb{E}(x<c)=\frac{a+c}{2}\quad\quad\ (\ddagger)$
        \item $S_s(p,v_s)=\left\{\begin{array}{ll}
          Sell  & \text{if }p\geq v_s \\
          Don't & \text{if }p < v_s
        \end{array}\right.$
        \item $\displaystyle{\max_p}\ u_b(p,k)=\displaystyle{\max_p}\ p^2\left(\frac{k}{2}-1\right)$
        \item FOC: $p\frac{k}{2}=p$\vspace{-6pt}
        \begin{align*}
          \text{SOC: }k-2\left\{\begin{array}{ll}
              <0,\ k\in(1,2)&\Rightarrow\text{concave}\\
              =0,\ k=2&\Rightarrow\text{flat}\\
              >0,\ k>2&\Rightarrow\text{convex}
          \end{array}\right.
        \end{align*}
      \end{enumerate}
      \vfill\null
    \end{multicols}
\end{frame}
\begin{frame}{PS11, Ex. 2 (A): Asymmetric/incomplete information (PBE)}
    Exercise 4.11 in Gibbons (p. 250). Difficult. A buyer and a seller have valuations $v_b$ and $v_s$. It is common knowledge that there are gains from trade (i.e., that $v_b > v_s$), but the size of the gains is private information, as follows: the seller’s valuation is uniformly distributed on [0,1]; the buyer’s valuation $v_b = kv_s$, where $k > 1$ is common knowledge; the seller knows $v_s$ (and hence $v_b$) but the buyer does not know $v_b$ (and hence $v_s$). Suppose the buyer makes a single offer, $p$, which the seller either accepts or rejects. What is the perfect Bayesian equilibrium when $k < 2$? When $k > 2$? (\href{https://www.jstor.org/stable/1911195}{See Samuelson 1984}.) \vspace{-8pt}
    \begin{multicols}{2}
      \begin{itemize}
        \item[Step 1:] Use the CDF to write up $\mathbb{P}(x<c)$. Use the mean to write up $\mathbb{E}(x<c)$.
        \item[Step 2:] The buyer offers a price \textit{p}. Write up the seller's strategy (best response).
        \item[Step 3:] Write up the buyer's problem.
        \item[Step 4:] Take the first-order and second-order condition wrt. \textit{p}.
        \item[Step 5:] Maximize buyer's utility for $k<2$.
        \item[Step 6:] \textbf{Maximize buyer's utility for $\bm{k>2}$.}
      \end{itemize}
      \vfill\null\columnbreak
      \begin{enumerate}
        \item \vspace{-2pt} Standard results for $x\sim U(a, b):$
        \item[CDF:] \vspace{-2pt} $F(x)=\frac{x-a}{b-a}\Rightarrow\mathbb{P}(x<c)=\frac{c-a}{b-a}\ \ (\dagger)$
        \item[Mean:] \vspace{-2pt}  $\mu=\frac{a+b}{2}\Rightarrow\mathbb{E}(x<c)=\frac{a+c}{2}\quad\quad\ (\ddagger)$
        \item \vspace{-2pt} $S_s(p,v_s)=\left\{\begin{array}{ll}
          Sell  & \text{if }p\geq v_s \\
          Don't & \text{if }p < v_s
        \end{array}\right.$
        \item \vspace{-2pt} $\displaystyle{\max_p}\ u_b(p,k)=\displaystyle{\max_p}\ p^2\left(\frac{k}{2}-1\right)$
        \item \vspace{-2pt} FOC: $p\frac{k}{2}=p$\vspace{-6pt}
        \begin{align*}
          \text{SOC: }k-2\left\{\begin{array}{ll}
              <0,\ k\in(1,2)&\Rightarrow\text{concave}\\
              =0,\ k=2&\Rightarrow\text{flat}\\
              >0,\ k>2&\Rightarrow\text{convex}
          \end{array}\right.
        \end{align*}
        \item \vspace{-6pt} $k\in(1,2)$: FOC, SOC $\Rightarrow p^*=0$
      \end{enumerate}
      \vfill\null
    \end{multicols}
\end{frame}
\begin{frame}{PS11, Ex. 2 (A): Asymmetric/incomplete information (PBE)}
    Exercise 4.11 in Gibbons (p. 250). Difficult. A buyer and a seller have valuations $v_b$ and $v_s$. It is common knowledge that there are gains from trade (i.e., that $v_b > v_s$), but the size of the gains is private information, as follows: the seller’s valuation is uniformly distributed on [0,1]; the buyer’s valuation $v_b = kv_s$, where $k > 1$ is common knowledge; the seller knows $v_s$ (and hence $v_b$) but the buyer does not know $v_b$ (and hence $v_s$). Suppose the buyer makes a single offer, $p$, which the seller either accepts or rejects. What is the perfect Bayesian equilibrium when $k < 2$? When $k > 2$? (\href{https://www.jstor.org/stable/1911195}{See Samuelson 1984}.) \vspace{-8pt}
    \begin{multicols}{2}
      \begin{itemize}
        \item[Step 1:] Use the CDF to write up $\mathbb{P}(x<c)$. Use the mean to write up $\mathbb{E}(x<c)$.
        \item[Step 2:] \vspace{-2pt} The buyer offers a price \textit{p}. Write up the seller's strategy (best response).
        \item[Step 3:] \vspace{-2pt} Write up the buyer's problem.
        \item[Step 4:] \vspace{-2pt} Take the FOC and SOC wrt. \textit{p}.
        \item[Step 5:] \vspace{-2pt} Maximize buyer's utility for $k<2$.
        \item[Step 6:] \vspace{-2pt} Maximize buyer's utility for $k>2$.
        \item[Step 7:] \vspace{-2pt} \textbf{Looking at the seller's strategy, will trade occur when $\bm{k>2}$?\\
        What about $\bm{k\in(1,2)}$? Have we seen something similar before?}
      \end{itemize}
      \vfill\null\columnbreak
      \begin{enumerate}
        \item \vspace{-2pt} Standard results for $x\sim U(a, b):$
        \item[CDF:] \vspace{-2pt} $F(x)=\frac{x-a}{b-a}\Rightarrow\mathbb{P}(x<c)=\frac{c-a}{b-a}\ \ (\dagger)$
        \item[Mean:] \vspace{-2pt}  $\mu=\frac{a+b}{2}\Rightarrow\mathbb{E}(x<c)=\frac{a+c}{2}\quad\quad\ (\ddagger)$
        \item \vspace{-2pt} $S_s(p,v_s)=\left\{\begin{array}{ll}
          Sell  & \text{if }p\geq v_s \\
          Don't & \text{if }p < v_s
        \end{array}\right.$
        \item \vspace{-2pt} $\displaystyle{\max_p}\ u_b(p,k)=\displaystyle{\max_p}\ p^2\left(\frac{k}{2}-1\right)$
        \item \vspace{-2pt} FOC: $p\frac{k}{2}=p$\vspace{-6pt}
        \begin{align*}
          \text{SOC: }k-2\left\{\begin{array}{ll}
              <0,\ k\in(1,2)&\Rightarrow\text{concave}\\
              =0,\ k=2&\Rightarrow\text{flat}\\
              >0,\ k>2&\Rightarrow\text{convex}
          \end{array}\right.
        \end{align*}
      \end{enumerate}
      \vfill\null
    \end{multicols}
\end{frame}
\begin{frame}{PS11, Ex. 2 (A): Asymmetric/incomplete information (PBE)}
    Exercise 4.11 in Gibbons (p. 250). Difficult. A buyer and a seller have valuations $v_b$ and $v_s$. It is common knowledge that there are gains from trade (i.e., that $v_b > v_s$), but the size of the gains is private information, as follows: the seller’s valuation is uniformly distributed on [0,1]; the buyer’s valuation $v_b = kv_s$, where $k > 1$ is common knowledge; the seller knows $v_s$ (and hence $v_b$) but the buyer does not know $v_b$ (and hence $v_s$). Suppose the buyer makes a single offer, $p$, which the seller either accepts or rejects. What is the perfect Bayesian equilibrium when $k < 2$? When $k > 2$? (\href{https://www.jstor.org/stable/1911195}{See Samuelson 1984}.) \vspace{-8pt}
    \begin{multicols}{2}
      \begin{itemize}
        \item[Step 1:] Use the CDF to write up $\mathbb{P}(x<c)$. Use the mean to write up $\mathbb{E}(x<c)$.
        \item[Step 2:] \vspace{-2pt} The buyer offers a price \textit{p}. Write up the seller's strategy (best response).
        \item[Step 3:] \vspace{-2pt} Write up the buyer's problem.
        \item[Step 4:] \vspace{-2pt} Take the FOC and SOC wrt. \textit{p}.
        \item[Step 5:] \vspace{-2pt} Maximize buyer's utility for $k<2$.
        \item[Step 6:] \vspace{-2pt} Maximize buyer's utility for $k>2$.
        \item[Step 7:] \vspace{-2pt} $\bm{k>2}$: As $v_s\in[0,1]$, seller will always accept the price $p^{**}=1$.\\
        \textbf{What about $\bm{k\in(1,2)}$? Have we seen something similar before?}
      \end{itemize}
      \vfill\null\columnbreak
      \begin{enumerate}
        \item \vspace{-2pt} Standard results for $x\sim U(a, b):$
        \item[CDF:] \vspace{-2pt} $F(x)=\frac{x-a}{b-a}\Rightarrow\mathbb{P}(x<c)=\frac{c-a}{b-a}\ \ (\dagger)$
        \item[Mean:] \vspace{-2pt}  $\mu=\frac{a+b}{2}\Rightarrow\mathbb{E}(x<c)=\frac{a+c}{2}\quad\quad\ (\ddagger)$
        \item \vspace{-2pt} $S_s(p,v_s)=\left\{\begin{array}{ll}
          Sell  & \text{if }p\geq v_s \\
          Don't & \text{if }p < v_s
        \end{array}\right.$
        \item \vspace{-2pt} $\displaystyle{\max_p}\ u_b(p,k)=\displaystyle{\max_p}\ p^2\left(\frac{k}{2}-1\right)$
        \item \vspace{-2pt} FOC: $p\frac{k}{2}=p$\vspace{-6pt}
        \begin{align*}
          \text{SOC: }k-2\left\{\begin{array}{ll}
              <0,\ k\in(1,2)&\Rightarrow\text{concave}\\
              =0,\ k=2&\Rightarrow\text{flat}\\
              >0,\ k>2&\Rightarrow\text{convex}
          \end{array}\right.
        \end{align*}
        \item \vspace{-6pt} $k\in(1,2)$: FOC, SOC $\Rightarrow p^*=0$
        \item \vspace{-2pt} $k>2$: $\max u_b$: $p\rightarrow\infty\Rightarrow p^{**}=1$
      \end{enumerate}
      \vfill\null
    \end{multicols}
\end{frame}
\begin{frame}{PS11, Ex. 2 (A): Asymmetric/incomplete information (PBE)}
    Exercise 4.11 in Gibbons (p. 250). Difficult. A buyer and a seller have valuations $v_b$ and $v_s$. It is common knowledge that there are gains from trade (i.e., that $v_b > v_s$), but the size of the gains is private information, as follows: the seller’s valuation is uniformly distributed on [0,1]; the buyer’s valuation $v_b = kv_s$, where $k > 1$ is common knowledge; the seller knows $v_s$ (and hence $v_b$) but the buyer does not know $v_b$ (and hence $v_s$). Suppose the buyer makes a single offer, $p$, which the seller either accepts or rejects. What is the perfect Bayesian equilibrium when $k < 2$? When $k > 2$? (\href{https://www.jstor.org/stable/1911195}{See Samuelson 1984}.) \vspace{-8pt}
    \begin{multicols}{2}
      \begin{itemize}
        \item[Step 1:] Use the CDF to write up $\mathbb{P}(x<c)$. Use the mean to write up $\mathbb{E}(x<c)$.
        \item[Step 2:] \vspace{-2pt} The buyer offers a price \textit{p}. Write up the seller's strategy (best response).
        \item[Step 3:] \vspace{-2pt} Write up the buyer's problem.
        \item[Step 4:] \vspace{-2pt} Take the FOC and SOC wrt. \textit{p}.
        \item[Step 5:] \vspace{-2pt} Maximize buyer's utility for $k<2$.
        \item[Step 6:] \vspace{-2pt} Maximize buyer's utility for $k>2$.
        \item[Step 7:] \vspace{-2pt} $\bm{k>2}$: As $v_s\in[0,1]$, seller will always accept the price $p^{**}=1$.\\
        $\bm{k\in(1,2)}$: Seller will not accept if $v_s>0$, though trade would benefit both under perfect information. Similar to Akerlof's \textit{'Lemons'}.
      \end{itemize}
      \vfill\null\columnbreak
      \begin{enumerate}
        \item \vspace{-2pt} Standard results for $x\sim U(a, b):$
        \item[CDF:] \vspace{-2pt} $F(x)=\frac{x-a}{b-a}\Rightarrow\mathbb{P}(x<c)=\frac{c-a}{b-a}\ \ (\dagger)$
        \item[Mean:] \vspace{-2pt}  $\mu=\frac{a+b}{2}\Rightarrow\mathbb{E}(x<c)=\frac{a+c}{2}\quad\quad\ (\ddagger)$
        \item \vspace{-2pt} $S_s(p,v_s)=\left\{\begin{array}{ll}
          Sell  & \text{if }p\geq v_s \\
          Don't & \text{if }p < v_s
        \end{array}\right.$
        \item \vspace{-2pt} $\displaystyle{\max_p}\ u_b(p,k)=\displaystyle{\max_p}\ p^2\left(\frac{k}{2}-1\right)$
        \item \vspace{-2pt} FOC: $p\frac{k}{2}=p$\vspace{-6pt}
        \begin{align*}
          \text{SOC: }k-2\left\{\begin{array}{ll}
              <0,\ k\in(1,2)&\Rightarrow\text{concave}\\
              =0,\ k=2&\Rightarrow\text{flat}\\
              >0,\ k>2&\Rightarrow\text{convex}
          \end{array}\right.
        \end{align*}
        \item \vspace{-6pt} $k\in(1,2)$: FOC, SOC $\Rightarrow p^*=0$
        \item \vspace{-2pt} $k>2$: $\max u_b$: $p\rightarrow\infty\Rightarrow p^{**}=1$
      \end{enumerate}
      \vfill\null
    \end{multicols}
\end{frame}
