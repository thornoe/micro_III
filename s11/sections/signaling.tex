\section{Signaling games in general}

\begin{frame}{PS11: Signaling games in general}
    \begin{multicols}{2}
      \textbf{Players:}\vspace{-4pt}
      \begin{itemize}
        \item 2 players: Sender (S) and receiver (R). E.g. firm and consumer, or employer and employee (Spence).
      \end{itemize}\vspace{-4pt}
      \textbf{Timing:}\vspace{-4pt}
      \begin{enumerate}
        \item Nature chooses the sender's type from $T=\{t_1,...\}$.
        \item S: The sender realizes her type and sends a signal from $M=\{m_1,...\}$, typically either $L$ (left) or $R$ (right).
        \item R: The receiver observes $m$ (but not the type $t$!) and forms his beliefs:\vspace{-4pt}
        \begin{align*}
          p=\mu(t_1|L)\text{ and }q=\mu(t_1|R)
        \end{align*}
        \item[] \vspace{-4pt} Consequently, for $S$ having two possible types:\vspace{-4pt}
        \begin{align*}
          1-p=\mu(t_2|L)\text{ and }1-q=\mu(t_2|R)
        \end{align*}
        \item \vspace{-4pt} R: The receiver chooses an action from $A=\{a_1,...\}$, e.g. $up$ or $down$.
        \item Payoffs are realized.
      \end{enumerate}
      \vfill\null\columnbreak
      \begin{figure}[!h]
        \center
        \def\svgwidth{\columnwidth}
        \import{figures/}{figure1.pdf_tex}
      \end{figure}
      \vfill\null
    \end{multicols}
\end{frame}

\begin{frame}{PS11: Signaling games in general}
    \begin{multicols}{2}
      \textbf{Players:}\vspace{-8pt}
      \begin{itemize}
        \item 2 players: Sender (S) and receiver (R). E.g. firm and consumer, or employer and employee (Spence).
      \end{itemize}\vspace{-6pt}
      \textbf{Timing:}\vspace{-8pt}
      \begin{enumerate}
        \item Nature chooses the sender's type from $T=\{t_1,...\}$.
        \item S: The sender realizes her type and sends a signal from $M=\{m_1,...\}$. Typically either $L$ (left) or $R$ (right).
        \item R: The receiver observes $m$ (but not the type $t$!) and forms his beliefs:\vspace{-5pt}
        \begin{align*}
          p=\mu(t_1|L)\text{ and }q=\mu(t_1|R)
        \end{align*}
        \item[] \vspace{-4pt} Consequently, for $S$ having two possible types:\vspace{-5pt}
        \begin{align*}
          1-p=\mu(t_2|L)\text{ and }1-q=\mu(t_2|R)
        \end{align*}
        \item \vspace{-4pt} R: The receiver chooses an action from $A=\{a_1,...\}$, e.g. $up$ or $down$.
        \item Payoffs are realized.
      \end{enumerate}\vspace{-6pt}
      \textbf{Four possible equilibria for two types:}\vspace{-6pt}
      \begin{itemize}
        \item Pooling on $L$ or pooling on $R$.
        \item Separating: $t_1$ plays $L$ and $t_2$ plays $R$ or the other way around.
      \end{itemize}
      \vfill\null\columnbreak
      \begin{figure}[!h]
        \center
        \def\svgwidth{\columnwidth}
        \import{figures/}{figure1.pdf_tex}
      \end{figure}
      \vfill\null
    \end{multicols}
\end{frame}

\begin{frame}{PS11: Signaling games in general}
    \begin{multicols}{2}
      \textbf{Players:}\vspace{-10pt}
      \begin{itemize}
        \item 2 players: Sender (S) and receiver (R). E.g. firm and consumer, or employer and employee (Spence).
      \end{itemize}\vspace{-8pt}
      \textbf{Timing:}\vspace{-10pt}
      \begin{enumerate}
        \item Nature chooses the sender's type from $T=\{t_1,...\}$.
        \item \vspace{-4pt} S: The sender realizes her type and sends a signal from $M=\{m_1,...\}$. Typically either $L$ (left) or $R$ (right).
        \item \vspace{-4pt} R: The receiver observes $m$ (but not the type $t$!) and forms his beliefs:\vspace{-8pt}
        \begin{align*}
          p=\mu(t_1|L)\text{ and }q=\mu(t_1|R)
        \end{align*}
        \item[] \vspace{-8pt} Consequently, for $S$ having two possible types:\vspace{-8pt}
        \begin{align*}
          1-p=\mu(t_2|L)\text{ and }1-q=\mu(t_2|R)
        \end{align*}
        \item \vspace{-8pt} R: The receiver chooses an action from $A=\{a_1,...\}$, e.g. $up$ or $down$.
        \item \vspace{-4pt} Payoffs are realized.
      \end{enumerate}\vspace{-8pt}
      \textbf{Four possible equilibria for two types:}\vspace{-8pt}
      \begin{itemize}
        \item Pooling on $L$ or pooling on $R$.
        \item \vspace{-4pt} Separating: $t_1$ plays $L$ and $t_2$ plays $R$ or the other way around.
      \end{itemize}
      \vfill\null\columnbreak
      \begin{figure}[!h]
        \center
        \def\svgwidth{\columnwidth}
        \import{figures/}{figure1.pdf_tex}
      \end{figure} \vspace{-6pt}
      \textbf{Cookbook: For each possible equilibrium go over signaling requirements 3 and 2:}\vspace{-6pt}
      \begin{itemize}
        \item[SR3:] R: Find the beliefs of R given S's equilibrium strategy. (In equilibrium, we only consider beliefs of R that are consistent with S's eq. strategy.)
        \item[SR2R:] \vspace{-2pt} R: Find the R's optimal strategy given beliefs about S's strategy.
        \item[SR2S:] \vspace{-2pt} S: Check whether S wants to deviate.
        \item[PBE:] \vspace{-2pt} If no deviation, it's a PBE.
      \end{itemize}
      \vfill\null
    \end{multicols}
\end{frame}
