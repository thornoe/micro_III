\section{PS11, Ex. 5: Signaling games (pooling PBE)}

\begin{frame}{PS11, Ex. 5.a: Signaling games (pooling PBE)}
    Exercise 4.3.a in Gibbons (p. 246). Specify a pooling perfect Bayesian equilibria in which both Sender types play $R$ in the following signaling game.\vspace{-8pt}
    \begin{multicols}{2}
      \vfill\null\columnbreak
      \begin{figure}[!h]
        \center
        \def\svgwidth{1.1\columnwidth}
        \import{figures/}{Gibbons_4_3_a.pdf_tex}
      \end{figure}
      \vfill\null
    \end{multicols}
\end{frame}


\begin{frame}{PS11, Ex. 5.b: Signaling games (pooling PBE)}
    Exercise 4.3.b in Gibbons (p. 246). The following three-type signaling game begins with a move by nature, not shown in the tree, that yields one of the three types with equal probability Specify a pooling perfect Bayesian equilibria in which all three Sender types play $L$.\vspace{-8pt}
    \begin{multicols}{2}
      \vfill\null\columnbreak
      \begin{figure}[!h]
        \center
        \def\svgwidth{1.1\columnwidth}
        \import{figures/}{Gibbons_4_3_b.pdf_tex}
      \end{figure}
      \vfill\null
    \end{multicols}
\end{frame}



\section{PS11, Ex. 6: Spence’s education signaling model (PBE)}

\begin{frame}{PS11, Ex. 6: Spence’s education signaling model (PBE)}
    Consider the following version of Spence’s education signaling model, where a firm is hiring a worker. Workers are characterized by their type $\theta$, which measures their ability. There are two worker types: $\theta\in\{\theta_L,\theta_H\}$. Nature chooses the worker’s type, with $p_H = \mathbb{P}[\theta = \theta_H]$ and $p_H = \mathbb{P}[\theta = \theta_H]=1-p_H$.\\\smallskip
    The worker observes his own type, but the firm does not. The worker can choose his level of education: $e\in\mathbb{R}^{+}$. The cost to him of acquiring this education is $c_\theta(e) = e/\theta$. Education is observed by the firm, who then forms beliefs about the workers type: $\mu(\theta|e)$. We assume that the marginal productivity of a worker is equal to his ability and that the company is in competition such it pays the marginal productivity: $w(e) = \mathbb{E}[\theta|e]$. Thus, the payoff to a worker conditional on his type and education is $u_\theta(e)=w(e)c_\theta(e)$. Suppose for this exercise that $\theta_H=3$ and $\theta_L=1$.\vspace{-4pt}
    \begin{itemize}
      \item[(a)] Find a separating pure strategy Perfect Bayesian Equilibrium.
      \item[(b)] Find a pooling pure strategy Perfect Bayesian Equilibrium.
    \end{itemize}
    \begin{multicols}{2}
      \vfill\null\columnbreak
      \vfill\null
    \end{multicols}
\end{frame}
