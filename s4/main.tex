\documentclass[8pt,apectratio=169]{beamer}

\usetheme[progressbar=frametitle]{metropolis}
\usepackage{appendixnumberbeamer}
\usepackage[style=authoryear, backend=bibtex8, natbib=true, maxcitenames=2]{biblatex}

\usepackage[utf8]{inputenc} % utf8x  defines more symbols, but may cause compatible problems
\usepackage{lmodern,textcomp} % Latin Modern fonts, contains €

\usepackage{graphicx}
\usepackage{import}

\usepackage{booktabs}
\usepackage[scale=2]{ccicons}

\usepackage{pgfplots}
\usepgfplotslibrary{dateplot}

\usepackage{xspace}
\newcommand{\themename}{\textbf{\textsc{metropolis}}\xspace}

% Math
\usepackage{amsmath}
\usepackage{bm} % bold symbol in math mode
\counterwithin*{equation}{section} % reset the equation number whenever section is stepped

% Optional packages
\usepackage{xcolor}
\usepackage{multicol}
\usepackage{multirow,array}
\usepackage{subcaption} % for subfigure and subtable
\usepackage{hyperref}
\usepackage{epigraph}
\usepackage[super,negative]{nth} % allows writing 1st, 2nd, 3rd with superscript
\usepackage{ulem} % use the "sout" tag to "strikethrough" text
\usepackage{cancel} % https://tex.stackexchange.com/questions/75525/how-to-write-crossed-out-math-in-latex
\usepackage{tcolorbox}

% Select what to do with command \comment:
  % \newcommand{\comment}[1]{}  %comments not shown
  % \newcommand{\comment}[1]{\par {\bfseries \color{blue} #1 \par}} %comments shown
% Select what to do with todonotes: i.e. \todo{}, \todo[inline]{}
  % \usepackage[disable]{todonotes} % notes not shown
  % \usepackage[draft]{todonotes}   % notes shown

%\numberwithin{equation}{section}

%\addbibresource{references}

\titlegraphic{\hfill \includegraphics[width=0.15 \textwidth]{figures/logo}}
\title{Microeconomics III: Problem Set 8\footnote{Slides created for exercise class 3 and 4, with reservation for possible errors.\\}}
\author{Thor Donsby Noe (\href{mailto:thor.noe@econ.ku.dk}{thor.noe@econ.ku.dk})
        \& Christopher Borberg (\href{mailto:christopher.borberg@econ.ku.dk}{christopher.borberg@econ.ku.dk})
        }
\date{November 13 2019} % \today
\institute{\normalsize Department of Economics, University of Copenhagen}

    % \definecolor{BlueTOL}{HTML}{222255}
    \definecolor{BrownTOL}{HTML}{666633}
    \definecolor{GreenTOL}{HTML}{225522}
    % \setbeamercolor{normal text}{fg=BlueTOL,bg=white}
    \setbeamercolor{alerted text}{fg=BrownTOL}
    \setbeamercolor{example text}{fg=GreenTOL}
    \setbeamercolor{background canvas}{bg=white}

    \setbeamercolor{block title alerted}{use=alerted text,
        fg=alerted text.fg,
        bg=alerted text.bg!80!alerted text.fg}
    \setbeamercolor{block body alerted}{use={block title alerted, alerted text},
        fg=alerted text.fg,
        bg=block title alerted.bg!50!alerted text.bg}
    \setbeamercolor{block title example}{use=example text,
        fg=example text.fg,
        bg=example text.bg!80!example text.fg}
    \setbeamercolor{block body example}{use={block title example, example text},
        fg=example text.fg,
        bg=block title example.bg!50!example text.bg}

\begin{document}
\maketitle

% ------------------------------------------------------------------------------
% ------ FRAME -----------------------------------------------------------------
% ------------------------------------------------------------------------------
\begin{frame}{Outline}
    \tableofcontents
\end{frame}


\section{PS4, Ex. 1 (A): MSNE and best-response functions}

\begin{frame}{PS4, Ex. 1 (A): MSNE and best-response functions}
  1. (A) Find all equilibria (pure and mixed) in the following games, first analytically and then through plotting the best-response functions.
  \begin{multicols}{2}
    \begin{itemize}
      \item[(a)]
    \end{itemize}
    \begin{table}
      \begin{tabular}{cl|c|c|}
          & \multicolumn{1}{c}{} & \multicolumn{2}{c}{Player 2}\\
          \parbox[t]{1mm}{\multirow{3}{*}{\rotatebox[origin=r]{90}{Player 1}}}
          & \multicolumn{1}{c}{} & \multicolumn{1}{c}{L (q)} & \multicolumn{1}{c}{L (1-q)} \\\cline{3-4}
          & T (p) & 3, 3 & 0, 0 \\\cline{3-4}
          & B (1-p) & 0, 0 & 4, 4 \\\cline{3-4}
      \end{tabular}
    \end{table}
  \vfill\null \columnbreak
    \begin{itemize}
      \item[(b)]
    \end{itemize}
    \begin{table}
      \begin{tabular}{cl|c|c|}
          & \multicolumn{1}{c}{} & \multicolumn{2}{c}{Player 2}\\
          \parbox[t]{1mm}{\multirow{3}{*}{\rotatebox[origin=r]{90}{Player 1}}}
          & \multicolumn{1}{c}{} & \multicolumn{1}{c}{L (q)} & \multicolumn{1}{c}{L (1-q)} \\\cline{3-4}
          & T (p) & 1, 1 & 0, 0 \\\cline{3-4}
          & B (1-p) & 1, 0 & 2, 1 \\\cline{3-4}
      \end{tabular}
    \end{table}
  \vfill\null
  \end{multicols}
    \begin{tabular}{|l|}
      \cline{1-1}
      \textbf{Hint}: Find the probabilities $q$ for which Player 1 is indifferent, e.g. $u_1(T,q)=u_1(B,q)$.\\
                      and the probabilities $p$ for which Player 2 is indifferent, e.g. $u_2(L,p)=u_2(R,p)$.\\\cline{1-1}
  \end{tabular}
\end{frame}

\begin{frame}{PS4, Ex. 1.a (A): MSNE and best-response functions}
  \begin{multicols}{2}
    \begin{itemize}
      \item[(a)] Find all equilibria (pure and mixed), first analytically and then through plotting the BR functions.
    \end{itemize}
    \begin{table}
      \begin{tabular}{cl|c|c|}
          & \multicolumn{1}{c}{} & \multicolumn{2}{c}{Player 2}\\
          \parbox[t]{1mm}{\multirow{3}{*}{\rotatebox[origin=r]{90}{Player 1}}}
          & \multicolumn{1}{c}{} & \multicolumn{1}{c}{L (q)} & \multicolumn{1}{c}{L (1-q)} \\\cline{3-4}
          & T (p) & 3, 3 & 0, 0 \\\cline{3-4}
          & B (1-p) & 0, 0 & 4, 4 \\\cline{3-4}
      \end{tabular}
    \end{table}
    \textbf{\textit{Highlight the best responses in pure strategies.}}
  \vfill\null \columnbreak
  \vfill\null
  \end{multicols}
\end{frame}
\begin{frame}{PS4, Ex. 1.a (A): MSNE and best-response functions}
  \begin{multicols}{2}
    \begin{itemize}
      \item[(a)] Find all equilibria (pure and mixed), first analytically and then through plotting the BR functions.
    \end{itemize}
    \begin{table}
      \begin{tabular}{cl|c|c|}
        & \multicolumn{1}{c}{} & \multicolumn{2}{c}{\color{blue}Player 2}\\
        \parbox[t]{1mm}{\multirow{3}{*}{\rotatebox[origin=r]{90}{\color{red}Player 1}}}
        & \multicolumn{1}{c}{} & \multicolumn{1}{c}{L (q)} & \multicolumn{1}{c}{L (1-q)} \\\cline{3-4}
        & T (p) & \textcolor{red}{3}, \textcolor{blue}{3} & 0, 0 \\\cline{3-4}
        & B (1-p) & 0, 0 & \textcolor{red}{4}, \textcolor{blue}{4} \\\cline{3-4}
      \end{tabular}
    \end{table}
    \textbf{\textit{For which values of q is Player 1 indifferent?}}\\\medskip
    Find $q$ such that Player 1 expects to have equal payoffs from playing $T$ and $B$:
    \begin{align*}
      E[u_1|T]&=E[u_1|B]\\
       &=
    \end{align*}
  \vfill\null \columnbreak
  \vfill\null
  \end{multicols}
\end{frame}
\begin{frame}{PS4, Ex. 1.a (A): MSNE and best-response functions}
  \begin{multicols}{2}
    \begin{itemize}
      \item[(a)] Find all equilibria (pure and mixed), first analytically and then through plotting the BR functions.
    \end{itemize}
    \begin{table}
      \begin{tabular}{cl|c|c|}
        & \multicolumn{1}{c}{} & \multicolumn{2}{c}{\color{blue}Player 2}\\
        \parbox[t]{1mm}{\multirow{3}{*}{\rotatebox[origin=r]{90}{\color{red}Player 1}}}
        & \multicolumn{1}{c}{} & \multicolumn{1}{c}{L (q)} & \multicolumn{1}{c}{L (1-q)} \\\cline{3-4}
        & T (p) & \textcolor{red}{3}, \textcolor{blue}{3} & 0, 0 \\\cline{3-4}
        & B (1-p) & 0, 0 & \textcolor{red}{4}, \textcolor{blue}{4} \\\cline{3-4}
      \end{tabular}
    \end{table}
    Find $q$ such that Player 1 expects to have equal payoffs from playing $T$ and $B$:
    \begin{align*}
      E[u_1|T]&=E[u_1|B]\\
      3q &= 4(1-q) \Leftrightarrow q = \frac{4}{7}
    \end{align*}
    \textbf{\textit{Write up all NE (pure and mixed).}}
    \begin{align*}
      NE=(p^{*},q^{*})=
    \end{align*}
  \vfill\null \columnbreak
  \vfill\null
  \end{multicols}
\end{frame}
\begin{frame}{PS4, Ex. 1.a (A): MSNE and best-response functions}
  \begin{multicols}{2}
    \begin{itemize}
      \item[(a)] Find all equilibria (pure and mixed), first analytically and then through plotting the BR functions.
    \end{itemize}
    \begin{table}
      \begin{tabular}{cl|c|c|}
        & \multicolumn{1}{c}{} & \multicolumn{2}{c}{\color{blue}Player 2}\\
        \parbox[t]{1mm}{\multirow{3}{*}{\rotatebox[origin=r]{90}{\color{red}Player 1}}}
        & \multicolumn{1}{c}{} & \multicolumn{1}{c}{L (q)} & \multicolumn{1}{c}{L (1-q)} \\\cline{3-4}
        & T (p) & \textcolor{red}{3}, \textcolor{blue}{3} & 0, 0 \\\cline{3-4}
        & B (1-p) & 0, 0 & \textcolor{red}{4}, \textcolor{blue}{4} \\\cline{3-4}
      \end{tabular}
    \end{table}
    Find $q$ such that Player 1 expects to have equal payoffs from playing $T$ and $B$:
    \begin{align*}
      E[u_1|T]&=E[u_1|B]\\
      3q &= 4(1-q) \Leftrightarrow q = \frac{4}{7}
    \end{align*}
    \textbf{\textit{Write up all NE (pure and mixed).}}\\\medskip
    The players have symmetric payoffs, thus:
    \begin{align*}
      NE=(p^{*},q^{*})=\left\{(0,0);(1,1);...\right\}
    \end{align*}
  \vfill\null \columnbreak
  \vfill\null
  \end{multicols}
\end{frame}
\begin{frame}{PS4, Ex. 1.a (A): MSNE and best-response functions}
  \begin{multicols}{2}
    \begin{itemize}
      \item[(a)] Find all equilibria (pure and mixed), first analytically and then through plotting the BR functions.
    \end{itemize}
    \begin{table}
      \begin{tabular}{cl|c|c|}
        & \multicolumn{1}{c}{} & \multicolumn{2}{c}{\color{blue}Player 2}\\
        \parbox[t]{1mm}{\multirow{3}{*}{\rotatebox[origin=r]{90}{\color{red}Player 1}}}
        & \multicolumn{1}{c}{} & \multicolumn{1}{c}{L (q)} & \multicolumn{1}{c}{R (1-q)} \\\cline{3-4}
        & T (p) & \textcolor{red}{3}, \textcolor{blue}{3} & 0, 0 \\\cline{3-4}
        & B (1-p) & 0, 0 & \textcolor{red}{4}, \textcolor{blue}{4} \\\cline{3-4}
      \end{tabular}
    \end{table}
    Find $q$ such that Player 1 expects to have equal payoffs from playing $T$ and $B$:
    \begin{align*}
      E[u_1|T]&=E[u_1|B]\\
      3q &= 4(1-q) \Leftrightarrow q = \frac{4}{7}
    \end{align*}
    The players have symmetric payoffs, thus:
    \begin{align*}
      NE=(p^{*},q^{*})=\left\{(0,0);(1,1);\left(\frac{4}{7},\frac{4}{7}\right)\right\}
    \end{align*}
    \textbf{\textit{Write up Player 1's best-response (BR) function, $\bm{p^{*}(q)}$}}
  \vfill\null \columnbreak
    \begin{align*}
      BR_1(q)=\left\{ \right.
    \end{align*}
  \vfill\null
  \end{multicols}
\end{frame}
\begin{frame}{PS4, Ex. 1.a (A): MSNE and best-response functions}
  \begin{multicols}{2}
    \begin{itemize}
      \item[(a)] Find all equilibria (pure and mixed), first analytically and then through plotting the BR functions.
    \end{itemize}
    \begin{table}
      \begin{tabular}{cl|c|c|}
        & \multicolumn{1}{c}{} & \multicolumn{2}{c}{\color{blue}Player 2}\\
        \parbox[t]{1mm}{\multirow{3}{*}{\rotatebox[origin=r]{90}{\color{red}Player 1}}}
        & \multicolumn{1}{c}{} & \multicolumn{1}{c}{L (q)} & \multicolumn{1}{c}{L (1-q)} \\\cline{3-4}
        & T (p) & \textcolor{red}{3}, \textcolor{blue}{3} & 0, 0 \\\cline{3-4}
        & B (1-p) & 0, 0 & \textcolor{red}{4}, \textcolor{blue}{4} \\\cline{3-4}
      \end{tabular}
    \end{table}
    Find $q$ such that Player 1 expects to have equal payoffs from playing $T$ and $B$:
    \begin{align*}
      E[u_1|T]&=E[u_1|B]\\
      3q &= 4(1-q) \Leftrightarrow q = \frac{4}{7}
    \end{align*}
    The players have symmetric payoffs, thus:
    \begin{align*}
      NE=(p^{*},q^{*})=\left\{(0,0);(1,1);\left(\frac{4}{7},\frac{4}{7}\right)\right\}
    \end{align*}
    \textbf{\textit{Plot Player 1's best-response (BR) function, $\bm{p^{*}(q)}$}}
  \vfill\null \columnbreak
    Write up and plot the BR functions:
    \vspace{-8pt}
    \begin{align*}
      BR_1(q)=\left\{ \begin{array}{lcl}
          p=0       & \text{if} & q<4/7 \\
          p\in[0,1] & \text{if} & q=4/7 \\
          p = 1     & \text{if} & q>4/7
      \end{array}\right.
    \end{align*}
    \vspace{-8pt}
    \includegraphics[width=\columnwidth]{figures/empty_plot_}
  \vfill\null
  \end{multicols}
\end{frame}
\begin{frame}{PS4, Ex. 1.a (A): MSNE and best-response functions}
  \begin{multicols}{2}
    \begin{itemize}
      \item[(a)] Find all equilibria (pure and mixed), first analytically and then through plotting the BR functions.
    \end{itemize}
    \begin{table}
      \begin{tabular}{cl|c|c|}
        & \multicolumn{1}{c}{} & \multicolumn{2}{c}{\color{blue}Player 2}\\
        \parbox[t]{1mm}{\multirow{3}{*}{\rotatebox[origin=r]{90}{\color{red}Player 1}}}
        & \multicolumn{1}{c}{} & \multicolumn{1}{c}{L (q)} & \multicolumn{1}{c}{L (1-q)} \\\cline{3-4}
        & T (p) & \textcolor{red}{3}, \textcolor{blue}{3} & 0, 0 \\\cline{3-4}
        & B (1-p) & 0, 0 & \textcolor{red}{4}, \textcolor{blue}{4} \\\cline{3-4}
      \end{tabular}
    \end{table}
    Find $q$ such that Player 1 expects to have equal payoffs from playing $T$ and $B$:
    \begin{align*}
      E[u_1|T]&=E[u_1|B]\\
      3q &= 4(1-q) \Leftrightarrow q = \frac{4}{7}
    \end{align*}
    The players have symmetric payoffs, thus:
    \begin{align*}
      NE=(p^{*},q^{*})=\left\{(0,0);(1,1);\left(\frac{4}{7},\frac{4}{7}\right)\right\}
    \end{align*}
    \textbf{\textit{Write up Player 2's BR function, $\bm{q^{*}(p)}$}}
  \vfill\null \columnbreak
    Write up and plot the BR functions:
    \vspace{-8pt}
    \begin{align*}
      BR_1(q)&=\left\{ \begin{array}{lcl}
          p=0       & \text{if} & q<4/7 \\
          p\in[0,1] & \text{if} & q=4/7 \\
          p = 1     & \text{if} & q>4/7
      \end{array}\right. \\
      BR_2(p)&=\left\{ \right.
    \end{align*}
    \vspace{-8pt}
    \includegraphics[width=\columnwidth]{figures/1a_}
  \vfill\null
  \end{multicols}
\end{frame}
\begin{frame}{PS4, Ex. 1.a (A): MSNE and best-response functions}
  \begin{multicols}{2}
    \begin{itemize}
      \item[(a)] Find all equilibria (pure and mixed), first analytically and then through plotting the BR functions.
    \end{itemize}
    \begin{table}
      \begin{tabular}{cl|c|c|}
        & \multicolumn{1}{c}{} & \multicolumn{2}{c}{\color{blue}Player 2}\\
        \parbox[t]{1mm}{\multirow{3}{*}{\rotatebox[origin=r]{90}{\color{red}Player 1}}}
        & \multicolumn{1}{c}{} & \multicolumn{1}{c}{L (q)} & \multicolumn{1}{c}{L (1-q)} \\\cline{3-4}
        & T (p) & \textcolor{red}{3}, \textcolor{blue}{3} & 0, 0 \\\cline{3-4}
        & B (1-p) & 0, 0 & \textcolor{red}{4}, \textcolor{blue}{4} \\\cline{3-4}
      \end{tabular}
    \end{table}
    Find $q$ such that Player 1 expects to have equal payoffs from playing $T$ and $B$:
    \begin{align*}
      E[u_1|T]&=E[u_1|B]\\
      3q &= 4(1-q) \Leftrightarrow q = \frac{4}{7}
    \end{align*}
    The players have symmetric payoffs, thus:
    \begin{align*}
      NE=(p^{*},q^{*})=\left\{(0,0);(1,1);\left(\frac{4}{7},\frac{4}{7}\right)\right\}
    \end{align*}
    \textbf{\textit{Plot Player 2's BR function, $\bm{q^{*}(p)}$}}
  \vfill\null \columnbreak
    Write up and plot the BR functions:
    \vspace{-8pt}
    \begin{align*}
      BR_1(q)=\left\{ \begin{array}{lcl}
          p=0       & \text{if} & q<4/7 \\
          p\in[0,1] & \text{if} & q=4/7 \\
          p = 1     & \text{if} & q>4/7
      \end{array}\right. \\
      BR_2(p)=\left\{ \begin{array}{lcl}
          q=0       & \text{if} & p<4/7  \\
          q\in[0,1] & \text{if} & p=4/7 \\
          q = 1     & \text{if} & p>4/7
      \end{array}\right.
    \end{align*}
    \vspace{-8pt}
    \includegraphics[width=\columnwidth]{figures/1a_}
  \vfill\null
  \end{multicols}
\end{frame}
\begin{frame}{PS4, Ex. 1.a (A): MSNE and best-response functions}
  \begin{multicols}{2}
    \begin{itemize}
      \item[(a)] Find all equilibria (pure and mixed), first analytically and then through plotting the BR functions.
    \end{itemize}
    \begin{table}
      \begin{tabular}{cl|c|c|}
        & \multicolumn{1}{c}{} & \multicolumn{2}{c}{\color{blue}Player 2}\\
        \parbox[t]{1mm}{\multirow{3}{*}{\rotatebox[origin=r]{90}{\color{red}Player 1}}}
        & \multicolumn{1}{c}{} & \multicolumn{1}{c}{L (q)} & \multicolumn{1}{c}{L (1-q)} \\\cline{3-4}
        & T (p) & \textcolor{red}{3}, \textcolor{blue}{3} & 0, 0 \\\cline{3-4}
        & B (1-p) & 0, 0 & \textcolor{red}{4}, \textcolor{blue}{4} \\\cline{3-4}
      \end{tabular}
    \end{table}
    Find $q$ such that Player 1 expects to have equal payoffs from playing $T$ and $B$:
    \begin{align*}
      E[u_1|T]&=E[u_1|B]\\
      3q &= 4(1-q) \Leftrightarrow q = \frac{4}{7}
    \end{align*}
    The players have symmetric payoffs, thus:
    \begin{align*}
      NE=(p^{*},q^{*})=\left\{(0,0);(1,1);\left(\frac{4}{7},\frac{4}{7}\right)\right\}
    \end{align*}
  \vfill\null \columnbreak
    Write up and plot the BR functions:
    \vspace{-8pt}
    \begin{align*}
      BR_1(q)=\left\{ \begin{array}{lcl}
          p=0       & \text{if} & q<4/7 \\
          p\in[0,1] & \text{if} & q=4/7 \\
          p = 1     & \text{if} & q>4/7
      \end{array}\right. \\
      BR_2(p)=\left\{ \begin{array}{lcl}
          q=0       & \text{if} & p<4/7  \\
          q\in[0,1] & \text{if} & p=4/7 \\
          q = 1     & \text{if} & p>4/7
      \end{array}\right.
    \end{align*}
    \vspace{-8pt}
    \includegraphics[width=\columnwidth]{figures/1a}
  \vfill\null
  \end{multicols}
\end{frame}

\begin{frame}{PS4, Ex. 1.b (A): MSNE and best-response functions}
  \begin{multicols}{2}
    \begin{itemize}
      \item[(b)] Find all equilibria (pure and mixed), first analytically and then through plotting the BR functions.
    \end{itemize}
    \vspace{-8pt}
    \begin{table}
      \begin{tabular}{cl|c|c|}
          & \multicolumn{1}{c}{} & \multicolumn{2}{c}{Player 2}\\
          \parbox[t]{1mm}{\multirow{3}{*}{\rotatebox[origin=r]{90}{Player 1}}}
          & \multicolumn{1}{c}{} & \multicolumn{1}{c}{L (q)} & \multicolumn{1}{c}{R (1-q)} \\\cline{3-4}
          & T (p) & 1, 1 & 0, 0 \\\cline{3-4}
          & B (1-p) & 1, 0 & 2, 1 \\\cline{3-4}
      \end{tabular}
    \end{table}
    \textbf{\textit{Highlight the best responses in pure strategies.}}
  \vfill\null \columnbreak
  \vfill\null
  \end{multicols}
\end{frame}
\begin{frame}{PS4, Ex. 1.b (A): MSNE and best-response functions}
  \begin{multicols}{2}
    \begin{itemize}
      \item[(b)] Find all equilibria (pure and mixed), first analytically and then through plotting the BR functions.
    \end{itemize}
    \vspace{-8pt}
    \begin{table}
      \begin{tabular}{cl|c|c|}
        & \multicolumn{1}{c}{} & \multicolumn{2}{c}{\color{blue}Player 2}\\
        \parbox[t]{1mm}{\multirow{3}{*}{\rotatebox[origin=r]{90}{\color{red}Player 1}}}
        & \multicolumn{1}{c}{} & \multicolumn{1}{c}{L (q)} & \multicolumn{1}{c}{R (1-q)} \\\cline{3-4}
        & T (p) & \textcolor{red}{1}, \textcolor{blue}{1} & 0, 0 \\\cline{3-4}
        & B (1-p) & \textcolor{red}{1}, 0 & \textcolor{red}{2}, \textcolor{blue}{1} \\\cline{3-4}
      \end{tabular}
    \end{table}
    \textbf{\textit{For which values of q is Player 1 indifferent?}}\\\medskip
    Find $q$ such that Player 1 expects to have equal payoffs from playing $T$ and $B$:
    \begin{align*}
      E[u_1|T]&=E[u_1|B]\\
       &=
    \end{align*}
  \vfill\null \columnbreak
  \vfill\null
  \end{multicols}
\end{frame}
\begin{frame}{PS4, Ex. 1.b (A): MSNE and best-response functions}
  \begin{multicols}{2}
    \begin{itemize}
      \item[(b)] Find all equilibria (pure and mixed), first analytically and then through plotting the BR functions.
    \end{itemize}
    \vspace{-8pt}
    \begin{table}
      \begin{tabular}{cl|c|c|}
        & \multicolumn{1}{c}{} & \multicolumn{2}{c}{\color{blue}Player 2}\\
        \parbox[t]{1mm}{\multirow{3}{*}{\rotatebox[origin=r]{90}{\color{red}Player 1}}}
        & \multicolumn{1}{c}{} & \multicolumn{1}{c}{L (q)} & \multicolumn{1}{c}{R (1-q)} \\\cline{3-4}
        & T (p) & \textcolor{red}{1}, \textcolor{blue}{1} & 0, 0 \\\cline{3-4}
        & B (1-p) & \textcolor{red}{1}, 0 & \textcolor{red}{2}, \textcolor{blue}{1} \\\cline{3-4}
      \end{tabular}
    \end{table}
    Find $q$ such that Player 1 expects to have equal payoffs from playing $T$ and $B$:
    \begin{align*}
      E[u_1|T]&=E[u_1|B]\\
      q &= q + 2(1-q) \Leftrightarrow q = 1
    \end{align*}
    \textbf{\textit{For which values of p is Player 2 indifferent?}}\\\medskip
    Find $p$ such that Player 2 expect to have equal payoffs from playing $L$ and $R$:
    \begin{align*}
      E[u_2|L]&=E[u_2|R]\\
       &=
    \end{align*}
  \vfill\null \columnbreak
  \vfill\null
  \end{multicols}
\end{frame}
\begin{frame}{PS4, Ex. 1.b (A): MSNE and best-response functions}
  \begin{multicols}{2}
    \begin{itemize}
      \item[(b)] Find all equilibria (pure and mixed), first analytically and then through plotting the BR functions.
    \end{itemize}
    \vspace{-8pt}
    \begin{table}
      \begin{tabular}{cl|c|c|}
        & \multicolumn{1}{c}{} & \multicolumn{2}{c}{\color{blue}Player 2}\\
        \parbox[t]{1mm}{\multirow{3}{*}{\rotatebox[origin=r]{90}{\color{red}Player 1}}}
        & \multicolumn{1}{c}{} & \multicolumn{1}{c}{L (q)} & \multicolumn{1}{c}{R (1-q)} \\\cline{3-4}
        & T (p) & \textcolor{red}{1}, \textcolor{blue}{1} & 0, 0 \\\cline{3-4}
        & B (1-p) & \textcolor{red}{1}, 0 & \textcolor{red}{2}, \textcolor{blue}{1} \\\cline{3-4}
      \end{tabular}
    \end{table}
    Find $q$ such that Player 1 expects to have equal payoffs from playing $T$ and $B$:
    \begin{align*}
      E[u_1|T]&=E[u_1|B]\\
      q &= q + 2(1-q) \Leftrightarrow q = 1
    \end{align*}
    Find $p$ such that Player 2 expect to have equal payoffs from playing $L$ and $R$:
    \begin{align*}
      E[u_2|L]&=E[u_2|R]\\
      p &= 1-p \Leftrightarrow p = \frac{1}{2}
    \end{align*}
    and chooses $q=1$ for $p>1/2$.\\\medskip
    \textbf{\textit{Write up all NE (pure and mixed).}}
  \vfill\null \columnbreak
  \vfill\null
  \end{multicols}
\end{frame}
\begin{frame}{PS4, Ex. 1.b (A): MSNE and best-response functions}
  \begin{multicols}{2}
    \begin{itemize}
      \item[(b)] Find all NE, first analytically:
    \end{itemize}
    \vspace{-8pt}
    \begin{table}
      \begin{tabular}{cl|c|c|}
        & \multicolumn{1}{c}{} & \multicolumn{2}{c}{\color{blue}Player 2}\\
        \parbox[t]{1mm}{\multirow{3}{*}{\rotatebox[origin=r]{90}{\color{red}Player 1}}}
        & \multicolumn{1}{c}{} & \multicolumn{1}{c}{L (q)} & \multicolumn{1}{c}{R (1-q)} \\\cline{3-4}
        & T (p) & \textcolor{red}{1}, \textcolor{blue}{1} & 0, 0 \\\cline{3-4}
        & B (1-p) & \textcolor{red}{1}, 0 & \textcolor{red}{2}, \textcolor{blue}{1} \\\cline{3-4}
      \end{tabular}
    \end{table}
    Player 1 is indifferent for:
    \begin{align*}
      E[u_1|T]&=E[u_1|B]\\
      q &= q + 2(1-q) \Leftrightarrow q = 1
    \end{align*}
    Player 2 is indifferent for:
    \begin{align*}
      E[u_2|L]&=E[u_2|R]\\
      p &= 1-p \Leftrightarrow p = \frac{1}{2}
    \end{align*}
    and chooses $q=1$ for $p>1/2$.\\\medskip
    The pure and mixed NE, $(p^{*},q^{*})$, are:
    \begin{align*}
      \left\{(0,0);(1,1);\left(p\in\left[\frac{1}{2},1\right),q=1\right)\right\}
    \end{align*}
  \vfill\null \columnbreak
  \vfill\null
  \end{multicols}
\end{frame}
\begin{frame}{PS4, Ex. 1.b (A): MSNE and best-response functions}
  \begin{multicols}{2}
    \begin{itemize}
      \item[(b)] Find all NE, first analytically:
    \end{itemize}
    \vspace{-8pt}
    \begin{table}
      \begin{tabular}{cl|c|c|}
        & \multicolumn{1}{c}{} & \multicolumn{2}{c}{\color{blue}Player 2}\\
        \parbox[t]{1mm}{\multirow{3}{*}{\rotatebox[origin=r]{90}{\color{red}Player 1}}}
        & \multicolumn{1}{c}{} & \multicolumn{1}{c}{L (q)} & \multicolumn{1}{c}{R (1-q)} \\\cline{3-4}
        & T (p) & \textcolor{red}{1}, \textcolor{blue}{1} & 0, 0 \\\cline{3-4}
        & B (1-p) & \textcolor{red}{1}, 0 & \textcolor{red}{2}, \textcolor{blue}{1} \\\cline{3-4}
      \end{tabular}
    \end{table}
    Player 1 is indifferent for:
    \begin{align*}
      E[u_1|T]&=E[u_1|B]\\
      q &= q + 2(1-q) \Leftrightarrow q = 1
    \end{align*}
    Player 2 is indifferent for:
    \begin{align*}
      E[u_2|L]&=E[u_2|R]\\
      p &= 1-p \Leftrightarrow p = \frac{1}{2}
    \end{align*}
    and chooses $q=1$ for $p>1/2$.\\\medskip
    The pure and mixed NE, $(p^{*},q^{*})$, are:
    \begin{align*}
      \left\{(0,0);(1,1);\left(p\in\left[\frac{1}{2},1\right),q=1\right)\right\}
    \end{align*}
    \textbf{\textit{Write up Player 1's best-response (BR) function, $\bm{p^{*}(q)}$}}
  \vfill\null \columnbreak
    \begin{align*}
      BR_1(q)=\left\{ \right.
    \end{align*}
  \vfill\null
  \end{multicols}
\end{frame}
\begin{frame}{PS4, Ex. 1.b (A): MSNE and best-response functions}
  \begin{multicols}{2}
    \begin{itemize}
      \item[(b)] Find all NE, first analytically:
    \end{itemize}
    \vspace{-8pt}
    \begin{table}
      \begin{tabular}{cl|c|c|}
        & \multicolumn{1}{c}{} & \multicolumn{2}{c}{\color{blue}Player 2}\\
        \parbox[t]{1mm}{\multirow{3}{*}{\rotatebox[origin=r]{90}{\color{red}Player 1}}}
        & \multicolumn{1}{c}{} & \multicolumn{1}{c}{L (q)} & \multicolumn{1}{c}{R (1-q)} \\\cline{3-4}
        & T (p) & \textcolor{red}{1}, \textcolor{blue}{1} & 0, 0 \\\cline{3-4}
        & B (1-p) & \textcolor{red}{1}, 0 & \textcolor{red}{2}, \textcolor{blue}{1} \\\cline{3-4}
      \end{tabular}
    \end{table}
    Player 1 is indifferent for:
    \begin{align*}
      E[u_1|T]&=E[u_1|B]\\
      q &= q + 2(1-q) \Leftrightarrow q = 1
    \end{align*}
    Player 2 is indifferent for:
    \begin{align*}
      E[u_2|L]&=E[u_2|R]\\
      p &= 1-p \Leftrightarrow p = \frac{1}{2}
    \end{align*}
    and chooses $q=1$ for $p>1/2$.\\\medskip
    The pure and mixed NE, $(p^{*},q^{*})$, are:
    \begin{align*}
      \left\{(0,0);(1,1);\left(p\in\left[\frac{1}{2},1\right),q=1\right)\right\}
    \end{align*}
    \textbf{\textit{Plot Player 1's best-response (BR) function, $\bm{p^{*}(q)}$}}
  \vfill\null \columnbreak
    Then through plotting the BR functions:
    \vspace{-8pt}
    \begin{align*}
      BR_1(q)=\left\{ \begin{array}{lcl}
          p=0       & \text{if} & q<1 \\
          p\in[0,1] & \text{if} & q=1
      \end{array}\right.
    \end{align*}
    \vspace{-8pt}
    \includegraphics[width=\columnwidth]{figures/empty_plot_}
  \vfill\null
  \end{multicols}
\end{frame}
\begin{frame}{PS4, Ex. 1.b (A): MSNE and best-response functions}
  \begin{multicols}{2}
    \begin{itemize}
      \item[(b)] Find all NE, first analytically:
    \end{itemize}
    \vspace{-8pt}
    \begin{table}
      \begin{tabular}{cl|c|c|}
        & \multicolumn{1}{c}{} & \multicolumn{2}{c}{\color{blue}Player 2}\\
        \parbox[t]{1mm}{\multirow{3}{*}{\rotatebox[origin=r]{90}{\color{red}Player 1}}}
        & \multicolumn{1}{c}{} & \multicolumn{1}{c}{L (q)} & \multicolumn{1}{c}{R (1-q)} \\\cline{3-4}
        & T (p) & \textcolor{red}{1}, \textcolor{blue}{1} & 0, 0 \\\cline{3-4}
        & B (1-p) & \textcolor{red}{1}, 0 & \textcolor{red}{2}, \textcolor{blue}{1} \\\cline{3-4}
      \end{tabular}
    \end{table}
    Player 1 is indifferent for:
    \begin{align*}
      E[u_1|T]&=E[u_1|B]\\
      q &= q + 2(1-q) \Leftrightarrow q = 1
    \end{align*}
    Player 2 is indifferent for:
    \begin{align*}
      E[u_2|L]&=E[u_2|R]\\
      p &= 1-p \Leftrightarrow p = \frac{1}{2}
    \end{align*}
    and chooses $q=1$ for $p>1/2$.\\\medskip
    The pure and mixed NE, $(p^{*},q^{*})$, are:
    \begin{align*}
      \left\{(0,0);(1,1);\left(p\in\left[\frac{1}{2},1\right),q=1\right)\right\}
    \end{align*}
    \textbf{\textit{Write up Player 2's BR function, $\bm{q^{*}(p)}$}}
  \vfill\null \columnbreak
    Then through plotting the BR functions:
    \vspace{-8pt}
    \begin{align*}
      BR_1(q)&=\left\{ \begin{array}{lcl}
          p=0       & \text{if} & q<1 \\
          p\in[0,1] & \text{if} & q=1
      \end{array}\right.\\
      BR_2(p)&=\left\{\right.\\
    \end{align*}
    \vspace{-8pt}
    \includegraphics[width=\columnwidth]{figures/1b_}
  \vfill\null
  \end{multicols}
\end{frame}
\begin{frame}{PS4, Ex. 1.b (A): MSNE and best-response functions}
  \begin{multicols}{2}
    \begin{itemize}
      \item[(b)] Find all NE, first analytically:
    \end{itemize}
    \vspace{-8pt}
    \begin{table}
      \begin{tabular}{cl|c|c|}
        & \multicolumn{1}{c}{} & \multicolumn{2}{c}{\color{blue}Player 2}\\
        \parbox[t]{1mm}{\multirow{3}{*}{\rotatebox[origin=r]{90}{\color{red}Player 1}}}
        & \multicolumn{1}{c}{} & \multicolumn{1}{c}{L (q)} & \multicolumn{1}{c}{R (1-q)} \\\cline{3-4}
        & T (p) & \textcolor{red}{1}, \textcolor{blue}{1} & 0, 0 \\\cline{3-4}
        & B (1-p) & \textcolor{red}{1}, 0 & \textcolor{red}{2}, \textcolor{blue}{1} \\\cline{3-4}
      \end{tabular}
    \end{table}
    Player 1 is indifferent for:
    \begin{align*}
      E[u_1|T]&=E[u_1|B]\\
      q &= q + 2(1-q) \Leftrightarrow q = 1
    \end{align*}
    Player 2 is indifferent for:
    \begin{align*}
      E[u_2|L]&=E[u_2|R]\\
      p &= 1-p \Leftrightarrow p = \frac{1}{2}
    \end{align*}
    and chooses $q=1$ for $p>1/2$.\\\medskip
    The pure and mixed NE, $(p^{*},q^{*})$, are:
    \begin{align*}
      \left\{(0,0);(1,1);\left(p\in\left[\frac{1}{2},1\right),q=1\right)\right\}
    \end{align*}
    \textbf{\textit{Plot Player 2's BR function, $\bm{q^{*}(p)}$}}
  \vfill\null \columnbreak
    Then through plotting the BR functions:
    \vspace{-8pt}
    \begin{align*}
      BR_1(q)&=\left\{ \begin{array}{lcl}
          p=0       & \text{if} & q<1 \\
          p\in[0,1] & \text{if} & q=1
      \end{array}\right. \\
      BR_2(p)&=\left\{ \begin{array}{lcl}
          q=0       & \text{if} & p<1/2  \\
          q\in[0,1] & \text{if} & p=1/2 \\
          q=1       & \text{if} & p>1/2
      \end{array}\right.
    \end{align*}
    \vspace{-8pt}
    \includegraphics[width=\columnwidth]{figures/1b_}
  \vfill\null
  \end{multicols}
\end{frame}
\begin{frame}{PS4, Ex. 1.b (A): MSNE and best-response functions}
  \begin{multicols}{2}
    \begin{itemize}
      \item[(b)] Find all NE, first analytically:
    \end{itemize}
    \vspace{-8pt}
    \begin{table}
      \begin{tabular}{cl|c|c|}
        & \multicolumn{1}{c}{} & \multicolumn{2}{c}{\color{blue}Player 2}\\
        \parbox[t]{1mm}{\multirow{3}{*}{\rotatebox[origin=r]{90}{\color{red}Player 1}}}
        & \multicolumn{1}{c}{} & \multicolumn{1}{c}{L (q)} & \multicolumn{1}{c}{R (1-q)} \\\cline{3-4}
        & T (p) & \textcolor{red}{1}, \textcolor{blue}{1} & 0, 0 \\\cline{3-4}
        & B (1-p) & \textcolor{red}{1}, 0 & \textcolor{red}{2}, \textcolor{blue}{1} \\\cline{3-4}
      \end{tabular}
    \end{table}
    Player 1 is indifferent for:
    \begin{align*}
      E[u_1|T]&=E[u_1|B]\\
      q &= q + 2(1-q) \Leftrightarrow q = 1
    \end{align*}
    Player 2 is indifferent for:
    \begin{align*}
      E[u_2|L]&=E[u_2|R]\\
      p &= 1-p \Leftrightarrow p = \frac{1}{2}
    \end{align*}
    and chooses $q=1$ for $p>1/2$.\\\medskip
    The pure and mixed NE, $(p^{*},q^{*})$, are:
    \begin{align*}
      \left\{(0,0);(1,1);\left(p\in\left[\frac{1}{2},1\right),q=1\right)\right\}
    \end{align*}
  \vfill\null \columnbreak
    Then through plotting the BR functions:
    \vspace{-8pt}
    \begin{align*}
      BR_1(q)&=\left\{ \begin{array}{lcl}
          p=0       & \text{if} & q<1 \\
          p\in[0,1] & \text{if} & q=1
      \end{array}\right. \\
      BR_2(p)&=\left\{ \begin{array}{lcl}
          q=0       & \text{if} & p<1/2  \\
          q\in[0,1] & \text{if} & p=1/2 \\
          q=1       & \text{if} & p>1/2
      \end{array}\right.
    \end{align*}
    \vspace{-8pt}
    \includegraphics[width=\columnwidth]{figures/1b}
  \vfill\null
  \end{multicols}
\end{frame}



\section{PS4, Ex. 2: Entry deterrence (backwards induction)}

\begin{frame}{PS4, Ex. 2: Entry deterrence (backwards induction)}
  \begin{multicols}{2}
    Consider the following dynamic game: firm 1 owns a shop in town A. Firm 2 decides whether to enter the market in town A. If firm 2 enters, firm 1 chooses whether to fight or accommodate the entrant. If firm 2 does not enter, firm 1 receives a profit of 2 and firm 2 gets 0. If firm 2 enters and firm 1 accommodates, they share the market and each of them receives a profit of 1. If firm 2 enters and firm 1 decides to fight, firm 2 suffers a loss of 1 (so that the payoff is -1), but fighting is costly for firm 1, lowering its payoff to 0.
    \begin{itemize}
      \item[(a)] Draw the game tree.
      \item[(b)] Solve the game by backwards induction.
    \end{itemize}
  \vfill\null \columnbreak
  \vfill\null
  \end{multicols}
\end{frame}
\begin{frame}{PS4, Ex. 2: Entry deterrence (backwards induction)}
  \begin{multicols}{2}
    Consider the following dynamic game: firm 1 owns a shop in town A. Firm 2 decides whether to enter the market in town A. If firm 2 enters, firm 1 chooses whether to fight or accommodate the entrant. If firm 2 does not enter, firm 1 receives a profit of 2 and firm 2 gets 0. If firm 2 enters and firm 1 accommodates, they share the market and each of them receives a profit of 1. If firm 2 enters and firm 1 decides to fight, firm 2 suffers a loss of 1 (so that the payoff is -1), but fighting is costly for firm 1, lowering its payoff to 0.
    \begin{itemize}
      \item[(a)] Draw the game tree.
      \item[(b)] Solve the game by backwards induction.
    \end{itemize}
  \vfill\null \columnbreak
    \begin{figure}[!h]
      \begin{center}
      \def\svgwidth{1.0\columnwidth}
      \import{figures/}{game_tree.pdf_tex}
      \end{center}
    \end{figure}
  \vfill\null
  \end{multicols}
\end{frame}
\begin{frame}{PS4, Ex. 2: Entry deterrence (backwards induction)}
  \begin{multicols}{2}
    \begin{itemize}
      \item[(a)] Draw the game tree.
      \item[(b)] Solve the game by backwards induction.
    \end{itemize}
    \textbf{Starting from the bottom}: If Firm 2 has entered the market in the \nth{1} round, then Firm 1 can choose to either fight or accommodate in the \nth{2} round.
  \vfill\null \columnbreak
    \begin{figure}[!h]
      \begin{center}
      \def\svgwidth{1.0\columnwidth}
      \import{figures/}{game_tree.pdf_tex}
      \end{center}
    \end{figure}
  \vfill\null
  \end{multicols}
\end{frame}
\begin{frame}{PS4, Ex. 2: Entry deterrence (backwards induction)}
  \begin{multicols}{2}
    \begin{itemize}
      \item[(a)] Draw the game tree.
      \item[(b)] Solve the game by backwards induction.
    \end{itemize}
    \textbf{Starting from the bottom}: If Firm 2 has entered the market in the \nth{1} round, then Firm 1 can choose to either fight or accommodate in the \nth{2} round.\\\medskip
    \textbf{Firm 1} will always accommodate, as it is more costly to fight $(1>0)$.
  \vfill\null \columnbreak
    \begin{figure}[!h]
      \begin{center}
      \def\svgwidth{1.0\columnwidth}
      \import{figures/}{game_tree.pdf_tex}
      \end{center}
    \end{figure}
  \vfill\null
  \end{multicols}
\end{frame}
\begin{frame}{PS4, Ex. 2: Entry deterrence (backwards induction)}
  \begin{multicols}{2}
    \begin{itemize}
      \item[(a)] Draw the game tree.
      \item[(b)] Solve the game by backwards induction.
    \end{itemize}
    \textbf{Starting from the bottom}: If Firm 2 has entered the market in the \nth{1} round, then Firm 1 can choose to either fight or accommodate in the \nth{2} round.\\\medskip
    \textbf{Firm 1} will always accommodate, as it is more costly to fight $(1>0)$.\\\medskip
    Knowing that Firm 1 is rational and will accommodate in the \nth{2} round, \textbf{Firm 2} (first mover), will always chose to enter in the \nth{1} round $(1>0)$, i.e. the backwards induction solution is the strategy profile:
      \begin{align*}
        (s_1,s_2)=(Accommodate,Enter)
      \end{align*}
  \vfill\null \columnbreak
    \begin{figure}[!h]
      \begin{center}
      \def\svgwidth{1.0\columnwidth}
      \import{figures/}{game_tree.pdf_tex}
      \end{center}
    \end{figure}
  \vfill\null
  \end{multicols}
\end{frame}
\begin{frame}{PS4, Ex. 2: Entry deterrence (backwards induction)}
  \begin{multicols}{2}
    \begin{itemize}
      \item[(a)] Draw the game tree.
      \item[(b)] Solve the game by backwards induction.
    \end{itemize}
    \textbf{Starting from the bottom}: If Firm 2 has entered the market in the \nth{1} round, then Firm 1 can choose to either fight or accommodate in the \nth{2} round.\\\medskip
    \textbf{Firm 1} will always accommodate, as it is more costly to fight $(1>0)$.\\\medskip
    Knowing that Firm 1 is rational and will accommodate in the \nth{2} round, \textbf{Firm 2} (first mover), will always chose to enter in the \nth{1} round $(1>0)$, i.e. the backwards induction solution is the strategy profile:
      \begin{align*}
        (s_1,s_2)=(Accommodate,Enter)
      \end{align*}
    \textbf{Intuition:} Firm 2 has \textit{first mover advantage}, thus, to "Fight" would not be a credible threat given Firm 1 is rational. I.e. Firm 2's decision can be reduced to the upper part of the game tree.
  \vfill\null \columnbreak
    \begin{figure}[!h]
      \begin{center}
      \def\svgwidth{1.0\columnwidth}
      \import{figures/}{game_tree.pdf_tex}
      \end{center}
    \end{figure}
    \begin{figure}[!h]
      \begin{center}
      \def\svgwidth{1.0\columnwidth}
      \import{figures/}{game_tree_reduced.pdf_tex}
      \end{center}
    \end{figure}
  \vfill\null
  \end{multicols}
\end{frame}
\begin{frame}{PS4, Ex. 2 extra: Choices off the equilibrium path}
  \begin{multicols}{2}
    \begin{itemize}
      \item[(c)] What is the solution now?
    \end{itemize}
    \textbf{Looking at the new choices}: If Firm 2 chooses to not enter in the \nth{1} round, then Firm 1 can choose to either continue as normal or shut down in the \nth{2} round, effectively handing over the whole market to Firm 2 instead.
  \vfill\null \columnbreak
    \begin{figure}[!h]
      \begin{center}
      \def\svgwidth{1.0\columnwidth}
      \import{figures/}{game_tree_extended.pdf_tex}
      \end{center}
    \end{figure}
  \vfill\null
  \end{multicols}
\end{frame}
\begin{frame}{PS4, Ex. 2 extra: Choices off the equilibrium path}
  \begin{multicols}{2}
    \begin{itemize}
      \item[(c)] What is the solution now?
    \end{itemize}
    \textbf{Looking at the new choices}: If Firm 2 chooses to not enter in the \nth{1} round, then Firm 1 can choose to either continue as normal or shut down in the \nth{2} round.\\\medskip
    \textbf{Firm 1} will always continue, as it will gain nothing in a shutdown $(2>0)$.\\\medskip
    Knowing that Firm 1 is rational and will choose to continue in the \nth{2} round, \textbf{Firm 2} (first mover), would get 0 by not entering in the \nth{1} round, so to enter in the \nth{1} round will be the best response $(1>0)$.\\\medskip
    What is the full strategy profile for the backwards induction solution?
  \vfill\null \columnbreak
    \begin{figure}[!h]
      \begin{center}
      \def\svgwidth{1.0\columnwidth}
      \import{figures/}{game_tree_extended_color.pdf_tex}
      \end{center}
    \end{figure}
  \vfill\null
  \end{multicols}
\end{frame}
\begin{frame}{PS4, Ex. 2 extra: Choices off the equilibrium path}
  \begin{multicols}{2}
    \begin{itemize}
      \item[(a)] What is the solution now?
    \end{itemize}
    \textbf{Looking at the new choices}: If Firm 2 chooses to not enter in the \nth{1} round, then Firm 1 can choose to either continue as normal or shut down in the \nth{2} round.\\\medskip
    \textbf{Firm 1} will always continue, as it will gain nothing in a shutdown $(2>0)$.\\\medskip
    Knowing that Firm 1 is rational and will choose to continue in the \nth{2} round, \textbf{Firm 2} (first mover), would get 0 by not entering in the \nth{1} round, so to enter in the \nth{1} round will be the best response $(1>0)$, i.e. the backwards induction solution is the full strategy profile:
      \begin{align*}
        (s_1,s_2)=("Accommodate""Continue","Enter")
      \end{align*}
    \textbf{Off the equilibrium path:} The strategy profile now reflect choices off the equilibrium path, this is done because firm 1's choices off the equilibrium path might be relevant to the equilibrium path.
  \vfill\null \columnbreak
    \begin{figure}[!h]
      \begin{center}
      \def\svgwidth{1.0\columnwidth}
      \import{figures/}{game_tree_extended_color.pdf_tex}
      \end{center}
    \end{figure}
  \vfill\null
  \end{multicols}
\end{frame}



\section{PS4, Ex. 3: The Focal Point (plotting BR functions)}

\begin{frame}{PS4, Ex. 3: The Focal Point (plotting BR functions)}
  \begin{multicols}{2}
    Thomas and Alice want to meet on a Friday night. There are two bars in their home town: “The Focal Point” and “The Other Place”. They have to decide independently where they go. If they meet in the same bar, they both get utility of 1. If they end up in different bars, they get utility of 0.
    \begin{itemize}
      \item[(a)] Find all equilibria (pure and mixed). Which equilibrium do you consider the most realistic? Where would you go if you were one of them?
      \item[(b)] Now assume that Thomas wants to meet Alice, but Alice does not want to meet Thomas. Thomas gets a payoff of 1 if he meets Alice, and -1 otherwise. Alice gets a payoff of -1 for meeting Thomas, and 1 otherwise. Find all equilibria (pure and mixed).
    \end{itemize}
  \vfill\null \columnbreak
  \begin{itemize}
    \item[(c)] Now assume again that Thomas and Alice both want to meet (so that payoffs are as in part (a)), but now there are $N$ bars in town, where $N$ can be very large. Show that there are $2^N-1$ equilibria (pure and mixed). Say that the bars have names: “The First Bar in Town”, “The Second Bar in Town”, and so on. Which equilibrium is the most realistic?
  \end{itemize}
  \includegraphics[width=\columnwidth]{figures/other_place}
  \vfill\null
  \end{multicols}
\end{frame}

\begin{frame}{PS4, Ex. 3.a: The Focal Point (plotting BR functions)}
  \begin{multicols}{2}
    Thomas and Alice want to meet on a Friday night. There are two bars in their home town: “The Focal Point” and “The Other Place”. They have to decide independently where they go. If they meet in the same bar, they both get utility of 1. If they end up in different bars, they get utility of 0.
    \begin{itemize}
      \item[(a)] Find all equilibria (pure and mixed). Which equilibrium do you consider the most realistic? Where would you go if you were one of them?
    \end{itemize}
    \begin{table}
      \begin{tabular}{cl|c|c|}
        & \multicolumn{1}{c}{} & \multicolumn{2}{c}{\color{blue}Thomas}\\
        \parbox[t]{1mm}{\multirow{3}{*}{\rotatebox[origin=r]{90}{\color{red}Alice}}}
        & \multicolumn{1}{c}{} & \multicolumn{1}{c}{F (q)} & \multicolumn{1}{c}{O (1-q)} \\\cline{3-4}
        & F (p) & \textcolor{red}{1}, \textcolor{blue}{1} & 0, 0 \\\cline{3-4}
        & O (1-p) & 0, 0 & \textcolor{red}{1}, \textcolor{blue}{1} \\\cline{3-4}
      \end{tabular}
    \end{table}
  \vfill\null \columnbreak
    \textbf{\textit{For which values of q is Alice indifferent?}}
    \begin{align*}
      E[u_A|Focal]&=E[u_A|Other]\\
       &=
    \end{align*}
  \vfill\null
  \end{multicols}
\end{frame}
\begin{frame}{PS4, Ex. 3.a: The Focal Point (plotting BR functions)}
  \begin{multicols}{2}
    Thomas and Alice want to meet on a Friday night. There are two bars in their home town: “The Focal Point” and “The Other Place”. They have to decide independently where they go. If they meet in the same bar, they both get utility of 1. If they end up in different bars, they get utility of 0.
    \begin{itemize}
      \item[(a)] Find all equilibria (pure and mixed). Which equilibrium do you consider the most realistic? Where would you go if you were one of them?
    \end{itemize}
    \begin{table}
      \begin{tabular}{cl|c|c|}
        & \multicolumn{1}{c}{} & \multicolumn{2}{c}{\color{blue}Thomas}\\
        \parbox[t]{1mm}{\multirow{3}{*}{\rotatebox[origin=r]{90}{\color{red}Alice}}}
        & \multicolumn{1}{c}{} & \multicolumn{1}{c}{F (q)} & \multicolumn{1}{c}{O (1-q)} \\\cline{3-4}
        & F (p) & \textcolor{red}{1}, \textcolor{blue}{1} & 0, 0 \\\cline{3-4}
        & O (1-p) & 0, 0 & \textcolor{red}{1}, \textcolor{blue}{1} \\\cline{3-4}
      \end{tabular}
    \end{table}
  \vfill\null \columnbreak
    Alice is indifferent for:
    \begin{align*}
      E[u_A|Focal]&=E[u_A|Other]\\
      q &= 1-q \Leftrightarrow q = \frac{1}{2}
    \end{align*}
    \textbf{\textit{Write up all NE (pure and mixed).}}
    \begin{align*}
      NE=(p^{*},q^{*})=\left\{...\right\}
    \end{align*}
  \vfill\null
  \end{multicols}
\end{frame}
\begin{frame}{PS4, Ex. 3.a: The Focal Point (plotting BR functions)}
  \begin{multicols}{2}
    Thomas and Alice want to meet on a Friday night. There are two bars in their home town: “The Focal Point” and “The Other Place”. They have to decide independently where they go. If they meet in the same bar, they both get utility of 1. If they end up in different bars, they get utility of 0.
    \begin{itemize}
      \item[(a)] Find all equilibria (pure and mixed). Which equilibrium do you consider the most realistic? Where would you go if you were one of them?
    \end{itemize}
    \begin{table}
      \begin{tabular}{cl|c|c|}
        & \multicolumn{1}{c}{} & \multicolumn{2}{c}{\color{blue}Thomas}\\
        \parbox[t]{1mm}{\multirow{3}{*}{\rotatebox[origin=r]{90}{\color{red}Alice}}}
        & \multicolumn{1}{c}{} & \multicolumn{1}{c}{F (q)} & \multicolumn{1}{c}{O (1-q)} \\\cline{3-4}
        & F (p) & \textcolor{red}{1}, \textcolor{blue}{1} & 0, 0 \\\cline{3-4}
        & O (1-p) & 0, 0 & \textcolor{red}{1}, \textcolor{blue}{1} \\\cline{3-4}
      \end{tabular}
    \end{table}
  \vfill\null \columnbreak
    Alice is indifferent for:
    \begin{align*}
      E[u_A|Focal]&=E[u_A|Other]\\
      q &= 1-q \Leftrightarrow q = \frac{1}{2}
    \end{align*}
    Taking advantage of symmetry:
    \begin{align*}
      NE=(p^{*},q^{*})=\left\{(0,0);(1,1);\left(\frac{1}{2},\frac{1}{2}\right)\right\}
    \end{align*}
    \textbf{\textit{Which is the most realistic?}}
  \vfill\null
  \end{multicols}
\end{frame}
\begin{frame}{PS4, Ex. 3.a: The Focal Point (plotting BR functions)}
  \begin{multicols}{2}
    Thomas and Alice want to meet on a Friday night. There are two bars in their home town: “The Focal Point” and “The Other Place”. They have to decide independently where they go. If they meet in the same bar, they both get utility of 1. If they end up in different bars, they get utility of 0.
    \begin{itemize}
      \item[(a)] Find all equilibria (pure and mixed). Which equilibrium do you consider the most realistic? Where would you go if you were one of them?
    \end{itemize}
    \begin{table}
      \begin{tabular}{cl|c|c|}
        & \multicolumn{1}{c}{} & \multicolumn{2}{c}{\color{blue}Thomas}\\
        \parbox[t]{1mm}{\multirow{3}{*}{\rotatebox[origin=r]{90}{\color{red}Alice}}}
        & \multicolumn{1}{c}{} & \multicolumn{1}{c}{F (q)} & \multicolumn{1}{c}{O (1-q)} \\\cline{3-4}
        & F (p) & \textcolor{red}{1}, \textcolor{blue}{1} & 0, 0 \\\cline{3-4}
        & O (1-p) & 0, 0 & \textcolor{red}{1}, \textcolor{blue}{1} \\\cline{3-4}
      \end{tabular}
    \end{table}
  \vfill\null \columnbreak
    Alice is indifferent for:
    \begin{align*}
      E[u_A|Focal]&=E[u_A|Other]\\
      q &= 1-q \Leftrightarrow q = \frac{1}{2}
    \end{align*}
    Taking advantage of symmetry:
    \begin{align*}
      NE=(p^{*},q^{*})=\left\{(0,0);(1,1);\left(\frac{1}{2},\frac{1}{2}\right)\right\}
    \end{align*}
    Which is the most realistic?\\\medskip
    $(\frac{1}{2},\frac{1}{2})$ seems unlikely as expected payoffs are $\frac{1}{2}$ while being 1 for $(0,0)$ and $(1,1)$.\\\medskip
    \textbf{\textit{Where would you go?}}
  \vfill\null
  \end{multicols}
\end{frame}
\begin{frame}{PS4, Ex. 3.a: The Focal Point (plotting BR functions)}
  \begin{multicols}{2}
    Thomas and Alice want to meet on a Friday night. There are two bars in their home town: “The Focal Point” and “The Other Place”. They have to decide independently where they go. If they meet in the same bar, they both get utility of 1. If they end up in different bars, they get utility of 0.
    \begin{itemize}
      \item[(a)] Find all equilibria (pure and mixed). Which equilibrium do you consider the most realistic? Where would you go if you were one of them?
    \end{itemize}
    \begin{table}
      \begin{tabular}{cl|c|c|}
        & \multicolumn{1}{c}{} & \multicolumn{2}{c}{\color{blue}Thomas}\\
        \parbox[t]{1mm}{\multirow{3}{*}{\rotatebox[origin=r]{90}{\color{red}Alice}}}
        & \multicolumn{1}{c}{} & \multicolumn{1}{c}{F (q)} & \multicolumn{1}{c}{O (1-q)} \\\cline{3-4}
        & F (p) & \textcolor{red}{1}, \textcolor{blue}{1} & 0, 0 \\\cline{3-4}
        & O (1-p) & 0, 0 & \textcolor{red}{1}, \textcolor{blue}{1} \\\cline{3-4}
      \end{tabular}
    \end{table}
  \vfill\null \columnbreak
    Alice is indifferent for:
    \begin{align*}
      E[u_A|Focal]&=E[u_A|Other]\\
      q &= 1-q \Leftrightarrow q = \frac{1}{2}
    \end{align*}
    Taking advantage of symmetry:
    \begin{align*}
      NE=(p^{*},q^{*})=\left\{(0,0);(1,1);\left(\frac{1}{2},\frac{1}{2}\right)\right\}
    \end{align*}
    Which is the most realistic?\\\medskip
    $(\frac{1}{2},\frac{1}{2})$ seems unlikely as expected payoffs are $\frac{1}{2}$ while being 1 for $(0,0)$ and $(1,1)$.\\\medskip
    Where would you go?\\\medskip
    I would go to the "The Focal Point" - it sounds like the place to meet.
  \vfill\null
  \end{multicols}
\end{frame}

\begin{frame}{PS4, Ex. 3.b: The Focal Point (plotting BR functions)}
  \begin{multicols}{2}
    \begin{itemize}
      \item[(b)] Now assume that Thomas wants to meet Alice, but Alice does not want to meet Thomas. Thomas gets a payoff of 1 if he meets Alice, and -1 otherwise. Alice gets a payoff of -1 for meeting Thomas, and 1 otherwise. Find all equilibria (pure and mixed).
    \end{itemize}
  \vfill\null \columnbreak
    \textbf{\textit{Write up the new matrix and highlight the best responses. What are the pure strategy NE?}}
  \vfill\null
  \end{multicols}
\end{frame}
\begin{frame}{PS4, Ex. 3.b: The Focal Point (plotting BR functions)}
  \begin{multicols}{2}
    \begin{itemize}
      \item[(b)] Now assume that Thomas wants to meet Alice, but Alice does not want to meet Thomas. Thomas gets a payoff of 1 if he meets Alice, and -1 otherwise. Alice gets a payoff of -1 for meeting Thomas, and 1 otherwise. Find all equilibria (pure and mixed).
    \end{itemize}
    \vspace{-8pt}
    \begin{table}
      \begin{tabular}{cl|c|c|}
        & \multicolumn{1}{c}{} & \multicolumn{2}{c}{\color{blue}Thomas}\\
        \parbox[t]{1mm}{\multirow{3}{*}{\rotatebox[origin=r]{90}{\color{red}Alice}}}
        & \multicolumn{1}{c}{} & \multicolumn{1}{c}{F (q)} & \multicolumn{1}{c}{O (1-q)} \\\cline{3-4}
        & F (p) & -1, \textcolor{blue}{1} & \textcolor{red}{1}, -1 \\\cline{3-4}
        & O (1-p) & \textcolor{red}{1}, -1 & -1, \textcolor{blue}{1} \\\cline{3-4}
      \end{tabular}
    \end{table}
    There exist no NE in pure strategies.
  \vfill\null \columnbreak
    \textbf{\textit{For which values of q is Alice indifferent?}}
  \vfill\null
  \end{multicols}
\end{frame}
\begin{frame}{PS4, Ex. 3.b: The Focal Point (plotting BR functions)}
  \begin{multicols}{2}
    \begin{itemize}
      \item[(b)] Now assume that Thomas wants to meet Alice, but Alice does not want to meet Thomas. Thomas gets a payoff of 1 if he meets Alice, and -1 otherwise. Alice gets a payoff of -1 for meeting Thomas, and 1 otherwise. Find all equilibria (pure and mixed).
    \end{itemize}
    \vspace{-8pt}
    \begin{table}
      \begin{tabular}{cl|c|c|}
        & \multicolumn{1}{c}{} & \multicolumn{2}{c}{\color{blue}Thomas}\\
        \parbox[t]{1mm}{\multirow{3}{*}{\rotatebox[origin=r]{90}{\color{red}Alice}}}
        & \multicolumn{1}{c}{} & \multicolumn{1}{c}{F (q)} & \multicolumn{1}{c}{O (1-q)} \\\cline{3-4}
        & F (p) & -1, \textcolor{blue}{1} & \textcolor{red}{1}, -1 \\\cline{3-4}
        & O (1-p) & \textcolor{red}{1}, -1 & -1, \textcolor{blue}{1} \\\cline{3-4}
      \end{tabular}
    \end{table}
    No PSNE. Alice is indifferent for:
    \begin{align*}
        E[u_A|Focal]&=E[u_A|Other]\\
        -q+(1-q)&=q-(1-q)\Leftrightarrow q=\frac{1}{2}
    \end{align*}
  \vfill\null \columnbreak
    \textbf{\textit{For which values of p is Thomas indifferent?}}
  \vfill\null
  \end{multicols}
\end{frame}
\begin{frame}{PS4, Ex. 3.b: The Focal Point (plotting BR functions)}
  \begin{multicols}{2}
    \begin{itemize}
      \item[(b)] Now assume that Thomas wants to meet Alice, but Alice does not want to meet Thomas. Thomas gets a payoff of 1 if he meets Alice, and -1 otherwise. Alice gets a payoff of -1 for meeting Thomas, and 1 otherwise. Find all equilibria (pure and mixed).
    \end{itemize}
    \vspace{-8pt}
    \begin{table}
      \begin{tabular}{cl|c|c|}
        & \multicolumn{1}{c}{} & \multicolumn{2}{c}{\color{blue}Thomas}\\
        \parbox[t]{1mm}{\multirow{3}{*}{\rotatebox[origin=r]{90}{\color{red}Alice}}}
        & \multicolumn{1}{c}{} & \multicolumn{1}{c}{F (q)} & \multicolumn{1}{c}{O (1-q)} \\\cline{3-4}
        & F (p) & -1, \textcolor{blue}{1} & \textcolor{red}{1}, -1 \\\cline{3-4}
        & O (1-p) & \textcolor{red}{1}, -1 & -1, \textcolor{blue}{1} \\\cline{3-4}
      \end{tabular}
    \end{table}
    No PSNE. Alice is indifferent for:
    \begin{align*}
        E[u_A|Focal]&=E[u_A|Other]\\
        -q+(1-q)&=q-(1-q)\Leftrightarrow q=\frac{1}{2}
    \end{align*}
    Thomas is indifferent for:
    \begin{align*}
        E[u_T|Focal]&=E[u_T|Other]\\
        p-(1-p)&=-p+(1-p)\Leftrightarrow p=\frac{1}{2}
    \end{align*}
  \vfill\null \columnbreak
    \textbf{\textit{Write up Alice's BR function, $\bm{p^{*}(q)}$}}
    \begin{align*}
      BR_A(q)=\left\{\right.
    \end{align*}
  \vfill\null
  \end{multicols}
\end{frame}
\begin{frame}{PS4, Ex. 3.b: The Focal Point (plotting BR functions)}
  \begin{multicols}{2}
    \begin{itemize}
      \item[(b)] Now assume that Thomas wants to meet Alice, but Alice does not want to meet Thomas. Thomas gets a payoff of 1 if he meets Alice, and -1 otherwise. Alice gets a payoff of -1 for meeting Thomas, and 1 otherwise. Find all equilibria (pure and mixed).
    \end{itemize}
    \vspace{-8pt}
    \begin{table}
      \begin{tabular}{cl|c|c|}
        & \multicolumn{1}{c}{} & \multicolumn{2}{c}{\color{blue}Thomas}\\
        \parbox[t]{1mm}{\multirow{3}{*}{\rotatebox[origin=r]{90}{\color{red}Alice}}}
        & \multicolumn{1}{c}{} & \multicolumn{1}{c}{F (q)} & \multicolumn{1}{c}{O (1-q)} \\\cline{3-4}
        & F (p) & -1, \textcolor{blue}{1} & \textcolor{red}{1}, -1 \\\cline{3-4}
        & O (1-p) & \textcolor{red}{1}, -1 & -1, \textcolor{blue}{1} \\\cline{3-4}
      \end{tabular}
    \end{table}
    No PSNE. Alice is indifferent for:
    \begin{align*}
        E[u_A|Focal]&=E[u_A|Other]\\
        -q+(1-q)&=q-(1-q)\Leftrightarrow q=\frac{1}{2}
    \end{align*}
    Thomas is indifferent for:
    \begin{align*}
        E[u_T|Focal]&=E[u_T|Other]\\
        p-(1-p)&=-p+(1-p)\Leftrightarrow p=\frac{1}{2}
    \end{align*}
  \vfill\null \columnbreak
    The BR functions are:
    \begin{align*}
      BR_A(q)&=\left\{ \begin{array}{lcl}
          p=1       & \text{if} & q<1/2 \\
          p\in[0,1] & \text{if} & q=1/2 \\
          p=0       & \text{if} & q>1/2
      \end{array}\right.\\
      BR_T(p)&=\left\{\right.
    \end{align*}
    \textbf{\textit{Write up Thomas' BR function, $\bm{q^{*}(p)}$}}
  \vfill\null
  \end{multicols}
\end{frame}
\begin{frame}{PS4, Ex. 3.b: The Focal Point (plotting BR functions)}
  \begin{multicols}{2}
    \begin{itemize}
      \item[(b)] Now assume that Thomas wants to meet Alice, but Alice does not want to meet Thomas. Thomas gets a payoff of 1 if he meets Alice, and -1 otherwise. Alice gets a payoff of -1 for meeting Thomas, and 1 otherwise. Find all equilibria (pure and mixed).
    \end{itemize}
    \vspace{-8pt}
    \begin{table}
      \begin{tabular}{cl|c|c|}
        & \multicolumn{1}{c}{} & \multicolumn{2}{c}{\color{blue}Thomas}\\
        \parbox[t]{1mm}{\multirow{3}{*}{\rotatebox[origin=r]{90}{\color{red}Alice}}}
        & \multicolumn{1}{c}{} & \multicolumn{1}{c}{F (q)} & \multicolumn{1}{c}{O (1-q)} \\\cline{3-4}
        & F (p) & -1, \textcolor{blue}{1} & \textcolor{red}{1}, -1 \\\cline{3-4}
        & O (1-p) & \textcolor{red}{1}, -1 & -1, \textcolor{blue}{1} \\\cline{3-4}
      \end{tabular}
    \end{table}
    No PSNE. Alice is indifferent for:
    \begin{align*}
        E[u_A|Focal]&=E[u_A|Other]\\
        -q+(1-q)&=q-(1-q)\Leftrightarrow q=\frac{1}{2}
    \end{align*}
    Thomas is indifferent for:
    \begin{align*}
        E[u_T|Focal]&=E[u_T|Other]\\
        p-(1-p)&=-p+(1-p)\Leftrightarrow p=\frac{1}{2}
    \end{align*}
  \vfill\null \columnbreak
    The BR functions are:
    \begin{align*}
      BR_A(q)=\left\{ \begin{array}{lcl}
          p=1       & \text{if} & q<1/2 \\
          p\in[0,1] & \text{if} & q=1/2 \\
          p=0       & \text{if} & q>1/2
      \end{array}\right. \\
      BR_T(p)=\left\{ \begin{array}{lcl}
          q=0       & \text{if} & p<1/2  \\
          q\in[0,1] & \text{if} & p=1/2 \\
          q=1       & \text{if} & p>1/2
      \end{array}\right.
    \end{align*}
  \vfill\null
  \end{multicols}
\end{frame}
\begin{frame}{PS4, Ex. 3.b: The Focal Point (plotting BR functions)}
  \begin{multicols}{2}
    \begin{itemize}
      \item[(b)] Now assume that Thomas wants to meet Alice, but Alice does not want to meet Thomas. Find all NE.
    \end{itemize}
    \vspace{-4pt}
    \begin{table}
      \begin{tabular}{cl|c|c|}
        & \multicolumn{1}{c}{} & \multicolumn{2}{c}{\color{blue}Thomas}\\
        \parbox[t]{1mm}{\multirow{3}{*}{\rotatebox[origin=r]{90}{\color{red}Alice}}}
        & \multicolumn{1}{c}{} & \multicolumn{1}{c}{F (q)} & \multicolumn{1}{c}{O (1-q)} \\\cline{3-4}
        & F (p) & -1, \textcolor{blue}{1} & \textcolor{red}{1}, -1 \\\cline{3-4}
        & O (1-p) & \textcolor{red}{1}, -1 & -1, \textcolor{blue}{1} \\\cline{3-4}
      \end{tabular}
    \end{table}
    No PSNE. Alice is indifferent for:
    \begin{align*}
        -q+(1-q)&=q-(1-q)\Leftrightarrow q=\frac{1}{2}
    \end{align*}
    Thomas is indifferent for:
    \begin{align*}
        p-(1-p)&=-p+(1-p)\Leftrightarrow p=\frac{1}{2}
    \end{align*}
    \begin{align*}
      BR_A(q)=\left\{ \begin{array}{lcl}
          p=1       & \text{if} & q<1/2 \\
          p\in[0,1] & \text{if} & q=1/2 \\
          p=0       & \text{if} & q>1/2
      \end{array}\right. \\
      BR_T(p)=\left\{ \begin{array}{lcl}
          q=0       & \text{if} & p<1/2  \\
          q\in[0,1] & \text{if} & p=1/2 \\
          q=1       & \text{if} & p>1/2
      \end{array}\right.
    \end{align*}
    \vfill\null \columnbreak
    \includegraphics[width=\columnwidth]{figures/empty_plot_}
    \textbf{\textit{Plot Alice's BR function, $\bm{p^{*}(q)}$}}
  \vfill\null
  \end{multicols}
\end{frame}
\begin{frame}{PS4, Ex. 3.b: The Focal Point (plotting BR functions)}
  \begin{multicols}{2}
    \begin{itemize}
      \item[(b)] Now assume that Thomas wants to meet Alice, but Alice does not want to meet Thomas. Find all NE.
    \end{itemize}
    \vspace{-4pt}
    \begin{table}
      \begin{tabular}{cl|c|c|}
        & \multicolumn{1}{c}{} & \multicolumn{2}{c}{\color{blue}Thomas}\\
        \parbox[t]{1mm}{\multirow{3}{*}{\rotatebox[origin=r]{90}{\color{red}Alice}}}
        & \multicolumn{1}{c}{} & \multicolumn{1}{c}{F (q)} & \multicolumn{1}{c}{O (1-q)} \\\cline{3-4}
        & F (p) & -1, \textcolor{blue}{1} & \textcolor{red}{1}, -1 \\\cline{3-4}
        & O (1-p) & \textcolor{red}{1}, -1 & -1, \textcolor{blue}{1} \\\cline{3-4}
      \end{tabular}
    \end{table}
    No PSNE. Alice is indifferent for:
    \begin{align*}
        -q+(1-q)&=q-(1-q)\Leftrightarrow q=\frac{1}{2}
    \end{align*}
    Thomas is indifferent for:
    \begin{align*}
        p-(1-p)&=-p+(1-p)\Leftrightarrow p=\frac{1}{2}
    \end{align*}
    \begin{align*}
      BR_A(q)=\left\{ \begin{array}{lcl}
          p=1       & \text{if} & q<1/2 \\
          p\in[0,1] & \text{if} & q=1/2 \\
          p=0       & \text{if} & q>1/2
      \end{array}\right. \\
      BR_T(p)=\left\{ \begin{array}{lcl}
          q=0       & \text{if} & p<1/2  \\
          q\in[0,1] & \text{if} & p=1/2 \\
          q=1       & \text{if} & p>1/2
      \end{array}\right.
    \end{align*}
    \vfill\null \columnbreak
    \includegraphics[width=\columnwidth]{figures/3b_}
    \textbf{\textit{Plot Thomas' BR function, $\bm{q^{*}(p)}$}}
  \vfill\null
  \end{multicols}
\end{frame}
\begin{frame}{PS4, Ex. 3.b: The Focal Point (plotting BR functions)}
  \begin{multicols}{2}
    \begin{itemize}
      \item[(b)] Now assume that Thomas wants to meet Alice, but Alice does not want to meet Thomas. Find all NE.
    \end{itemize}
    \vspace{-4pt}
    \begin{table}
      \begin{tabular}{cl|c|c|}
        & \multicolumn{1}{c}{} & \multicolumn{2}{c}{\color{blue}Thomas}\\
        \parbox[t]{1mm}{\multirow{3}{*}{\rotatebox[origin=r]{90}{\color{red}Alice}}}
        & \multicolumn{1}{c}{} & \multicolumn{1}{c}{F (q)} & \multicolumn{1}{c}{O (1-q)} \\\cline{3-4}
        & F (p) & -1, \textcolor{blue}{1} & \textcolor{red}{1}, -1 \\\cline{3-4}
        & O (1-p) & \textcolor{red}{1}, -1 & -1, \textcolor{blue}{1} \\\cline{3-4}
      \end{tabular}
    \end{table}
    No PSNE. Alice is indifferent for:
    \begin{align*}
        -q+(1-q)&=q-(1-q)\Leftrightarrow q=\frac{1}{2}
    \end{align*}
    Thomas is indifferent for:
    \begin{align*}
        p-(1-p)&=-p+(1-p)\Leftrightarrow p=\frac{1}{2}
    \end{align*}
    \begin{align*}
      BR_A(q)=\left\{ \begin{array}{lcl}
          p=1       & \text{if} & q<1/2 \\
          p\in[0,1] & \text{if} & q=1/2 \\
          p=0       & \text{if} & q>1/2
      \end{array}\right. \\
      BR_T(p)=\left\{ \begin{array}{lcl}
          q=0       & \text{if} & p<1/2  \\
          q\in[0,1] & \text{if} & p=1/2 \\
          q=1       & \text{if} & p>1/2
      \end{array}\right.
    \end{align*}
    \vfill\null \columnbreak
    \includegraphics[width=\columnwidth]{figures/3b}
    \textbf{\textit{Write up all NE (pure and mixed).}}
    \begin{align*}
      NE=(p^{*},q^{*})=
    \end{align*}
  \vfill\null
  \end{multicols}
\end{frame}
\begin{frame}{PS4, Ex. 3.b: The Focal Point (plotting BR functions)}
  \begin{multicols}{2}
    \begin{itemize}
      \item[(b)] Now assume that Thomas wants to meet Alice, but Alice does not want to meet Thomas. Find all NE.
    \end{itemize}
    \vspace{-4pt}
    \begin{table}
      \begin{tabular}{cl|c|c|}
        & \multicolumn{1}{c}{} & \multicolumn{2}{c}{\color{blue}Thomas}\\
        \parbox[t]{1mm}{\multirow{3}{*}{\rotatebox[origin=r]{90}{\color{red}Alice}}}
        & \multicolumn{1}{c}{} & \multicolumn{1}{c}{F (q)} & \multicolumn{1}{c}{O (1-q)} \\\cline{3-4}
        & F (p) & -1, \textcolor{blue}{1} & \textcolor{red}{1}, -1 \\\cline{3-4}
        & O (1-p) & \textcolor{red}{1}, -1 & -1, \textcolor{blue}{1} \\\cline{3-4}
      \end{tabular}
    \end{table}
    No PSNE. Alice is indifferent for:
    \begin{align*}
        -q+(1-q)&=q-(1-q)\Leftrightarrow q=\frac{1}{2}
    \end{align*}
    Thomas is indifferent for:
    \begin{align*}
        p-(1-p)&=-p+(1-p)\Leftrightarrow p=\frac{1}{2}
    \end{align*}
    \begin{align*}
      BR_A(q)=\left\{ \begin{array}{lcl}
          p=1       & \text{if} & q<1/2 \\
          p\in[0,1] & \text{if} & q=1/2 \\
          p=0       & \text{if} & q>1/2
      \end{array}\right. \\
      BR_T(p)=\left\{ \begin{array}{lcl}
          q=0       & \text{if} & p<1/2  \\
          q\in[0,1] & \text{if} & p=1/2 \\
          q=1       & \text{if} & p>1/2
      \end{array}\right.
    \end{align*}
    \vfill\null \columnbreak
    \includegraphics[width=\columnwidth]{figures/3b}
    The only NE is the Mixed Strategy NE:
    \begin{align*}
      (p^{*},q^{*})=\left(\frac{1}{2},\frac{1}{2}\right)
    \end{align*}
  \vfill\null
  \end{multicols}
\end{frame}

\begin{frame}{PS4, Ex. 3.c: The Focal Point (plotting BR functions)}
    \begin{itemize}
      \item[(c)] Now assume again that Thomas and Alice both want to meet (so that payoffs are as in part (a)), but now there are $N$ bars in town, where $N$ can be very large. Show that there are $2^N-1$ equilibria (pure and mixed). Say that the bars have names: “The First Bar in Town”, “The Second Bar in Town”, and so on. Which equilibrium is the most realistic?
    \end{itemize}
  \vfill\null
\end{frame}
\begin{frame}{PS4, Ex. 3.c: The Focal Point (plotting BR functions)}
    \begin{itemize}
      \item[(c)] Now assume again that Thomas and Alice both want to meet (so that payoffs are as in part (a)), but now there are $N$ bars in town, where $N$ can be very large. Show that there are $2^N-1$ equilibria (pure and mixed). Say that the bars have names: “The First Bar in Town”, “The Second Bar in Town”, and so on.
    \end{itemize}
  \begin{multicols}{2}
    \textbf{For N=2:} We have $3=2^N-1$ equilibria:
    \begin{align*}
      (p^{*},q^{*})&=\left\{(0,0);(1,1);\left(\frac{1}{2},\frac{1}{2}\right)\right\}
    \end{align*}
  \vfill\null \columnbreak
    \vspace{-12pt}
    \begin{table}
      \begin{tabular}{cl|c|c|}
        & \multicolumn{1}{c}{} & \multicolumn{2}{c}{\color{blue}Thomas}\\
        \parbox[t]{1mm}{\multirow{3}{*}{\rotatebox[origin=r]{90}{\color{red}Alice}}}
        & \multicolumn{1}{c}{} & \multicolumn{1}{c}{$Bar_1$ (q)} & \multicolumn{1}{c}{$Bar_2$ (1-q)} \\\cline{3-4}
        & $Bar_1$ (p) & \textcolor{red}{1}, \textcolor{blue}{1} & 0, 0 \\\cline{3-4}
        & $Bar_2$ (1-p) & 0, 0 & \textcolor{red}{1}, \textcolor{blue}{1} \\\cline{3-4}
      \end{tabular}
    \end{table}
  \vfill\null
  \end{multicols}
    \textbf{\textit{What about N=3?}}
  \vfill\null
\end{frame}
\begin{frame}{PS4, Ex. 3.c: The Focal Point (plotting BR functions)}
    \begin{itemize}
      \item[(c)] Now assume again that Thomas and Alice both want to meet (so that payoffs are as in part (a)), but now there are $N$ bars in town, where $N$ can be very large. Show that there are $2^N-1$ equilibria (pure and mixed). Say that the bars have names: “The First Bar in Town”, “The Second Bar in Town”, and so on.
    \end{itemize}
  \begin{multicols}{2}
    \textbf{For N=2:} We have found $3=2^N-1$ equilibria:
    \begin{align*}
      (p^{*},q^{*})&=\left\{(0,0);(1,1);\left(\frac{1}{2},\frac{1}{2}\right)\right\}
    \end{align*}
  \vfill\null \columnbreak
    \vspace{-12pt}
    \begin{table}
      \begin{tabular}{cl|c|c|}
        & \multicolumn{1}{c}{} & \multicolumn{2}{c}{\color{blue}Thomas}\\
        \parbox[t]{1mm}{\multirow{3}{*}{\rotatebox[origin=r]{90}{\color{red}Alice}}}
        & \multicolumn{1}{c}{} & \multicolumn{1}{c}{$Bar_1$ (q)} & \multicolumn{1}{c}{$Bar_2$ (1-q)} \\\cline{3-4}
        & $Bar_1$ (p) & \textcolor{red}{1}, \textcolor{blue}{1} & 0, 0 \\\cline{3-4}
        & $Bar_2$ (1-p) & 0, 0 & \textcolor{red}{1}, \textcolor{blue}{1} \\\cline{3-4}
      \end{tabular}
    \end{table}
  \vfill\null
  \end{multicols}
    \textbf{\textit{What about N=3?}}
    \vspace{-12pt}
    \begin{table}
      \begin{tabular}{cl|c|c|c|}
        & \multicolumn{1}{c}{} & \multicolumn{3}{c}{\color{blue}Thomas}\\
        & \multicolumn{1}{c}{} & \multicolumn{1}{c}{$Bar_1$ ($q_1$)} & \multicolumn{1}{c}{$Bar_2$ ($q_2$)} & \multicolumn{1}{c}{$Bar_3$ (1-$q_1$-$q_2$)} \\\cline{3-5}
        \parbox[t]{1mm}{\multirow{3}{*}{\rotatebox[origin=c]{90}{\color{red}Alice}}}
        & $Bar_1$ ($p_1$) & \textcolor{red}{1}, \textcolor{blue}{1} & 0, 0 & 0, 0 \\\cline{3-5}
        & $Bar_2$ ($p_2$) & 0, 0 & \textcolor{red}{1}, \textcolor{blue}{1} & 0, 0 \\\cline{3-5}
        & $Bar_3$ (1-$p_1$-$p_2$) & 0, 0 & 0, 0 & \textcolor{red}{1}, \textcolor{blue}{1} \\\cline{3-5}
      \end{tabular}
    \end{table}
  \vfill\null
\end{frame}
\begin{frame}{PS4, Ex. 3.c: The Focal Point (plotting BR functions)}
    \begin{itemize}
      \item[(c)] Now assume again that Thomas and Alice both want to meet (so that payoffs are as in part (a)), but now there are $N$ bars in town, where $N$ can be very large. Show that there are $2^N-1$ equilibria (pure and mixed). Say that the bars have names: “The First Bar in Town”, “The Second Bar in Town”, and so on.
    \end{itemize}
    \vspace{-6pt}
  \begin{multicols}{2}
    \textbf{For N=2:} We have $3=2^N-1$ equilibria:
    \begin{align*}
      (p^{*},q^{*})&=\left\{(0,0);(1,1);\left(\frac{1}{2},\frac{1}{2}\right)\right\}
    \end{align*}
  \vfill\null \columnbreak
    \vspace{-12pt}
    \begin{table}
      \begin{tabular}{cl|c|c|}
        & \multicolumn{1}{c}{} & \multicolumn{2}{c}{\color{blue}Thomas}\\
        \parbox[t]{1mm}{\multirow{3}{*}{\rotatebox[origin=r]{90}{\color{red}Alice}}}
        & \multicolumn{1}{c}{} & \multicolumn{1}{c}{$Bar_1$ (q)} & \multicolumn{1}{c}{$Bar_2$ (1-q)} \\\cline{3-4}
        & $Bar_1$ (p) & \textcolor{red}{1}, \textcolor{blue}{1} & 0, 0 \\\cline{3-4}
        & $Bar_2$ (1-p) & 0, 0 & \textcolor{red}{1}, \textcolor{blue}{1} \\\cline{3-4}
      \end{tabular}
    \end{table}
  \vfill\null
  \end{multicols}
    \vspace{-20pt}
    \textbf{For N=3:} We have $7=2^N-1$ equilibria, $(p_1^{*},p_2^{*},q_1^{*},q_2^{*})$:
    \begin{align*}
      \left\{(0,0,0,0);(0,1,0,1);(1,0,1,0)
      ;\left(\frac{1}{2},\frac{1}{2},\frac{1}{2},\frac{1}{2}\right)
      ;\left(\frac{1}{2},0,\frac{1}{2},0\right)
      ;\left(0,\frac{1}{2},0,\frac{1}{2}\right)
      ;\left(\frac{1}{3},\frac{1}{3},\frac{1}{3},\frac{1}{3}\right)
      \right\}
    \end{align*}
    \vspace{-12pt}
    \begin{table}
      \begin{tabular}{cl|c|c|c|}
        & \multicolumn{1}{c}{} & \multicolumn{3}{c}{\color{blue}Thomas}\\
        & \multicolumn{1}{c}{} & \multicolumn{1}{c}{$Bar_1$ ($q_1$)} & \multicolumn{1}{c}{$Bar_2$ ($q_2$)} & \multicolumn{1}{c}{$Bar_3$ (1-$q_1$-$q_2$)} \\\cline{3-5}
        \parbox[t]{1mm}{\multirow{3}{*}{\rotatebox[origin=c]{90}{\color{red}Alice}}}
        & $Bar_1$ ($p_1$) & \textcolor{red}{1}, \textcolor{blue}{1} & 0, 0 & 0, 0 \\\cline{3-5}
        & $Bar_2$ ($p_2$) & 0, 0 & \textcolor{red}{1}, \textcolor{blue}{1} & 0, 0 \\\cline{3-5}
        & $Bar_3$ (1-$p_1$-$p_2$) & 0, 0 & 0, 0 & \textcolor{red}{1}, \textcolor{blue}{1} \\\cline{3-5}
      \end{tabular}
    \end{table}
    \textbf{\textit{What about any N?}}
  \vfill\null
\end{frame}
\begin{frame}{PS4, Ex. 3.c: The Focal Point (plotting BR functions)}
    \begin{itemize}
      \item[(c)] Now assume again that Thomas and Alice both want to meet (so that payoffs are as in part (a)), but now there are $N$ bars in town, where $N$ can be very large. Show that there are $2^N-1$ equilibria (pure and mixed). Say that the bars have names: “The First Bar in Town”, “The Second Bar in Town”, and so on.
    \end{itemize}
    \vspace{-6pt}
  \begin{multicols}{2}
    \textbf{For N=2:} We have $3=2^N-1$ equilibria:
    \begin{align*}
      (p^{*},q^{*})&=\left\{(0,0);(1,1);\left(\frac{1}{2},\frac{1}{2}\right)\right\}
    \end{align*}
  \vfill\null \columnbreak
    \vspace{-12pt}
    \begin{table}
      \begin{tabular}{cl|c|c|}
        & \multicolumn{1}{c}{} & \multicolumn{2}{c}{\color{blue}Thomas}\\
        \parbox[t]{1mm}{\multirow{3}{*}{\rotatebox[origin=r]{90}{\color{red}Alice}}}
        & \multicolumn{1}{c}{} & \multicolumn{1}{c}{$Bar_1$ (q)} & \multicolumn{1}{c}{$Bar_2$ (1-q)} \\\cline{3-4}
        & $Bar_1$ (p) & \textcolor{red}{1}, \textcolor{blue}{1} & 0, 0 \\\cline{3-4}
        & $Bar_2$ (1-p) & 0, 0 & \textcolor{red}{1}, \textcolor{blue}{1} \\\cline{3-4}
      \end{tabular}
    \end{table}
  \vfill\null
  \end{multicols}
    \vspace{-20pt}
    \textbf{For N=3:} We have $7=2^N-1$ equilibria, $(p_1^{*},p_2^{*},q_1^{*},q_2^{*})$:
    \begin{align*}
      \left\{(0,0,0,0);(0,1,0,1);(1,0,1,0)
      ;\left(\frac{1}{2},\frac{1}{2},\frac{1}{2},\frac{1}{2}\right)
      ;\left(\frac{1}{2},0,\frac{1}{2},0\right)
      ;\left(0,\frac{1}{2},0,\frac{1}{2}\right)
      ;\left(\frac{1}{3},\frac{1}{3},\frac{1}{3},\frac{1}{3}\right)
      \right\}
    \end{align*}
    \vspace{-12pt}
    \begin{table}
      \begin{tabular}{cl|c|c|c|}
        & \multicolumn{1}{c}{} & \multicolumn{3}{c}{\color{blue}Thomas}\\
        & \multicolumn{1}{c}{} & \multicolumn{1}{c}{$Bar_1$ ($q_1$)} & \multicolumn{1}{c}{$Bar_2$ ($q_2$)} & \multicolumn{1}{c}{$Bar_3$ (1-$q_1$-$q_2$)} \\\cline{3-5}
        \parbox[t]{1mm}{\multirow{3}{*}{\rotatebox[origin=c]{90}{\color{red}Alice}}}
        & $Bar_1$ ($p_1$) & \textcolor{red}{1}, \textcolor{blue}{1} & 0, 0 & 0, 0 \\\cline{3-5}
        & $Bar_2$ ($p_2$) & 0, 0 & \textcolor{red}{1}, \textcolor{blue}{1} & 0, 0 \\\cline{3-5}
        & $Bar_3$ (1-$p_1$-$p_2$) & 0, 0 & 0, 0 & \textcolor{red}{1}, \textcolor{blue}{1} \\\cline{3-5}
      \end{tabular}
    \end{table}
    \textbf{For any N:} It is plausible that the geometric continues for $N>3$. Note that we're asked to "show" not "proof", thus, providing two examples is sufficient.
\end{frame}
\begin{frame}{PS4, Ex. 3.c: The Focal Point (plotting BR functions)}
    \begin{itemize}
      \item[(c)] Which equilibrium is the most realistic?
    \end{itemize}
    \textbf{For N=3:} We have $7=2^N-1$ equilibria, $(p_1^{*},p_2^{*},q_1^{*},q_2^{*})$:
    \begin{align*}
      \left\{(0,0,0,0);(0,1,0,1);(1,0,1,0)
      ;\left(\frac{1}{2},\frac{1}{2},\frac{1}{2},\frac{1}{2}\right)
      ;\left(\frac{1}{2},0,\frac{1}{2},0\right)
      ;\left(0,\frac{1}{2},0,\frac{1}{2}\right)
      ;\left(\frac{1}{3},\frac{1}{3},\frac{1}{3},\frac{1}{3}\right)
      \right\}
    \end{align*}
    \vspace{-18pt}
    \begin{table}
      \begin{tabular}{cl|c|c|c|}
        & \multicolumn{1}{c}{} & \multicolumn{3}{c}{\color{blue}Thomas}\\
        & \multicolumn{1}{c}{} & \multicolumn{1}{c}{$Bar_1$ ($q_1$)} & \multicolumn{1}{c}{$Bar_2$ ($q_2$)} & \multicolumn{1}{c}{$Bar_3$ (1-$q_1$-$q_2$)} \\\cline{3-5}
        \parbox[t]{1mm}{\multirow{3}{*}{\rotatebox[origin=c]{90}{\color{red}Alice}}}
        & $Bar_1$ ($p_1$) & \textcolor{red}{1}, \textcolor{blue}{1} & 0, 0 & 0, 0 \\\cline{3-5}
        & $Bar_2$ ($p_2$) & 0, 0 & \textcolor{red}{1}, \textcolor{blue}{1} & 0, 0 \\\cline{3-5}
        & $Bar_3$ (1-$p_1$-$p_2$) & 0, 0 & 0, 0 & \textcolor{red}{1}, \textcolor{blue}{1} \\\cline{3-5}
      \end{tabular}
    \end{table}
    \textbf{\textit{Look at the expected payoffs from the pure and mixed equilibria when N=3...}}
\end{frame}
\begin{frame}{PS4, Ex. 3.c: The Focal Point (plotting BR functions)}
    \begin{itemize}
      \item[(c)] Which equilibrium is the most realistic?
    \end{itemize}
    \textbf{For N=3:} We have $7=2^N-1$ equilibria, $(p_1^{*},p_2^{*},q_1^{*},q_2^{*})$:
    \begin{align*}
      \left\{(0,0,0,0);(0,1,0,1);(1,0,1,0)
      ;\left(\frac{1}{2},\frac{1}{2},\frac{1}{2},\frac{1}{2}\right)
      ;\left(\frac{1}{2},0,\frac{1}{2},0\right)
      ;\left(0,\frac{1}{2},0,\frac{1}{2}\right)
      ;\left(\frac{1}{3},\frac{1}{3},\frac{1}{3},\frac{1}{3}\right)
      \right\}
    \end{align*}
    \vspace{-18pt}
    \begin{table}
      \begin{tabular}{cl|c|c|c|}
        & \multicolumn{1}{c}{} & \multicolumn{3}{c}{\color{blue}Thomas}\\
        & \multicolumn{1}{c}{} & \multicolumn{1}{c}{$Bar_1$ ($q_1$)} & \multicolumn{1}{c}{$Bar_2$ ($q_2$)} & \multicolumn{1}{c}{$Bar_3$ (1-$q_1$-$q_2$)} \\\cline{3-5}
        \parbox[t]{1mm}{\multirow{3}{*}{\rotatebox[origin=c]{90}{\color{red}Alice}}}
        & $Bar_1$ ($p_1$) & \textcolor{red}{1}, \textcolor{blue}{1} & 0, 0 & 0, 0 \\\cline{3-5}
        & $Bar_2$ ($p_2$) & 0, 0 & \textcolor{red}{1}, \textcolor{blue}{1} & 0, 0 \\\cline{3-5}
        & $Bar_3$ (1-$p_1$-$p_2$) & 0, 0 & 0, 0 & \textcolor{red}{1}, \textcolor{blue}{1} \\\cline{3-5}
      \end{tabular}
    \end{table}
    In the three PSNE, the expected payoffs are: $\left(E[u_A|q_1^{*},q_2^{*})],E[u_T|p_1^{*},p_2^{*})]\right)=$
    \begin{align*}
      \left\{(1-q_1-q_2,1-p_1-p_2);(q_2,p_2);(q_1,p_1)\right\}\sim
      \left\{(1,1);(1,1);(1,1)\right\}
    \end{align*}
    \textbf{\textit{What are the expected payoffs in the four MSNE?}}
  \vfill\null
\end{frame}
\begin{frame}{PS4, Ex. 3.c: The Focal Point (plotting BR functions)}
    \begin{itemize}
      \item[(c)] Which equilibrium is the most realistic?
    \end{itemize}
    \textbf{For N=3:} We have $7=2^N-1$ equilibria, $(p_1^{*},p_2^{*},q_1^{*},q_2^{*})$:
    \begin{align*}
      \left\{(0,0,0,0);(0,1,0,1);(1,0,1,0)
      ;\left(\frac{1}{2},\frac{1}{2},\frac{1}{2},\frac{1}{2}\right)
      ;\left(\frac{1}{2},0,\frac{1}{2},0\right)
      ;\left(0,\frac{1}{2},0,\frac{1}{2}\right)
      ;\left(\frac{1}{3},\frac{1}{3},\frac{1}{3},\frac{1}{3}\right)
      \right\}
    \end{align*}
    \vspace{-12pt}
    \begin{table}
      \begin{tabular}{cl|c|c|c|}
        & \multicolumn{1}{c}{} & \multicolumn{3}{c}{\color{blue}Thomas}\\
        & \multicolumn{1}{c}{} & \multicolumn{1}{c}{$Bar_1$ ($q_1$)} & \multicolumn{1}{c}{$Bar_2$ ($q_2$)} & \multicolumn{1}{c}{$Bar_3$ (1-$q_1$-$q_2$)} \\\cline{3-5}
        \parbox[t]{1mm}{\multirow{3}{*}{\rotatebox[origin=c]{90}{\color{red}Alice}}}
        & $Bar_1$ ($p_1$) & \textcolor{red}{1}, \textcolor{blue}{1} & 0, 0 & 0, 0 \\\cline{3-5}
        & $Bar_2$ ($p_2$) & 0, 0 & \textcolor{red}{1}, \textcolor{blue}{1} & 0, 0 \\\cline{3-5}
        & $Bar_3$ (1-$p_1$-$p_2$) & 0, 0 & 0, 0 & \textcolor{red}{1}, \textcolor{blue}{1} \\\cline{3-5}
      \end{tabular}
    \end{table}
    In the three PSNE, the expected payoffs are: $\left(E[u_A|q_1^{*},q_2^{*})],E[u_T|p_1^{*},p_2^{*})]\right)=$
    \begin{align*}
      \left\{(1-q_1-q_2,1-p_1-p_2);(q_2,p_2);(q_1,p_1)\right\}\sim
      \left\{(1,1);(1,1);(1,1)\right\}
    \end{align*}
    In the four MSNE, the expected payoffs are: $\left(E[u_A|q_1^{*},q_2^{*})],E[u_T|p_1^{*},p_2^{*})]\right)=$
    \begin{align*}
     &\left\{\left(\frac{q_1+q_2}{2},\frac{p_1+p_2}{2}\right)
      ;\left(\frac{1-q_2}{2},\frac{1-p_2}{2}\right)
      ;\left(\frac{1-q_1}{2},\frac{1-p_1}{2}\right)
      ;\left(\frac{1}{3},\frac{1}{3}\right)\right\} \\
     \sim
     &\left\{\left(\frac{1}{2},\frac{1}{2}\right)
      ;\left(\frac{1}{2},\frac{1}{2}\right)
      ;\left(\frac{1}{2},\frac{1}{2}\right)
      ;\left(\frac{1}{3},\frac{1}{3}\right)\right\}
    \end{align*}
    \textbf{\textit{Which equilibria are the most realistic - and which is the least realistic?}}
\end{frame}
\begin{frame}{PS4, Ex. 3.c: The Focal Point (plotting BR functions)}
    \begin{itemize}
      \item[(c)] Which equilibrium is the most realistic?
    \end{itemize}
    \textbf{For N=3:} We have $7=2^N-1$ equilibria, $(p_1^{*},p_2^{*},q_1^{*},q_2^{*})$:
    \begin{align*}
      \left\{(0,0,0,0);(0,1,0,1);(1,0,1,0)
      ;\left(\frac{1}{2},\frac{1}{2},\frac{1}{2},\frac{1}{2}\right)
      ;\left(\frac{1}{2},0,\frac{1}{2},0\right)
      ;\left(0,\frac{1}{2},0,\frac{1}{2}\right)
      ;\left(\frac{1}{3},\frac{1}{3},\frac{1}{3},\frac{1}{3}\right)
      \right\}
    \end{align*}
    \vspace{-12pt}
    \begin{table}
      \begin{tabular}{cl|c|c|c|}
        & \multicolumn{1}{c}{} & \multicolumn{3}{c}{\color{blue}Thomas}\\
        & \multicolumn{1}{c}{} & \multicolumn{1}{c}{$Bar_1$ ($q_1$)} & \multicolumn{1}{c}{$Bar_2$ ($q_2$)} & \multicolumn{1}{c}{$Bar_3$ (1-$q_1$-$q_2$)} \\\cline{3-5}
        \parbox[t]{1mm}{\multirow{3}{*}{\rotatebox[origin=c]{90}{\color{red}Alice}}}
        & $Bar_1$ ($p_1$) & \textcolor{red}{1}, \textcolor{blue}{1} & 0, 0 & 0, 0 \\\cline{3-5}
        & $Bar_2$ ($p_2$) & 0, 0 & \textcolor{red}{1}, \textcolor{blue}{1} & 0, 0 \\\cline{3-5}
        & $Bar_3$ (1-$p_1$-$p_2$) & 0, 0 & 0, 0 & \textcolor{red}{1}, \textcolor{blue}{1} \\\cline{3-5}
      \end{tabular}
    \end{table}
    \vspace{-4pt}
    In the three PSNE, the expected payoffs are: $\left(E[u_A|q_1^{*},q_2^{*})],E[u_T|p_1^{*},p_2^{*})]\right)=$
    \begin{align*}
      \left\{(1-q_1-q_2,1-p_1-p_2);(q_2,p_2);(q_1,p_1)\right\}\sim
      \left\{(1,1);(1,1);(1,1)\right\}
    \end{align*}
    In the four MSNE, the expected payoffs are: $\left(E[u_A|q_1^{*},q_2^{*})],E[u_T|p_1^{*},p_2^{*})]\right)=$
    \begin{align*}
     &\left\{\left(\frac{q_1+q_2}{2},\frac{p_1+p_2}{2}\right)
      ;\left(\frac{1-q_2}{2},\frac{1-p_2}{2}\right)
      ;\left(\frac{1-q_1}{2},\frac{1-p_1}{2}\right)
      ;\left(\frac{1}{3},\frac{1}{3}\right)\right\} \\
     \sim
     &\left\{\left(\frac{1}{2},\frac{1}{2}\right)
      ;\left(\frac{1}{2},\frac{1}{2}\right)
      ;\left(\frac{1}{2},\frac{1}{2}\right)
      ;\left(\frac{1}{3},\frac{1}{3}\right)\right\}
    \end{align*}
    PSNE have higher expected payoffs but without communication it's not clear which one to go for. Due to coordination issues, the MSNE can be just as good, even though expected payoffs are reciprocal to the number of actions that a MSNE is split between.
\end{frame}


\section{Guide: Examine which equilibria are the most realistic in a static game}

\begin{frame}{Guide: Examine which equilibria are the most realistic in a static game}
  \begin{multicols}{2}
    \begin{enumerate}
      \item Look at the pareto optimal solutions:\\
      \begin{itemize}
        \normalsize
          \item[a.] If exactly one Pure Strategy Nash Equilibrium (PSNE) is pareto optimal, rational players should pick this sd olution.
          \item[b.] If there are multiple pareto optimal PSNE, there is a risk of miscoordination, as players can't tell which PSNE the other player is going for. Thus, playing a mix of these can be just as good as arbitrarily picking a pure strategy.
      \end{itemize}
      \item Look at the punishment in the case of miscoordination:\\
      \begin{itemize}
        \normalsize
          \item E.g. If Player 1 thinks they are going for a certain PSNE, but they miscoordinate and Player 2 plays something else, how hard will Player 1 be punished?
      \end{itemize}
    \end{enumerate}
    \vfill\null \columnbreak
    \begin{itemize}
      \item[3.] Use the two first points to talk about what rational players would do?
      % \item[4.] And what if the players were risk averse or risk loving?\\
      % \begin{itemize}
      %   \normalsize
      %     \item[ad. 1] If there is a risk of miscoordination between multiple pareto optimal PSNE, then a risk averse Player 1 could benefit from playing a mix of these to have a positive probability of hitting the PSNE regardless of which pure strategy Player 2 would go for.
      %     %If there are risks of miscoordination between multiple pareto optimal PSNE, then a risk averse player would most likely benefit from playing a mix of these, in order to lower the variance of his payoff.
      %     \item[ad. 2] If a strategy has severe punishment for miscoordination, a risk averse player would most likely want to look for a strategy with lower possible punishment.
      %     %If a strategy has severe punishment for miscoordination, a risk averse player would most likely want to look for another strategy where he isn't punished as hard by miscoordination.
      % \end{itemize}
      \item[4.] Finally, consider what would happen if one player could send a message? Or they had just played the game with mixed strategies, and by chance landed on a pareto optimal PSNE?
    \end{itemize}
    \vfill\null
  \end{multicols}
\end{frame}


\section{PS4, Ex. 4: Generalized Battle of the Sexes (plotting BR functions)}

\begin{frame}{PS4, Ex. 4: Generalized Battle of the Sexes (plotting BR functions)}
  \begin{multicols}{2}
    Consider the following Generalized Battle of the Sexes game, with $N > 1$:
    \begin{table}
      \begin{tabular}{cl|c|c|}
          & \multicolumn{1}{c}{} & \multicolumn{2}{c}{Player 2}\\
          \parbox[t]{1mm}{\multirow{3}{*}{\rotatebox[origin=r]{90}{Player 1}}}
          & \multicolumn{1}{c}{} & \multicolumn{1}{c}{C1 (q)} & \multicolumn{1}{c}{C2 (1-q)} \\\cline{3-4}
          & C1 (p)    & N, 1 & 0, 0 \\\cline{3-4}
          & C2 (1-p)  & 0, 0 & 1, N \\\cline{3-4}
      \end{tabular}
    \end{table}
  \vfill\null \columnbreak
  \begin{itemize}
    \item[(a)] How can you interpret the parameter $N$?
    \item[(b)] Solve for the mixed strategy Nash equilibrium (MSNE). When $N$ becomes very large, what happens to the probability of successful coordination?
  \end{itemize}
  \vfill\null
  \end{multicols}
\end{frame}
\begin{frame}{PS4, Ex. 4.a: Generalized Battle of the Sexes (plotting BR functions)}
  \begin{multicols}{2}
    \begin{itemize}
      \item[(a)] How can you interpret $N > 1$?
    \end{itemize}
    Formally: $N$ is the factor of additional utility for one's most preferred outcome.\\\medskip
    Informally: $N$ is a measure for the conflict of interests.
    \begin{itemize}
      \item[(b)] Find the MSNE.
    \end{itemize}
    \vspace{-8pt}
    \begin{table}
      \begin{tabular}{cl|c|c|}
          & \multicolumn{1}{c}{} & \multicolumn{2}{c}{\color{blue}Player 2}\\
          \parbox[t]{1mm}{\multirow{3}{*}{\rotatebox[origin=r]{90}{\color{red}Player 1}}}
          & \multicolumn{1}{c}{} & \multicolumn{1}{c}{C1 (q)} & \multicolumn{1}{c}{C2 (1-q)} \\\cline{3-4}
          & C1 (p)    & \textcolor{red}{N}, \textcolor{blue}{1} & 0, 0 \\\cline{3-4}
          & C2 (1-p)  & 0, 0 & \textcolor{red}{1}, \textcolor{blue}{N} \\\cline{3-4}
      \end{tabular}
    \end{table}
    \vspace{-8pt}
  \vfill\null \columnbreak
    \textbf{\textit{For which values of q is Player 1 indifferent?}}
  \vfill\null
  \end{multicols}
\end{frame}
\begin{frame}{PS4, Ex. 4.a: Generalized Battle of the Sexes (plotting BR functions)}
  \begin{multicols}{2}
    \begin{itemize}
      \item[(a)] How can you interpret $N > 1$?
    \end{itemize}
    Formally: $N$ is the factor of additional utility for one's most preferred outcome.\\\medskip
    Informally: $N$ is a measure for the conflict of interests.
    \begin{itemize}
      \item[(b)] Find the MSNE.
    \end{itemize}
    \vspace{-8pt}
    \begin{table}
      \begin{tabular}{cl|c|c|}
          & \multicolumn{1}{c}{} & \multicolumn{2}{c}{\color{blue}Player 2}\\
          \parbox[t]{1mm}{\multirow{3}{*}{\rotatebox[origin=r]{90}{\color{red}Player 1}}}
          & \multicolumn{1}{c}{} & \multicolumn{1}{c}{C1 (q)} & \multicolumn{1}{c}{C2 (1-q)} \\\cline{3-4}
          & C1 (p)    & \textcolor{red}{N}, \textcolor{blue}{1} & 0, 0 \\\cline{3-4}
          & C2 (1-p)  & 0, 0 & \textcolor{red}{1}, \textcolor{blue}{N} \\\cline{3-4}
      \end{tabular}
    \end{table}
    Player 1 is indifferent for:
    \begin{align*}
      E[u_1|C1]&=E[u_1|C2]\\
      Nq &= 1-q \Leftrightarrow q = \frac{1}{1+N}
    \end{align*}
    \vspace{-8pt}
  \vfill\null \columnbreak
    \textbf{\textit{For which values of p is Player 2 indifferent?}}
  \vfill\null
  \end{multicols}
\end{frame}
\begin{frame}{PS4, Ex. 4.b: Generalized Battle of the Sexes (plotting BR functions)}
  \begin{multicols}{2}
    \begin{itemize}
      \item[(b)] Find the MSNE.
    \end{itemize}
    \vspace{-8pt}
    \begin{table}
      \begin{tabular}{cl|c|c|}
          & \multicolumn{1}{c}{} & \multicolumn{2}{c}{\color{blue}Player 2}\\
          \parbox[t]{1mm}{\multirow{3}{*}{\rotatebox[origin=r]{90}{\color{red}Player 1}}}
          & \multicolumn{1}{c}{} & \multicolumn{1}{c}{C1 (q)} & \multicolumn{1}{c}{C2 (1-q)} \\\cline{3-4}
          & C1 (p)    & \textcolor{red}{N}, \textcolor{blue}{1} & 0, 0 \\\cline{3-4}
          & C2 (1-p)  & 0, 0 & \textcolor{red}{1}, \textcolor{blue}{N} \\\cline{3-4}
      \end{tabular}
    \end{table}
    Player 1 is indifferent for:
    \begin{align*}
      E[u_1|C1]&=E[u_1|C2]\\
      Nq &= 1-q \Leftrightarrow q = \frac{1}{1+N}
    \end{align*}
    Player 2 is indifferent for:
    \begin{align*}
      E[u_2|C1]&=E[u_2|C2]\\
      p &= N(1-p) \Leftrightarrow p = \frac{N}{1+N}
    \end{align*}
    \begin{align*}
      BR_1(q)=p^{*}(q)=\left\{\right. \\
      BR_2(p)=q^{*}(p)=\left\{\right.
    \end{align*}
  \vfill\null \columnbreak
    \textbf{\textit{Find the best-response functions.}}
  \vfill\null
  \end{multicols}
\end{frame}
\begin{frame}{PS4, Ex. 4.b: Generalized Battle of the Sexes (plotting BR functions)}
  \begin{multicols}{2}
    \begin{itemize}
      \item[(b)] Find the MSNE.
    \end{itemize}
    \vspace{-12pt}
    \begin{table}
      \begin{tabular}{cl|c|c|}
          & \multicolumn{1}{c}{} & \multicolumn{2}{c}{\color{blue}Player 2}\\
          \parbox[t]{1mm}{\multirow{3}{*}{\rotatebox[origin=r]{90}{\color{red}Player 1}}}
          & \multicolumn{1}{c}{} & \multicolumn{1}{c}{C1 (q)} & \multicolumn{1}{c}{C2 (1-q)} \\\cline{3-4}
          & C1 (p)    & \textcolor{red}{N}, \textcolor{blue}{1} & 0, 0 \\\cline{3-4}
          & C2 (1-p)  & 0, 0 & \textcolor{red}{1}, \textcolor{blue}{N} \\\cline{3-4}
      \end{tabular}
    \end{table}
    Player 1 is indifferent for:
    \vspace{-2pt}
    \begin{align*}
      E[u_1|C1]&=E[u_1|C2]\\
      Nq &= 1-q \Leftrightarrow q = \frac{1}{1+N}
    \end{align*}
    Player 2 is indifferent for:
    \vspace{-2pt}
    \begin{align*}
      E[u_2|C1]&=E[u_2|C2]\\
      p &= N(1-p) \Leftrightarrow p = \frac{N}{1+N}
    \end{align*}
    \vspace{-8pt}
    \begin{align*}
      BR_1(q)=\left\{ \begin{array}{lcl}
          p=0       & \text{if} & q<\frac{1}{1+N} \\
          p\in[0,1] & \text{if} & q=\frac{1}{1+N} \\
          p=1       & \text{if} & q>\frac{1}{1+N}
      \end{array}\right. \\
      BR_2(p)=\left\{ \begin{array}{lcl}
          q=0       & \text{if} & p<\frac{N}{1+N}  \\
          q\in[0,1] & \text{if} & p=\frac{N}{1+N} \\
          q=1       & \text{if} & p>\frac{N}{1+N}
      \end{array}\right.
    \end{align*}
  \vfill\null \columnbreak
    \textbf{\textit{Write the mixed strategy NE, $\bm{(p^{*},q^{*})}$.}}
  \vfill\null
  \end{multicols}
\end{frame}
\begin{frame}{PS4, Ex. 4.b: Generalized Battle of the Sexes (plotting BR functions)}
  \begin{multicols}{2}
    \begin{itemize}
      \item[(b)] Find the MSNE.
    \end{itemize}
    \vspace{-12pt}
    \begin{table}
      \begin{tabular}{cl|c|c|}
          & \multicolumn{1}{c}{} & \multicolumn{2}{c}{\color{blue}Player 2}\\
          \parbox[t]{1mm}{\multirow{3}{*}{\rotatebox[origin=r]{90}{\color{red}Player 1}}}
          & \multicolumn{1}{c}{} & \multicolumn{1}{c}{C1 (q)} & \multicolumn{1}{c}{C2 (1-q)} \\\cline{3-4}
          & C1 (p)    & \textcolor{red}{N}, \textcolor{blue}{1} & 0, 0 \\\cline{3-4}
          & C2 (1-p)  & 0, 0 & \textcolor{red}{1}, \textcolor{blue}{N} \\\cline{3-4}
      \end{tabular}
    \end{table}
    Player 1 is indifferent for:
    \vspace{-4pt}
    \begin{align*}
      Nq &= 1-q \Leftrightarrow q = \frac{1}{1+N}
    \end{align*}
    Player 2 is indifferent for:
    \vspace{-4pt}
    \begin{align*}
      p &= N(1-p) \Leftrightarrow p = \frac{N}{1+N}
    \end{align*}
    \vspace{-10pt}
    \begin{align*}
      BR_1(q)=\left\{ \begin{array}{lcl}
          p=0       & \text{if} & q<\frac{1}{1+N} \\
          p\in[0,1] & \text{if} & q=\frac{1}{1+N} \\
          p=1       & \text{if} & q>\frac{1}{1+N}
      \end{array}\right. \\
      BR_2(p)=\left\{ \begin{array}{lcl}
          q=0       & \text{if} & p<\frac{N}{1+N}  \\
          q\in[0,1] & \text{if} & p=\frac{N}{1+N} \\
          q=1       & \text{if} & p>\frac{N}{1+N}
      \end{array}\right.
    \end{align*}
    \vspace{-4pt}
    \begin{align*}
      NE=\left\{(0,0);(1,1);\left(\frac{1}{N+1},\frac{N}{N+1}\right)\right\}
    \end{align*}
  \vfill\null \columnbreak
  \vfill\null
  \end{multicols}
\end{frame}
\begin{frame}{PS4, Ex. 4.b: Generalized Battle of the Sexes (plotting BR functions)}
  \begin{multicols}{2}
    \begin{itemize}
      \item[(b)] Find the MSNE.
    \end{itemize}
    \vspace{-12pt}
    \begin{table}
      \begin{tabular}{cl|c|c|}
          & \multicolumn{1}{c}{} & \multicolumn{2}{c}{\color{blue}Player 2}\\
          \parbox[t]{1mm}{\multirow{3}{*}{\rotatebox[origin=r]{90}{\color{red}Player 1}}}
          & \multicolumn{1}{c}{} & \multicolumn{1}{c}{C1 (q)} & \multicolumn{1}{c}{C2 (1-q)} \\\cline{3-4}
          & C1 (p)    & \textcolor{red}{N}, \textcolor{blue}{1} & 0, 0 \\\cline{3-4}
          & C2 (1-p)  & 0, 0 & \textcolor{red}{1}, \textcolor{blue}{N} \\\cline{3-4}
      \end{tabular}
    \end{table}
    Player 1 is indifferent for:
    \vspace{-4pt}
    \begin{align*}
      Nq &= 1-q \Leftrightarrow q = \frac{1}{1+N}
    \end{align*}
    Player 2 is indifferent for:
    \vspace{-4pt}
    \begin{align*}
      p &= N(1-p) \Leftrightarrow p = \frac{N}{1+N}
    \end{align*}
    \vspace{-10pt}
    \begin{align*}
      BR_1(q)=\left\{ \begin{array}{lcl}
          p=0       & \text{if} & q<\frac{1}{1+N} \\
          p\in[0,1] & \text{if} & q=\frac{1}{1+N} \\
          p=1       & \text{if} & q>\frac{1}{1+N}
      \end{array}\right. \\
      BR_2(p)=\left\{ \begin{array}{lcl}
          q=0       & \text{if} & p<\frac{N}{1+N}  \\
          q\in[0,1] & \text{if} & p=\frac{N}{1+N} \\
          q=1       & \text{if} & p>\frac{N}{1+N}
      \end{array}\right.
    \end{align*}
    \vspace{-4pt}
    \begin{align*}
      NE=\left\{(0,0);(1,1);\left(\frac{N}{N+1},\frac{1}{N+1}\right)\right\}
    \end{align*}
  \vfill\null \columnbreak
    When $N\rightarrow\infty$, what happens to the probability of successful coordination?\\\medskip
    \includegraphics[width=\columnwidth]{figures/4b_empty}
    \textbf{\textit{To illustrate it, plot Player 1's BR function, $\bm{p^{*}(q)}$, e.g. for $\bm{N=9}$.}}
  \vfill\null
  \end{multicols}
\end{frame}
\begin{frame}{PS4, Ex. 4.b: Generalized Battle of the Sexes (plotting BR functions)}
  \begin{multicols}{2}
    \begin{itemize}
      \item[(b)] Find the MSNE.
    \end{itemize}
    \vspace{-12pt}
    \begin{table}
      \begin{tabular}{cl|c|c|}
          & \multicolumn{1}{c}{} & \multicolumn{2}{c}{\color{blue}Player 2}\\
          \parbox[t]{1mm}{\multirow{3}{*}{\rotatebox[origin=r]{90}{\color{red}Player 1}}}
          & \multicolumn{1}{c}{} & \multicolumn{1}{c}{C1 (q)} & \multicolumn{1}{c}{C2 (1-q)} \\\cline{3-4}
          & C1 (p)    & \textcolor{red}{N}, \textcolor{blue}{1} & 0, 0 \\\cline{3-4}
          & C2 (1-p)  & 0, 0 & \textcolor{red}{1}, \textcolor{blue}{N} \\\cline{3-4}
      \end{tabular}
    \end{table}
    Player 1 is indifferent for:
    \begin{align*}
      Nq &= 1-q \Leftrightarrow q = \frac{1}{1+N}
    \end{align*}
    Player 2 is indifferent for:
    \begin{align*}
      p &= N(1-p) \Leftrightarrow p = \frac{N}{1+N}
    \end{align*}
    \vspace{-10pt}
    \begin{align*}
      BR_1(q)=\left\{ \begin{array}{lcl}
          p=0       & \text{if} & q<\frac{1}{1+N} \\
          p\in[0,1] & \text{if} & q=\frac{1}{1+N} \\
          p=1       & \text{if} & q>\frac{1}{1+N}
      \end{array}\right. \\
      BR_2(p)=\left\{ \begin{array}{lcl}
          q=0       & \text{if} & p<\frac{N}{1+N}  \\
          q\in[0,1] & \text{if} & p=\frac{N}{1+N} \\
          q=1       & \text{if} & p>\frac{N}{1+N}
      \end{array}\right.
    \end{align*}
    \vspace{-6pt}
    \begin{align*}
      NE=\left\{(0,0);(1,1);\left(\frac{N}{N+1},\frac{1}{N+1}\right)\right\}
    \end{align*}
  \vfill\null \columnbreak
    When $N\rightarrow\infty$, what happens to the probability of successful coordination?\\\medskip
    \includegraphics[width=\columnwidth]{figures/4b_}
    \textbf{\textit{Plot Player 2's BR function, $\bm{q^{*}(p)}$, for the same large value of N (e.g. $\bm{N=9}$).}}
  \vfill\null
  \end{multicols}
\end{frame}
\begin{frame}{PS4, Ex. 4.b: Generalized Battle of the Sexes (plotting BR functions)}
  \begin{multicols}{2}
    \begin{itemize}
      \item[(b)] Find the MSNE.
    \end{itemize}
    \vspace{-12pt}
    \begin{table}
      \begin{tabular}{cl|c|c|}
          & \multicolumn{1}{c}{} & \multicolumn{2}{c}{\color{blue}Player 2}\\
          \parbox[t]{1mm}{\multirow{3}{*}{\rotatebox[origin=r]{90}{\color{red}Player 1}}}
          & \multicolumn{1}{c}{} & \multicolumn{1}{c}{C1 (q)} & \multicolumn{1}{c}{C2 (1-q)} \\\cline{3-4}
          & C1 (p)    & \textcolor{red}{N}, \textcolor{blue}{1} & 0, 0 \\\cline{3-4}
          & C2 (1-p)  & 0, 0 & \textcolor{red}{1}, \textcolor{blue}{N} \\\cline{3-4}
      \end{tabular}
    \end{table}
    Player 1 is indifferent for:
    \vspace{-6pt}
    \begin{align*}
      Nq &= 1-q \Leftrightarrow q = \frac{1}{1+N}
    \end{align*}
    Player 2 is indifferent for:
    \vspace{-6pt}
    \begin{align*}
      p &= N(1-p) \Leftrightarrow p = \frac{N}{1+N}
    \end{align*}
    \vspace{-12pt}
    \begin{align*}
      BR_1(q)=\left\{ \begin{array}{lcl}
          p=0       & \text{if} & q<\frac{1}{1+N} \\
          p\in[0,1] & \text{if} & q=\frac{1}{1+N} \\
          p=1       & \text{if} & q>\frac{1}{1+N}
      \end{array}\right. \\
      BR_2(p)=\left\{ \begin{array}{lcl}
          q=0       & \text{if} & p<\frac{N}{1+N}  \\
          q\in[0,1] & \text{if} & p=\frac{N}{1+N} \\
          q=1       & \text{if} & p>\frac{N}{1+N}
      \end{array}\right.
    \end{align*}
    \vspace{-10pt}
    \begin{align*}
      NE=\left\{(0,0);(1,1);\left(\frac{N}{N+1},\frac{1}{N+1}\right)\right\}
    \end{align*}
  \vfill\null \columnbreak
    When $N\rightarrow\infty$, what happens to the probability of successful coordination?\\\medskip
    \includegraphics[width=\columnwidth]{figures/4b}
    \textbf{\textit{In the MSNE, what happens to $\bm{p^{*}}$ and $\bm{q^{*}}$ when $\bm{N\rightarrow\infty}$? What happens to the expected payoffs in the MSNE?}}
  \vfill\null
  \end{multicols}
\end{frame}
\begin{frame}{PS4, Ex. 4.b: Generalized Battle of the Sexes (plotting BR functions)}
  \begin{multicols}{2}
    \begin{itemize}
      \item[(b)] Find the MSNE.
    \end{itemize}
    \vspace{-18pt}
    \begin{table}
      \begin{tabular}{cl|c|c|}
          & \multicolumn{1}{c}{} & \multicolumn{2}{c}{\color{blue}Player 2}\\
          \parbox[t]{1mm}{\multirow{3}{*}{\rotatebox[origin=r]{90}{\color{red}Player 1}}}
          & \multicolumn{1}{c}{} & \multicolumn{1}{c}{C1 (q)} & \multicolumn{1}{c}{C2 (1-q)} \\\cline{3-4}
          & C1 (p)    & \textcolor{red}{N}, \textcolor{blue}{1} & 0, 0 \\\cline{3-4}
          & C2 (1-p)  & 0, 0 & \textcolor{red}{1}, \textcolor{blue}{N} \\\cline{3-4}
      \end{tabular}
    \end{table}
    Player 1 is indifferent for:
    \vspace{-6pt}
    \begin{align*}
      Nq &= 1-q \Leftrightarrow q = \frac{1}{1+N}
    \end{align*}
    Player 2 is indifferent for:
    \vspace{-6pt}
    \begin{align*}
      p &= N(1-p) \Leftrightarrow p = \frac{N}{1+N}
    \end{align*}
    \vspace{-16pt}
    \begin{align*}
      BR_1(q)=\left\{ \begin{array}{lcl}
          p=0       & \text{if} & q<\frac{1}{1+N} \\
          p\in[0,1] & \text{if} & q=\frac{1}{1+N} \\
          p=1       & \text{if} & q>\frac{1}{1+N}
      \end{array}\right. \\
      BR_2(p)=\left\{ \begin{array}{lcl}
          q=0       & \text{if} & p<\frac{N}{1+N}  \\
          q\in[0,1] & \text{if} & p=\frac{N}{1+N} \\
          q=1       & \text{if} & p>\frac{N}{1+N}
      \end{array}\right.
    \end{align*}
    \vspace{-12pt}
    \begin{align*}
      NE=\left\{(0,0);(1,1);\left(\frac{N}{N+1},\frac{1}{N+1}\right)\right\}
    \end{align*}
  \vfill\null \columnbreak
    When $N\rightarrow\infty$, what happens to the probability of successful coordination?\\\medskip
    \vspace{-4pt}
    \includegraphics[width=\columnwidth]{figures/4b}
    \vspace{-18pt}
    \begin{align*}
        MSNE:(p^{*},q^{*},u_1^{*},u_2^{*})\xrightarrow[N\to\infty]{}(1,0,0,0)
    \end{align*}
    When $N$ is large, coordination is difficult as Player 1 plays C1 most of the time and player 2 plays C2 most of the time.
  \vfill\null
  \end{multicols}
\end{frame}


\section{Take Home Assignment 1 (theorems and backwards induction)}

\begin{frame}{Take Home Assignment 1, Ex. 1-2: Theorems}
  \begin{itemize}
    \item[(1)] \textbf{Nash's theorem (\href{https://rbsc.princeton.edu/sites/default/files/Non-Cooperative_Games_Nash.pdf}{John Nash, 1950}):}
  \end{itemize}
    \begin{tabular}{|l|}
      \cline{1-1}
        All finite games (finite number of players with finitely many strategies)\\
        have at least one Nash Equilibrium. Some of these game may only have an\\
        equilibrium in mixed strategies.\\\cline{1-1}
    \end{tabular}
  \vfill\null
\end{frame}
\begin{frame}{Take Home Assignment 1, Ex. 1-2: Theorems}
  \begin{itemize}
    \item[(1)] \textbf{Nash's theorem (\href{https://rbsc.princeton.edu/sites/default/files/Non-Cooperative_Games_Nash.pdf}{John Nash, 1950}):}
  \end{itemize}
    \begin{tabular}{|l|}
      \cline{1-1}
        All finite games (finite number of players with finitely many strategies)\\
        have at least one Nash Equilibrium. Some of these game may only have an\\
        equilibrium in mixed strategies.\\\cline{1-1}
    \end{tabular}\\\bigskip
  Refinement:
  \begin{itemize}
    \item[(2)] \textbf{The Oddness Theorem (\href{http://www.dklevine.com/archive/refs4402.pdf}{Robert Wilson, 1971}; \href{https://link.springer.com/article/10.1007\%2FBF01737572\#page-1 }{John Charles Harsanyi, 1973}):}
  \end{itemize}
    \begin{tabular}{|l|}
      \cline{1-1}
        Almost all finite games (finite number of players with finitely many strategies)\\
        have at a finite number of Nash Equilibria, and that number is also odd.\\\cline{1-1}
    \end{tabular}
  \vfill\null
\end{frame}
\begin{frame}{Take Home Assignment 1, Ex. 1-2: Theorems}
  \begin{itemize}
    \item[(1)] \textbf{Nash's theorem (\href{https://rbsc.princeton.edu/sites/default/files/Non-Cooperative_Games_Nash.pdf}{John Nash, 1950}):}
  \end{itemize}
    \begin{tabular}{|l|}
      \cline{1-1}
        All finite games (finite number of players with finitely many strategies)\\
        have at least one Nash Equilibrium. Some of these game may only have an\\
        equilibrium in mixed strategies.\\\cline{1-1}
    \end{tabular}\\\bigskip
  Refinement:
  \begin{itemize}
    \item[(2)] \textbf{The Oddness Theorem (\href{http://www.dklevine.com/archive/refs4402.pdf}{Robert Wilson, 1971}; \href{https://link.springer.com/article/10.1007\%2FBF01737572\#page-1 }{John Charles Harsanyi, 1973}):}
  \end{itemize}
    \begin{tabular}{|l|}
      \cline{1-1}
        Almost all finite games (finite number of players with finitely many strategies)\\
        have at a finite number of Nash Equilibria, and that number is also odd.\\\cline{1-1}
    \end{tabular}\\\medskip
    An exception is when one player is indifferent for a \textit{pure} strategy of the other player, e.g. the games we have seen in
    \begin{itemize}
        \item Exercise 2 of the Take Home Assignment.
        \item Exercise 1.b of Problem Set 4.
        \item Exercise 7.b and 7.c of Problem Set 3.
    \end{itemize}
    In these cases we get an infinite set of equilibria, i.e. the real numbers in an interval.
  \vfill\null
\end{frame}

\begin{frame}{Take Home Assignment 1, Ex. 3: Backwards induction}
  \begin{itemize}
    \item[3.] A dynamic game.
  \end{itemize}
  \begin{figure}[!h]
    \center
    \def\svgwidth{\columnwidth}
    \import{figures/}{TH1_Ex3.pdf_tex}
  \end{figure}
\end{frame}
\begin{frame}{Take Home Assignment 1, Ex. 3.a: Backwards induction}
  \begin{itemize}
    \item[(3a)] The Backwards Induction (BI) solution.
  \end{itemize}
  \begin{figure}[!h]
    \center
    \def\svgwidth{\columnwidth}
    \import{figures/}{TH1_Ex3_color.pdf_tex}
  \end{figure}
  The BI solution is the complete strategy profile - on and off the equilibrium path:
  \begin{align*}
    (best\ responses\ for\ player\ 1,\ best\ responses\ for\ player\ 2)=(s_1,s_2)=(Rrr',R'L'')
  \end{align*}
\end{frame}
\begin{frame}{Take Home Assignment 1, Ex. 3.b: Backwards induction}
  \begin{itemize}
    \item[(3b)] Player 1: Any improvement from deleting a strategy.
  \end{itemize}
  \begin{figure}[!h]
    \center
    \def\svgwidth{\columnwidth}
    \import{figures/}{TH1_Ex3b_suboptimal.pdf_tex}
  \end{figure}
  Most of you answered that Player 1 can improve his outcome by deleting $r$ to get 3.\\\medskip
  This is true, but it is suboptimal (for Player 1 at least).
\end{frame}
\begin{frame}{Take Home Assignment 1, Ex. 3.b: Backwards induction}
  \begin{itemize}
    \item[(3b)] Player 1: The best improvement from deleting a strategy.
  \end{itemize}
  \begin{figure}[!h]
    \center
    \def\svgwidth{\columnwidth}
    \import{figures/}{TH1_Ex3b_improved.pdf_tex}
  \end{figure}
  Player 1 can do better by deleting $r'$. Then he can use the \textit{\textbf{first mover advantage}} to choose the left side of the tree and use the \textit{\textbf{last mover advantage}} to pick $r$ and get $4$.\\\medskip
  Player 2 has limited agency in the middle stage, but picks $L'$ to avoid negative payoffs.
\end{frame}
\begin{frame}{Take Home Assignment 1, Ex. 3.c: Backwards induction}
  \begin{itemize}
    \item[(3c)] Player 2: Show there can be no improvement from deleting a strategy,
  \end{itemize}
  \begin{figure}[!h]
    \center
    \def\svgwidth{\columnwidth}
    \import{figures/}{TH1_Ex3_color.pdf_tex}
  \end{figure}
  The foolproof way: Delete $L',R',L'',R''$ one at the time and solve all 4 new games.\\\medskip
  The smart way: Argue that the only possible improvement for Player 2 would be to end up in $(3,3)$, but Player 1 would never choose $l$ over $r$.
\end{frame}


\section{PS4, Ex. 5: North-Atlantic, 1943 (MSNE)}

\begin{frame}{PS4, Ex. 5: North-Atlantic, 1943 (MSNE)}
  North-Atlantic, 1943. An allied convoy, counting 100 ships, is heading east and it can choose between a northern route where icebergs are known to be numerous or a more southern route. The northern route is dangerous - because of the icebergs - and it is estimated that 6 ships will get lost due to icebergs. Below the surface, the wolf-pack lures. If the u-boats catch the convoy on the southern route, it is a field day, and 40 ships from the convoy are estimated to get lost. If the u-boats catch the convoy on the northern route, they do not have as much time hunting down the convoy - due to petrol shortages - and they are only expected to be able to sink 20 ships from the convoy. The wolf-pack does not have time to check both locations, north and south. Each headquarter (allied or nazi) has to decide whether to go north or south. Unfortunately, there is no radar etc, so one cannot observe the move of the enemy before taking a decision. Each headquarter has a simple payoff function. For the allied headquarter it equals the number of ships making it across the Atlantic. For the nazi headquarter payoff equals the number of ships lost by the allies.
  \begin{itemize}
    \item[(a)] Write down this strategic situation in a bi-matrix.
    \item[(b)] Find the Nash Equilibrium (equilibria?)
    \item[(c)] In equilibrium, what is the expected number of ships that make it across the Atlantic?
  \end{itemize}
  \vfill\null
\end{frame}

\begin{frame}{PS4, Ex. 5.a: North-Atlantic, 1943 (MSNE)}
      North-Atlantic, 1943. An allied convoy, counting 100 ships, is heading east and it can choose between a northern route where icebergs are known to be numerous or a more southern route. The northern route is dangerous - because of the icebergs - and it is estimated that 6 ships will get lost due to icebergs. Below the surface, the wolf-pack lures. If the u-boats catch the convoy on the southern route, it is a field day, and 40 ships from the convoy are estimated to get lost. If the u-boats catch the convoy on the northern route, they do not have as much time hunting down the convoy - due to petrol shortages - and they are only expected to be able to sink 20 ships from the convoy. The wolf-pack does not have time to check both locations, north and south. Each headquarter (allied or nazi) has to decide whether to go north or south. Unfortunately, there is no radar etc, so one cannot observe the move of the enemy before taking a decision. Each headquarter has a simple payoff function. For the allied headquarter it equals the number of ships making it across the Atlantic. For the nazi headquarter payoff equals the number of ships lost by the allies.
    \begin{itemize}
      \item[(a)] Write down this strategic situation in a bi-matrix.
    \end{itemize}
    Let the chance of the nazis going north be noted by q, and the chance they go south be noted by 1-q. The chance the Allied go north is then noted by p, and the chance they go south is noted by 1-p.
  \vfill\null
\end{frame}
\begin{frame}{PS4, Ex. 5.a: North-Atlantic, 1943 (MSNE)}
    North-Atlantic, 1943. An allied convoy, counting 100 ships, is heading east and it can choose between a northern route where icebergs are known to be numerous or a more southern route. The northern route is dangerous - because of the icebergs - and it is estimated that 6 ships will get lost due to icebergs. Below the surface, the wolf-pack lures. If the u-boats catch the convoy on the southern route, it is a field day, and 40 ships from the convoy are estimated to get lost. If the u-boats catch the convoy on the northern route, they do not have as much time hunting down the convoy - due to petrol shortages - and they are only expected to be able to sink 20 ships from the convoy. The wolf-pack does not have time to check both locations, north and south. Each headquarter (allied or nazi) has to decide whether to go north or south. Unfortunately, there is no radar etc, so one cannot observe the move of the enemy before taking a decision. Each headquarter has a simple payoff function. For the allied headquarter it equals the number of ships making it across the Atlantic. For the nazi headquarter payoff equals the number of ships lost by the allies.
    \begin{itemize}
      \item[(a)] Write down this strategic situation in a bi-matrix:
    \end{itemize}
    \vspace{-8pt}
    \begin{table}
      \begin{tabular}{cl|c|c|}
          & \multicolumn{1}{c}{} & \multicolumn{2}{c}{Nazis}\\
          \parbox[t]{1mm}{\multirow{3}{*}{\rotatebox[origin=r]{90}{Allied}}}
          & \multicolumn{1}{c}{} & \multicolumn{1}{c}{North (q)} & \multicolumn{1}{c}{South (1-q)} \\\cline{3-4}
          & North (p)    & 74, 26 & 94, 6 \\\cline{3-4}
          & South (1-p)  & 100, 0 & 60, 40 \\\cline{3-4}
      \end{tabular}
    \end{table}
    \begin{itemize}
      \item[(b)] Find the Nash Equilibrium (equilibria?)
    \end{itemize}
  \vfill\null
\end{frame}

\begin{frame}{PS4, Ex. 5.b: North-Atlantic, 1943 (MSNE)}
    \begin{itemize}
      \item[(a)] Write down this strategic situation in a bi-matrix:
    \end{itemize}
    \vspace{-12pt}
    \begin{table}
      \begin{tabular}{cl|c|c|}
          & \multicolumn{1}{c}{} & \multicolumn{2}{c}{\color{blue}Nazis}\\
          \parbox[t]{1mm}{\multirow{3}{*}{\rotatebox[origin=r]{90}{\color{red}Allied}}}
          & \multicolumn{1}{c}{} & \multicolumn{1}{c}{North (q)} & \multicolumn{1}{c}{South (1-q)} \\\cline{3-4}
          & North (p)    & 74, \textcolor{blue}{26} & \textcolor{red}{94}, 6 \\\cline{3-4}
          & South (1-p)  & \textcolor{red}{100}, 0 & 60, \textcolor{blue}{40} \\\cline{3-4}
      \end{tabular}
    \end{table}
    \begin{itemize}
      \item[(b)] Find the Nash Equilibrium (equilibria?): \\\medskip
      It's a zero-sum (100-sum) type game like matching-pennies or rock-paper-scissors.\\
      Thus, no pure strategy NE (PSNE), but a mixed strategy NE (MSNE) must exist.\\\medskip
      \textbf{\textit{Find q such that the Allied are indifferent.}}
    \end{itemize}
  \vfill\null
\end{frame}
\begin{frame}{PS4, Ex. 5.b: North-Atlantic, 1943 (MSNE)}
    \begin{itemize}
      \item[(a)] Write down this strategic situation in a bi-matrix: \\ It's a zero-sum (100-sum) type game:
    \end{itemize}
    \vspace{-12pt}
    \begin{table}
      \begin{tabular}{cl|c|c|}
          & \multicolumn{1}{c}{} & \multicolumn{2}{c}{\color{blue}Nazis}\\
          \parbox[t]{1mm}{\multirow{3}{*}{\rotatebox[origin=r]{90}{\color{red}Allied}}}
          & \multicolumn{1}{c}{} & \multicolumn{1}{c}{North (q)} & \multicolumn{1}{c}{South (1-q)} \\\cline{3-4}
          & North (p)    & 74, \textcolor{blue}{26} & \textcolor{red}{94}, 6 \\\cline{3-4}
          & South (1-p)  & \textcolor{red}{100}, 0 & 60, \textcolor{blue}{40} \\\cline{3-4}
      \end{tabular}
    \end{table}
    \begin{itemize}
      \item[(b)] Find the Nash Equilibrium (equilibria?): There are no PSNE, find the MSNE:
    \end{itemize}
    The Allied are indifferent for:
    \begin{align*}
      E[u_A|North]&=E[u_A|South]\\
      74q + 94(1-q) &= 100q + 60(1-q) \Leftrightarrow ... \Leftrightarrow q = \frac{17}{30}
    \end{align*}
    \textbf{\textit{Find p such that the Nazis are indifferent.}}
  \vfill\null
\end{frame}
\begin{frame}{PS4, Ex. 5.b: North-Atlantic, 1943 (MSNE)}
    \begin{itemize}
      \item[(a)] Write down this strategic situation in a bi-matrix: \\ It's a zero-sum (100-sum) type game:
    \end{itemize}
    \vspace{-12pt}
    \begin{table}
      \begin{tabular}{cl|c|c|}
          & \multicolumn{1}{c}{} & \multicolumn{2}{c}{\color{blue}Nazis}\\
          \parbox[t]{1mm}{\multirow{3}{*}{\rotatebox[origin=r]{90}{\color{red}Allied}}}
          & \multicolumn{1}{c}{} & \multicolumn{1}{c}{North (q)} & \multicolumn{1}{c}{South (1-q)} \\\cline{3-4}
          & North (p)    & 74, \textcolor{blue}{26} & \textcolor{red}{94}, 6 \\\cline{3-4}
          & South (1-p)  & \textcolor{red}{100}, 0 & 60, \textcolor{blue}{40} \\\cline{3-4}
      \end{tabular}
    \end{table}
    \begin{itemize}
      \item[(b)] Find the Nash Equilibrium (equilibria?): There are no PSNE, find the MSNE:
    \end{itemize}
    The Allied are indifferent for:
    \begin{align*}
      E[u_A|North]&=E[u_A|South]\\
      74q + 94(1-q) &= 100q + 60(1-q) \Leftrightarrow ... \Leftrightarrow q = \frac{17}{30}
    \end{align*}
    The Nazis are indifferent for:
    \begin{align*}
      E[u_N|North]&=E[u_N|South]\\
      26p &= 6p + 40(1-p) \Leftrightarrow ... \Leftrightarrow p = \frac{2}{3}
    \end{align*}
    \textbf{\textit{Write up all Nash Equilibria, $\bm{(p^{*},q^{*})}$.}}
  \vfill\null
\end{frame}
\begin{frame}{PS4, Ex. 5.b: North-Atlantic, 1943 (MSNE)}
    \begin{itemize}
      \item[(a)] Write down this strategic situation in a bi-matrix: \\ It's a zero-sum (100-sum) type game:
    \end{itemize}
    \vspace{-12pt}
    \begin{table}
      \begin{tabular}{cl|c|c|}
          & \multicolumn{1}{c}{} & \multicolumn{2}{c}{\color{blue}Nazis}\\
          \parbox[t]{1mm}{\multirow{3}{*}{\rotatebox[origin=r]{90}{\color{red}Allied}}}
          & \multicolumn{1}{c}{} & \multicolumn{1}{c}{North (q)} & \multicolumn{1}{c}{South (1-q)} \\\cline{3-4}
          & North (p)    & 74, \textcolor{blue}{26} & \textcolor{red}{94}, 6 \\\cline{3-4}
          & South (1-p)  & \textcolor{red}{100}, 0 & 60, \textcolor{blue}{40} \\\cline{3-4}
      \end{tabular}
    \end{table}
    \begin{itemize}
      \item[(b)] Find the Nash Equilibrium (equilibria?): There are no PSNE, find the MSNE:
    \end{itemize}
    The Allied are indifferent for:
    \begin{align*}
      E[u_A|North]&=E[u_A|South]\\
      74q + 94(1-q) &= 100q + 60(1-q) \Leftrightarrow ... \Leftrightarrow q = \frac{17}{30}
    \end{align*}
    The Nazis are indifferent for:
    \begin{align*}
      E[u_N|North]&=E[u_N|South]\\
      26p &= 6p + 40(1-p) \Leftrightarrow ... \Leftrightarrow p = \frac{2}{3}
    \end{align*}
    The unique NE is:
    \begin{align*}
       NE=(p^{*},q^{*})=\left(\frac{2}{3},\frac{17}{30}\right)
    \end{align*}
    \vspace{-10pt}
    \begin{itemize}
      \item[(c)] In equilibrium, what is the expected number of ships that make it across the Atlantic?
    \end{itemize}
  \vfill\null
\end{frame}

\begin{frame}{PS4, Ex. 5.c: North-Atlantic, 1943 (MSNE)}
    \begin{table}
      \begin{tabular}{cl|c|c|}
          & \multicolumn{1}{c}{} & \multicolumn{2}{c}{\color{blue}Nazis}\\
          \parbox[t]{1mm}{\multirow{3}{*}{\rotatebox[origin=r]{90}{\color{red}Allied}}}
          & \multicolumn{1}{c}{} & \multicolumn{1}{c}{North (q)} & \multicolumn{1}{c}{South (1-q)} \\\cline{3-4}
          & North (p)    & 74, \textcolor{blue}{26} & \textcolor{red}{94}, 6 \\\cline{3-4}
          & South (1-p)  & \textcolor{red}{100}, 0 & 60, \textcolor{blue}{40} \\\cline{3-4}
      \end{tabular}
    \end{table}
    \vspace{-8pt}
    \begin{align*}
       NE=(p^{*},q^{*})=\left(\frac{2}{3},\frac{17}{30}\right)
    \end{align*}
    \vspace{-10pt}
    \begin{itemize}
      \item[(c)] In equilibrium, what is the expected number of ships that make it across the Atlantic?
    \end{itemize}
    \textbf{\textit{First, write up the Allied's expected utility of playing North and South respectively.}}
  \vfill\null
\end{frame}
\begin{frame}{PS4, Ex. 5.c: North-Atlantic, 1943 (MSNE)}
    \begin{table}
      \begin{tabular}{cl|c|c|}
          & \multicolumn{1}{c}{} & \multicolumn{2}{c}{\color{blue}Nazis}\\
          \parbox[t]{1mm}{\multirow{3}{*}{\rotatebox[origin=r]{90}{\color{red}Allied}}}
          & \multicolumn{1}{c}{} & \multicolumn{1}{c}{North (q)} & \multicolumn{1}{c}{South (1-q)} \\\cline{3-4}
          & North (p)    & 74, \textcolor{blue}{26} & \textcolor{red}{94}, 6 \\\cline{3-4}
          & South (1-p)  & \textcolor{red}{100}, 0 & 60, \textcolor{blue}{40} \\\cline{3-4}
      \end{tabular}
    \end{table}
    \vspace{-8pt}
    \begin{align*}
       NE=(p^{*},q^{*})=\left(\frac{2}{3},\frac{17}{30}\right)
    \end{align*}
    \vspace{-10pt}
    \begin{itemize}
      \item[(c)] In equilibrium, what is the expected number of ships that make it across the Atlantic?
    \end{itemize}
    In general, the Allied's expected utility of playing North and South respectively:
    \begin{align*}
      E[u_A|North]&=74q + 94(1-q)=94-20q\\
      E[u_A|South]&=100q + 60(1-q)=60+40q
    \end{align*}
    \textbf{\textit{Then write up the Allied's expected utility in the equilibrium: $\bm{E[u_A|p^{*},q^{*}]}$.}}
  \vfill\null
\end{frame}
\begin{frame}{PS4, Ex. 5.c: North-Atlantic, 1943 (MSNE)}
    \begin{table}
      \begin{tabular}{cl|c|c|}
          & \multicolumn{1}{c}{} & \multicolumn{2}{c}{\color{blue}Nazis}\\
          \parbox[t]{1mm}{\multirow{3}{*}{\rotatebox[origin=r]{90}{\color{red}Allied}}}
          & \multicolumn{1}{c}{} & \multicolumn{1}{c}{North (q)} & \multicolumn{1}{c}{South (1-q)} \\\cline{3-4}
          & North (p)    & 74, \textcolor{blue}{26} & \textcolor{red}{94}, 6 \\\cline{3-4}
          & South (1-p)  & \textcolor{red}{100}, 0 & 60, \textcolor{blue}{40} \\\cline{3-4}
      \end{tabular}
    \end{table}
    \vspace{-8pt}
    \begin{align*}
       NE=(p^{*},q^{*})=\left(\frac{2}{3},\frac{17}{30}\right)
    \end{align*}
    \vspace{-10pt}
    \begin{itemize}
      \item[(c)] In equilibrium, what is the expected number of ships that make it across the Atlantic?
    \end{itemize}
    In general, the Allied's expected utility of playing North and South respectively:
    \begin{align}
      E[u_A|North]&=74q + 94(1-q)=94-20q\label{north}\\
      E[u_A|South]&=100q + 60(1-q)=60+40q\label{south}
    \end{align}
    In equilibrium, the Allied's expected utility:
    \begin{align*}
      E[u_A|p^{*},q^{*}] &= p^{*}E[u_A|North]+(1-p^{*})E[u_A|South]\\
        &= p^{*}(94-20q^{*})+(1-p^{*})(60+40q^{*}),\quad\quad\text{using eq. \eqref{north} and \eqref{south}}\\
        &= \frac{2}{3}\left(94-20\frac{17}{30}\right)+\frac{1}{3}\left(60+40\frac{17}{30}\right)\\
        &\approx 73.48 % from typing (2/3)(94-20(17/30)+(1/3)(60+40(17/30) on Google
    \end{align*}
  \vfill\null
\end{frame}


\section{PS4, Ex. 6: Stopping the bike thief (MSNE)}

\begin{frame}{PS4, Ex. 6: Stopping the bike thief (MSNE)}
  \begin{multicols}{2}
    As in Problem Set 2, there are $N\geq2$ people observing someone trying to steal a parked bike. Each of the witnesses would like the thief to be stopped, but prefers not to do it him/herself (because it is unpleasant and perhaps even dangerous). More precisely, if the thief is stopped by someone else, each of the witnesses gets a utility of $v > 0$. Every person who stops the thief gets a utility of $v-c>0$, where $c$ is the cost of interaction with the thief. Finally, if nobody stops the thief and the bike gets stolen, every witness gets a utility of $0$. The witnesses decide whether or not to stop the thief simultaneously and independently.
  \vfill\null \columnbreak
    \begin{itemize}
      \item[a)] Solve for a symmetric mixed strategy equilibrium of this game, where each witness stops the thief with probability $p\in(0,1)$.
      \item[b)] Discuss what happens to $p$ as the number of witness becomes very large. What happens then to the probability that the thief will get stopped? What is the intuition for this result?
    \end{itemize}
    \includegraphics[width=\columnwidth]{figures/bike_thief}
  \vfill\null
  \end{multicols}
\end{frame}

\begin{frame}{PS4, Ex. 6.a: Stopping the bike thief (MSNE)}
    Payoffs for player $i\neq j$:
    \begin{align*}
      u_i(s_i,s_j)=
      \left\{ \begin{array}{rcl}
      v > 0 & \mbox{if} & \mbox{$i$ does nothing and $j$ stops the thief} \\
      v-c>0 & \mbox{if} & \mbox{$i$ stops the thief} \\
      0     & \mbox{if} & \mbox{nobody stops the thief}
      \end{array}\right.
    \end{align*}
    \vspace{-12pt}
    \begin{itemize}
      \item[a)] Solve for a symmetric mixed strategy equilibrium of this game, where each witness stops the thief with probability $p\in(0,1)$.
    \end{itemize}
    \textbf{\textit{Taking advantage of symmetry, find the probability p such that person i is indifferent between stopping the thief or not. That is, her expected payoff from stopping the thief equals her expected payoff from someone else stopping the thief.}}
  \vfill\null
\end{frame}
\begin{frame}{PS4, Ex. 6.a: Stopping the bike thief (MSNE)}
    Payoffs for player $i\neq j$:
    \begin{align*}
      u_i(s_i,s_j)=
      \left\{ \begin{array}{rcl}
      v > 0 & \mbox{if} & \mbox{$i$ does nothing and $j$ stops the thief} \\
      v-c>0 & \mbox{if} & \mbox{$i$ stops the thief} \\
      0     & \mbox{if} & \mbox{nobody stops the thief}
      \end{array}\right.
    \end{align*}
    \vspace{-12pt}
    \begin{itemize}
      \item[a)] Solve for a symmetric mixed strategy equilibrium of this game, where each witness stops the thief with probability $p\in(0,1)$.
    \end{itemize}
    Person $i$ is indifferent between stopping the thief or not when her expected payoff from stopping the thief equals her expected payoff from someone else stopping the thief:
    \begin{align*}
        E[u_i|i\text{ stops thief}]&=Prob[j\text{ stops thief}]\times E[u_i|j\text{ stops thief}],\quad i\neq j\in \mathcal{J}=1,...,1-N\\
        E[u_i|i\text{ stops thief}]&=\left(1-Prob[\text{nobody in $\mathcal{J}$ stops thief}]\right)\times E[u_i|j\text{ stops thief}]
    \end{align*}
  \vfill\null
\end{frame}
\begin{frame}{PS4, Ex. 6.a: Stopping the bike thief (MSNE)}
    Payoffs for player $i\neq j$:
    \begin{align*}
      u_i(s_i,s_j)=
      \left\{ \begin{array}{rcl}
      v > 0 & \mbox{if} & \mbox{$i$ does nothing and $j$ stops the thief} \\
      v-c>0 & \mbox{if} & \mbox{$i$ stops the thief} \\
      0     & \mbox{if} & \mbox{nobody stops the thief}
      \end{array}\right.
    \end{align*}
    \vspace{-12pt}
    \begin{itemize}
      \item[a)] Solve for a symmetric mixed strategy equilibrium of this game, where each witness stops the thief with probability $p\in(0,1)$.
    \end{itemize}
    Person $i$ is indifferent between stopping the thief or not when her expected payoff from stopping the thief equals her expected payoff from someone else stopping the thief:
    \begin{align*}
        E[u_i|i\text{ stops thief}]&=Prob[j\text{ stops thief}]\times E[u_i|j\text{ stops thief}],\quad i\neq j\in \mathcal{J}=1,...,1-N\\
        E[u_i|i\text{ stops thief}]&=\left(1-Prob[\text{nobody in $\mathcal{J}$ stops thief}]\right)\times E[u_i|j\text{ stops thief}]\\
        v-c&=\left(1-(1-p)^{N-1}\right)\times v\\
        \frac{c}{v}&=(1-p)^{N-1}\\
        p^{*}&=1-\left(\frac{c}{v}\right)^{\frac{1}{N-1}},
        \quad\quad\quad\quad\quad\quad\quad\quad\quad\quad\quad\ 0<\frac{c}{v}<1
    \end{align*}
    \textbf{\textit{Is there a mixed strategy NE?}}
  \vfill\null
\end{frame}
\begin{frame}{PS4, Ex. 6.a: Stopping the bike thief (MSNE)}
    Payoffs for player $i\neq j$:
    \begin{align*}
      u_i(s_i,s_j)=
      \left\{ \begin{array}{rcl}
      v > 0 & \mbox{if} & \mbox{$i$ does nothing and $j$ stops the thief} \\
      v-c>0 & \mbox{if} & \mbox{$i$ stops the thief} \\
      0     & \mbox{if} & \mbox{nobody stops the thief}
      \end{array}\right.
    \end{align*}
    \vspace{-12pt}
    \begin{itemize}
      \item[a)] Solve for a symmetric mixed strategy equilibrium of this game, where each witness stops the thief with probability $p\in(0,1)$.
    \end{itemize}
    Person $i$ is indifferent between stopping the thief or not when her expected payoff from stopping the thief equals her expected payoff from someone else stopping the thief:
    \begin{align}
        E[u_i|i\text{ stops thief}]&=Prob[j\text{ stops thief}]\times E[u_i|j\text{ stops thief}],\quad i\neq j\in \mathcal{J}=1,...,1-N\nonumber\\
        E[u_i|i\text{ stops thief}]&=\left(1-Prob[\text{nobody in $\mathcal{J}$ stops thief}]\right)\times E[u_i|j\text{ stops thief}]\nonumber\\
        v-c&=\left(1-(1-p)^{N-1}\right)\times v\nonumber\\
        \frac{c}{v}&=(1-p)^{N-1}\label{c over v}\\
        p^{*}&=1-\left(\frac{c}{v}\right)^{\frac{1}{N-1}},
        \quad\quad\quad\quad\quad\quad\quad\quad\quad\quad\quad\ 0<\frac{c}{v}<1\nonumber
    \end{align}
    The MSNE is where each persons stops the thief with probability $p^{*}=1-\left(\frac{c}{v}\right)^{\frac{1}{N-1}}$
  \vfill\null
\end{frame}

\begin{frame}{PS4, Ex. 6.b: Stopping the bike thief (MSNE)}
    The MSNE is where each persons stops the thief with probability $p^{*}=1-\left(\frac{c}{v}\right)^{\frac{1}{N-1}}$
    \begin{itemize}
      \item[b)] Discuss what happens to $p$ as the number of witness becomes very large. What happens then to the probability that the thief will get stopped? What is the intuition for this result?
    \end{itemize}
    \textbf{\textit{Calculate the probability that the thief is stopped.}}
  \vfill\null
\end{frame}
\begin{frame}{PS4, Ex. 6.b: Stopping the bike thief (MSNE)}
    The MSNE is where each persons stops the thief with probability $p^{*}=1-\left(\frac{c}{v}\right)^{\frac{1}{N-1}}$
    \begin{itemize}
      \item[b)] Discuss what happens to $p$ as the number of witness becomes very large. What happens then to the probability that the thief will get stopped? What is the intuition for this result?
    \end{itemize}
    \textbf{\textit{Calculate the probability that the thief is stopped.}}
    \begin{align*}
        Prob[\text{the thief is stopped}]&=1-Prob[\text{nobody stops the thief}]
        &\vdots
    \end{align*}
  \vfill\null
\end{frame}
\begin{frame}{PS4, Ex. 6.b: Stopping the bike thief (MSNE)}
    The MSNE is where each persons stops the thief with probability $p^{*}=1-\left(\frac{c}{v}\right)^{\frac{1}{N-1}}$
    \begin{itemize}
      \item[b)] Discuss what happens to $p$ as the number of witness becomes very large. What happens then to the probability that the thief will get stopped? What is the intuition for this result?
    \end{itemize}
    \vspace{-10pt}
    \begin{align*}
        Prob[\text{the thief is stopped}]&=1-Prob[\text{nobody stops the thief}]\\
            &=1-(1-p^{*})^{N}\\
            &=1-(1-p^{*})^{N-1}(1-p^{*})\\
            &=1-\frac{c}{v}(1-p^{*}),&&\text{inserting eq. \eqref{c over v} from 6.a}\\
            &=1-\frac{c}{v}\left(1-1+\left(\frac{c}{v}\right)^{\frac{1}{N-1}}\right),&&\text{inserting the MSNE}\\
            &=1-\frac{c}{v}^{\left(1+\frac{1}{N-1}\right)}
    \end{align*}
    \textbf{\textit{When $\bm{N\rightarrow\infty}$, what happens to the probability of the thief being caught?}}
  \vfill\null
\end{frame}
\begin{frame}{PS4, Ex. 6.b: Stopping the bike thief (MSNE)}
    The MSNE is where each persons stops the thief with probability $p^{*}=1-\left(\frac{c}{v}\right)^{\frac{1}{N-1}}$
    \begin{itemize}
      \item[b)] Discuss what happens to $p$ as the number of witness becomes very large. What happens then to the probability that the thief will get stopped? What is the intuition for this result?
    \end{itemize}
    \vspace{-10pt}
    \begin{align*}
        Prob[\text{the thief is stopped}]&=1-Prob[\text{nobody stops the thief}]\\
            &=1-(1-p^{*})^{N}\\
            &=1-(1-p^{*})^{N-1}(1-p^{*})\\
            &=1-\frac{c}{v}(1-p^{*}),&&\text{inserting eq. \eqref{c over v} from 6.a}\\
            &=1-\frac{c}{v}\left(1-1+\left(\frac{c}{v}\right)^{\frac{1}{N-1}}\right),&&\text{inserting the MSNE}\\
            &=1-\frac{c}{v}^{\left(1+\frac{1}{N-1}\right)}\xrightarrow[N\to\infty]{}1-\frac{c}{v}
    \end{align*}
    When $N\rightarrow\infty$, the probability of the thief being caught decreases due to an increase in the incentive to freeride. Remember: there's a cost to stopping the thief yourself:
    \begin{align*}
        E[u_i|p^{*}]&=v\cdot Prob[\text{the thief is stopped}]-c\cdot p^{*}\\
            &=v\cdot \left(1-(1-p^{*})^{N}\right)-c\cdot p^{*}\\
            &= v-c^{\left(1+\frac{1}{N-1}\right)}-c+\left(\frac{c^2}{v}\right)^{\frac{1}{N-1}}
            \xrightarrow[N\to\infty]{} v-2c+1
    \end{align*}
  \vfill\null
\end{frame}


\section{PS4, Ex. 7: To keep or split (backwards induction)}

\begin{frame}{PS4, Ex. 7: To keep or split (backwards induction)}
  \begin{multicols}{2}
    Consider the following 2 × 2 game where payoffs are monetary:
    \begin{table}
      \begin{tabular}{l|c|c|}
          \multicolumn{1}{c}{} & \multicolumn{1}{c}{L} & \multicolumn{1}{c}{R} \\\cline{2-3}
          T & 3, 3 & 0, 4 \\\cline{2-3}
          B & 4, 0 & 1, 1 \\\cline{2-3}
      \end{tabular}
    \end{table}
    Before this game is played, Player 1 can choose whether, after the game is played, players should keep their own payoffs or split the aggregate payoff evenly between them.
  \vfill\null \columnbreak
    \begin{itemize}
      \item[(a)] Draw the game tree of this two-stage game (assuming that Players 1’s choice of whether to split payoffs is revealed to Player 2 before the second stage).
      \item[(b)] Solve by backwards induction.
    \end{itemize}
  \vfill\null
  \end{multicols}
\end{frame}

\begin{frame}{PS4, Ex. 7.a: To keep or split (backwards induction)}
  \begin{multicols}{2}
    \begin{itemize}
      \item[(a)] Draw the game tree:
    \end{itemize}
    \begin{table}
    \begin{tabular}{l|c|c|}
    \multicolumn{1}{c}{} & \multicolumn{1}{c}{L} & \multicolumn{1}{c}{R} \\\cline{2-3}
    T & 3, 3 & 0, 4 \\\cline{2-3}
    B & 4, 0 & 1, 1 \\\cline{2-3}
    \end{tabular}
    \end{table}
    \nth{2} stage: The above is the static game for a keep game, find the static game for a  split game and draw the full game tree.
  \vfill\null \columnbreak
  \vfill\null
  \end{multicols}
\end{frame}
\begin{frame}{PS4, Ex. 7.a: To keep or split (backwards induction)}
  \begin{multicols}{2}
    \begin{itemize}
      \item[(a)] Draw the game tree:
    \end{itemize}
    \nth{1} stage: Player 1 chooses Keep or Split. Player 2 observes the choice.\\\medskip
    \nth{2} stage: They play the static game and payoffs are realized.
    \begin{figure}[!h]
      \center
      \def\svgwidth{.5\columnwidth}
      \import{figures/}{7a.pdf_tex}
    \end{figure}
    \vspace{-9pt}
    \begin{table}
      \begin{tabular}{l|c|c|}
        \multicolumn{1}{c}{} & \multicolumn{1}{c}{L} & \multicolumn{1}{c}{R} \\\cline{2-3}
        T & 3, 3 & 0, 4 \\\cline{2-3}
        B & 4, 0 & 1, 1 \\\cline{2-3}
      \end{tabular}\
      \begin{tabular}{l|c|c|}
        \multicolumn{1}{c}{} & \multicolumn{1}{c}{L'} & \multicolumn{1}{c}{R'} \\\cline{2-3}
        T' & 3, 3 & 2, 2 \\\cline{2-3}
        B' & 2, 2 & 1, 1 \\\cline{2-3}
      \end{tabular}
    \end{table}
  \vfill\null \columnbreak
    \begin{itemize}
    \item[(b)] Solve by backwards induction:
    \end{itemize}
  \vfill\null
  \end{multicols}
\end{frame}

\begin{frame}{PS4, Ex. 7.b: To keep or split (backwards induction)}
  \begin{multicols}{2}
    \begin{itemize}
      \item[(a)] Draw the game tree:
    \end{itemize}
    \nth{1} stage: Player 1 chooses Keep or Split. Player 2 observes the choice.\\\medskip
    \nth{2} stage: They play the static game and payoffs are realized.
    \begin{figure}[!h]
      \center
      \def\svgwidth{.5\columnwidth}
      \import{figures/}{7a_color.pdf_tex}
    \end{figure}
    \vspace{-9pt}
    \begin{table}
      \begin{tabular}{l|c|c|}
        \multicolumn{1}{c}{} & \multicolumn{1}{c}{L} & \multicolumn{1}{c}{\textcolor{blue}{R}} \\\cline{2-3}
        T & 3, 3 & 0, \textcolor{blue}{4} \\\cline{2-3}
        \textcolor{red}{B} & \textcolor{red}{4}, 0 & \textcolor{red}{1}, \textcolor{blue}{1} \\\cline{2-3}
        \end{tabular}\
        \begin{tabular}{l|c|c|}
          \multicolumn{1}{c}{} & \multicolumn{1}{c}{\textcolor{blue}{L'}} & \multicolumn{1}{c}{R'} \\\cline{2-3}
          \textcolor{red}{T'} & \textcolor{red}{3}, \textcolor{blue}{3} & \textcolor{red}{2}, 2 \\\cline{2-3}
          B' & 2, \textcolor{blue}{2} & 1, 1 \\\cline{2-3}
        \end{tabular}
    \end{table}
  \vfill\null \columnbreak
    \begin{itemize}
      \item[(b)] Solve by backwards induction:
    \end{itemize}
    \nth{2} stage: Each bi-matrix has a unique NE that can be founds using IESDS.\\\medskip
    \nth{1} stage: Player 1's choice can be reduced to choosing between the subgame NE in each bi-matrix:
    \begin{figure}[!h]
      \center
      \def\svgwidth{.5\columnwidth}
      \import{figures/}{7b.pdf_tex}
    \end{figure}
    BI gives the subgame perfect NE:
    \begin{align*}
      SPNE=(Split\ B\ T',\ R\ L')
    \end{align*}
  \vfill\null
  \end{multicols}
\end{frame}

\begin{frame}{PS4, Ex. 7: To keep or split (backwards induction)}
  Alternatively, draw the game tree in extensive form to find $SPNE:(Split\ B\ T',\ R\ L')$\\\medskip
  \nth{1} stage: Player 1 chooses Keep or Split. Player 2 observes the choice.\\\medskip
  \nth{2} stage: Player 2 chooses $L$ or $R$ ($L'$ or $R'$). The action is private information.\\\medskip
  \nth{3} stage: Player 1 chooses $T$ or $B$ ($T'$ or $B'$) without knowing what Player 2 did.
  \vspace{-4pt}
  \begin{figure}[!h]
    \center
    \def\svgwidth{\columnwidth}
    \import{figures/}{7b_extensive_form.pdf_tex}
  \end{figure}
  \vspace{-4pt}
  The order of stage 2 and 3 is arbitrary, but the \nth{2} stage must be private information.
  %(you can swap stage 2 and 3, then the new \nth{2} stage would be private information.)
\end{frame}


% \section{Problem Set 3, Ex. 5: Luxembourg as a rogue state (static game)}
%
% \begin{frame}{Problem Set 3, Ex. 5: Luxembourg as a rogue state (static game)}
%   \begin{multicols}{2}
%     Assume that Luxembourg has turned into a rogue state. It is close to acquiring nuclear weapons, which would threaten the stability in the whole region. The Vatican ($V$) and Denmark ($D$) are preparing an attack on Luxembourg’s nuclear research facilities to stop or slow down its nuclear program. The probability that the attack will be a success is
%     \begin{align*}
%       p(s_V,s_D)=s_V+s_D-s_vs_D,
%     \end{align*}
%     where $s_i\in[0,1]$ is the share of its military capacity that country $i\ (i\in\{V,D\})$ uses in the attack. If the attack is successful then each country receives a payoff of 1. The cost of participating in the attack for country $i$ is
%     \begin{align*}
%       c_i(s_i)=s_i^2
%     \end{align*}
%     The objective of each country is to maximize its expected payoff from the attack minus the cost.
%   \vfill\null\columnbreak
%     \begin{itemize}
%       \item[(a)] Suppose that the Vatican and Denmark choose the shares of military capacity to use in the attack simultaneously and independently. Find the Nash equilibrium (NE) of this game.
%       \item[(b)] Find the social optimum (SO) under the condition that the two countries use the same share of their military capacity. I.e., find the $\bar{s}_V=\bar{s}_D=\bar{s}$ that maximizes aggregate payoff from the attack minus costs. Compare with the equilibrium from question (a) and give an intuitive explanation of your findings.
%     \end{itemize}
%     \hfill \includegraphics[width=0.20 \textwidth]{figures/nuclear}
%   \vfill\null
%   \end{multicols}
% \end{frame}
%
% \begin{frame}{Problem Set 3, Ex. 5.a: Luxembourg as a rogue state (static game)}
%   \begin{multicols}{2}
%     Assume that Luxembourg has turned into a rogue state. It is close to acquiring nuclear weapons, which would threaten the stability in the whole region. The Vatican ($V$) and Denmark ($D$) are preparing an attack on Luxembourg’s nuclear research facilities to stop or slow down its nuclear program. The probability that the attack will be a success is
%     \begin{align*}
%       p(s_V,s_D)=s_V+s_D-s_vs_D,
%     \end{align*}
%     where $s_i\in[0,1]$ is the share of its military capacity that country $i\ (i\in\{V,D\})$ uses in the attack. If the attack is successful then each country receives a payoff of 1. The cost of participating in the attack for country $i$ is
%     \begin{align*}
%       c_i(s_i)=s_i^2
%     \end{align*}
%     The objective of each country is to maximize its expected payoff from the attack minus the cost.
%   \vfill\null\columnbreak
%     \begin{itemize}
%       \item[(a)] Suppose that the Vatican and Denmark choose the shares of military capacity to use in the attack simultaneously and independently. Find the Nash equilibrium (NE) of this game.
%     \end{itemize}
%     Write expected payoff for player $i\neq j$.
%   \vfill\null
%   \end{multicols}
% \end{frame}
% \begin{frame}{Problem Set 3, Ex. 5.a: Luxembourg as a rogue state (static game)}
%   \begin{multicols}{2}
%     Assume that Luxembourg has turned into a rogue state. It is close to acquiring nuclear weapons, which would threaten the stability in the whole region. The Vatican ($V$) and Denmark ($D$) are preparing an attack on Luxembourg’s nuclear research facilities to stop or slow down its nuclear program. The probability that the attack will be a success is
%     \begin{align*}
%       p(s_V,s_D)=s_V+s_D-s_vs_D,
%     \end{align*}
%     where $s_i\in[0,1]$ is the share of its military capacity that country $i\ (i\in\{V,D\})$ uses in the attack. If the attack is successful then each country receives a payoff of 1. The cost of participating in the attack for country $i$ is
%     \begin{align*}
%       c_i(s_i)=s_i^2
%     \end{align*}
%     The objective of each country is to maximize its expected payoff from the attack minus the cost.
%   \vfill\null\columnbreak
%     \begin{itemize}
%       \item[(a)] Suppose that the Vatican and Denmark choose the shares of military capacity to use in the attack simultaneously and independently. Find the Nash equilibrium (NE) of this game.
%     \end{itemize}
%     Expected payoff for player $i\neq j$:
%     \begin{align*}
%       u_i(s_i,s_j)=\underbrace{s_i+s_j-s_is_j}_\text{Probability of success}-\underbrace{s_i^2}_\text{Cost}
%     \end{align*}
%     Find the best-response function for $i$.
%   \vfill\null
%   \end{multicols}
% \end{frame}
% \begin{frame}{Problem Set 3, Ex. 5.a: Luxembourg as a rogue state (static game)}
%   \begin{multicols}{2}
%     Assume that Luxembourg has turned into a rogue state. It is close to acquiring nuclear weapons, which would threaten the stability in the whole region. The Vatican ($V$) and Denmark ($D$) are preparing an attack on Luxembourg’s nuclear research facilities to stop or slow down its nuclear program. The probability that the attack will be a success is
%     \begin{align*}
%       p(s_V,s_D)=s_V+s_D-s_vs_D,
%     \end{align*}
%     where $s_i\in[0,1]$ is the share of its military capacity that country $i\ (i\in\{V,D\})$ uses in the attack. If the attack is successful then each country receives a payoff of 1. The cost of participating in the attack for country $i$ is
%     \begin{align*}
%       c_i(s_i)=s_i^2
%     \end{align*}
%     The objective of each country is to maximize its expected payoff from the attack minus the cost.
%   \vfill\null\columnbreak
%     \begin{itemize}
%       \item[(a)] Suppose that the Vatican and Denmark choose the shares of military capacity to use in the attack simultaneously and independently. Find the Nash equilibrium (NE) of this game.
%     \end{itemize}
%     Expected payoff for player $i\neq j$:
%     \begin{align*}
%       u_i(s_i,s_j)=\underbrace{s_i+s_j-s_is_j}_\text{Probability of success}-\underbrace{s_i^2}_\text{Cost}
%     \end{align*}
%     Find the best-response function for $i$:
%     \begin{align*}
%       FOC:\ \frac{\delta u_i}{\delta s_i}=1+0-s_j-2s_i&=0\\
%        s_i&=\frac{1-s_j}{2}
%     \end{align*}
%     What is the NE?\\\medskip
%     (Hint: is the game symmetric?)
%   \vfill\null
%   \end{multicols}
% \end{frame}
% \begin{frame}{Problem Set 3, Ex. 5.a: Luxembourg as a rogue state (static game)}
%   \begin{multicols}{2}
%     Assume that Luxembourg has turned into a rogue state. It is close to acquiring nuclear weapons, which would threaten the stability in the whole region. The Vatican ($V$) and Denmark ($D$) are preparing an attack on Luxembourg’s nuclear research facilities to stop or slow down its nuclear program. The probability that the attack will be a success is
%     \begin{align*}
%       p(s_V,s_D)=s_V+s_D-s_vs_D,
%     \end{align*}
%     where $s_i\in[0,1]$ is the share of its military capacity that country $i\ (i\in\{V,D\})$ uses in the attack. If the attack is successful then each country receives a payoff of 1. The cost of participating in the attack for country $i$ is
%     \begin{align*}
%       c_i(s_i)=s_i^2
%     \end{align*}
%     The objective of each country is to maximize its expected payoff from the attack minus the cost.
%   \vfill\null\columnbreak
%     \begin{itemize}
%       \item[(a)] Find the NE in the static game:
%     \end{itemize}
%     Expected payoff for player $i\neq j$:
%     \begin{align*}
%       u_i(s_i,s_j)=\underbrace{s_i+s_j-s_is_j}_\text{Probability of success}-\underbrace{s_i^2}_\text{Cost}
%     \end{align*}
%     Find the best-response function for $i$:
%     \begin{align*}
%       FOC:\ \frac{\delta u_i}{\delta s_i}=1+0-s_j-2s_i&=0\\
%        s_i&=\frac{1-s_j}{2}
%     \end{align*}
%     Taking advantage of symmetry $s_i^{*}=s_j^{*}$:
%     \begin{align*}
%        s_i^{*}&=\frac{1-s_i^{*}}{2}\\
%       2s_i^{*}+s_i^{*}&=1\\
%        s_i^{*}&=\frac{1}{3}\equiv s^{NE}
%     \end{align*}
%     i.e. $NE=\left\{(s_D^{*},s_V^{*})=(\frac{1}{3},\frac{1}{3})\right\}$
%   \vfill\null
%   \end{multicols}
% \end{frame}
%
% \begin{frame}{Problem Set 3, Ex. 5.b: Luxembourg as a rogue state (static game)}
%   \begin{multicols}{2}
%     \begin{itemize}
%       \item[(a)] Find the NE in the static game:
%     \end{itemize}
%     Expected payoff for player $i\neq j$:
%     \begin{align*}
%       u_i(s_i,s_j)=\underbrace{s_i+s_j-s_is_j}_\text{Probability of success}-\underbrace{s_i^2}_\text{Cost}
%     \end{align*}
%     Find the best-response function for $i$:
%     \begin{align*}
%       FOC:\ \frac{\delta u_i}{\delta s_i}=1+0-s_j-2s_i&=0\\
%        s_i&=\frac{1-s_j}{2}
%     \end{align*}
%     Taking advantage of symmetry $s_i^{*}=s_j^{*}$:
%     \begin{align*}
%        s_i^{*}&=\frac{1-s_i^{*}}{2}\\
%       2s_i^{*}+s_i^{*}&=1\\
%        s_i^{*}&=\frac{1}{3}\equiv s^{NE}
%     \end{align*}
%     i.e. $NE=\left\{(s_D^{*},s_V^{*})=(\frac{1}{3},\frac{1}{3})\right\}$
%   \vfill\null\columnbreak
%     \begin{itemize}
%       \item[(b)] Find the social optimum (SO) under the condition that the two countries use the same share of their military capacity. I.e., find the $\bar{s}_V=\bar{s}_D=\bar{s}$ that maximizes aggregate payoff from the attack minus costs. Compare with the equilibrium from question (a) and give an intuitive explanation of your findings.
%     \end{itemize}
%   \vfill\null
%   \end{multicols}
% \end{frame}
% \begin{frame}{Problem Set 3, Ex. 5.b: Luxembourg as a rogue state (static game)}
%   \begin{multicols}{2}
%     \begin{itemize}
%       \item[(a)] Find the NE in the static game:
%     \end{itemize}
%     Expected payoff for player $i\neq j$:
%     \begin{align*}
%       u_i(s_i,s_j)=\underbrace{s_i+s_j-s_is_j}_\text{Probability of success}-\underbrace{s_i^2}_\text{Cost}
%     \end{align*}
%     Find the best-response function for $i$:
%     \begin{align*}
%       FOC:\ \frac{\delta u_i}{\delta s_i}=1+0-s_j-2s_i&=0\\
%        s_i&=\frac{1-s_j}{2}
%     \end{align*}
%     Taking advantage of symmetry $s_i^{*}=s_j^{*}$:
%     \begin{align*}
%        s_i^{*}&=\frac{1-s_i^{*}}{2}\\
%       2s_i^{*}+s_i^{*}&=1\\
%        s_i^{*}&=\frac{1}{3}\equiv s^{NE}
%     \end{align*}
%     i.e. $NE=\left\{(s_D^{*},s_V^{*})=(\frac{1}{3},\frac{1}{3})\right\}$
%   \vfill\null\columnbreak
%     \begin{itemize}
%       \item[(b)] Find the social optimum (SO) under the condition that the two countries use the same share of their military capacity. I.e., find the $\bar{s}_V=\bar{s}_D=\bar{s}$ that maximizes aggregate payoff from the attack minus costs. Compare with the equilibrium from question (a) and give an intuitive explanation of your findings.
%     \end{itemize}
%     Write expected payoff for player $i\neq j$.
%   \vfill\null
%   \end{multicols}
% \end{frame}
% \begin{frame}{Problem Set 3, Ex. 5.b: Luxembourg as a rogue state (static game)}
%   \begin{multicols}{2}
%     \begin{itemize}
%       \item[(a)] Find the NE in the static game:
%     \end{itemize}
%     Expected payoff for player $i\neq j$:
%     \begin{align*}
%       u_i(s_i,s_j)=\underbrace{s_i+s_j-s_is_j}_\text{Probability of success}-\underbrace{s_i^2}_\text{Cost}
%     \end{align*}
%     Find the best-response function for $i$:
%     \begin{align*}
%       FOC:\ \frac{\delta u_i}{\delta s_i}=1+0-s_j-2s_i&=0\\
%        s_i&=\frac{1-s_j}{2}
%     \end{align*}
%     Taking advantage of symmetry $s_i^{*}=s_j^{*}$:
%     \begin{align*}
%        s_i^{*}&=\frac{1-s_i^{*}}{2}\\
%       2s_i^{*}+s_i^{*}&=1\\
%        s_i^{*}&=\frac{1}{3}\equiv s^{NE}
%     \end{align*}
%     i.e. $NE=\left\{(s_D^{*},s_V^{*})=(\frac{1}{3},\frac{1}{3})\right\}$
%   \vfill\null\columnbreak
%     \begin{itemize}
%       \item[(b)] Find the social optimum (SO) under the condition that the two countries use the same share of their military capacity. I.e., find the $\bar{s}_V=\bar{s}_D=\bar{s}$ that maximizes aggregate payoff from the attack minus costs. Compare with the equilibrium from question (a) and give an intuitive explanation of your findings.
%     \end{itemize}
%     Expected payoff for $i$, $\bar{s}_D=\bar{s}_V=\bar{s}$:
%     \begin{align*}
%       u_i(\bar{s})&=\underbrace{\bar{s}+\bar{s}-\bar{s}\bar{s}}_\text{Probability of success}-\underbrace{\bar{s}^2}_\text{Cost}\\
%                   &=2\bar{s}-2\bar{s}^2
%     \end{align*}
%     Find the social planner target function.
%   \vfill\null
%   \end{multicols}
% \end{frame}
% \begin{frame}{Problem Set 3, Ex. 5.b: Luxembourg as a rogue state (static game)}
%   \begin{multicols}{2}
%     \begin{itemize}
%       \item[(a)] Find the NE in the static game:
%     \end{itemize}
%     Expected payoff for player $i\neq j$:
%     \begin{align*}
%       u_i(s_i,s_j)=\underbrace{s_i+s_j-s_is_j}_\text{Probability of success}-\underbrace{s_i^2}_\text{Cost}
%     \end{align*}
%     Find the best-response function for $i$:
%     \begin{align*}
%       FOC:\ \frac{\delta u_i}{\delta s_i}=1+0-s_j-2s_i&=0\\
%        s_i&=\frac{1-s_j}{2}
%     \end{align*}
%     Taking advantage of symmetry $s_i^{*}=s_j^{*}$:
%     \begin{align*}
%        s_i^{*}&=\frac{1-s_i^{*}}{2}\\
%       2s_i^{*}+s_i^{*}&=1\\
%        s_i^{*}&=\frac{1}{3}\equiv s^{NE}
%     \end{align*}
%     i.e. $NE=\left\{(s_D^{*},s_V^{*})=(\frac{1}{3},\frac{1}{3})\right\}$
%   \vfill\null\columnbreak
%     \begin{itemize}
%       \item[(b)] Find the social optimum (SO) under the condition that the two countries use the same share of their military capacity. I.e., find the $\bar{s}_V=\bar{s}_D=\bar{s}$ that maximizes aggregate payoff from the attack minus costs. Compare with the equilibrium from question (a) and give an intuitive explanation of your findings.
%     \end{itemize}
%     Expected payoff for $i$, $\bar{s}_D=\bar{s}_V=\bar{s}$:
%     \vspace{-6pt}
%     \begin{align*}
%       u_i(\bar{s})&=\underbrace{\bar{s}+\bar{s}-\bar{s}\bar{s}}_\text{Probability of success}-\underbrace{\bar{s}^2}_\text{Cost}\\
%                   &=2\bar{s}-2\bar{s}^2
%     \end{align*}
%     \vspace{-6pt}
%     The social planner target function:
%     \begin{align*}
%       \pi^S(\bar{s})&=\underbrace{2}_\text{Countries}(2\bar{s}-2\bar{s}^2)=4\bar{s}-4\bar{s}^2
%     \end{align*}
%     \vspace{-6pt}
%     Find the social optimum (SO).
%   \vfill\null
%   \end{multicols}
% \end{frame}
% \begin{frame}{Problem Set 3, Ex. 5.b: Luxembourg as a rogue state (static game)}
%   \begin{multicols}{2}
%     \begin{itemize}
%       \item[(a)] Find the NE in the static game:
%     \end{itemize}
%     Expected payoff for player $i\neq j$:
%     \begin{align*}
%       u_i(s_i,s_j)=\underbrace{s_i+s_j-s_is_j}_\text{Probability of success}-\underbrace{s_i^2}_\text{Cost}
%     \end{align*}
%     Find the best-response function for $i$:
%     \begin{align*}
%       FOC:\ \frac{\delta u_i}{\delta s_i}=1+0-s_j-2s_i&=0\\
%        s_i&=\frac{1-s_j}{2}
%     \end{align*}
%     Taking advantage of symmetry $s_i^{*}=s_j^{*}$:
%     \begin{align*}
%        s_i^{*}&=\frac{1-s_i^{*}}{2}\\
%       2s_i^{*}+s_i^{*}&=1\\
%        s_i^{*}&=\frac{1}{3}\equiv s^{NE}
%     \end{align*}
%     i.e. $NE=\left\{(s_D^{*},s_V^{*})=(\frac{1}{3},\frac{1}{3})\right\}$
%   \vfill\null\columnbreak
%     \begin{itemize}
%       \item[(b)] Find the SO given shares are equal:
%     \end{itemize}
%     Expected payoff for $i$, $\bar{s}_D=\bar{s}_V=\bar{s}$:
%     \begin{align*}
%       u_i(\bar{s})&=\underbrace{\bar{s}+\bar{s}-\bar{s}\bar{s}}_\text{Probability of success}-\underbrace{\bar{s}^2}_\text{Cost}\\
%                   &=2\bar{s}-2\bar{s}^2
%     \end{align*}
%     Social planner target function:
%     \begin{align*}
%       \pi^S(\bar{s})&=\underbrace{2}_\text{Countries}(2\bar{s}-2\bar{s}^2)=4\bar{s}-4\bar{s}^2
%     \end{align*}
%     Find the social optimum (SO):
%     \begin{align*}
%       FOC:\ \frac{\delta\pi^S}{\delta s_i}=4-8\bar{S}&=0\\
%        \bar{S}&=\frac{4}{8}=\frac{1}{2}>\frac{1}{3}
%     \end{align*}
%     Compare with the equilibrium from question (a) and give an intuitive explanation of your findings.
%   \vfill\null
%   \end{multicols}
% \end{frame}
% \begin{frame}{Problem Set 3, Ex. 5.b: Luxembourg as a rogue state (static game)}
%   \begin{multicols}{2}
%     \begin{itemize}
%       \item[(a)] Find the NE in the static game:
%     \end{itemize}
%     Expected payoff for player $i\neq j$:
%     \begin{align*}
%       u_i(s_i,s_j)=\underbrace{s_i+s_j-s_is_j}_\text{Probability of success}-\underbrace{s_i^2}_\text{Cost}
%     \end{align*}
%     Find the best-response function for $i$:
%     \begin{align*}
%       FOC:\ \frac{\delta u_i}{\delta s_i}=1+0-s_j-2s_i&=0\\
%        s_i&=\frac{1-s_j}{2}
%     \end{align*}
%     Taking advantage of symmetry $s_i^{*}=s_j^{*}$:
%     \begin{align*}
%        s_i^{*}&=\frac{1-s_i^{*}}{2}\\
%       2s_i^{*}+s_i^{*}&=1\\
%        s_i^{*}&=\frac{1}{3}\equiv s^{NE}
%     \end{align*}
%     i.e. $NE=\left\{(s_D^{*},s_V^{*})=(\frac{1}{3},\frac{1}{3})\right\}$
%   \vfill\null\columnbreak
%     \begin{itemize}
%       \item[(b)] Find the SO given shares are equal:
%     \end{itemize}
%     Expected payoff for $i$, $\bar{s}_D=\bar{s}_V=\bar{s}$:
%     \begin{align*}
%       u_i(\bar{s})&=\underbrace{\bar{s}+\bar{s}-\bar{s}\bar{s}}_\text{Probability of success}-\underbrace{\bar{s}^2}_\text{Cost}\\
%                   &=2\bar{s}-2\bar{s}^2
%     \end{align*}
%     Social planner target function:
%     \begin{align*}
%       \pi^S(\bar{s})&=\underbrace{2}_\text{Countries}(2\bar{s}-2\bar{s}^2)=4\bar{s}-4\bar{s}^2
%     \end{align*}
%     Find the social optimum (SO):
%     \begin{align*}
%       FOC:\ \frac{\delta\pi^S}{\delta s_i}=4-8\bar{S}&=0\\
%        \bar{S}&=\frac{4}{8}=\frac{1}{2}>\frac{1}{3}
%     \end{align*}
%     The SO is higher than the NE as the positive externality is not rewarded, which leads to an incentive to free ride.
%   \vfill\null
%   \end{multicols}
% \end{frame}


\section{PS4, Ex. 8: Building a playground (Stackelberg game)}

\begin{frame}{PS4, Ex. 8: Building a playground (Stackelberg game)}
  \begin{multicols}{2}
    Two neighbors are building a common playground for their children. The time spent on the project by neighbor $i$ is $x_i \geq 0,\ i = 1, 2$. The resulting quality of the playground is
    \begin{align*}
      q(x_1,x_2)=x_1+x_2-x_1x_2
    \end{align*}
    Spending time on the project is costly. More precisely, the cost function of the neighbors are:
    \begin{align*}
      C_i(x_i)=x_i^2,\ \ \ i=1,2
    \end{align*}
    The payoff of neighbor $i$, $U_i$, is equal to the quality of the playground minus his cost.
    \includegraphics[width=\columnwidth]{figures/playground}
  \vfill\null \columnbreak
    \begin{itemize}
      \item[(a)] Suppose the neighbors decide how much time to spend on the project simultaneously and independently. Derive the best response functions. Find the Nash equilibrium of this game.
      \item[(b)] Suppose now that the game is played in two stages. First, neighbor 1 decides how much time to spend on the project. Neighbor 2 observes this and then chooses how much time to put in himself. Find the backwards induction outcome of this game.
      \item[(c)] Compare the games from (a) and (b) with respect to the payoff that each neighbor obtains. Give an intuitive explanation of your results.
    \end{itemize}
  \vfill\null
  \end{multicols}
\end{frame}

\begin{frame}{PS4, Ex. 8.a: Building a playground (Stackelberg game)}
    \begin{itemize}
    \item[(a)] Suppose the neighbors decide how much time to spend on the project simultaneously and independently. Derive the best response functions. Find the Nash equilibrium of this game.
    \end{itemize}
    \vfill\null
  \begin{multicols}{2}
    \begin{itemize}
      \item[(Step 1)] Write up the payoff function
    \end{itemize}
    \vfill\null \columnbreak
    Information so far
    \begin{itemize}
      \item[1] Quality: \begin{math}q(x_1,x_2)=x_1+x_2-x_1x_2 \end{math}
      \item[2] Cost: \begin{math}C_i(x_i)=x_i^2,\ \ \ i=1,2  \end{math}
      \item[3] Payoff: Quality-Cost
    \end{itemize}
    \vfill\null
  \end{multicols}
\end{frame}
\begin{frame}{PS4, Ex. 8.a: Building a playground (Stackelberg game)}
    \begin{itemize}
    \item[(a)] Suppose the neighbors decide how much time to spend on the project simultaneously and independently. Derive the best response functions. Find the Nash equilibrium of this game.
    \end{itemize}
    \vfill\null
  \begin{multicols}{2}
    \begin{itemize}
      \item[(Step 1)] Write up the payoff function
       \item[(Step 2)] Write up the FOC and find the best response function
    \end{itemize}
    \vfill\null \columnbreak
    Information so far
    \begin{itemize}
      \item[1] \begin{math}Quality: q(x_1,x_2)=x_1+x_2-x_1x_2 \end{math}
      \item[2]\begin{math} Cost: C_i(x_i)=x_i^2,\ \ \ i=1,2  \end{math}
      \item[3] \begin{math}Payoff: U_i=x_1+x_2-x_1x_2-x_i^2\ \ \ i=1,2  \end{math}
    \end{itemize}
    \vfill\null
  \end{multicols}
\end{frame}
\begin{frame}{PS4, Ex. 8.a: Building a playground (Stackelberg game)}
    \begin{itemize}
    \item[(a)] Suppose the neighbors decide how much time to spend on the project simultaneously and independently. Derive the best response functions. Find the Nash equilibrium of this game.
    \end{itemize}
    \vfill\null
  \begin{multicols}{2}
    \begin{itemize}
      \item[(Step 1)] Write up the payoff function
      \item[(Step 2)] Write up the FOC and find the best response function
      \item[(Step 3)] This is a symmetric game, so the BR are the same for both players, use this to find the NE by substituting (6) into (5) and isolating \begin{math}x_1\end{math}
    \end{itemize}
    \vfill\null \columnbreak
    Information so far
    \begin{itemize}
      \item[1] \begin{math}Quality: q(x_1,x_2)=x_1+x_2-x_1x_2 \end{math}
      \item[2] \begin{math}Cost: C_i(x_i)=x_i^2,\ \ \ i=1,2  \end{math}
      \item[3] \begin{math}Payoff: U_i=x_1+x_2-x_1x_2-x_i^2\ \ \ i=1,2  \end{math}
      \item[4] \begin{math}FOC: 1-x_j-2x_i=0  \end{math}
      \item[5] \begin{math}BR_1: x_1=(1-x_2)/2 \end{math}
      \item[6] \begin{math}BR_2: x_2=(1-x_1)/2 \end{math}
    \end{itemize}
    \vfill\null
  \end{multicols}
\end{frame}
\begin{frame}{PS4, Ex. 8.a: Building a playground (Stackelberg game)}
    \begin{itemize}
    \item[(a)] Suppose the neighbors decide how much time to spend on the project simultaneously and independently. Derive the best response functions. Find the Nash equilibrium of this game.
    \end{itemize}
    \vfill\null
  \begin{multicols}{2}
    \begin{itemize}
      \item[(Step 1)] Write up the payoff function
      \item[(Step 2)] Write up the FOC and find the best response function
      \item[(Step 3)] This is a symmetric game, so the BR are the same for both players, use this to find the NE by substituting (6) into (5) and isolating \begin{math}x_1 \end{math}
      \item[(NE)]\begin{math} x_1=(1-(1-x_1)/2)/2 \Rightarrow x_1=\frac{1}{3}\end{math} \
      \begin{math} x_2=(1-(1-x_2)/2)/2 \Rightarrow x_2=\frac{1}{3}\end{math} \ \begin{math}NE: (\frac{1}{3},\frac{1}{3})\end{math}
    \end{itemize}
    \vfill\null \columnbreak
    Information so far
    \begin{itemize}
      \item[1] \begin{math}Quality: q(x_1,x_2)=x_1+x_2-x_1x_2 \end{math}
      \item[2] \begin{math}Cost: C_i(x_i)=x_i^2,\ \ \ i=1,2  \end{math}
      \item[3] \begin{math}Payoff: U_i=x_1+x_2-x_1x_2-x_i^2\ \ \ i=1,2  \end{math}
      \item[4] \begin{math}FOC: 1-x_j-2x_i=0  \end{math}
      \item[5] \begin{math}BR_1: x_1=(1-x_2)/2 \end{math}
      \item[6] \begin{math}BR_2: x_2=(1-x_1)/2 \end{math}
    \end{itemize}
    \vfill\null
  \end{multicols}
\end{frame}

\begin{frame}{PS4, Ex. 8.b: Building a playground (Stackelberg game)}
    \begin{itemize}
    \item[(b)] Suppose now that the game is played in two stages. First, neighbor 1 decides how much time to spend on the project. Neighbor 2 observes this and then chooses how much time to put in himself. Find the backwards induction outcome of this game.
    \end{itemize}
    \vfill\null
  \begin{multicols}{2}
    \begin{itemize}
      \item[(Step 1)] Write up the new payoff function for player one, where he takes player 2s best response as given. In order words, write his payoff as a function of \begin{math}x_1\end{math} and \begin{math}BR_2(x_1)\end{math}
    \end{itemize}
    \vfill\null \columnbreak
    Information so far
    \begin{itemize}
      \item[1] \begin{math}Quality: q(x_1,x_2)=x_1+x_2-x_1x_2 \end{math}
      \item[2] \begin{math}Cost: C_i(x_i)=x_i^2,\ \ \ i=1,2  \end{math}
      \item[3] \begin{math}U_1(x_1,x_2)=x_1+x_2-x_1x_2-x_1^2 \end{math}
      \item[4] \begin{math}BR_2: x_2=(1-x_1)/2 \end{math}
    \end{itemize}
    \vfill\null
  \end{multicols}
\end{frame}
\begin{frame}{PS4, Ex. 8.b: Building a playground (Stackelberg game)}
    \begin{itemize}
    \item[(b)] Suppose now that the game is played in two stages. First, neighbor 1 decides how much time to spend on the project. Neighbor 2 observes this and then chooses how much time to put in himself. Find the backwards induction outcome of this game.
    \end{itemize}
    \vfill\null
  \begin{multicols}{2}
    \begin{itemize}
      \item[(Step 1)] Write up the new payoff function for player one, where he takes player 2s best response as given. In order words, write his payoff as a function of \begin{math}x_1\end{math} and \begin{math}BR_2(x_1)\end{math}
      \item[(Step 2)] Write up the FOC and find the best response function for player 1, as a function of \begin{math}x_1\end{math} and \begin{math}BR_2(x_1)\end{math}
    \end{itemize}
    \vfill\null \columnbreak
    Information so far
    \begin{itemize}
      \item[1] \begin{math}Quality: q(x_1,x_2)=x_1+x_2-x_1x_2 \end{math}
      \item[2] \begin{math}Cost: C_i(x_i)=x_i^2,\ \ \ i=1,2  \end{math}
      \item[3] \begin{math}U_1(x_1,x_2)=x_1+x_2-x_1x_2-x_1^2 \end{math}
      \item[4] \begin{math}BR_2: x_2=(1-x_1)/2 \end{math}
      \item[5] \begin{math}U_1(x_1,BR_2(x_1)): x_1+(1-x_1)/2-x_1(1-x_1)/2-x_1^2 \end{math}
    \end{itemize}
    \vfill\null
  \end{multicols}
\end{frame}
\begin{frame}{PS4, Ex. 8.b: Building a playground (Stackelberg game)}
    \begin{itemize}
    \item[(b)] Suppose now that the game is played in two stages. First, neighbor 1 decides how much time to spend on the project. Neighbor 2 observes this and then chooses how much time to put in himself. Find the backwards induction outcome of this game.
    \end{itemize}
    \vfill\null
  \begin{multicols}{2}
    \begin{itemize}
      \item[(Step 1)] Write up the new payoff function for player one, where he takes player 2s best response as given. In order words, write his payoff as a function of \begin{math}x_1\end{math} and \begin{math}BR_2(x_1)\end{math}
      \item[(Step 2)] Write up the FOC and find the best response function for player 1, as a function of \begin{math}x_1\end{math} and \begin{math}BR_2(x_1)\end{math}
      \item[(Step 3)] Use the value for $x_1$ to find $x_2$ and write up the SPNE
    \end{itemize}
    \vfill\null \columnbreak
    Information so far
    \begin{itemize}
      \item[1] \begin{math}Quality: q(x_1,x_2)=x_1+x_2-x_1x_2 \end{math}
      \item[2] \begin{math}Cost: C_i(x_i)=x_i^2,\ \ \ i=1,2  \end{math}
      \item[3] \begin{math}U_1(x_1,x_2)=x_1+x_2-x_1x_2-x_1^2 \end{math}
      \item[4] \begin{math}BR_2: x_2=(1-x_1)/2 \end{math}
      \item[5] \begin{math}U_1(x_1,BR_2(x_1)): x_1+(1-x_1)/2-x_1(1-x_1)/2-x_1^2 \end{math}
      \item[6] \begin{math}FOC_1: 1-\frac{1}{2}-\frac{1}{2}-x_1=0 \end{math}
      \item[7] \begin{math}BR_1: x_1=0 \end{math}
    \end{itemize}
    \vfill\null
  \end{multicols}
\end{frame}
\begin{frame}{PS4, Ex. 8.b: Building a playground (Stackelberg game)}
    \begin{itemize}
    \item[(b)] Suppose now that the game is played in two stages. First, neighbor 1 decides how much time to spend on the project. Neighbor 2 observes this and then chooses how much time to put in himself. Find the backwards induction outcome of this game.
    \end{itemize}
    \vfill\null
  \begin{multicols}{2}
    \begin{itemize}
      \item[(Step 1)] Write up the new payoff function for player one, where he takes player 2s best response as given. In other words, write his payoff as a function of \begin{math}x_1\end{math} and \begin{math}BR_2(x_1)\end{math}
      \item[(Step 2)] Write up the FOC and find the best response function for player 1, as a function of \begin{math}x_1\end{math} and \begin{math}BR_2(x_1)\end{math}
      \item[(Step 3)] Use the value for $x_1$ to find $x_2$ and write up the SPNE
      \item[(SPNE)]\begin{math} x_1=0 \end{math} \
      \begin{math} x_2=(1-0)/2 \Rightarrow x_2=\frac{1}{2}\end{math} \ \begin{math}SPNE: (0,\frac{1}{2})\end{math}
    \end{itemize}
    \vfill\null \columnbreak
    Information so far
    \begin{itemize}
      \item[1] \begin{math}Quality: q(x_1,x_2)=x_1+x_2-x_1x_2 \end{math}
      \item[2] \begin{math}Cost: C_i(x_i)=x_i^2,\ \ \ i=1,2  \end{math}
      \item[3] \begin{math}U_1(x_1,x_2)=x_1+x_2-x_1x_2-x_1^2 \end{math}
      \item[4] \begin{math}BR_2: x_2=(1-x_1)/2 \end{math}
      \item[5] \begin{math}U_1(x_1,BR_2(x_1)): x_1+(1-x_1)/2-x_1(1-x_1)/2-x_1^2 \end{math}
      \item[6] \begin{math}FOC_1: 1-\frac{1}{2}-\frac{1}{2}-x_1=0 \end{math}
      \item[7] \begin{math}BR_1: x_1=0 \end{math}
    \end{itemize}
    \vfill\null
  \end{multicols}
\end{frame}

\begin{frame}{PS4, Ex. 8.c: Building a playground (Stackelberg game)}
    \begin{itemize}
    \item[(c)] Compare the games from (a) and (b) with respect to the payoff that each neighbor obtains. Give an intuitive explanation of your results.
    \end{itemize}
    \vfill\null
  \begin{multicols}{2}
    \begin{itemize}
      \item[(Step 1)] What are the payoffs for each player in the two games? What is the total utility?
    \end{itemize}
    \vfill\null \columnbreak
    Information so far
    \begin{itemize}
      \item[G1] \begin{math}NE=\left(\frac{1}{3},\frac{1}{3}\right)\end{math}
      \item[G2] \begin{math}SPNE=\left(0,\frac{1}{2}\right)\end{math}
      \item[Utility] \begin{math}U_i=x_1+x_2-x_1x_2-x_i^2,\ \ \ i=1,2  \end{math}
    \end{itemize}
    \vfill\null
  \end{multicols}
\end{frame}
\begin{frame}{PS4, Ex. 8.c: Building a playground (Stackelberg game)}
    \begin{itemize}
    \item[(c)] Compare the games from (a) and (b) with respect to the payoff that each neighbor obtains. Give an intuitive explanation of your results.
    \end{itemize}
    \vfill\null
  \begin{multicols}{2}
    \begin{itemize}
      \item[(Step 1)] What are the payoffs for each player in the two games? What is the total utility?
      \item[(Step 2)] Compare and explain.
      \item[(Bonus)] If bargaining is possible, does a pareto improvement exist to the outcome in the \nth{2} game?
    \end{itemize}
    \vfill\null \columnbreak
    Information so far
    \begin{itemize}
      \item[G1] \begin{math}NE=\left(\frac{1}{3},\frac{1}{3}\right)\end{math}
      \item[G2] \begin{math}SPNE=\left(0,\frac{1}{2}\right)\end{math}
      \item[Utility] \begin{math}U_i=x_1+x_2-x_1x_2-x_i^2,\ \ \ i=1,2  \end{math}
      \item[G1] \begin{math}U_1=\frac{1}{3}+\frac{1}{3}-\frac{1}{3}^2-\frac{1}{3}^2=\frac{4}{9}\end{math}
      \item[G1] \begin{math}U_2=\frac{1}{3}+\frac{1}{3}-\frac{1}{3}^2-\frac{1}{3}^2=\frac{4}{9}\end{math}
      \item[G1] \begin{math}U_T=\frac{4}{9}+\frac{4}{9}=\frac{8}{9}\end{math}
      \item[G2] \begin{math}U_1'=0+\frac{1}{2}-0\cdot\frac{1}{2}-0^2=\frac{1}{2}\end{math}
      \item[G2] \begin{math}U_2'=0+\frac{1}{3}-0\cdot\frac{1}{2}-\frac{1}{2}^2=\frac{1}{4}\end{math}
      \item[G2] \begin{math}U_T'=\frac{1}{2}+\frac{1}{4}=\frac{3}{4}\end{math}
    \end{itemize}
    \vfill\null
  \end{multicols}
\end{frame}
\begin{frame}{PS4, Ex. 8.c: Building a playground (Stackelberg game)}
    \begin{itemize}
        \item[(c)] Compare the games from (a) and (b) with respect to the payoff that each neighbor obtains. Give an intuitive explanation of your results.
    \end{itemize}
    \vfill\null
  \begin{multicols}{2}
    \begin{itemize}
      \item[(Player 1)] Gets a higher payoff when he gets to choose first (first mover advantage). He chooses to freeride, relying on Player 2 to pick up the slack.
      \item[(Player 2)] Gets a lower payoff when she chooses second. Even though Player 1 freerides, it is still optimal for her to pick up some of the slack.
      \item[(Total U)] Overall utility is lower in the \nth{2} game. With bargaining, P2 could offer P1 compensation in order to remove the freeride opportunity.\\
      E.g. In game 2, if P2 could offer P1 1/9 in order for them to choose at the same time instead. P1 would accept the offer and get the payoff $5/9>1/2$ and P2 would transfer 1/9 and be left with $3/9>1/4$.
      %\item[(Player 2)] Gets a lower payoff when he goes second. Even though player 1 freerides, it is still optimal for him to pick up some of the slack. Given the option to make an offer to remove to freeride opportunity, p2 could make an offer that p1 would accept \\ Example: In game 2, if p2 could offer p1 1/9 in order for them to choose at the same time instead, player 2 would offer it and have a payoff of $3/9>1/4$ and player 1 would accept it and have a payoff of $5/9>1/2$.
    \end{itemize}
    \vfill\null \columnbreak
    Information so far
    \begin{itemize}
      \item[G1] \begin{math}NE=\left(\frac{1}{3},\frac{1}{3}\right)\end{math}
      \item[G2] \begin{math}SPNE=\left(0,\frac{1}{2}\right)\end{math}
      \item[Utility] \begin{math}U_i=x_1+x_2-x_1x_2-x_i^2,\ \ \ i=1,2  \end{math}
      \item[G1] \begin{math}U_1=\frac{1}{3}+\frac{1}{3}-\frac{1}{3}^2-\frac{1}{3}^2=\frac{4}{9}\end{math}
      \item[G1] \begin{math}U_2=\frac{1}{3}+\frac{1}{3}-\frac{1}{3}^2-\frac{1}{3}^2=\frac{4}{9}\end{math}
      \item[G1] \begin{math}U_T=\frac{4}{9}+\frac{4}{9}=\frac{8}{9}\end{math}
      \item[G2] \begin{math}U_1'=0+\frac{1}{2}-0\cdot\frac{1}{2}-0^2=\frac{1}{2}\end{math}
      \item[G2] \begin{math}U_2'=0+\frac{1}{3}-0\cdot\frac{1}{2}-\frac{1}{2}^2=\frac{1}{4}\end{math}
      \item[G2] \begin{math}U_T'=\frac{1}{2}+\frac{1}{4}=\frac{3}{4}\end{math}
      \item[$\rightarrow$] The Coase Theorem (\href{https://en.wikipedia.org/wiki/Coase_theorem}{Coase, 1960}).
    \end{itemize}
    \vfill\null
  \end{multicols}
\end{frame}



% \section{Code examples} % out-comment: ctrl-shift-7 or ctrl-shift-* (use cmd for Mac)
%
% \begin{frame}{Code examples}
%   \begin{multicols}{2}
%     Game tree: % In general, I recommend drawing game trees in the hand as it is the fastest and resembles the exam situation. If you write your assignments on the computer, you can take a picture or leave space to draw the figures after printing. For the slides, I draw the game trees in Inkscape, which is a great piece of free software – when you have gotten used to it… Editing an existing game tree can be a quite straightforward start, but exporting the illustration to a LaTeX document can again be a bit cumbersome. If you’re persistent, you can find “7b_extensive_form.svg” in the figures folder of the zip-file and edit it with Inkscape. As you see, you can use LaTeX code such as $x_1$. Then you save it as type: “Portable Document Format (*.pdf)” and choose “Omit text in PDF and create LaTeX file” and “Use exported object’s size”, which creates two new files (*.pdf and *.pdf_tex). Both must be uploaded to Overleaf to even see how the figures looks, as the files make no sense on their own. To add them to your document, search for “svg” in the main.tex file and re-use my code.
%     \begin{figure}[!h]
%       \center
%       \def\svgwidth{\columnwidth}
%       \import{figures/}{game_tree.pdf_tex}
%     \end{figure}
%   \vfill\null \columnbreak
%     Matrix, no player names:
%     \vspace{-10pt}
%     \begin{table} % as opposed to matrices with player names, each line does not start with "&" as there's no empty column for the name-box. Otherwise, see the explanations below.
%       \begin{tabular}{l|c|c|}
%         \multicolumn{1}{c}{} & \multicolumn{1}{c}{L (q)} & \multicolumn{1}{c}{R (1-q)} \\\cline{2-3}
%         T (p)   &  &  \\\cline{2-3}
%         B (1-p) &  &  \\\cline{2-3}
%       \end{tabular}
%     \end{table}
%     Matrix, no colors:
%     \vspace{-10pt}
%     \begin{table}
%       \begin{tabular}{cl|c|c|} % the number of total columns and which have vertical lines between them (left-align or center text).
%         & \multicolumn{1}{c}{} & \multicolumn{2}{c}{Player 2}\\ % "2" is the number of columns in the matrix that the 2nd player name spans over
%         \parbox[t]{1mm}{\multirow{3}{*}{\rotatebox[origin=r]{90}{Player 1}}} % "3" is the number of rows the 1st player name spans over (including the one with the column names)
%         & \multicolumn{1}{c}{} & \multicolumn{1}{c}{L (q)} & \multicolumn{1}{c}{R (1-q)} \\\cline{3-4} % column names use the "\multicolumn" command to not draw vertical lines between them.
%         & T (p)   &  &  \\\cline{3-4} % a horizontal line is drawn after the line break using "\cline{x-y}" where x and y are the column numbers of the cells to be underlined.
%         & B (1-p) &  &  \\\cline{3-4}
%       \end{tabular}
%     \end{table}
%     Matrix, with colors:
%     \vspace{-10pt}
%     \begin{table}
%       \begin{tabular}{cl|c|c|}
%         & \multicolumn{1}{c}{} & \multicolumn{2}{c}{\color{blue}Player 2}\\
%         \parbox[t]{1mm}{\multirow{3}{*}{\rotatebox[origin=r]{90}{\color{red}Player 1}}}
%         & \multicolumn{1}{c}{} & \multicolumn{1}{c}{L (q)} & \multicolumn{1}{c}{R (1-q)} \\\cline{3-4}
%         & T (p)   & \textcolor{red}{1}, \textcolor{blue}{1} &   \\\cline{3-4}
%         & B (1-p) &  &  \\\cline{3-4}
%       \end{tabular}
%     \end{table}
%   \vfill\null
%   \end{multicols}
% \end{frame}



\end{document}
