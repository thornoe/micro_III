\documentclass[8pt,apectratio=169]{beamer}

\usetheme[progressbar=frametitle]{metropolis}
\usepackage{appendixnumberbeamer}
\usepackage[style=authoryear, backend=bibtex8, natbib=true, maxcitenames=2]{biblatex}

\usepackage[utf8]{inputenc} % utf8x  defines more symbols, but may cause compatible problems
\usepackage{lmodern,textcomp} % Latin Modern fonts, contains €

\usepackage{graphicx}
\usepackage{import}

\usepackage{booktabs}
\usepackage[scale=2]{ccicons}

\usepackage{pgfplots}
\usepgfplotslibrary{dateplot}

\usepackage{xspace}
\newcommand{\themename}{\textbf{\textsc{metropolis}}\xspace}

% Math
\usepackage{amsmath}
\usepackage{bm} % bold symbol in math mode
\counterwithin*{equation}{section} % reset the equation number whenever section is stepped

% Optional packages
\usepackage{xcolor}
\usepackage{multicol}
\usepackage{multirow,array}
\usepackage{subcaption} % for subfigure and subtable
\usepackage{hyperref}
\usepackage{epigraph}
\usepackage[super,negative]{nth} % allows writing 1st, 2nd, 3rd with superscript
\usepackage{ulem} % use the "sout" tag to "strikethrough" text
\usepackage{cancel} % https://tex.stackexchange.com/questions/75525/how-to-write-crossed-out-math-in-latex
\usepackage{tcolorbox}

% Select what to do with command \comment:
  % \newcommand{\comment}[1]{}  %comments not shown
  % \newcommand{\comment}[1]{\par {\bfseries \color{blue} #1 \par}} %comments shown
% Select what to do with todonotes: i.e. \todo{}, \todo[inline]{}
  % \usepackage[disable]{todonotes} % notes not shown
  % \usepackage[draft]{todonotes}   % notes shown

%\numberwithin{equation}{section}

%\addbibresource{references}

\titlegraphic{\hfill \includegraphics[width=0.15 \textwidth]{figures/logo}}
\title{Microeconomics III: Problem Set 8\footnote{Slides created for exercise class 3 and 4, with reservation for possible errors.\\}}
\author{Thor Donsby Noe (\href{mailto:thor.noe@econ.ku.dk}{thor.noe@econ.ku.dk})
        \& Christopher Borberg (\href{mailto:christopher.borberg@econ.ku.dk}{christopher.borberg@econ.ku.dk})
        }
\date{November 13 2019} % \today
\institute{\normalsize Department of Economics, University of Copenhagen}

    % \definecolor{BlueTOL}{HTML}{222255}
    \definecolor{BrownTOL}{HTML}{666633}
    \definecolor{GreenTOL}{HTML}{225522}
    % \setbeamercolor{normal text}{fg=BlueTOL,bg=white}
    \setbeamercolor{alerted text}{fg=BrownTOL}
    \setbeamercolor{example text}{fg=GreenTOL}
    \setbeamercolor{background canvas}{bg=white}

    \setbeamercolor{block title alerted}{use=alerted text,
        fg=alerted text.fg,
        bg=alerted text.bg!80!alerted text.fg}
    \setbeamercolor{block body alerted}{use={block title alerted, alerted text},
        fg=alerted text.fg,
        bg=block title alerted.bg!50!alerted text.bg}
    \setbeamercolor{block title example}{use=example text,
        fg=example text.fg,
        bg=example text.bg!80!example text.fg}
    \setbeamercolor{block body example}{use={block title example, example text},
        fg=example text.fg,
        bg=block title example.bg!50!example text.bg}

\begin{document}
\maketitle

% Select what to do with command \intuition{}:
  \newcommand{\intuition}[1]{#1} % intuition shown
  %\newcommand{\intuition}[1]{[...]}  % intuition not shown


% ------------------------------------------------------------------------------
% ------ FRAME -----------------------------------------------------------------
% ------------------------------------------------------------------------------
\begin{frame}{Outline}
    \tableofcontents
\end{frame}



\section{PS5, Ex. 1 (A): Dynamic game (backwards induction)}

\begin{frame}{PS5, Ex. 1 (A): Dynamic game (backwards induction)}
  \begin{multicols}{2}
    \begin{itemize}
      \item Consider the dynamic game shown in extensive form. Solve it by backwards induction.
    \end{itemize}
    \vfill\null \columnbreak
    \begin{figure}[!h]
      \center
      \def\svgwidth{.8\columnwidth}
      \import{figures/}{1_.pdf_tex}
    \end{figure}
    \vfill\null
  \end{multicols}
\end{frame}

\begin{frame}{PS5, Ex. 1 (A): Dynamic game (backwards induction)}
  \begin{multicols}{2}
    \begin{itemize}
      \item Consider the dynamic game shown in extensive form. Solve it by backwards induction.
    \end{itemize}
    The backwards induction solution is the full strategy profile given by the subgame perfect NE:
    \begin{align*}
      BI=\{s_1^{*},s_2^{*}\}=\{(R, R''),\ L'\}
    \end{align*}
    \vfill\null \columnbreak
    \begin{figure}[!h]
      \center
      \def\svgwidth{.8\columnwidth}
      \import{figures/}{1.pdf_tex}
    \end{figure}
    \vfill\null
  \end{multicols}
\end{frame}



\section{PS5, Ex. 2 (A): Dynamic game (strategy sets)}

\begin{frame}{PS5, Ex. 2 (A): Extended Battle of the Sexes Game (strategy sets)}
  \begin{multicols}{2}
    Consider the game in the figure.
    \begin{itemize}
      \item[(a)] Write up the strategy sets of the players.
      \item[(b)] Write up the normal form (bi-matrix).
      \item[(c)] Find the (Pure Strategy) Nash Equilibria.
      \item[(d)] Find the backwards induction outcome.
    \end{itemize}
    \vfill\null \columnbreak
    \begin{figure}[!h]
      \center
      \def\svgwidth{\columnwidth}
      \import{figures/}{2_.pdf_tex}
    \end{figure}
    \vfill\null
  \end{multicols}
\end{frame}

\begin{frame}{PS5, Ex. 2.a (A): Extended Battle of the Sexes Game (strategy sets)}
  \begin{multicols}{2}
    \begin{itemize}
      \item[(a)] Write up the strategy sets of the players.
    \end{itemize}
    The two strategy sets are:
    \begin{align*}
      S_1=\{\ &(G, O', O'');\ (G, O', F'');\\
              &(G, F', O'');\ (G, F', F'');\\
              &(S, O', O'');\ (S, O', F'');\\
              &(S, F', O'');\ (S, F', F'')\ \}\\
      S_2=\{\ &O\ ;\ F\ \}
    \end{align*}
    \begin{itemize}
      \item[(b)] \textbf{\textit{Write up the normal form (bi-matrix).}}
    \end{itemize}
    \vfill\null \columnbreak
    \begin{figure}[!h]
      \center
      \def\svgwidth{\columnwidth}
      \import{figures/}{2_.pdf_tex}
    \end{figure}
    \vfill\null
  \end{multicols}
\end{frame}

\begin{frame}{PS5, Ex. 2.b (A): Extended Battle of the Sexes Game (strategy sets)}
  \begin{multicols}{2}
    \begin{itemize}
      \item[(b)] Write up the normal form (bi-matrix).
    \end{itemize}
    \begin{table}
      \begin{tabular}{cl|c|c|}
        & \multicolumn{1}{c}{} & \multicolumn{2}{c}{\color{blue}Player 2}\\
        & \multicolumn{1}{c}{} & \multicolumn{1}{c}{O} & \multicolumn{1}{c}{F} \\\cline{3-4}
        \parbox[t]{1mm}{\multirow{8}{*}{\rotatebox[origin=c]{90}{\color{red}Player 1}}}
        & (G, O', O'') & \textcolor{red}{3}, \textcolor{blue}{1} & 0, 0 \\\cline{3-4}
        & (G, O', F'') & \textcolor{red}{3}, 1 & 1, \textcolor{blue}{3} \\\cline{3-4}
        & (G, F', O'') & 0, \textcolor{blue}{0} & 0, \textcolor{blue}{0} \\\cline{3-4}
        & (G, F', F'') & 0, 0 & 1, \textcolor{blue}{3} \\\cline{3-4}
        & (S, O', O'') & 2, \textcolor{blue}{2} & \textcolor{red}{2}, \textcolor{blue}{2} \\\cline{3-4}
        & (S, O', F'') & 2, \textcolor{blue}{2} & \textcolor{red}{2}, \textcolor{blue}{2} \\\cline{3-4}
        & (S, F', O'') & 2, \textcolor{blue}{2} & \textcolor{red}{2}, \textcolor{blue}{2} \\\cline{3-4}
        & (S, F', F'') & 2, \textcolor{blue}{2} & \textcolor{red}{2}, \textcolor{blue}{2} \\\cline{3-4}
      \end{tabular}
    \end{table}
    \begin{itemize}
      \item[(c)] \textbf{\textit{Find the (Pure Strategy) Nash Equilibria.}}
    \end{itemize}
    \vfill\null \columnbreak
    \begin{figure}[!h]
      \center
      \def\svgwidth{\columnwidth}
      \import{figures/}{2_.pdf_tex}
    \end{figure}
    \vfill\null
  \end{multicols}
\end{frame}

\begin{frame}{PS5, Ex. 2.c (A): Extended Battle of the Sexes Game (strategy sets)}
  \begin{multicols}{2}
    \begin{itemize}
      \item[(c)] Find the (Pure Strategy) Nash Equilibria.
    \end{itemize}
    The five PSNE are:
    \begin{align*}
      PSNE=\{\  &(G, O', O''), O\ ;\ (S, O', O''), F\ ;\\
              &(S, O', F''), F\ ;\ (S, F', O''), F\ ;\\
              &(S, F', F''), F\ \}
    \end{align*}
    \vspace{-16pt}
    \begin{table}
      \begin{tabular}{cl|c|c|}
        & \multicolumn{1}{c}{} & \multicolumn{2}{c}{\color{blue}Player 2}\\
        & \multicolumn{1}{c}{} & \multicolumn{1}{c}{O} & \multicolumn{1}{c}{F} \\\cline{3-4}
        \parbox[t]{1mm}{\multirow{8}{*}{\rotatebox[origin=c]{90}{\color{red}Player 1}}}
        & (G, O', O'') & \textcolor{red}{3}, \textcolor{blue}{1} & 0, 0 \\\cline{3-4}
        & (G, O', F'') & \textcolor{red}{3}, 1 & 1, \textcolor{blue}{3} \\\cline{3-4}
        & (G, F', O'') & 0, \textcolor{blue}{0} & 0, \textcolor{blue}{0} \\\cline{3-4}
        & (G, F', F'') & 0, 0 & 1, \textcolor{blue}{3} \\\cline{3-4}
        & (S, O', O'') & 2, \textcolor{blue}{2} & \textcolor{red}{2}, \textcolor{blue}{2} \\\cline{3-4}
        & (S, O', F'') & 2, \textcolor{blue}{2} & \textcolor{red}{2}, \textcolor{blue}{2} \\\cline{3-4}
        & (S, F', O'') & 2, \textcolor{blue}{2} & \textcolor{red}{2}, \textcolor{blue}{2} \\\cline{3-4}
        & (S, F', F'') & 2, \textcolor{blue}{2} & \textcolor{red}{2}, \textcolor{blue}{2} \\\cline{3-4}
      \end{tabular}
    \end{table}
    \begin{itemize}
      \item[(d)] \textbf{\textit{Find the backwards induction outcome.}}
    \end{itemize}
    \vfill\null \columnbreak
    \begin{figure}[!h]
      \center
      \def\svgwidth{\columnwidth}
      \import{figures/}{2_.pdf_tex}
    \end{figure}
    \vfill\null
  \end{multicols}
\end{frame}

\begin{frame}{PS5, Ex. 2.d (A): Extended Battle of the Sexes Game (strategy sets)}
  \begin{multicols}{2}
    \begin{itemize}
      \item[(d)] Find the backwards induction outcome.
    \end{itemize}
    BI gives the unique Subgame Perfect NE:
    \begin{align*}
      SPNE=\{s_1^{*},s_2^{*}\}=\{(S, O', F''), F\}
    \end{align*}
    The PSNE $\{(G, O', O''), O\}$ is not subgame perfect as player 1's strategy is weakly dominated by $(G, O', F'')$. A SPNE needs to be rational on and off the equilibrium path, thus, $O''$ is an empty threat.
    \vspace{-8pt}
    \begin{table}
      \begin{tabular}{cl|c|c|}
        & \multicolumn{1}{c}{} & \multicolumn{2}{c}{\color{blue}Player 2}\\
        & \multicolumn{1}{c}{} & \multicolumn{1}{c}{O} & \multicolumn{1}{c}{F} \\\cline{3-4}
        \parbox[t]{1mm}{\multirow{8}{*}{\rotatebox[origin=c]{90}{\color{red}Player 1}}}
        & (G, O', O'') & \textcolor{red}{3}, \textcolor{blue}{1} & 0, 0 \\\cline{3-4}
        & (G, O', F'') & \textcolor{red}{3}, 1 & 1, \textcolor{blue}{3} \\\cline{3-4}
        & (G, F', O'') & 0, \textcolor{blue}{0} & 0, \textcolor{blue}{0} \\\cline{3-4}
        & (G, F', F'') & 0, 0 & 1, \textcolor{blue}{3} \\\cline{3-4}
        & (S, O', O'') & 2, \textcolor{blue}{2} & \textcolor{red}{2}, \textcolor{blue}{2} \\\cline{3-4}
        & (S, O', F'') & 2, \textcolor{blue}{2} & \textcolor{red}{2}, \textcolor{blue}{2} \\\cline{3-4}
        & (S, F', O'') & 2, \textcolor{blue}{2} & \textcolor{red}{2}, \textcolor{blue}{2} \\\cline{3-4}
        & (S, F', F'') & 2, \textcolor{blue}{2} & \textcolor{red}{2}, \textcolor{blue}{2} \\\cline{3-4}
      \end{tabular}
    \end{table}
    \vfill\null \columnbreak
    \begin{figure}[!h]
      \center
      \def\svgwidth{\columnwidth}
      \import{figures/}{2.pdf_tex}
    \end{figure}
    \vfill\null
  \end{multicols}
\end{frame}



\section{PS5, Ex. 3 (A): Stackelberg Duopoly (empty threats)}

\begin{frame}{PS5, Ex. 3 (A): Stackelberg Duopoly (empty threats)}
  \begin{multicols}{2}
    Consider the Stackelberg game we saw in the lecture (Stackelberg Duopoly presented in the end of Lecture 3). We solved for the backwards induction outcome. But if we were to look for Nash Equilibria, there are many. In fact there is an infinite amount. But they rely on ‘empty threats’. To see why, consider the following equilibrium. The \textit{follower} says to the \textit{leader}: I want you to produce $\widehat{q_1}$ (where $\widehat{q_1} < a$) and then I will produce $\widehat{q_2} = BR_2(\widehat{q_1})$. If you produce $q_1\neq\widehat{q_1}$ then I will set $q_2=a-q_1$ such that you make zero profit.
    \vfill\null \columnbreak
    \begin{itemize}
      \item[(a)] Write up the equilibria described above formally, using for the strategies the notation of the ‘Simple Dynamic Game’.
      \item[(b)] Explain why this kind of equilibrium does not survive backwards induction unless $\widehat{q_1}=BR_1\left(BR_2(q_1)\right)$, which is the Stackelberg outcome we derived in the lecture.
    \end{itemize}
    \vfill\null
  \end{multicols}
\end{frame}

\begin{frame}{PS5, Ex. 3.a (A): Stackelberg Duopoly (empty threats)}
  \begin{multicols}{2}
    \begin{itemize}
      \item[(a)] Write up the equilibria described above formally, using for the strategies the notation of the ‘Simple Dynamic Game’.
    \end{itemize}
    In the equilibrium, player 1 plays any quantity $\widehat{q_1}$ he is dictated, player 2 then play his BR.\\\medskip
    To finalize the equilibrium, we need players 2's response off the equilibrium path, i.e. if player one plays $q_1\neq\widehat{q_1}$ \\\medskip
    The NE consists of the strategies:
    \begin{align*}
    q_1^*&=\widehat{q_1},\quad\textit{where }q_1^*<a \\
    q_2^*&=\left\{
      \begin{array}{lcl}
          BR_2(\widehat{q_1}) & \text{if} & q_1=\widehat{q_1} \\
          a-q_1 & \text{if} & q_1\neq\widehat{q_1}
      \end{array}\right.
    \end{align*}
    \vfill\null \columnbreak
    \begin{itemize}
      \item[(b)] \textbf{\textit{Explain why this kind of equilibrium does not survive backwards induction unless $\bm{\widehat{q_1}=BR_1\left(BR_2(q_1)\right)}$, which is the Stackelberg outcome we derived in the lecture.}}
    \end{itemize}
    \vfill\null
  \end{multicols}
\end{frame}

\begin{frame}{PS5, Ex. 3.b (A): Stackelberg Duopoly (empty threats)}
  \begin{multicols}{2}
    \begin{itemize}
      \item[(a)] Write up the equilibria described above formally, using for the strategies the notation of the ‘Simple Dynamic Game’.
    \end{itemize}
    In the equilibrium, player 1 plays any quantity $\widehat{q_1}$ he is dictated, player 2 then play his BR.\\\medskip
    To finalize the equilibrium, we need players 2's response off the equilibrium path, i.e. if player one plays $q_1\neq\widehat{q_1}$ \\\medskip
    The NE consists of the strategies:
    \begin{align*}
    q_1^*&=\widehat{q_1},\quad\textit{where }q_1^*<a \\
    q_2^*&=\left\{
      \begin{array}{lcl}
          BR_2(\widehat{q_1}) & \text{if} & q_1=\widehat{q_1} \\
          a-q_1 & \text{if} & q_1\neq\widehat{q_1}
      \end{array}\right.
    \end{align*}
    \vfill\null \columnbreak
    \begin{itemize}
      \item[(b)] Explain why this kind of equilibrium does not survive backwards induction unless $\widehat{q_1}=BR_1\left(BR_2(q_1)\right)$, which is the Stackelberg outcome we derived in the lecture.
    \end{itemize}
     For the case where player 2 selects $\widehat{q_1}$ to be the quantity selected in the Stackelberg game, the outcome will be exactly as in the Stackelberg game. \\\medskip
     For the case where player 2 selects another outcome, player 1 will know that no matter his choice of $q_1$, player 2 will always maximize his profits, selecting his $q_2$ using his best response function. Thus, the threat is empty.
    \vfill\null
  \end{multicols}
\end{frame}



\section{PS5, Ex. 4: The Mutated Seabass (backwards induction)}

\begin{frame}{PS5, Ex. 4: The Mutated Seabass (backwards induction)}
    Consider a game where two evil organizations, rather prosaically named A and B, are battling for world domination. The battle takes the form of a three-stage game. Organization A is on the verge of acquiring a new powerful weapon, the \textit{mutated seabass}. In stage 1 of the game, they decide whether to acquire the weapon or not. Their choice is observed by organization B. In stage 2, organization B decides whether to attack organization A. If an attack occurs, the game stops. If no attack occurs, it moves to stage 3, where organization A decides whether or not to attack organization B. The payoffs are as follows. If no-one attacks the other, the payoffs to both organizations are 0. If B attacks A, then the payoffs to both organizations are .1. The same if A attacks B, without having acquired the seabass weapon. If, on the other hand, A acquires the weapon, the payoffs from A attacking B are 2 to A and -2 to B.
    \begin{itemize}
      \item[(a)] Draw the game tree that corresponds to the game. What are the strategies of the players?
      \item[(b)] What is the backwards induction outcome?
      \item[(c)] What is the intuition for the outcome? What role do you think it plays that B observes if A acquires the weapon or not?
    \end{itemize}
  \vfill\null
\end{frame}

\begin{frame}{PS5, Ex. 4.a: The Mutated Seabass (backwards induction)}
  \begin{itemize}
    \item[(a)] Draw the game tree that corresponds to the game. \textbf{\textit{What are the strategies of the players?}}
  \end{itemize}
  \vspace{-8pt}
  \begin{figure}[!h]
    \center
    \def\svgwidth{\columnwidth}
    \import{figures/}{4a.pdf_tex}
  \end{figure}
  \vfill\null
\end{frame}
\begin{frame}{PS5, Ex. 4.a: The Mutated Seabass (backwards induction)}
  \begin{itemize}
    \item[(a)] Draw the game tree that corresponds to the game. What are the strategies of the players?
  \end{itemize}
  \vspace{-8pt}
  \begin{figure}[!h]
    \center
    \def\svgwidth{\columnwidth}
    \import{figures/}{4a.pdf_tex}
  \end{figure}
  \vspace{-8pt}
  \begin{align*}
    S_A=\{ &(Acquire, a, a'); (Acquire, a, na'); (Acquire, na, a'); (Acquire, na, na');\\
            &(Not\ acquire, a, a'); (Not\ acquire, a, na'); (Not\ acquire, na, a'); (Not\ acquire, na, na') \}\\
    S_B=\{ &(A, A'); (A, NA'); (NA, A'); (NA, NA') \}
  \end{align*}
  \vspace{-12pt}
  \begin{itemize}
    \item[(b)] \textbf{\textit{What is the backwards induction outcome?}}
  \end{itemize}
  \vfill\null
\end{frame}

\begin{frame}{PS5, Ex. 4.b: The Mutated Seabass (backwards induction)}
  \begin{itemize}
    \item[(b)] What is the backwards induction outcome?
    \begin{itemize}\normalsize
      \item[\nth{3} stage:] Org. A will choose to attack if having acquired the weapon and not attack if not having acquired the weapon.
      \item[\nth{2} stage:] Org. B will choose to attack if Org. A has acquired the weapon and not attack if they have not acquired the weapon.
      \item[\nth{1} stage:] Org. A will commit to not acquiring the weapon in order to signal peaceful intentions to Org. B, i.e. leading to the payoffs $(0,0)$.
    \end{itemize}
  \end{itemize}
  \vspace{-8pt}
  \begin{figure}[!h]
    \center
    \def\svgwidth{\columnwidth}
    \import{figures/}{4b.pdf_tex}
  \end{figure}
  \vspace{-4pt}
  $BI=SPNE=\{S_A,S_B\}=\{ (Not\ acquire, a, na'),(A, NA') \}$
  \vfill\null
\end{frame}
\begin{frame}{PS5, Ex. 4.b: The Mutated Seabass (backwards induction)}
  \vspace{-4pt}
  \begin{itemize}
    \item[]
    \begin{itemize}\normalsize
      \item[\nth{3} stage:] Org. A will choose to attack if having acquired the weapon and not attack if not having acquired the weapon.
      \item[\nth{2} stage:] Org. B will choose to attack if Org. A has acquired the weapon and not attack if they have not acquired the weapon.
      \item[\nth{1} stage:] Org. A will commit to not acquiring the weapon in order to signal peaceful intentions to Org. B, i.e. leading to the payoffs $(0,0)$.
    \end{itemize}
  \end{itemize}
  \vspace{-8pt}
  \begin{figure}[!h]
    \center
    \def\svgwidth{\columnwidth}
    \import{figures/}{4b.pdf_tex}
  \end{figure}
  \vspace{-4pt}
  $BI=SPNE=\{S_A,S_B\}=\{ (Not\ acquire, a, na'),(A, NA') \}$
  \vspace{-2pt}
  \begin{itemize}
    \item[(c)] \textbf{\textit{What is the intuition for the outcome?}}
  \end{itemize}
  \vfill\null
\end{frame}

\begin{frame}{PS5, Ex. 4.c: The Mutated Seabass (backwards induction)}
    \begin{itemize}
      \item[(c)] What is the intuition for the outcome?
      \begin{itemize}\normalsize
        \item[\nth{3} stage:] \intuition{Org. A does only benefit from attacking if having acquired the weapon.}
        \item[\nth{2} stage:] \intuition{Org. B will only choose to attack if Org. A has acquired the weapon.}
        \item[\nth{1} stage:] \intuition{Due to the order of the game, not acquiring the weapon is a precommitment that sends the credible signal that Org. A will not attack in the last stage.}
      \end{itemize}
    \end{itemize}
    \vspace{-8pt}
    \begin{figure}[!h]
      \center
      \def\svgwidth{\columnwidth}
      \import{figures/}{4b.pdf_tex}
    \end{figure}
    \textbf{\textit{What role do you think it plays that B observes if A acquires the weapon or not?\\
    I.e. what is the outcome if Organization A cannot send a signal in the \nth{1} stage?}}
  \vfill\null
\end{frame}
\begin{frame}{PS5, Ex. 4.c: The Mutated Seabass (backwards induction)}
    \begin{itemize}
      \item[(c)] What role do you think it plays that B observes if A acquires the weapon or not? I.e. what is the outcome if Organization A cannot send a signal in the \nth{1} stage?
      \begin{itemize}\normalsize
        \item[\nth{3} stage:] [unchanged] Org. A will choose to attack if having acquired the weapon and not attack if not having acquired the weapon.
        \item[\nth{2} stage:] Knowing that Org. A will attack if having acquired the weapon, Org. B chooses to attack first, giving the payoffs $(-1,-1)$ regardless of stage one.
        \item[\nth{1} stage:] Org. A cannot affect the outcome, but acquires it in case Org. B deviates.
      \end{itemize}
    \end{itemize}
    \vspace{-8pt}
    \begin{figure}[!h]
      \center
      \def\svgwidth{\columnwidth}
      \import{figures/}{4c.pdf_tex}
    \end{figure}
    \vspace{-2pt}
    $SPNE=\{S_A,S_B\}=\{ (Acquire, a, na'),A \}$
  \vfill\null
\end{frame}
\begin{frame}{PS5, Ex. 4.c: The Mutated Seabass (backwards induction)}
    \begin{itemize}
      \item[(c)] Fool-proof alternative solution method:
      \begin{itemize}\normalsize
        \item[\nth{3} stage:] [unchanged] Org. A will choose to attack if having acquired the weapon and not attack if not having acquired the weapon.
        \item[Reduce:] Substitute in Org A's best actions and outcomes from the \nth{3} stage to get the reduced form game which is the static game below:
      \end{itemize}
    \end{itemize}
    \vspace{-6pt}
    \begin{figure}[!h]
      \center
      \def\svgwidth{.8\columnwidth}
      \import{figures/}{4c_reduced.pdf_tex}
    \end{figure}
    \vspace{-4pt}
    \begin{itemize}
      \item[]
      \begin{itemize}\normalsize
        \item[Bi-matrix:] Write up the normal form of the reduced game and highlight best responses:
      \end{itemize}
    \end{itemize}
    \vspace{-8pt}
    \begin{table}
      \begin{tabular}{cl|c|c|}
        & \multicolumn{1}{c}{} & \multicolumn{2}{c}{\color{blue}Org. B}\\
        & \multicolumn{1}{c}{} & \multicolumn{1}{c}{A} & \multicolumn{1}{c}{NA} \\\cline{3-4}
        \parbox[t]{1mm}{\multirow{2}{*}{\rotatebox[origin=c]{90}{\color{red}Org. A}}}
        & $(Acquire, a, na')$ & \textcolor{red}{-1}, \textcolor{blue}{-1} & \textcolor{red}{2}, -2 \\\cline{3-4}
        & $(Not\ acquire, a, na')$ & \textcolor{red}{-1}, -1 & 0, \textcolor{blue}{0} \\\cline{3-4}
      \end{tabular}
    \end{table}
    \vspace{-2pt}
    The strategy $(Not\ acquire, a, na')$ is weakly dominated by $(Acquire, a, na')$ and the unique Subgame Perfect NE is crystal clear: $\{S_A,S_B\}=\{(Acquire, a, na'),A\}$
    \vfill\null
\end{frame}



\section{PS5, Ex. 5: Three player game (backwards induction)}

\begin{frame}{PS5, Ex. 5: Three player game (backwards induction)}
    Consider the game below where player 1 chooses the matrix (A or B), player 2 chooses the row (C or D), and player 3 chooses the column (E or F). In each cell, the first number gives the payoff of Player 1, the second number the payoff of Player 2, and the third number the payoff of Player 3.
    \begin{table}
      \begin{tabular}{l|c|c|}
        \multicolumn{1}{c}{} & \multicolumn{1}{c}{E} & \multicolumn{1}{c}{F} \\\cline{2-3}
        C & 5, 2, 2 & 2, 1, 1 \\\cline{2-3}
        D & 0, 1, 1 & 1, 0, 0 \\\cline{2-3}
        \multicolumn{1}{c}{} & \multicolumn{2}{c}{A}
      \end{tabular}\quad
      \begin{tabular}{l|c|c|}
        \multicolumn{1}{c}{} & \multicolumn{1}{c}{E} & \multicolumn{1}{c}{F} \\\cline{2-3}
        C & 6, 0, 1 & 3, 1, 2 \\\cline{2-3}
        D & 1, 1, 0 & 2, 2, 1 \\\cline{2-3}
        \multicolumn{1}{c}{} & \multicolumn{2}{c}{B}
      \end{tabular}
    \end{table}
    \begin{itemize}
      \item[(a)] Suppose first that the game is static, such that all three players move simultaneously. Find all the pure-strategy Nash Equilibria.
      \item[(b)] Now suppose the game is dynamic: Player 1 moves first, and then, after having observed his move, Player 2 moves, and, finally, after having observed the first two moves, Player 3 moves. Draw the game tree and solve by backwards induction.
      \item[(c)] Discuss the differences in the results you find
    \end{itemize}
  \vfill\null
\end{frame}

\begin{frame}{PS5, Ex. 5.a: Three player game (backwards induction)}
    \begin{itemize}
      \item[(a)] Suppose first that the game is static, such that all three players move simultaneously. Find all the pure-strategy Nash Equilibria.
    \end{itemize}
    \begin{table}
      \begin{tabular}{l|c|c|}
        \multicolumn{1}{c}{} & \multicolumn{1}{c}{E} & \multicolumn{1}{c}{F} \\\cline{2-3}
        C & 5, \textcolor{red}{2}, \textcolor{blue}{2} & 2, \textcolor{red}{1}, 1 \\\cline{2-3}
        D & 0, 1, \textcolor{blue}{1} & 1, 0, 0 \\\cline{2-3}
        \multicolumn{1}{c}{} & \multicolumn{2}{c}{A}
      \end{tabular}\quad
      \begin{tabular}{l|c|c|}
        \multicolumn{1}{c}{} & \multicolumn{1}{c}{E} & \multicolumn{1}{c}{\color{blue}F} \\\cline{2-3}
        C & \textcolor{green}{6}, 0, 1 & \textcolor{green}{3}, 1, \textcolor{blue}{2} \\\cline{2-3}
        \color{red}D & \textcolor{green}{1}, \textcolor{red}{1}, 0 & \textcolor{green}{2}, \textcolor{red}{2}, \textcolor{blue}{1} \\\cline{2-3}
        \multicolumn{1}{c}{} & \multicolumn{2}{c}{\color{green}B}
      \end{tabular}
    \end{table}
    Iterated Elimination of Strictly Dominated Strategies (IESDS): For player 1, A is strictly dominated by B. In the reduced form game, C and E are strictly dominated for Player 2 and 3 respectively.\\\medskip
    $IESDS \Rightarrow$ Pure Strategy NE: $PSNE=\{S_1,S_2,S_3\}=\{B,D,F\}$ with outcome (2,2,1).
  \vfill\null
\end{frame}

\begin{frame}{PS5, Ex. 5.b: Three player game (backwards induction)}
    Consider the game below where player 1 chooses the matrix (A or B), player 2 chooses the row (C or D), and player 3 chooses the column (E or F). In each cell, the first number gives the payoff of Player 1, the second number the payoff of Player 2, and the third number the payoff of Player 3.
    \begin{table}
      \begin{tabular}{l|c|c|}
        \multicolumn{1}{c}{} & \multicolumn{1}{c}{E} & \multicolumn{1}{c}{F} \\\cline{2-3}
        C & 5, 2, 2 & 2, 1, 1 \\\cline{2-3}
        D & 0, 1, 1 & 1, 0, 0 \\\cline{2-3}
        \multicolumn{1}{c}{} & \multicolumn{2}{c}{A}
      \end{tabular}\quad
      \begin{tabular}{l|c|c|}
        \multicolumn{1}{c}{} & \multicolumn{1}{c}{E} & \multicolumn{1}{c}{F} \\\cline{2-3}
        C & 6, 0, 1 & 3, 1, 2 \\\cline{2-3}
        D & 1, 1, 0 & 2, 2, 1 \\\cline{2-3}
        \multicolumn{1}{c}{} & \multicolumn{2}{c}{B}
      \end{tabular}
    \end{table}
    \begin{itemize}
      \item[(b)] Now suppose the game is dynamic: Player 1 moves first, and then, after having observed his move, Player 2 moves, and, finally, after having observed the first two moves, Player 3 moves. \textbf{\textit{Draw the game tree}} and solve by backwards induction.
    \end{itemize}
  \vfill\null
\end{frame}
\begin{frame}{PS5, Ex. 5.b: Three player game (backwards induction)}
    \begin{itemize}
      \item[(b)] Now suppose the game is dynamic: Player 1 moves first, and then, after having observed his move, Player 2 moves, and, finally, after having observed the first two moves, Player 3 moves. Draw the game tree and \textbf{\textit{solve by backwards induction}}.
    \end{itemize}
    \vspace{-10pt}
    \begin{figure}[!h]
      \center
      \def\svgwidth{\columnwidth}
      \import{figures/}{5b_.pdf_tex}
    \end{figure}
  \vfill\null
\end{frame}
\begin{frame}{PS5, Ex. 5.b: Three player game (backwards induction)}
    \begin{itemize}
      \item[(b)] Now suppose the game is dynamic: Player 1 moves first, and then, after having observed his move, Player 2 moves, and, finally, after having observed the first two moves, Player 3 moves. Draw the game tree and solve by backwards induction.
    \end{itemize}
    \vspace{-10pt}
    \begin{figure}[!h]
      \center
      \def\svgwidth{\columnwidth}
      \import{figures/}{5b.pdf_tex}
    \end{figure}
    $BI \Rightarrow SPNE=\{S_1,S_2,S_3\}=\{A,(C,D'),(E,E',F'',F''')\}$ with outcome (5,2,2).
  \vfill\null
\end{frame}

\begin{frame}{PS5, Ex. 5.c: Three player game (backwards induction)}
    \begin{itemize}
      \item[(a)] $IESDS \Rightarrow PSNE=\{S_1,S_2,S_3\}=\{B,D,F\}$ with outcome (2,2,1).
      \item[(b)] $BI \Rightarrow SPNE=\{S_1,S_2,S_3\}=\{A,(C,D'),(E,E',F'',F''')\}$ w. outcome (5,2,2).
    \end{itemize}
    \vspace{-10pt}
    \begin{figure}[!h]
      \center
      \def\svgwidth{\columnwidth}
      \import{figures/}{5b.pdf_tex}
    \end{figure}
    \begin{itemize}
      \item[(c)] \textbf{\textit{Discuss the differences in the results you find.}}
    \end{itemize}
  \vfill\null
\end{frame}
\begin{frame}{PS5, Ex. 5.c: Three player game (backwards induction)}
    \begin{itemize}
      \item[(a)] $IESDS \Rightarrow PSNE=\{S_1,S_2,S_3\}=\{B,D,F\}$ with outcome (2,2,1).
      \item[(b)] $BI \Rightarrow SPNE=\{S_1,S_2,S_3\}=\{A,(C,D'),(E,E',F'',F''')\}$ w. outcome (5,2,2).
    \end{itemize}
    \vspace{-6pt}
    \begin{figure}[!h]
      \center
      \def\svgwidth{\columnwidth}
      \import{figures/}{5b.pdf_tex}
    \end{figure}
    \begin{itemize}
      \item[(c)] Discuss the differences in the results you find.
    \end{itemize}
    \textbf{In the static game:} \intuition{$(A,C,E)$ with outcome (5,2,2) cannot be a solution. Player 2 and 3 will not play $C$ and $E$ as they expect player 1 to play $B$ instead and get (6,0,1).}\\\medskip
    \textbf{In the dynamic game:} \intuition{Player 1 can expect at higher payoff on the left side of the tree than on the right side, thus, commits to $A$, allowing Player 2 and 3 to play $C$ and $E$.}
  \vfill\null
\end{frame}



\section{PS5, Ex. 6: Stackelberg assignment (backwards induction)}

\begin{frame}{PS5, Ex. 6: Stackelberg assignment (backwards induction)}
  \begin{multicols}{2}
    Two students are working together on the next assignment. Student $i, i = 1, 2,$ exerts an effort $y_i\geq0$. The resulting quality of the assignment is
    \begin{align*}
      q(y_1, y_2) = y_1y_2.
    \end{align*}
    Exerting effort is costly, but the costs differ, since one student likes game theory more than the other. More precisely, the cost functions are
    \begin{align*}
      C_1(y_1) &= \frac{1}{3}(y_1)^3,\\
      C_2(y_2) &= (y_2)^2.
    \end{align*}
    The payoff for student $i, U_i,$ is equal to the quality of the assignment less his cost of effort.
    \begin{align*}
      U_1(y_1,y_2)&=q(y_1,y_2)-C_1(y_1),\\
      U_2(y_1,y_2)&=q(y_1,y_2)-C_2(y_2).
    \end{align*}
    \vfill\null \columnbreak
    \textbf{(a)} Consider the game where both of them choose their effort levels simultaneously and independently. Derive the best response functions. Find the (pure strategy) Nash equilibrium $(y_1^{NE}, y_2^{NE})$ with $y_1^{NE}, y_2^{NE} > 0$.\\
    \textbf{(b)} Suppose now that Student 1 chooses his effort first, then sends the assignment on to Student 2. Student 2 observes how much exert Student 1 has exerted, makes his own choice of effort, and then submits. Solve by backwards induction.\\
    \textbf{(c)} Compare the outcomes in (a) and (b) with respect to the payoffs of the students. Which game does each of the two students prefer? Give an intuitive explanation of your answer.\\
    \textbf{(d)} Find the socially optimal levels of effort $(y_1^{SO}, y_2^{SO})$, i.e., the levels that maximize the sum of the two students’ payoffs. Calculate the payoff that the two students get in the social optimum.
    \vfill\null
  \end{multicols}
\end{frame}

\begin{frame}{PS5, Ex. 6.a: Stackelberg assignment (backwards induction)}
  \begin{itemize}
    \item[(a)] Consider the game where both of them choose their effort levels simultaneously and independently. Derive the best response functions. Find the (pure strategy) Nash equilibrium $(y_1^{NE}, y_2^{NE})$ with $y_1^{NE}, y_2^{NE} > 0$.
  \end{itemize}
    \vfill\null
  \begin{multicols}{2}
    \vfill\null \columnbreak
    Information so far:\\\medskip
    Quality: $q(y_1, y_2) = y_1y_2.$\\
    Costs: $C_1(y_1) = \frac{1}{3}(y_1)^3,\ \ C_2(y_2) = (y_2)^2.$\\
    Payoffs: $U_i(y_i,y_j) = q(y_i,y_j)-C_i(y_i).$
  \end{multicols}
\end{frame}

\begin{frame}{PS5, Ex. 6.a: Stackelberg assignment (backwards induction)}
    \begin{itemize}
    \item[(a)] Consider the game where both of them choose their effort levels simultaneously and independently. Derive the best response functions. Find the (pure strategy) Nash equilibrium $(y_1^{NE}, y_2^{NE})$ with $y_1^{NE}, y_2^{NE} > 0$.
    \end{itemize}
    \vfill\null
  \begin{multicols}{2}
    \begin{itemize}
      \item[(Step 1)] Write up the payoff functions
    \end{itemize}
    \vfill\null \columnbreak
    Information so far:
    \begin{itemize}
    \item[1] Quality: $q(y_1, y_2) = y_1y_2.$\\
    \item[2] Costs: $C_1(y_1) = \frac{1}{3}(y_1)^3,\ \ C_2(y_2) = (y_2)^2.$\\
    \item[3] $Payoff_i$: $U_i(y_i,y_j) = q(y_1,y_2)-C_i(y_i).$ \\
    \end{itemize}
    \vfill\null
  \end{multicols}
\end{frame}

\begin{frame}{PS5, Ex. 6.a: Stackelberg assignment (backwards induction)}
    \begin{itemize}
    \item[(a)] Consider the game where both of them choose their effort levels simultaneously and independently. Derive the best response functions. Find the (pure strategy) Nash equilibrium $(y_1^{NE}, y_2^{NE})$ with $y_1^{NE}, y_2^{NE} > 0$.
    \end{itemize}
    \vfill\null
  \begin{multicols}{2}
    \begin{itemize}
      \item[(Step 1)] Write up the payoff functions
      \item[(Step 2)] Write up the FOC and find the best response functions
    \end{itemize}
    \vfill\null \columnbreak
    Information so far:
    \begin{itemize}
    \item[1] Quality: $q(y_1, y_2) = y_1y_2.$\\
    \item[2] Costs: $C_1(y_1) = \frac{1}{3}(y_1)^3,\ \ C_2(y_2) = (y_2)^2.$\\
    \item[3] $Payoff_i$: $U_i(y_i,y_j) = q(y_1,y_2)-C_i(y_i).$ \\
    \item[4] $Payoff_1(y_1,y_2)$: $U_1(y_1,y_2) = y_1*y_2-\frac{1}{3}y_1^3$ \\
    \item[5] $Payoff_2(y_1,y_2)$: $U_1(y_1,y_2) = y_1*y_2-y_2^2$ \\
    \end{itemize}
    \vfill\null
  \end{multicols}
\end{frame}

\begin{frame}{PS5, Ex. 6.a: Stackelberg assignment (backwards induction)}
    \begin{itemize}
    \item[(a)] Consider the game where both of them choose their effort levels simultaneously and independently. Derive the best response functions. Find the (pure strategy) Nash equilibrium $(y_1^{NE}, y_2^{NE})$ with $y_1^{NE}, y_2^{NE} > 0$.
    \end{itemize}
    \vfill\null
  \begin{multicols}{2}
    \begin{itemize}
      \item[(Step 1)] Write up the payoff functions
      \item[(Step 2)] Write up the FOC and find the best response functions
      \item[(Step 3)] This is not a symmetric game, so you have to substitute $BR_i$ into $BR_j$ and then isolate $y_i$ and insert it into $BR_j$:
    \end{itemize}
    \vfill\null \columnbreak
    Information so far:
    \begin{itemize}
    \item[1] Quality: $q(y_1, y_2) = y_1y_2.$\\
    \item[2] Costs: $C_1(y_1) = \frac{1}{3}(y_1)^3,\ \ C_2(y_2) = (y_2)^2.$\\
    \item[3] $Payoff_i$: $U_i(y_i,y_j) = q(y_1,y_2)-C_i(y_i).$ \\
    \item[4] $Payoff_1(y_1,y_2)$: $U_1(y_1,y_2) = y_1*y_2-\frac{1}{3}y_1^3$ \\
    \item[5] $Payoff_2(y_1,y_2)$: $U_1(y_1,y_2) = y_1*y_2-y_2^2$ \\
    \item[6] $BR_1(y_2)$: $y_1 = y_2^{1/2}$ \\
    \item[7] $BR_2(y_1)$: $y_2 = \frac{y_1}{2}$ \\
    \end{itemize}
    \vfill\null
  \end{multicols}
\end{frame}

\begin{frame}{PS5, Ex. 6.a: Stackelberg assignment (backwards induction)}
    \begin{itemize}
    \item[(a)] Consider the game where both of them choose their effort levels simultaneously and independently. Derive the best response functions. Find the (pure strategy) Nash equilibrium $(y_1^{NE}, y_2^{NE})$ with $y_1^{NE}, y_2^{NE} > 0$.
    \end{itemize}
    \vfill\null
  \begin{multicols}{2}
    \begin{itemize}
      \item[(Step 1)] Write up the payoff functions
      \item[(Step 2)] Write up the FOC and find the best response functions
      \item[(Step 3)] This is not a symmetric game, so you have to substitute $BR_i$ into $BR_j$ and then isolate $y_i$ and insert it into $BR_j$:
      \begin{align*}
          y_1&=\frac{y_1}{2}^{\frac{1}{2}} \Rightarrow y_1-2y_1^2=0 \Rightarrow y_1=\frac{1}{2} \\
          y_2&=\frac{\frac{1}{2}}{2}=\frac{1}{4}
      \end{align*}
      \item[NE:] \begin{math} \left(\frac{1}{2},\frac{1}{4}\right)\end{math}
    \end{itemize}
    \vfill\null \columnbreak
    Information so far:
    \begin{itemize}
    \item[1] Quality: $q(y_1, y_2) = y_1y_2.$\\
    \item[2] Costs: $C_1(y_1) = \frac{1}{3}(y_1)^3,\ \ C_2(y_2) = (y_2)^2.$\\
    \item[3] $Payoff_i$: $U_i(y_i,y_j) = q(y_1,y_2)-C_i(y_i).$ \\
    \item[4] $Payoff_1(y_1,y_2)$: $U_1(y_1,y_2) = y_1*y_2-\frac{1}{3}y_1^3$ \\
    \item[5] $Payoff_2(y_1,y_2)$: $U_1(y_1,y_2) = y_1*y_2-y_2^2$ \\
    \item[6] $BR_1(y_2)$: $y_1 = y_2^{1/2}$ \\
    \item[7] $BR_2(y_1)$: $y_2 = \frac{y_1}{2}$ \\
    \end{itemize}
    \vfill\null
  \end{multicols}
\end{frame}

\begin{frame}{PS5, Ex. 6.b: Stackelberg assignment (backwards induction)}
    \begin{itemize}
    \item[(b)] Suppose now that Student 1 chooses his effort first, then sends the assignment on to Student 2. Student 2 observes how much exert Student 1 has exerted, makes his own choice of effort, and then submits. Solve by backwards induction.
    \end{itemize}
    \vfill\null
  \begin{multicols}{2}
    \vfill\null \columnbreak
    Information so far:
    \begin{itemize}
        \item[1] Quality: $q(y_1, y_2) = y_1y_2.$
        \item[2] Costs: $C_1(y_1) = \frac{1}{3}(y_1)^3,\ \ C_2(y_2) = (y_2)^2.$
        \item[3] $Payoff_1(y_1,y_2)$: $U_1(y_1,y_2) = y_1*y_2-\frac{1}{3}y_1^3$ \\
        \item[4] $BR_2(y_1)$: $y_2 = \frac{y_1}{2}$ \\
    \end{itemize}
    \vfill\null
  \end{multicols}
\end{frame}
\begin{frame}{PS5, Ex. 6.b: Stackelberg assignment (backwards induction)}
    \begin{itemize}
    \item[(b)] Suppose now that Student 1 chooses his effort first, then sends the assignment on to Student 2. Student 2 observes how much exert Student 1 has exerted, makes his own choice of effort, and then submits. Solve by backwards induction.
    \end{itemize}
    \vfill\null
  \begin{multicols}{2}
    \begin{itemize}
      \item[(Step 1)] Write up the new payoff function for player one, where he takes player 2s best response as given. In other words, write his payoff as a function of \begin{math}y_1\end{math} and \begin{math}BR_2(y_1)\end{math}
    \end{itemize}
    \vfill\null \columnbreak
    Information so far:
    \begin{itemize}
        \item[1] Quality: $q(y_1, y_2) = y_1y_2.$\\
        \item[2] Costs: $C_1(y_1) = \frac{1}{3}(y_1)^3,\ \ C_2(y_2) = (y_2)^2.$\\
        \item[3] $Payoff_1(y_1,y_2)$: $U_1(y_1,y_2) = y_1*y_2-\frac{1}{3}y_1^3$ \\
        \item[4] $BR_2(y_1)$: $y_2 = \frac{y_1}{2}$ \\
    \end{itemize}
    \vfill\null
  \end{multicols}
\end{frame}

\begin{frame}{PS5, Ex. 6.b: Stackelberg assignment (backwards induction)}
    \begin{itemize}
    \item[(b)] Suppose now that Student 1 chooses his effort first, then sends the assignment on to Student 2. Student 2 observes how much exert Student 1 has exerted, makes his own choice of effort, and then submits. Solve by backwards induction.
    \end{itemize}
    \vfill\null
  \begin{multicols}{2}
    \begin{itemize}
      \item[(Step 1)] Write up the new payoff function for player one, where he takes player 2s best response as given. In other words, write his payoff as a function of \begin{math}y_1\end{math} and \begin{math}BR_2(y_1)\end{math}
      \item[(Step 2)] Write up the FOC and find the best response function for player 1, as a function of \begin{math}y_1\end{math} and \begin{math}BR_2(y_1)\end{math}
    \end{itemize}
    \vfill\null \columnbreak
    Information so far:
    \begin{itemize}
        \item[1] Quality: $q(y_1, y_2) = y_1y_2.$\\
        \item[2] Costs: $C_1(y_1) = \frac{1}{3}(y_1)^3,\ \ C_2(y_2) = (y_2)^2.$\\
        \item[3] $Payoff_1(y_1,y_2)$: $U_1(y_1,y_2) = y_1*y_2-\frac{1}{3}y_1^3$ \\
        \item[4] $BR_2(y_1)$: $y_2 = \frac{y_1}{2}$ \\
        \item[5] $Payoff_1(y_1,BR_2(y_1))$: $U_1(y_1,\frac{y_1}{2}) = y_1*\frac{y_1}{2}-\frac{1}{3}y_1^3$ \\
    \end{itemize}
    \vfill\null
  \end{multicols}
\end{frame}

\begin{frame}{PS5, Ex. 6.b: Stackelberg assignment (backwards induction)}
    \begin{itemize}
    \item[(b)] Suppose now that Student 1 chooses his effort first, then sends the assignment on to Student 2. Student 2 observes how much exert Student 1 has exerted, makes his own choice of effort, and then submits. Solve by backwards induction.
    \end{itemize}
    \vfill\null
  \begin{multicols}{2}
    \begin{itemize}
      \item[(Step 1)] Write up the new payoff function for player one, where he takes player 2s best response as given. In other words, write his payoff as a function of \begin{math}y_1\end{math} and \begin{math}BR_2(y_1)\end{math}
      \item[(Step 2)] Write up the FOC and find the best response function for player 1, as a function of \begin{math}y_1\end{math} and \begin{math}BR_2(y_1)\end{math}:
      \begin{align*}
          FOC:            &\ y_1 - y_1^2 = 0\\
          BR_1(BR_2(y_1)):&\ y_1 = y_1^2\\
              \Rightarrow &\ y_1=0\vee y_1=1
      \end{align*}
      Of the two roots, $y_1=1$ is the best response as the payoff is positive.
      \item[(Step 3)] Use the value for $y_1$ to find $y_2$ and write up the SPNE:
    \end{itemize}
    \vfill\null \columnbreak
    Information so far:
    \begin{itemize}
        \item[1] Quality: $q(y_1, y_2) = y_1y_2.$\\
        \item[2] Costs: $C_1(y_1) = \frac{1}{3}(y_1)^3,\ \ C_2(y_2) = (y_2)^2.$\\
        \item[3] $Payoff_1(y_1,y_2)$: $U_1(y_1,y_2) = y_1*y_2-\frac{1}{3}y_1^3$ \\
        \item[4] $BR_2(y_1)$: $y_2 = \frac{y_1}{2}$ \\
        \item[5] $Payoff_1(y_1,BR_2(y_1))$: $U_1(y_1,\frac{y_1}{2}) = y_1*\frac{y_1}{2}-\frac{1}{3}y_1^3$ \\
    \end{itemize}
    \vfill\null
  \end{multicols}
\end{frame}

\begin{frame}{PS5, Ex. 6.b: Stackelberg assignment (backwards induction)}
    \begin{itemize}
    \item[(b)] Suppose now that Student 1 chooses his effort first, then sends the assignment on to Student 2. Student 2 observes how much exert Student 1 has exerted, makes his own choice of effort, and then submits. Solve by backwards induction.
    \end{itemize}
    \vfill\null
  \begin{multicols}{2}
    \begin{itemize}
      \item[(Step 1)] Write up the new payoff function for player one, where he takes player 2s best response as given. In other words, write his payoff as a function of \begin{math}y_1\end{math} and \begin{math}BR_2(y_1)\end{math}
      \item[(Step 2)] Write up the FOC and find the best response function for player 1, as a function of \begin{math}y_1\end{math} and \begin{math}BR_2(y_1)\end{math}
      \item[(Step 3)] Use the value for $y_1$ to find $y_2$ and write up the SPNE:
      \begin{align*}
          y_1&=1\\
          y_2&=\frac{y_1}{2}=\frac{1}{2}
      \end{align*}
      \item[SPNE:] \begin{math}\left(1,\frac{1}{2}\right)\end{math}
    \end{itemize}
    \vfill\null \columnbreak
    Information so far:
    \begin{itemize}
        \item[1] Quality: $q(y_1, y_2) = y_1y_2.$\\
        \item[2] Costs: $C_1(y_1) = \frac{1}{3}(y_1)^3,\ \ C_2(y_2) = (y_2)^2.$\\
        \item[3] $Payoff_1(y_1,y_2)$: $U_1(y_1,y_2) = y_1*y_2-\frac{1}{3}y_1^3$ \\
        \item[4] $BR_2(y_1)$: $y_2 = \frac{y_1}{2}$ \\
        \item[5] $Payoff_1(y_1,BR_2(y_1))$: $U_1(y_1,\frac{y_1}{2}) = y_1*\frac{y_1}{2}-\frac{1}{3}y_1^3$ \\
        \item[6] $FOC$: $y_1 - y_1^2 = 0$ \\
        \item[7] $BR_1(BR_2(y_1))$: $y_1 = y_1^2 \Rightarrow y_1=1$ \\
    \end{itemize}
    \vfill\null
  \end{multicols}
\end{frame}

\begin{frame}{PS5, Ex. 6.c: Stackelberg assignment (backwards induction)}
  \begin{itemize}
    \item[(c)] Compare the outcomes in (a) and (b) with respect to the payoffs of the students. Which game does each of the two students prefer? Give an intuitive explanation of your answer.
  \end{itemize}
  \begin{multicols}{2}
    \begin{itemize}
        \item[(Step 1)] Calculate the payoffs.
    \end{itemize}
    \vfill\null \columnbreak
    Relevant information:
    \begin{itemize}
        \item[(a)] \begin{math} NE=\left(\frac{1}{2},\frac{1}{4}\right)\end{math}
        \item[(b)] \begin{math} SPNE=\left(1,\frac{1}{2}\right)\end{math}
    \end{itemize}
  \end{multicols}
\end{frame}
\begin{frame}{PS5, Ex. 6.c: Stackelberg assignment (backwards induction)}
  \begin{itemize}
    \item[(c)] Compare the outcomes in (a) and (b) with respect to the payoffs of the students. Which game does each of the two students prefer? Give an intuitive explanation of your answer.
  \end{itemize}
  \begin{multicols}{2}
    \begin{itemize}
        \item[(Step 1)] Calculate the payoffs.
        \item[(Step 2)] Which game do they prefer and why?
    \end{itemize}
    \vfill\null \columnbreak
    Relevant information:
    \begin{itemize}
        \item[(a)] \begin{math} NE=\left(\frac{1}{2},\frac{1}{4}\right)\end{math}
        \item[(a)] \begin{math} Payoffs=\left(\frac{1}{12},\frac{1}{16}\right)\end{math}
        \item[(b)] \begin{math} SPNE=\left(1,\frac{1}{2}\right)\end{math}
        \item[(b)] \begin{math} Payoffs=\left(\frac{1}{6},\frac{1}{4}\right)\end{math}
    \end{itemize}
  \end{multicols}
\end{frame}
\begin{frame}{PS5, Ex. 6.c: Stackelberg assignment (backwards induction)}
  \begin{itemize}
    \item[(c)] Compare the outcomes in (a) and (b) with respect to the payoffs of the students. Which game does each of the two students prefer? Give an intuitive explanation of your answer.
  \end{itemize}
  \begin{multicols}{2}
    \begin{itemize}
        \item[(Step 1)] Calculate the payoffs.
        \item[(Step 2)] Which game do they prefer and why?
        \item[(Player 1)] \intuition{This is a case of last mover advantage; the payoff function means that Player 1 has an incentive to set his effort high, in order to motivate player 2 to do the same.}
        \item[(Player 2)] \intuition{Gets most of the extra benefit, since he can optimize his own effort, without it affecting player 1's effort.}
        \item[(Pref)] Both players prefer the dynamic game! But is it optimal?
    \end{itemize}
    \vfill\null \columnbreak
    Relevant information:
    \begin{itemize}
        \item[(a)] \begin{math} NE=\left(\frac{1}{2},\frac{1}{4}\right)\end{math}
        \item[(a)] \begin{math} Payoffs=\left(\frac{1}{12},\frac{1}{16}\right)\end{math}
        \item[(b)] \begin{math} SPNE=\left(1,\frac{1}{2}\right)\end{math}
        \item[(b)] \begin{math} Payoffs=\left(\frac{1}{6},\frac{1}{4}\right)\end{math}
    \end{itemize}
  \end{multicols}
\end{frame}

\begin{frame}{PS5, Ex. 6.d: Stackelberg assignment (backwards induction)}
  \begin{itemize}
    \item[(d)] Find the socially optimal levels of effort $(y_1^{SO}, y_2^{SO})$, i.e., the levels that maximize the sum of the two students’ payoffs. Calculate the payoff that the two students get in the social optimum.
  \end{itemize}
  \begin{multicols}{2}
    \vfill\null \columnbreak
    Information so far:\\\medskip
    \begin{itemize}
        \item[1] Quality: $q(y_1, y_2) = y_1y_2.$\\
        \item[2] Costs: $C_1(y_1) = \frac{1}{3}(y_1)^3,\ \ C_2(y_2) = (y_2)^2.$\\
        \item[3] $Payoff_1(y_1,y_2)$: $U_1(y_1,y_2) = y_1*y_2-\frac{1}{3}y_1^3$ \\
        \item[4] $Payoff_2(y_1,y_2)$: $U_1(y_1,y_2) = y_1*y_2-y_2^2$ \\
    \end{itemize}
  \end{multicols}
\end{frame}

\begin{frame}{PS5, Ex. 6.d: Stackelberg assignment (backwards induction)}
  \begin{itemize}
    \item[(d)] Find the socially optimal levels of effort $(y_1^{SO}, y_2^{SO})$, i.e., the levels that maximize the sum of the two students’ payoffs. Calculate the payoff that the two students get in the social optimum.
  \end{itemize}
  \begin{multicols}{2}
    \begin{itemize}
      \item[(Step 1)] Write up the combined payoff function for the players
    \end{itemize}
    \vfill\null \columnbreak
    Information so far:\\\medskip
    \begin{itemize}
        \item[1] Quality: $q(y_1, y_2) = y_1y_2.$\\
        \item[2] Costs: $C_1(y_1) = \frac{1}{3}(y_1)^3,\ \ C_2(y_2) = (y_2)^2.$\\
        \item[3] $Payoff_1(y_1,y_2)$: $U_1(y_1,y_2) = y_1*y_2-\frac{1}{3}y_1^3$ \\
        \item[4] $Payoff_2(y_1,y_2)$: $U_1(y_1,y_2) = y_1*y_2-y_2^2$ \\
    \end{itemize}
  \end{multicols}
\end{frame}

\begin{frame}{PS5, Ex. 6.d: Stackelberg assignment (backwards induction)}
  \begin{itemize}
    \item[(d)] Find the socially optimal levels of effort $(y_1^{SO}, y_2^{SO})$, i.e., the levels that maximize the sum of the two students’ payoffs. Calculate the payoff that the two students get in the social optimum.
  \end{itemize}
  \begin{multicols}{2}
    \begin{itemize}
      \item[(Step 1)] Write up the combined payoff function for the players
      \item[(Step 2)] Write up the FOCs
    \end{itemize}
    \vfill\null \columnbreak
    Information so far:\\\medskip
    \begin{itemize}
        \item[1] Quality: $q(y_1, y_2) = y_1y_2.$\\
        \item[2] Costs: $C_1(y_1) = \frac{1}{3}(y_1)^3,\ \ C_2(y_2) = (y_2)^2.$\\
        \item[3] $Payoff_1(y_1,y_2)$: $U_1(y_1,y_2) = y_1*y_2-\frac{1}{3}y_1^3$ \\
        \item[4] $Payoff_2(y_1,y_2)$: $U_1(y_1,y_2) = y_1*y_2-y_2^2$ \\
        \item[5] $Payoff_T(y_1,y_2)$: $U_1(y_1,y_2)+U_2(y_1,y_2) = 2*y_1*y_2-\frac{1}{3}y_1^3 - y_2^2$ \\
    \end{itemize}
  \end{multicols}
\end{frame}

\begin{frame}{PS5, Ex. 6.d: Stackelberg assignment (backwards induction)}
  \begin{itemize}
    \item[(d)] Find the socially optimal levels of effort $(y_1^{SO}, y_2^{SO})$, i.e., the levels that maximize the sum of the two students’ payoffs. Calculate the payoff that the two students get in the social optimum.
  \end{itemize}
  \begin{multicols}{2}
    \begin{itemize}
      \item[(Step 1)] Write up the combined payoff function for the players
      \item[(Step 2)] Write up the FOCs
      \item[(Step 3)] Subsitute $FOC_i$ into $FOC_j$ and isolate $y_i$, then insert $y_i$ into $FOC_j$ and calculate the payoffs:
    \end{itemize}
    \vfill\null \columnbreak
    Information so far:\\\medskip
    \begin{itemize}
        \item[1] Quality: $q(y_1, y_2) = y_1y_2.$\\
        \item[2] Costs: $C_1(y_1) = \frac{1}{3}(y_1)^3,\ \ C_2(y_2) = (y_2)^2.$\\
        \item[3] $Payoff_1(y_1,y_2)$: $U_1(y_1,y_2) = y_1*y_2-\frac{1}{3}y_1^3$ \\
        \item[4] $Payoff_2(y_1,y_2)$: $U_1(y_1,y_2) = y_1*y_2-y_2^2$ \\
        \item[5] $Payoff_T(y_1,y_2)$: $U_1(y_1,y_2)+U_2(y_1,y_2) = 2*y_1*y_2-\frac{1}{3}y_1^3 - y_2^2$ \\
        \item[5] $FOC_{y_1}: 0=2y_2-y_1^2 \Rightarrow y_1=(2y_2)^{\frac{1}{2}}$ \\
        \item[5] $FOC_{y_2}: 0=2y_1-2y_2 \Rightarrow y_2=y_1$ \\
    \end{itemize}
  \end{multicols}
\end{frame}

\begin{frame}{PS5, Ex. 6.d: Stackelberg assignment (backwards induction)}
  \begin{itemize}
    \item[(d)] Find the socially optimal levels of effort $(y_1^{SO}, y_2^{SO})$, i.e., the levels that maximize the sum of the two students’ payoffs. Calculate the payoff that the two students get in the social optimum.
  \end{itemize}
  \begin{multicols}{2}
    \begin{itemize}
      \item[(Step 1)] Write up the combined payoff function for the players
      \item[(Step 2)] Write up the FOCs
      \item[(Step 3)] Subsitute $FOC_i$ into $FOC_j$ and isolate $y_i$, then insert $y_i$ into $FOC_j$ and calculate the payoffs:
      \begin{align*}
          y_1=(2y_1)^{\frac{1}{2}} \Rightarrow y_1=2 \\
          y_2=y_1=2
      \end{align*}
      \item[SO:] The social optimum is \begin{math} (2,2)\end{math}
        \begin{align*}
          Payoff_1=2*2-\frac{1}{3}2^2= \frac{4}{3} \\
          Payoff_2=2*2-2^2=0
        \end{align*}
    \item[(Interpret)]
    \end{itemize}
    \vfill\null \columnbreak
    Information so far:\\\medskip
    \begin{itemize}
        \item[1] Quality: $q(y_1, y_2) = y_1y_2.$\\
        \item[2] Costs: $C_1(y_1) = \frac{1}{3}(y_1)^3,\ \ C_2(y_2) = (y_2)^2.$\\
        \item[3] $Payoff_1(y_1,y_2)$: $U_1(y_1,y_2) = y_1*y_2-\frac{1}{3}y_1^3$ \\
        \item[4] $Payoff_2(y_1,y_2)$: $U_1(y_1,y_2) = y_1*y_2-y_2^2$ \\
        \item[5] $Payoff_T(y_1,y_2)$: $U_1(y_1,y_2)+U_2(y_1,y_2) = 2*y_1*y_2-\frac{1}{3}y_1^3 - y_2^2$ \\
        \item[5] $FOC_{y_1}: 0=2y_2-y_1^2 \Rightarrow y_1=(2y_2)^{\frac{1}{2}}$ \\
        \item[5] $FOC_{y_2}: 0=2y_1-2y_2 \Rightarrow y_2=y_1$ \\
    \end{itemize}
  \end{multicols}
\end{frame}

\begin{frame}{PS5, Ex. 6.d: Stackelberg assignment (backwards induction)}
  \begin{itemize}
    \item[(d)] Find the socially optimal levels of effort $(y_1^{SO}, y_2^{SO})$, i.e., the levels that maximize the sum of the two students’ payoffs. Calculate the payoff that the two students get in the social optimum.
  \end{itemize}
  \begin{multicols}{2}
    \begin{itemize}
      \item[(Step 1)] Write up the combined payoff function for the players
      \item[(Step 2)] Write up the FOCs
      \item[(Step 3)] Subsitute $FOC_i$ into $FOC_j$ and isolate $y_i$, then insert $y_i$ into $FOC_j$ and calculate the payoffs:
      \begin{align*}
          y_1=(2y_1)^{\frac{1}{2}} \Rightarrow y_1=2 \\
          y_2=y_1=2
      \end{align*}
      \item[SO:] The social optimum is \begin{math} (2,2)\end{math}
        \begin{align*}
          Payoff_1=2*2-\frac{1}{3}2^2= \frac{4}{3} \\
          Payoff_2=2*2-2^2=0
        \end{align*}
    \item[(Interpret)] \intuition{Since player 1s effort is cheaper for small values of effort, he could pay player 2 to increase P2s effort, whilst increasing both of their payoffs.}
    \end{itemize}
    \vfill\null \columnbreak
    Information so far:\\\medskip
    \begin{itemize}
        \item[1] Quality: $q(y_1, y_2) = y_1y_2.$\\
        \item[2] Costs: $C_1(y_1) = \frac{1}{3}(y_1)^3,\ \ C_2(y_2) = (y_2)^2.$\\
        \item[3] $Payoff_1(y_1,y_2)$: $U_1(y_1,y_2) = y_1*y_2-\frac{1}{3}y_1^3$ \\
        \item[4] $Payoff_2(y_1,y_2)$: $U_1(y_1,y_2) = y_1*y_2-y_2^2$ \\
        \item[5] $Payoff_T(y_1,y_2)$: $U_1(y_1,y_2)+U_2(y_1,y_2) = 2*y_1*y_2-\frac{1}{3}y_1^3 - y_2^2$ \\
        \item[5] $FOC_{y_1}: 0=2y_2-y_1^2 \Rightarrow y_1=(2y_2)^{\frac{1}{2}}$ \\
        \item[5] $FOC_{y_2}: 0=2y_1-2y_2 \Rightarrow y_2=y_1$ \\
    \end{itemize}
  \end{multicols}
\end{frame}



\section{PS5, Ex. 7: Dynamic game (proper subgames)}

\begin{frame}{PS5, Ex. 7: Dynamic game (proper subgames)}
  \begin{multicols}{2}
    Consider the game from exercise 1.
    \begin{itemize}
      \item[(a)] Write down the strategies of the two players. How many proper subgames are there (so not including the entire game itself)?
      \item[(b)] Write down the Subgame-Perfect Nash Equilibrium (SPNE). Compare the SPNE to the backwards-induction outcome which you found in question 1.
      \item[(c)] Write down the normal form of the game and find all (pure strategy) Nash equilibria. Compare to the set of SPNE and comment.
    \end{itemize}
    \vfill\null \columnbreak
    \begin{figure}[!h]
      \center
      \def\svgwidth{.8\columnwidth}
      \import{figures/}{1_.pdf_tex}
    \end{figure}
    \vfill\null
  \end{multicols}
\end{frame}

\begin{frame}{PS5, Ex. 7.a: Dynamic game (proper subgames)}
  \begin{multicols}{2}
    \begin{itemize}
      \item[(a)] Write down the strategies of the two players. \textbf{\textit{How many proper subgames are there (so not including the entire game itself)?}}
    \end{itemize}
    The two strategy sets are:
    \begin{align*}
      S_1=\{\ &(L, L''); (L, R''); (R, L''); (R, R'')\ \}\\
      S_2=\{\ &L'\ ;\ R'\ \}
    \end{align*}
    \vfill\null \columnbreak
    \begin{figure}[!h]
      \center
      \def\svgwidth{.8\columnwidth}
      \import{figures/}{1_.pdf_tex}
    \end{figure}
    \vfill\null
  \end{multicols}
\end{frame}
\begin{frame}{PS5, Ex. 7.a: Dynamic game (proper subgames)}
  \begin{multicols}{2}
    \begin{itemize}
      \item[(a)] Write down the strategies of the two players. How many proper subgames are there (so not including the entire game itself)?
    \end{itemize}
    The two strategy sets are:
    \begin{align*}
      S_1=\{\ &(L, L''); (L, R''); (R, L''); (R, R'')\ \}\\
      S_2=\{\ &L'\ ;\ R'\ \}
    \end{align*}
    There are two proper subgames.
    \begin{itemize}
      \item[(b)] \textbf{\textit{Write down the Subgame-Perfect Nash Equilibrium (SPNE). Compare the SPNE to the backwards-induction outcome which you found in question 1.}}
    \end{itemize}
    \vfill\null \columnbreak
    \begin{figure}[!h]
      \center
      \def\svgwidth{\columnwidth}
      \import{figures/}{7a.pdf_tex}
    \end{figure}
    \vfill\null
  \end{multicols}
\end{frame}

\begin{frame}{PS5, Ex. 7.b: Dynamic game (proper subgames)}
  \begin{multicols}{2}
    \begin{itemize}
      \item[(b)] Write down the Subgame-Perfect Nash Equilibrium (SPNE). Compare the SPNE to the backwards-induction outcome which you found in question 1.
      \vspace{6pt}
      \begin{itemize}\normalsize
        \item[Turquoise game:] $SPNE_T=\{s_1^{*}\}=\{R''\}$
        \item[Purple game:] $SPNE_P=\{s_1^{*},s_2^{*}\}=\{R'',L'\}$
        \item[Entire game:] $SPNE_E=\{(R\ R''),L'\}$
      \end{itemize}
    \end{itemize}
    Which coincides with the backwards induction solution from Exercise 1!
    \begin{itemize}
      \item[(c)] \textbf{\textit{Write down the normal form of the game}} and find all (pure strategy) Nash equilibria. Compare to the set of SPNE and comment.
    \end{itemize}
    \vfill\null \columnbreak
    \begin{figure}[!h]
      \center
      \def\svgwidth{\columnwidth}
      \import{figures/}{7b.pdf_tex}
    \end{figure}
    \vfill\null
  \end{multicols}
\end{frame}

\begin{frame}{PS5, Ex. 7.c: Dynamic game (proper subgames)}
  \begin{multicols}{2}
    \begin{itemize}
      \item[(c)] Write down the normal form of the game and \textbf{\textit{find all (pure strategy) Nash equilibria}}. Compare to the set of SPNE and comment.
    \end{itemize}
    \vspace{-10pt}
    \begin{table}
      \begin{tabular}{cl|c|c|}
        & \multicolumn{1}{c}{} & \multicolumn{2}{c}{Player 2}\\
        & \multicolumn{1}{c}{} & \multicolumn{1}{c}{L'} & \multicolumn{1}{c}{R'} \\\cline{3-4}
        \parbox[t]{1mm}{\multirow{4}{*}{\rotatebox[origin=c]{90}{Player 1}}}
        & (L, L'') & 1, 4 & 1, 4 \\\cline{3-4}
        & (L, R'') & 1, 4 & 1, 4 \\\cline{3-4}
        & (R, L'') & 1, 0 & 0, 2 \\\cline{3-4}
        & (R, R'') & 2, 3 & 0, 2 \\\cline{3-4}
      \end{tabular}
    \end{table}
    \vfill\null \columnbreak
    \begin{figure}[!h]
      \center
      \def\svgwidth{\columnwidth}
      \import{figures/}{7b.pdf_tex}
    \end{figure}
    \vfill\null
  \end{multicols}
\end{frame}
\begin{frame}{PS5, Ex. 7.c: Dynamic game (proper subgames)}
  \begin{multicols}{2}
    \begin{itemize}
      \item[(c)] Write down the normal form of the game and find all (pure strategy) Nash equilibria. \textbf{\textit{Compare to the set of SPNE and comment.}}
    \end{itemize}
    \vspace{-10pt}
    \begin{table}
      \begin{tabular}{cl|c|c|}
        & \multicolumn{1}{c}{} & \multicolumn{2}{c}{\color{blue}Player 2}\\
        & \multicolumn{1}{c}{} & \multicolumn{1}{c}{L'} & \multicolumn{1}{c}{R'} \\\cline{3-4}
        \parbox[t]{1mm}{\multirow{4}{*}{\rotatebox[origin=c]{90}{\color{red}Player 1}}}
        & (L, L'') & 1, \textcolor{blue}{4} & \textcolor{red}{1}, \textcolor{blue}{4} \\\cline{3-4}
        & (L, R'') & 1, \textcolor{blue}{4} & \textcolor{red}{1}, \textcolor{blue}{4} \\\cline{3-4}
        & (R, L'') & 1, 0 & 0, \textcolor{blue}{2} \\\cline{3-4}
        & (R, R'') & \textcolor{red}{2}, \textcolor{blue}{3} & 0, 2 \\\cline{3-4}
      \end{tabular}
    \end{table}
    \vspace{-10pt}
    \begin{align*}
      PSNE=\{\ &(L, L''),R';\\
               &(L, R''),R';\\
               &(R, R''),L'\ \}
    \end{align*}
    \vfill\null \columnbreak
    \begin{figure}[!h]
      \center
      \def\svgwidth{\columnwidth}
      \import{figures/}{7b.pdf_tex}
    \end{figure}
    \vfill\null
  \end{multicols}
\end{frame}
\begin{frame}{PS5, Ex. 7.c: Dynamic game (proper subgames)}
  \begin{multicols}{2}
    \begin{itemize}
      \item[(c)] Write down the normal form of the game and find all (pure strategy) Nash equilibria. Compare to the set of SPNE and comment.
    \end{itemize}
    \vspace{-10pt}
    \begin{table}
      \begin{tabular}{cl|c|c|}
        & \multicolumn{1}{c}{} & \multicolumn{2}{c}{\color{blue}Player 2}\\
        & \multicolumn{1}{c}{} & \multicolumn{1}{c}{L'} & \multicolumn{1}{c}{R'} \\\cline{3-4}
        \parbox[t]{1mm}{\multirow{4}{*}{\rotatebox[origin=c]{90}{\color{red}Player 1}}}
        & (L, L'') & 1, \textcolor{blue}{4} & \textcolor{red}{1}, \textcolor{blue}{4} \\\cline{3-4}
        & (L, R'') & 1, \textcolor{blue}{4} & \textcolor{red}{1}, \textcolor{blue}{4} \\\cline{3-4}
        & (R, L'') & 1, 0 & 0, \textcolor{blue}{2} \\\cline{3-4}
        & (R, R'') & \textcolor{red}{2}, \textcolor{blue}{3} & 0, 2 \\\cline{3-4}
      \end{tabular}
    \end{table}
    \vspace{-10pt}
    \begin{align*}
      PSNE=\{\ &(L, L''),R';\\
               &(L, R''),R';\\
               &(R, R''),L'\ \}
    \end{align*}
    How is it, that $(L, L''),R'$ and $(L, R''),R'$ are not Subgame Perfect Nash Equilibria?\\\medskip
    \intuition{They rely on empty threats (Player 2 playing $R'$ and Player 1 playing $L''$).}
    \vfill\null \columnbreak
    \begin{figure}[!h]
      \center
      \def\svgwidth{\columnwidth}
      \import{figures/}{7b.pdf_tex}
    \end{figure}
    \vfill\null
  \end{multicols}
\end{frame}



\section{PS5, Ex. 8: Dynamic game (bi-matrix)}

\begin{frame}{PS5, Ex. 8: Dynamic game (bi-matrix)}
  For the game given in extensive form below, answer the following four questions:
  \begin{itemize}
    \item[(a)] What are the strategy sets of each player?
    \item[(b)] How many proper subgames are there (so not including the entire game itself)?
    \item[(c)] Find the backwards induction outcome and write down the SPNE.
    \item[(d)] Write down this game as a bi-matrix and find all pure strategy Nash equilibria.]
  \end{itemize}
    '' is used to notate player 2’s choices given player 1 plays R. $S_2(R)=\{L'';R''\}$
  \begin{figure}[!h]
    \center
    \def\svgwidth{.8\columnwidth}
    \import{figures/}{8.pdf_tex}
  \end{figure}
  \vfill\null
\end{frame}

\begin{frame}{PS5, Ex. 8.a: Dynamic game (bi-matrix)}
  \begin{itemize}
    \item[(a)] What are the strategy sets of each player?
  \end{itemize}
  \begin{figure}[!h]
    \center
    \def\svgwidth{.8\columnwidth}
    \import{figures/}{8.pdf_tex}
  \end{figure}
  $S_1=\{L;R\}$\\
  $S_2=\{(L',L'');(L', R'');(C',L'');(C', R'');(R',L'');(R', R'')\}$
  \begin{itemize}
    \item[(b)] \textbf{\textit{How many proper subgames are there (so not including the entire game itself)?}}
  \end{itemize}
  \vfill\null
\end{frame}

\begin{frame}{PS5, Ex. 8.b: Dynamic game (bi-matrix)}
  \begin{itemize}
    \item[(b)] How many proper subgames are there (so not including the entire game itself)?
  \end{itemize}
  There are two:
  \begin{figure}[!h]
    \center
    \def\svgwidth{.9\columnwidth}
    \import{figures/}{8b.pdf_tex}
  \end{figure}
  \begin{itemize}
    \item[(C)] \textbf{\textit{Find the backwards induction outcome and write down the SPNE.}}
  \end{itemize}
  \vfill\null
\end{frame}

\begin{frame}{PS5, Ex. 8.c: Dynamic game (bi-matrix)}
  \begin{itemize}
    \item[(c)] Find the backwards induction outcome and write down the SPNE.
  \end{itemize}
  Looking at the last tier of subgames, P2 would play respectively L' and R'. Knowing this strategy of P2, P1's payoff for L is 1 and R is 2, so he will choose R:
  \begin{align*}
      \Rightarrow SPNE=\{R,(L',R'')\}
  \end{align*}
  \begin{figure}[!h]
    \center
    \def\svgwidth{.9\columnwidth}
    \import{figures/}{8c.pdf_tex}
  \end{figure}
  \begin{itemize}
    \item[(d)] \textbf{\textit{Write down this game as a bi-matrix and find all pure strategy Nash equilibria.}}
  \end{itemize}
  \vfill\null
\end{frame}

\begin{frame}{PS5, Ex. 8.d: Dynamic game (bi-matrix)}
  \begin{itemize}
    \item[(d)] Write down this game as a bi-matrix and find all pure strategy Nash equilibria.
  \end{itemize}
  \begin{figure}[!h]
    \center
    \def\svgwidth{.8\columnwidth}
    \import{figures/}{8d.pdf_tex}
  \end{figure}
  \begin{table}
    \begin{tabular}{cl|c|c|c|c|c|c|}
      & \multicolumn{1}{c}{} & \multicolumn{6}{c}{\color{blue}Player 2}\\
      \parbox[t]{1mm}{\multirow{3}{*}{\rotatebox[origin=r]{90}{\color{red}Player 1}}}
      & \multicolumn{1}{c}{} & \multicolumn{1}{c}{L',L''} & \multicolumn{1}{c}{L',R''} & \multicolumn{1}{c}{C',L''} & \multicolumn{1}{c}{C',R''} & \multicolumn{1}{c}{R',L''} & \multicolumn{1}{c}{R',R''} \\\cline{3-8}
      & L & \textcolor{red}{1}, \textcolor{blue}{5} & 1, \textcolor{blue}{5} & \textcolor{red}{9}, 4 & \textcolor{red}{9}, 4 & \textcolor{red}{5}, 1 & \textcolor{red}{5}, 1 \\\cline{3-8}
      & R & 0, 0 & \textcolor{red}{2}, \textcolor{blue}{3} & 0, 0 & 2, \textcolor{blue}{3} & 0, 0 & 2, \textcolor{blue}{3} \\\cline{3-8}
    \end{tabular}
  \end{table}
    PSNE are $\{L,(L'L'');R,(L'R'')\}$
  \vfill\null
\end{frame}


\section{PS5, Ex. 9: Dynamic games (proper subgames / indifference)}

\begin{frame}{PS5, Ex. 9: Dynamic games (proper subgames / indifference)}
  (Voluntary, if there is enough time) Answer the following questions for the two games below:
  \begin{itemize}
    \item What are the strategies of the players?
    \item How many proper subgames are there (so not including the entire game itself)?
    \item Find all SPNE.
  \end{itemize}
\end{frame}

\begin{frame}{PS5, Ex. 9.a: Dynamic games (proper subgames)}
  \begin{multicols}{2}
    (a) \textbf{\textit{Answer the following questions:}}
    \begin{itemize}
      \item What are the strategies of the players?
      \item How many proper subgames are there (so not including the entire game itself)?
      \item Find all SPNE.
    \end{itemize}
    \vfill\null \columnbreak
    \includegraphics[width=1.2\columnwidth]{figures/Set_5_figure_1}
    \vfill\null
  \end{multicols}
\end{frame}
\begin{frame}{PS5, Ex. 9.a: Dynamic games (proper subgames)}
  \begin{multicols}{2}
    (a)
    \begin{itemize}
      \item Strategy sets:
      \begin{align*}
        S_1=\{&L;R\}\\
        S_2=\{&L;R\}\\
        S_3=\{&(L,L,L);(L,L,R);\\
              &(L,R,L);(L,R,R);\\
              &(R,L,L);(R,L,R);\\
              &(R,R,L);(R,R,R);\}
      \end{align*}
      \item There are 4 proper subgames.
      \item $SPNE=\{R,L,(R,R,L)\}$ \\
            with outcome (5,4,4).
    \end{itemize}
    \vfill\null \columnbreak
    \includegraphics[width=1.2\columnwidth]{figures/Set_5_figure_1}
    \vfill\null
  \end{multicols}
\end{frame}

\begin{frame}{PS5, Ex. 9.b: Dynamic games (indifference)}
  \begin{multicols}{2}
    (b) \textbf{\textit{Answer the following questions:}}
    \begin{itemize}
      \item What are the strategies of the players?
      \item How many proper subgames are there (so not including the entire game itself)?
      \item Find all SPNE.
    \end{itemize}
    \vfill\null \columnbreak
    \includegraphics[width=1.2\columnwidth]{figures/Set_5_figure_2}
    \vfill\null
  \end{multicols}
\end{frame}
\begin{frame}{PS5, Ex. 9.b: Dynamic games (indifference)}
  \begin{multicols}{2}
    (b)
    \begin{itemize}
      \item Strategy sets (same as in 9.a):
      \begin{align*}
        S_1=\{&L;R\}\\
        S_2=\{&L;R\}\\
        S_3=\{&(L,L,L);(L,L,R);\\
              &(L,R,L);(L,R,R);\\
              &(R,L,L);(R,L,R);\\
              &(R,R,L);(R,R,R);\}
      \end{align*}
      \item There are 4 proper subgames.
      \item \textbf{\textit{At two different nodes, a player can be indifferent. Solve for each case.}}
    \end{itemize}
    \vfill\null \columnbreak
    \includegraphics[width=1.2\columnwidth]{figures/Set_5_figure_2}
    \vfill\null
  \end{multicols}
\end{frame}
\begin{frame}{PS5, Ex. 9.b: Dynamic games (indifference)}
  \begin{multicols}{2}
    (b)
    \begin{itemize}
      \item Strategy sets (same as in 9.a):
      \begin{align*}
        S_1=\{&L;R\}\\
        S_2=\{&L;R\}\\
        S_3=\{&(L,L,L);(L,L,R);\\
              &(L,R,L);(L,R,R);\\
              &(R,L,L);(R,L,R);\\
              &(R,R,L);(R,R,R);\}
      \end{align*}
      \item There are 4 proper subgames.
      \item At two different nodes, a player can be indifferent. Solving for each case:
      \begin{align*}
        SPNE=\{ &L,L,(L,L,L);\\
                &L,L,(L,L,R);\\
                &R,R,(L,L,R)\}
      \end{align*}
    \end{itemize}
    \vfill\null \columnbreak
    \includegraphics[width=1.2\columnwidth]{figures/Set_5_figure_2}
    \vfill\null
  \end{multicols}
\end{frame}




% \section{Code examples} % out-comment: ctrl-shift-7 or ctrl-shift-* (use cmd for Mac)

% \begin{frame}{Code examples}
%   \begin{multicols}{2}
%     % Game tree:
%     \begin{figure}[!h]
%       \center
%       \def\svgwidth{.8\columnwidth}
%       \import{figures/}{1_.pdf_tex}
%     \end{figure}
%   \vfill\null \columnbreak
%     Matrix, no player names:
%     \vspace{-10pt}
%     \begin{table} % as opposed to matrices with player names, each line does not start with "&" as there's no empty column for the name-box. Otherwise, see the explanations below.
%       \begin{tabular}{l|c|c|}
%         \multicolumn{1}{c}{} & \multicolumn{1}{c}{L (q)} & \multicolumn{1}{c}{R (1-q)} \\\cline{2-3}
%         T (p)   &  &  \\\cline{2-3}
%         B (1-p) &  &  \\\cline{2-3}
%       \end{tabular}
%     \end{table}
%     Matrix, no colors:
%     \vspace{-10pt}
%     \begin{table}
%       \begin{tabular}{cl|c|c|} % the number of total columns and which have vertical lines between them (left-align or center text).
%         & \multicolumn{1}{c}{} & \multicolumn{2}{c}{Player 2}\\ % "2" is the number of columns in the matrix that the 2nd player name spans over
%         \parbox[t]{1mm}{\multirow{3}{*}{\rotatebox[origin=r]{90}{Player 1}}} % "3" is the number of rows the 1st player name spans over (including the one with the column names)
%         & \multicolumn{1}{c}{} & \multicolumn{1}{c}{L (q)} & \multicolumn{1}{c}{R (1-q)} \\\cline{3-4} % column names use the "\multicolumn" command to not draw vertical lines between them.
%         & T (p)   &  &  \\\cline{3-4} % a horizontal line is drawn after the line break using "\cline{x-y}" where x and y are the column numbers of the cells to be underlined.
%         & B (1-p) &  &  \\\cline{3-4}
%       \end{tabular}
%     \end{table}
%     Matrix, with colors:
%     \vspace{-10pt}
%     \begin{table}
%       \begin{tabular}{cl|c|c|}
%         & \multicolumn{1}{c}{} & \multicolumn{2}{c}{\color{blue}Player 2}\\
%         \parbox[t]{1mm}{\multirow{3}{*}{\rotatebox[origin=r]{90}{\color{red}Player 1}}}
%         & \multicolumn{1}{c}{} & \multicolumn{1}{c}{L (q)} & \multicolumn{1}{c}{R (1-q)} \\\cline{3-4}
%         & T (p)   & \textcolor{red}{1}, \textcolor{blue}{1} &   \\\cline{3-4}
%         & B (1-p) &  &  \\\cline{3-4}
%       \end{tabular}
%     \end{table}
%   \vfill\null
%   \end{multicols}
% \end{frame}



\end{document}
