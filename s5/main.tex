\documentclass[8pt,apectratio=169]{beamer}

\usetheme[progressbar=frametitle]{metropolis}
\usepackage{appendixnumberbeamer}
\usepackage[style=authoryear, backend=bibtex8, natbib=true, maxcitenames=2]{biblatex}

\usepackage[utf8]{inputenc} % utf8x  defines more symbols, but may cause compatible problems
\usepackage{lmodern,textcomp} % Latin Modern fonts, contains €

\usepackage{graphicx}
\usepackage{import}

\usepackage{booktabs}
\usepackage[scale=2]{ccicons}

\usepackage{pgfplots}
\usepgfplotslibrary{dateplot}

\usepackage{xspace}
\newcommand{\themename}{\textbf{\textsc{metropolis}}\xspace}

% Math
\usepackage{amsmath}
\usepackage{bm} % bold symbol in math mode
\counterwithin*{equation}{section} % reset the equation number whenever section is stepped

% Optional packages
\usepackage{xcolor}
\usepackage{multicol}
\usepackage{multirow,array}
\usepackage{subcaption} % for subfigure and subtable
\usepackage{hyperref}
\usepackage{epigraph}
\usepackage[super,negative]{nth} % allows writing 1st, 2nd, 3rd with superscript
\usepackage{ulem} % use the "sout" tag to "strikethrough" text
\usepackage{cancel} % https://tex.stackexchange.com/questions/75525/how-to-write-crossed-out-math-in-latex
\usepackage{tcolorbox}

% Select what to do with command \comment:
  % \newcommand{\comment}[1]{}  %comments not shown
  % \newcommand{\comment}[1]{\par {\bfseries \color{blue} #1 \par}} %comments shown
% Select what to do with todonotes: i.e. \todo{}, \todo[inline]{}
  % \usepackage[disable]{todonotes} % notes not shown
  % \usepackage[draft]{todonotes}   % notes shown

%\numberwithin{equation}{section}

%\addbibresource{references}

\titlegraphic{\hfill \includegraphics[width=0.15 \textwidth]{figures/logo}}
\title{Microeconomics III: Problem Set 8\footnote{Slides created for exercise class 3 and 4, with reservation for possible errors.\\}}
\author{Thor Donsby Noe (\href{mailto:thor.noe@econ.ku.dk}{thor.noe@econ.ku.dk})
        \& Christopher Borberg (\href{mailto:christopher.borberg@econ.ku.dk}{christopher.borberg@econ.ku.dk})
        }
\date{November 13 2019} % \today
\institute{\normalsize Department of Economics, University of Copenhagen}

    % \definecolor{BlueTOL}{HTML}{222255}
    \definecolor{BrownTOL}{HTML}{666633}
    \definecolor{GreenTOL}{HTML}{225522}
    % \setbeamercolor{normal text}{fg=BlueTOL,bg=white}
    \setbeamercolor{alerted text}{fg=BrownTOL}
    \setbeamercolor{example text}{fg=GreenTOL}
    \setbeamercolor{background canvas}{bg=white}

    \setbeamercolor{block title alerted}{use=alerted text,
        fg=alerted text.fg,
        bg=alerted text.bg!80!alerted text.fg}
    \setbeamercolor{block body alerted}{use={block title alerted, alerted text},
        fg=alerted text.fg,
        bg=block title alerted.bg!50!alerted text.bg}
    \setbeamercolor{block title example}{use=example text,
        fg=example text.fg,
        bg=example text.bg!80!example text.fg}
    \setbeamercolor{block body example}{use={block title example, example text},
        fg=example text.fg,
        bg=block title example.bg!50!example text.bg}

\begin{document}
\maketitle

% ------------------------------------------------------------------------------
% ------ FRAME -----------------------------------------------------------------
% ------------------------------------------------------------------------------
\begin{frame}{Outline}
    \tableofcontents
\end{frame}



\section{Kahoot!}

\begin{frame}{Kahoot: A exercises}
  Form a group for each table:
  \begin{itemize}
    \item Get prepared to answer the three A exercises as a team (5 min).
  \end{itemize}
  \includegraphics[width=\textwidth]{figures/kahoot}
\end{frame}



\section{PS5, Ex. 1 (A): Dynamic game (backwards induction)}

\begin{frame}{PS5, Ex. 1 (A): Dynamic game (backwards induction)}
  \begin{multicols}{2}
    \begin{itemize}
      \item Consider the dynamic game shown in extensive form. Solve it by backwards induction.
    \end{itemize}
    \vfill\null \columnbreak
    \begin{figure}[!h]
      \center
      \def\svgwidth{.8\columnwidth}
      \import{figures/}{1_.pdf_tex}
    \end{figure}
    \vfill\null
  \end{multicols}
\end{frame}

\begin{frame}{PS5, Ex. 1 (A): Dynamic game (backwards induction)}
  \begin{multicols}{2}
    \begin{itemize}
      \item Consider the dynamic game shown in extensive form. Solve it by backwards induction.
    \end{itemize}
    The backwards induction solution is the full strategy profile given by the subgame perfect NE:
    \begin{align*}
      SPNE=(s_1^{*},s_2^{*})=(R\ R'',\ L')
    \end{align*}
    \vfill\null \columnbreak
    \begin{figure}[!h]
      \center
      \def\svgwidth{.8\columnwidth}
      \import{figures/}{1.pdf_tex}
    \end{figure}
    \vfill\null
  \end{multicols}
\end{frame}



\section{PS5, Ex. 2 (A): Dynamic game (strategy sets)}

\begin{frame}{PS5, Ex. 2 (A): Extended Battle of the Sexes Game (strategy sets)}
  \begin{multicols}{2}
    Consider the game in the figure.
    \begin{itemize}
      \item[(a)] Write up the strategy sets of the players.
      \item[(b)] Write up the normal form (including the bi-matrix).
      \item[(c)] Find the Nash Equilibria.
      \item[(d)] Find the backwards induction outcome
    \end{itemize}
    \vfill\null \columnbreak
    \begin{figure}[!h]
      \center
      \def\svgwidth{\columnwidth}
      \import{figures/}{2_.pdf_tex}
    \end{figure}
    \vfill\null
  \end{multicols}
\end{frame}

\begin{frame}{PS5, Ex. 2.a (A): Extended Battle of the Sexes Game (strategy sets)}
  \begin{multicols}{2}
    \begin{itemize}
      \item[(a)] Write up the strategy sets of the players.
    \end{itemize}
    \vfill\null \columnbreak
    \begin{figure}[!h]
      \center
      \def\svgwidth{\columnwidth}
      \import{figures/}{2_.pdf_tex}
    \end{figure}
    \vfill\null
  \end{multicols}
\end{frame}
\begin{frame}{PS5, Ex. 2.a (A): Extended Battle of the Sexes Game (strategy sets)}
  \begin{multicols}{2}
    \begin{itemize}
      \item[(a)] Write up the strategy sets of the players.
    \end{itemize}
    The two strategy sets are:
    \begin{align*}
      S_1=\{\ &(G\ O'\ O''),\ (G\ O'\ F''),\\
              &(G\ F'\ O''),\ (G\ F'\ F''),\\
              &(S\ O'\ O''),\ (S\ O'\ F''),\\
              &(S\ F'\ O''),\ (S\ F'\ F'')\ \}\\
      S_2=\{\ &(O),\ (F)\ \}
    \end{align*}
    \vfill\null \columnbreak
    \begin{figure}[!h]
      \center
      \def\svgwidth{\columnwidth}
      \import{figures/}{2_.pdf_tex}
    \end{figure}
    \vfill\null
  \end{multicols}
\end{frame}

\begin{frame}{PS5, Ex. 2.b (A): Extended Battle of the Sexes Game (strategy sets)}
  \begin{multicols}{2}
    \begin{itemize}
      \item[(b)] Write up the normal form (bi-matrix).
    \end{itemize}
    \vfill\null \columnbreak
    \begin{figure}[!h]
      \center
      \def\svgwidth{\columnwidth}
      \import{figures/}{2_.pdf_tex}
    \end{figure}
    \vfill\null
  \end{multicols}
\end{frame}
\begin{frame}{PS5, Ex. 2.b (A): Extended Battle of the Sexes Game (strategy sets)}
  \begin{multicols}{2}
    \begin{itemize}
      \item[(b)] Write up the normal form (bi-matrix).
    \end{itemize}
    \begin{table}
      \begin{tabular}{cl|c|c|}
        & \multicolumn{1}{c}{} & \multicolumn{2}{c}{\color{blue}Player 2}\\
        & \multicolumn{1}{c}{} & \multicolumn{1}{c}{O} & \multicolumn{1}{c}{F} \\\cline{3-4}
        \parbox[t]{1mm}{\multirow{8}{*}{\rotatebox[origin=c]{90}{\color{red}Player 1}}}
        & (G\ O'\ O'') & \textcolor{red}{3}, \textcolor{blue}{1} & 0, 0 \\\cline{3-4}
        & (G\ O'\ F'') & \textcolor{red}{3}, 1 & 1, \textcolor{blue}{3} \\\cline{3-4}
        & (G\ F'\ O'') & 0, \textcolor{blue}{0} & 0, \textcolor{blue}{0} \\\cline{3-4}
        & (G\ F'\ F'') & 0, 0 & 1, \textcolor{blue}{3} \\\cline{3-4}
        & (S\ O'\ O'') & 2, \textcolor{blue}{2} & \textcolor{red}{2}, \textcolor{blue}{2} \\\cline{3-4}
        & (S\ O'\ F'') & 2, \textcolor{blue}{2} & \textcolor{red}{2}, \textcolor{blue}{2} \\\cline{3-4}
        & (S\ F'\ O'') & 2, \textcolor{blue}{2} & \textcolor{red}{2}, \textcolor{blue}{2} \\\cline{3-4}
        & (S\ F'\ F'') & 2, \textcolor{blue}{2} & \textcolor{red}{2}, \textcolor{blue}{2} \\\cline{3-4}
      \end{tabular}
    \end{table}
    \vfill\null \columnbreak
    \begin{figure}[!h]
      \center
      \def\svgwidth{\columnwidth}
      \import{figures/}{2_.pdf_tex}
    \end{figure}
    \vfill\null
  \end{multicols}
\end{frame}

\begin{frame}{PS5, Ex. 2.c (A): Extended Battle of the Sexes Game (strategy sets)}
  \begin{multicols}{2}
    \begin{itemize}
      \item[(c)] Find the Nash Equilibria.
    \end{itemize}
    \begin{table}
      \begin{tabular}{cl|c|c|}
        & \multicolumn{1}{c}{} & \multicolumn{2}{c}{\color{blue}Player 2}\\
        & \multicolumn{1}{c}{} & \multicolumn{1}{c}{O} & \multicolumn{1}{c}{F} \\\cline{3-4}
        \parbox[t]{1mm}{\multirow{8}{*}{\rotatebox[origin=c]{90}{\color{red}Player 1}}}
        & (G\ O'\ O'') & \textcolor{red}{3}, \textcolor{blue}{1} & 0, 0 \\\cline{3-4}
        & (G\ O'\ F'') & \textcolor{red}{3}, 1 & 1, \textcolor{blue}{3} \\\cline{3-4}
        & (G\ F'\ O'') & 0, \textcolor{blue}{0} & 0, \textcolor{blue}{0} \\\cline{3-4}
        & (G\ F'\ F'') & 0, 0 & 1, \textcolor{blue}{3} \\\cline{3-4}
        & (S\ O'\ O'') & 2, \textcolor{blue}{2} & \textcolor{red}{2}, \textcolor{blue}{2} \\\cline{3-4}
        & (S\ O'\ F'') & 2, \textcolor{blue}{2} & \textcolor{red}{2}, \textcolor{blue}{2} \\\cline{3-4}
        & (S\ F'\ O'') & 2, \textcolor{blue}{2} & \textcolor{red}{2}, \textcolor{blue}{2} \\\cline{3-4}
        & (S\ F'\ F'') & 2, \textcolor{blue}{2} & \textcolor{red}{2}, \textcolor{blue}{2} \\\cline{3-4}
      \end{tabular}
    \end{table}
    \vfill\null \columnbreak
    \begin{figure}[!h]
      \center
      \def\svgwidth{\columnwidth}
      \import{figures/}{2_.pdf_tex}
    \end{figure}
    \vfill\null
  \end{multicols}
\end{frame}
\begin{frame}{PS5, Ex. 2.c (A): Extended Battle of the Sexes Game (strategy sets)}
  \begin{multicols}{2}
    \begin{itemize}
      \item[(c)] Find the Nash Equilibria.
    \end{itemize}
    The five NE are:
    \begin{align*}
      NE=\{\ &(G\ O'\ O'',\ O);\ (S\ O'\ O'',\ F);\\
            &(S\ O'\ F'',\ F);\ (S\ F'\ O'',\ F);\\
            &(S\ F'\ F'',\ F)\ \}\\
    \end{align*}
    \vspace{-30pt}
    \begin{table}
      \begin{tabular}{cl|c|c|}
        & \multicolumn{1}{c}{} & \multicolumn{2}{c}{\color{blue}Player 2}\\
        & \multicolumn{1}{c}{} & \multicolumn{1}{c}{O} & \multicolumn{1}{c}{F} \\\cline{3-4}
        \parbox[t]{1mm}{\multirow{8}{*}{\rotatebox[origin=c]{90}{\color{red}Player 1}}}
        & (G\ O'\ O'') & \textcolor{red}{3}, \textcolor{blue}{1} & 0, 0 \\\cline{3-4}
        & (G\ O'\ F'') & \textcolor{red}{3}, 1 & 1, \textcolor{blue}{3} \\\cline{3-4}
        & (G\ F'\ O'') & 0, \textcolor{blue}{0} & 0, \textcolor{blue}{0} \\\cline{3-4}
        & (G\ F'\ F'') & 0, 0 & 1, \textcolor{blue}{3} \\\cline{3-4}
        & (S\ O'\ O'') & 2, \textcolor{blue}{2} & \textcolor{red}{2}, \textcolor{blue}{2} \\\cline{3-4}
        & (S\ O'\ F'') & 2, \textcolor{blue}{2} & \textcolor{red}{2}, \textcolor{blue}{2} \\\cline{3-4}
        & (S\ F'\ O'') & 2, \textcolor{blue}{2} & \textcolor{red}{2}, \textcolor{blue}{2} \\\cline{3-4}
        & (S\ F'\ F'') & 2, \textcolor{blue}{2} & \textcolor{red}{2}, \textcolor{blue}{2} \\\cline{3-4}
      \end{tabular}
    \end{table}
    \vfill\null \columnbreak
    \begin{figure}[!h]
      \center
      \def\svgwidth{\columnwidth}
      \import{figures/}{2_.pdf_tex}
    \end{figure}
    \vfill\null
  \end{multicols}
\end{frame}

\begin{frame}{PS5, Ex. 2.d (A): Extended Battle of the Sexes Game (strategy sets)}
  \begin{multicols}{2}
    \begin{itemize}
      \item[(d)] Find the backwards induction outcome
    \end{itemize}
    \begin{table}
      \begin{tabular}{cl|c|c|}
        & \multicolumn{1}{c}{} & \multicolumn{2}{c}{\color{blue}Player 2}\\
        & \multicolumn{1}{c}{} & \multicolumn{1}{c}{O} & \multicolumn{1}{c}{F} \\\cline{3-4}
        \parbox[t]{1mm}{\multirow{8}{*}{\rotatebox[origin=c]{90}{\color{red}Player 1}}}
        & (G\ O'\ O'') & \textcolor{red}{3}, \textcolor{blue}{1} & 0, 0 \\\cline{3-4}
        & (G\ O'\ F'') & \textcolor{red}{3}, 1 & 1, \textcolor{blue}{3} \\\cline{3-4}
        & (G\ F'\ O'') & 0, \textcolor{blue}{0} & 0, \textcolor{blue}{0} \\\cline{3-4}
        & (G\ F'\ F'') & 0, 0 & 1, \textcolor{blue}{3} \\\cline{3-4}
        & (S\ O'\ O'') & 2, \textcolor{blue}{2} & \textcolor{red}{2}, \textcolor{blue}{2} \\\cline{3-4}
        & (S\ O'\ F'') & 2, \textcolor{blue}{2} & \textcolor{red}{2}, \textcolor{blue}{2} \\\cline{3-4}
        & (S\ F'\ O'') & 2, \textcolor{blue}{2} & \textcolor{red}{2}, \textcolor{blue}{2} \\\cline{3-4}
        & (S\ F'\ F'') & 2, \textcolor{blue}{2} & \textcolor{red}{2}, \textcolor{blue}{2} \\\cline{3-4}
      \end{tabular}
    \end{table}
    \vfill\null \columnbreak
    \begin{figure}[!h]
      \center
      \def\svgwidth{\columnwidth}
      \import{figures/}{2_.pdf_tex}
    \end{figure}
    \vfill\null
  \end{multicols}
\end{frame}
\begin{frame}{PS5, Ex. 2.d (A): Extended Battle of the Sexes Game (strategy sets)}
  \begin{multicols}{2}
    \begin{itemize}
      \item[(d)] Find the backwards induction outcome
    \end{itemize}
    BI gives the unique SPNE:
    \begin{align*}
      SPNE=(s_1^{*},s_2^{*})=(S\ O'\ F'',\ F)
    \end{align*}
    The NE $(G\ O'\ O'',\ O)$ is not subgame perfect as player 1's strategy is weakly dominated by $(G\ O'\ F'')$. A SPNE needs to be rational on and off the equilibrium path, thus, $O''$ is an empty threat.
    \vspace{-8pt}
    \begin{table}
      \begin{tabular}{cl|c|c|}
        & \multicolumn{1}{c}{} & \multicolumn{2}{c}{\color{blue}Player 2}\\
        & \multicolumn{1}{c}{} & \multicolumn{1}{c}{O} & \multicolumn{1}{c}{F} \\\cline{3-4}
        \parbox[t]{1mm}{\multirow{8}{*}{\rotatebox[origin=c]{90}{\color{red}Player 1}}}
        & (G\ O'\ O'') & \textcolor{red}{3}, \textcolor{blue}{1} & 0, 0 \\\cline{3-4}
        & (G\ O'\ F'') & \textcolor{red}{3}, 1 & 1, \textcolor{blue}{3} \\\cline{3-4}
        & (G\ F'\ O'') & 0, \textcolor{blue}{0} & 0, \textcolor{blue}{0} \\\cline{3-4}
        & (G\ F'\ F'') & 0, 0 & 1, \textcolor{blue}{3} \\\cline{3-4}
        & (S\ O'\ O'') & 2, \textcolor{blue}{2} & \textcolor{red}{2}, \textcolor{blue}{2} \\\cline{3-4}
        & (S\ O'\ F'') & 2, \textcolor{blue}{2} & \textcolor{red}{2}, \textcolor{blue}{2} \\\cline{3-4}
        & (S\ F'\ O'') & 2, \textcolor{blue}{2} & \textcolor{red}{2}, \textcolor{blue}{2} \\\cline{3-4}
        & (S\ F'\ F'') & 2, \textcolor{blue}{2} & \textcolor{red}{2}, \textcolor{blue}{2} \\\cline{3-4}
      \end{tabular}
    \end{table}
    \vfill\null \columnbreak
    \begin{figure}[!h]
      \center
      \def\svgwidth{\columnwidth}
      \import{figures/}{2.pdf_tex}
    \end{figure}
    \vfill\null
  \end{multicols}
\end{frame}



\section{PS5, Ex. 3 (A): Stackelberg game}

\begin{frame}{PS5, Ex. 3 (A): Stackelberg game}
  \begin{multicols}{2}
    \vfill\null \columnbreak
    \vfill\null
  \end{multicols}
\end{frame}

\begin{frame}{PS5, Ex. 3.a (A): Stackelberg game}
  \begin{multicols}{2}
    \vfill\null \columnbreak
    \vfill\null
  \end{multicols}
\end{frame}

\begin{frame}{PS5, Ex. 3.b (A): Stackelberg game}
  \begin{multicols}{2}
    \vfill\null \columnbreak
    \vfill\null
  \end{multicols}
\end{frame}



\section{PS5, Ex. 4: The Mutated Seabass (backwards induction)}

\begin{frame}{PS5, Ex. 4: The Mutated Seabass (backwards induction)}
    Consider a game where two evil organizations, rather prosaically named A and B, are battling for world domination. The battle takes the form of a three-stage game. Organization A is on the verge of acquiring a new powerful weapon, the \textit{mutated seabass}. In stage 1 of the game, they decide whether to acquire the weapon or not. Their choice is observed by organization B. In stage 2, organization B decides whether to attack organization A. If an attack occurs, the game stops. If no attack occurs, it moves to stage 3, where organization A decides whether or not to attack organization B. The payoffs are as follows. If no-one attacks the other, the payoffs to both organizations are 0. If B attacks A, then the payoffs to both organizations are .1. The same if A attacks B, without having acquired the seabass weapon. If, on the other hand, A acquires the weapon, the payoffs from A attacking B are 2 to A and -2 to B.
    \begin{itemize}
      \item[(a)] Draw the game tree that corresponds to the game. What are the strategies of the players?
      \item[(b)] What is the backwards induction outcome?
      \item[(c)] What is the intuition for the outcome? What role do you think it plays that B observes if A acquires the weapon or not?
    \end{itemize}
  \vfill\null
\end{frame}

\begin{frame}{PS5, Ex. 4.a: The Mutated Seabass (backwards induction)}
  \begin{itemize}
    \item[(a)] Draw the game tree that corresponds to the game. What are the strategies of the players?
  \end{itemize}
  \vspace{-8pt}
  \begin{align*}
    S_A=\{ &(Acquire, a, a'), (Acquire, a, na'), (Acquire, na, a'), (Acquire, na, na'),\\
            &(Not\ acquire, a, a'), (Not\ acquire, a, na'), (Not\ acquire, na, a'), (Not\ acquire, na, na') \}\\
    S_B=\{ &(A, A'), (A, NA'), (NA, A'), (NA, NA') \}
  \end{align*}
  \vspace{-8pt}
  \begin{figure}[!h]
    \center
    \def\svgwidth{\columnwidth}
    \import{figures/}{4a.pdf_tex}
  \end{figure}
  \vfill\null
\end{frame}

\begin{frame}{PS5, Ex. 4.b: The Mutated Seabass (backwards induction)}
  \begin{itemize}
    \item[(b)] What is the backwards induction outcome?
    \begin{itemize}\normalsize
      \item[\nth{3} stage:] Org. A will choose to attack if having acquired the weapon and not attack if not having acquired the weapon.
      \item[\nth{2} stage:] Org. B will choose to attack if Org. A has acquired the weapon and not attack if they have not acquired the weapon.
      \item[\nth{1} stage:] Org. A will choose to not acquire the weapon in order to signal peaceful intentions to Org. B, i.e. giving the payoffs $(0,0)$.
      \item[SPNE:] $\{S_A;S_B\}=\{ (Not\ acquire, a, na');(A, NA') \}$
    \end{itemize}
  \end{itemize}
  \vspace{-8pt}
  \begin{figure}[!h]
    \center
    \def\svgwidth{\columnwidth}
    \import{figures/}{4b.pdf_tex}
  \end{figure}
  \vfill\null
\end{frame}

\begin{frame}{PS5, Ex. 4.c: The Mutated Seabass (backwards induction)}
    \begin{itemize}
      \item[(c)] What is the intuition for the outcome? What role do you think it plays that B observes if A acquires the weapon or not?
      \begin{itemize}\normalsize
        \item[\nth{3} stage:] Org. A does only benefit from attacking if having acquired the weapon.
        \item[\nth{2} stage:] Org. B will only choose to attack if Org. A has acquired the weapon.
        \item[\nth{1} stage:] Not acquiring the weapon is a credible signal that Org. A will not attack.
      \end{itemize}
      \item[] What if Org. A cannot send a signal in the \nth{1} stage?
    \end{itemize}
    \vspace{-8pt}
    \begin{figure}[!h]
      \center
      \def\svgwidth{\columnwidth}
      \import{figures/}{4b.pdf_tex}
    \end{figure}
    \vfill\null
  \vfill\null
\end{frame}
\begin{frame}{PS5, Ex. 4.c: The Mutated Seabass (backwards induction)}
    \begin{itemize}
      \item[(c)] What is the intuition for the outcome? What role do you think it plays that B observes if A acquires the weapon or not?
      \begin{itemize}\normalsize
        \item[\nth{3} stage:] [unchanged] Org. A will choose to attack if having acquired the weapon and not attack if not having acquired the weapon.
        \item[\nth{2} stage:] Knowing that Org. A will attack if having acquired the weapon, Org. B chooses to attack first, giving the payoffs $(-1,-1)$.
        \item[\nth{1} stage:] Org. A cannot affect the outcome, but acquires it in case Org. B deviates.
        \item[SPNE:] $\{S_A;S_B\}=\{ (Acquire, a, na');A \}$
      \end{itemize}
    \end{itemize}
    \vspace{-10pt}
    \begin{figure}[!h]
      \center
      \def\svgwidth{\columnwidth}
      \import{figures/}{4c.pdf_tex}
    \end{figure}
    \vfill\null
  \vfill\null
\end{frame}


\section{PS5, Ex. 5: Three player game (backwards induction)}

\begin{frame}{PS5, Ex. 5: Three player game (backwards induction)}
    Consider the game below where player 1 chooses the matrix (A or B), player 2 chooses the row (C or D), and player 3 chooses the column (E or F). In each cell, the first number gives the payoff of Player 1, the second number the payoff of Player 2, and the third number the payoff of Player 3.
    \begin{table}
      \begin{tabular}{l|c|c|}
        \multicolumn{1}{c}{} & \multicolumn{1}{c}{E} & \multicolumn{1}{c}{F} \\\cline{2-3}
        C & 5, 2, 2 & 2, 1, 1 \\\cline{2-3}
        D & 0, 1, 1 & 1, 0, 0 \\\cline{2-3}
        \multicolumn{1}{c}{} & \multicolumn{2}{c}{A}
      \end{tabular}\quad
      \begin{tabular}{l|c|c|}
        \multicolumn{1}{c}{} & \multicolumn{1}{c}{E} & \multicolumn{1}{c}{F} \\\cline{2-3}
        C & 6, 0, 1 & 3, 1, 2 \\\cline{2-3}
        D & 1, 1, 0 & 2, 2, 1 \\\cline{2-3}
        \multicolumn{1}{c}{} & \multicolumn{2}{c}{B}
      \end{tabular}
    \end{table}
    \begin{itemize}
      \item[(a)] Suppose first that the game is static, such that all three players move simultaneously. Find all the pure-strategy Nash Equilibria.
      \item[(b)] Now suppose the game is dynamic: Player 1 moves first, and then, after having observed his move, Player 2 moves, and, finally, after having observed the first two moves, Player 3 moves. Draw the game tree and solve by backwards induction.
      \item[(c)] Discuss the differences in the results you find
    \end{itemize}
  \vfill\null
\end{frame}

\begin{frame}{PS5, Ex. 5.a: Three player game (backwards induction)}
    \begin{itemize}
      \item[(a)] Suppose first that the game is static, such that all three players move simultaneously. Find all the pure-strategy Nash Equilibria.
    \end{itemize}
    \begin{table}
      \begin{tabular}{l|c|c|}
        \multicolumn{1}{c}{} & \multicolumn{1}{c}{E} & \multicolumn{1}{c}{F} \\\cline{2-3}
        C & 5, \textcolor{blue}{2}, \textcolor{green}{2} & 2, \textcolor{blue}{1}, 1 \\\cline{2-3}
        D & 0, 1, \textcolor{green}{1} & 1, 0, 0 \\\cline{2-3}
        \multicolumn{1}{c}{} & \multicolumn{2}{c}{A}
      \end{tabular}\quad
      \begin{tabular}{l|c|c|}
        \multicolumn{1}{c}{} & \multicolumn{1}{c}{E} & \multicolumn{1}{c}{F} \\\cline{2-3}
        C & \textcolor{red}{6}, 0, 1 & \textcolor{red}{3}, 1, \textcolor{green}{2} \\\cline{2-3}
        D & \textcolor{red}{1}, \textcolor{blue}{1}, 0 & \textcolor{red}{2}, \textcolor{blue}{2}, \textcolor{green}{1} \\\cline{2-3}
        \multicolumn{1}{c}{} & \multicolumn{2}{c}{B}
      \end{tabular}
    \end{table}
  \vfill\null
\end{frame}



\section{PS5, Ex. 6: }

\begin{frame}{PS5, Ex. 6: }
  \begin{multicols}{2}
    \vfill\null \columnbreak
    \vfill\null
  \end{multicols}
\end{frame}

\begin{frame}{PS5, Ex. 6.a: }
  \begin{multicols}{2}
    \vfill\null \columnbreak
    \vfill\null
  \end{multicols}
\end{frame}



\section{PS5, Ex. 7: }

\begin{frame}{PS5, Ex. 7: }
  \begin{multicols}{2}
    \vfill\null \columnbreak
    \vfill\null
  \end{multicols}
\end{frame}

\begin{frame}{PS5, Ex. 7.a: }
  \begin{multicols}{2}
    \vfill\null \columnbreak
    \vfill\null
  \end{multicols}
\end{frame}



\section{PS5, Ex. 8: }

\begin{frame}{PS5, Ex. 8: }
  \begin{multicols}{2}
    \vfill\null \columnbreak
    \vfill\null
  \end{multicols}
\end{frame}

\begin{frame}{PS5, Ex. 8.a: }
  \begin{multicols}{2}
    \vfill\null \columnbreak
    \vfill\null
  \end{multicols}
\end{frame}



\section{PS5, Ex. 9: }

\begin{frame}{PS5, Ex. 9: }
  \begin{multicols}{2}
    \vfill\null \columnbreak
    \vfill\null
  \end{multicols}
\end{frame}

\begin{frame}{PS5, Ex. 9.a: }
  \begin{multicols}{2}
    \vfill\null \columnbreak
    \vfill\null
  \end{multicols}
\end{frame}



\section{Code examples} % out-comment: ctrl-shift-7 or ctrl-shift-* (use cmd for Mac)

\begin{frame}{Code examples}
  \begin{multicols}{2}
    % Game tree:
    \begin{figure}[!h]
      \center
      \def\svgwidth{.8\columnwidth}
      \import{figures/}{1_.pdf_tex}
    \end{figure}
  \vfill\null \columnbreak
    Matrix, no player names:
    \vspace{-10pt}
    \begin{table}
      \begin{tabular}{l|c|c|}
        \multicolumn{1}{c}{} & \multicolumn{1}{c}{L (q)} & \multicolumn{1}{c}{R (1-q)} \\\cline{2-3}
        T (p)   &  &  \\\cline{2-3}
        B (1-p) &  &  \\\cline{2-3}
      \end{tabular}
    \end{table}
    Matrix, no colors:
    \vspace{-10pt}
    \begin{table}
      \begin{tabular}{cl|c|c|}
        & \multicolumn{1}{c}{} & \multicolumn{2}{c}{Player 2}\\
        \parbox[t]{1mm}{\multirow{3}{*}{\rotatebox[origin=r]{90}{Player 1}}}
        & \multicolumn{1}{c}{} & \multicolumn{1}{c}{L (q)} & \multicolumn{1}{c}{R (1-q)} \\\cline{3-4}
        & T (p)   &  &  \\\cline{3-4}
        & B (1-p) &  &  \\\cline{3-4}
      \end{tabular}
    \end{table}
    Matrix, with colors:
    \vspace{-10pt}
    \begin{table}
      \begin{tabular}{cl|c|c|}
        & \multicolumn{1}{c}{} & \multicolumn{2}{c}{\color{blue}Player 2}\\
        \parbox[t]{1mm}{\multirow{3}{*}{\rotatebox[origin=r]{90}{\color{red}Player 1}}}
        & \multicolumn{1}{c}{} & \multicolumn{1}{c}{L (q)} & \multicolumn{1}{c}{R (1-q)} \\\cline{3-4}
        & T (p)   & \textcolor{red}{}, \textcolor{blue}{} &   \\\cline{3-4}
        & B (1-p) &  &  \\\cline{3-4}
      \end{tabular}
    \end{table}
  \vfill\null
  \end{multicols}
\end{frame}


\end{document}
