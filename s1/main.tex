\documentclass[8pt,apectratio=169]{beamer}

\usetheme[progressbar=frametitle]{metropolis}
\usepackage{appendixnumberbeamer}
\usepackage[style=authoryear, backend=bibtex8, natbib=true, maxcitenames=2]{biblatex}

\usepackage[utf8]{inputenc} % utf8x  defines more symbols, but may cause compatible problems
\usepackage{lmodern,textcomp} % Latin Modern fonts, contains €

\usepackage{graphicx}
\usepackage{import}

\usepackage{booktabs}
\usepackage[scale=2]{ccicons}

\usepackage{pgfplots}
\usepgfplotslibrary{dateplot}

\usepackage{xspace}
\newcommand{\themename}{\textbf{\textsc{metropolis}}\xspace}

% Math
\usepackage{amsmath}
\usepackage{bm} % bold symbol in math mode
\counterwithin*{equation}{section} % reset the equation number whenever section is stepped

% Optional packages
\usepackage{xcolor}
\usepackage{multicol}
\usepackage{multirow,array}
\usepackage{subcaption} % for subfigure and subtable
\usepackage{hyperref}
\usepackage{epigraph}
\usepackage[super,negative]{nth} % allows writing 1st, 2nd, 3rd with superscript
\usepackage{ulem} % use the "sout" tag to "strikethrough" text
\usepackage{cancel} % https://tex.stackexchange.com/questions/75525/how-to-write-crossed-out-math-in-latex
\usepackage{tcolorbox}

% Select what to do with command \comment:
  % \newcommand{\comment}[1]{}  %comments not shown
  % \newcommand{\comment}[1]{\par {\bfseries \color{blue} #1 \par}} %comments shown
% Select what to do with todonotes: i.e. \todo{}, \todo[inline]{}
  % \usepackage[disable]{todonotes} % notes not shown
  % \usepackage[draft]{todonotes}   % notes shown

%\numberwithin{equation}{section}

%\addbibresource{references}

\titlegraphic{\hfill \includegraphics[width=0.15 \textwidth]{figures/logo}}
\title{Microeconomics III: Problem Set 8\footnote{Slides created for exercise class 3 and 4, with reservation for possible errors.\\}}
\author{Thor Donsby Noe (\href{mailto:thor.noe@econ.ku.dk}{thor.noe@econ.ku.dk})
        \& Christopher Borberg (\href{mailto:christopher.borberg@econ.ku.dk}{christopher.borberg@econ.ku.dk})
        }
\date{November 13 2019} % \today
\institute{\normalsize Department of Economics, University of Copenhagen}

    % \definecolor{BlueTOL}{HTML}{222255}
    \definecolor{BrownTOL}{HTML}{666633}
    \definecolor{GreenTOL}{HTML}{225522}
    % \setbeamercolor{normal text}{fg=BlueTOL,bg=white}
    \setbeamercolor{alerted text}{fg=BrownTOL}
    \setbeamercolor{example text}{fg=GreenTOL}
    \setbeamercolor{background canvas}{bg=white}

    \setbeamercolor{block title alerted}{use=alerted text,
        fg=alerted text.fg,
        bg=alerted text.bg!80!alerted text.fg}
    \setbeamercolor{block body alerted}{use={block title alerted, alerted text},
        fg=alerted text.fg,
        bg=block title alerted.bg!50!alerted text.bg}
    \setbeamercolor{block title example}{use=example text,
        fg=example text.fg,
        bg=example text.bg!80!example text.fg}
    \setbeamercolor{block body example}{use={block title example, example text},
        fg=example text.fg,
        bg=block title example.bg!50!example text.bg}

\begin{document}
\maketitle

% ------------------------------------------------------------------------------
% ------ FRAME -----------------------------------------------------------------
% ------------------------------------------------------------------------------
\begin{frame}{Outline}
\tableofcontents
\end{frame}


\section{Welcome}

\begin{frame}{About me}
  \begin{multicols}{2}
    Thesis on the electricity market:
      \begin{itemize}
        \item[-] \textit{auction design}
      \end{itemize}
    Research Assistant: Estimating the blue Net National Product
      \begin{itemize}
        \item[-] \textit{valuation of public goods}
      \end{itemize}
    Erasmus student at Universitat de Barcelona - School of Economics.
      \begin{itemize}
        \item[-] \textit{public policy, taxation, regulation, privatization, institutions}
      \end{itemize}
    Student assistant at LO (now FH - Danish Trade Union Confederation)
      \begin{itemize}
        \item[-] \textit{labour economics, family economics, industrial organization}
      \end{itemize}
  \columnbreak
  \includegraphics[width=0.5 \textwidth]{graphics/GNNP}
  \end{multicols}
\end{frame}


\section{Motivation}

\begin{frame}{Relevance of game theory}
\begin{multicols}{2}
From the course description:
\begin{itemize}
  \item[1.] This course furthers the \textit{\textbf{introduction of game theory}} and its applications in economic models.
  \item[2.] The student who successfully completes the course will learn the basics of game theory and will be \textit{\textbf{enabled to work further with advanced game theory}}.
\end{itemize}
\columnbreak
\vfill\null
\end{multicols}
\end{frame}

\begin{frame}{Relevance of game theory}
\begin{multicols}{2}
\color{lightgray}
From the course description:
\begin{itemize}\color{lightgray}
  \item[\textcolor{lightgray}{1.}] This course furthers the \textit{\textbf{introduction of game theory}} and its applications in economic models.
  \item[\textcolor{lightgray}{2.}] The student who successfully completes the course will learn the basics of game theory and will be \textit{\textbf{enabled to work further with advanced game theory}}.
\end{itemize}
\columnbreak
\begin{itemize}
  \item[3.] The student will also learn how economic problems involving strategic situations can be \textit{\textbf{modeled}} using game theory, as well as how these models are \textit{\textbf{solved}}.
  \item[4.] The course intention is that the student \textit{\textbf{becomes able to work with}} modern economic theory, for instance within the areas of industrial organization, macroeconomics, international economics, labor economics, public economics, political economics and financial economics.
\end{itemize}
\end{multicols}
\end{frame}

\begin{frame}{Game theory in current master's courses (2019/2020)}
\begin{multicols}{2}
Courses where game theory is central:
\begin{itemize}
  \item Mechanism Design \item Contract Theory \item Auctions - Theory  and Practice, Incentives and Organizations \item Industrial Organization \item Advanced Industrial Organization  \item Strategic Management \item Advanced Strategic Management \item Behavioral Finance (F) \item Foundations of Behavioral Economics \item Behavioral and Experimental Economics (summerschool) \item Science of Behavior Change
\end{itemize}
\columnbreak
Courses where game theory plays a part:
\begin{itemize}
  \item Public Finance (taxation) \item Labour Economics \item Health Economics \item Political Economics \item Advanced Development Economics - Micro Aspects
\end{itemize}
Strategic, logical thinking is also useful for macroeconomists and econometricians
\end{multicols}
\end{frame}


\section{Overview of the course}

\begin{frame}{Overview of the course}
\begin{multicols}{2}
Form:
  \begin{itemize}
    \item 12 lectures + conclusion
    \item 12 problem sets in 12 exercise classes
    \item 3 mandatory assignments delivered by \href{mailto:thor.noe@econ.ku.dk}{mail} or in my pigeon box in the hall of building 26 no later than: \\
    Oct. 4, Nov. 1, Nov. 29 (Fridays)
  \end{itemize}
Question lesson on December 18, 10:00-12:00 in 7-0-28
\columnbreak
\vfill\null
\end{multicols}
\end{frame}

\begin{frame}{Overview of the course}
\begin{multicols}{2}
\color{lightgray}
Form:
\begin{itemize}\color{lightgray}
  \item[\textcolor{lightgray}{\textbullet}] 12 lectures + conclusion
  \item[\textcolor{lightgray}{\textbullet}] 12 problem sets in 12 exercise classes
  \item[\textcolor{lightgray}{\textbullet}] 3 mandatory assignments delivered by mail or in my pigeon box in the hall of building 26: \\
  Oct. 4, Nov. 1, Nov. 29
\end{itemize}
Question lesson on December 18, 10:00-12:00 in 7-0-28
\vfill\null
\columnbreak
\color{black}
Course content
\begin{enumerate}
  \item Static games with complete information (PS 1-3)
  \item Dynamic games with complete information (PS 4-6)
  \item Static games with incomplete information (PS 3, 8-10)
  \item Dynamic games with incomplete information (PS 6-7, 10-11)
  \item Psychological Game Theory (PS 12)
\end{enumerate}
\vfill\null
\end{multicols}
\end{frame}

\begin{frame}{Exam}
\begin{multicols}{2}
Form:
\begin{itemize}
  \item Two hours without aids on Peter Bangs Vej 36
\end{itemize}
Content:
\begin{itemize}
  \item \textit{\textbf{Cook-book solutions}} trained in the problem sets
  \item \textit{\textbf{Reflection}} and \textit{\textbf{discussion}}
\end{itemize}
\columnbreak
\vfill\null
\end{multicols}
\end{frame}

\begin{frame}{Exam}
\begin{multicols}{2}\color{lightgray}
Form:
\begin{itemize}\color{lightgray}
  \item[\textcolor{lightgray}{\textbullet}] Two hours without aids on Peter Bangs Vej 36
\end{itemize}
Content:
\begin{itemize}\color{lightgray}
  \item[\textcolor{lightgray}{\textbullet}] \textit{\textbf{Cook-book solutions}} from the problem sets
  \item[\textcolor{lightgray}{\textbullet}] \textit{\textbf{Reeflection}} and \textit{\textbf{discussion}}
\end{itemize}
\vfill\null
\columnbreak\color{black}
Important to complete all problem sets - but also to attend the lectures and read the curriculum:
\begin{itemize}
  \item Last winter 33\% failed the exam
  \item General lack in the ability to \textit{\textbf{interpret, explain, and give examples}} from the real world
\end{itemize}
We will train this in class as well
\end{multicols}
\end{frame}

\begin{frame}{Learning outcome}
\begin{multicols}{2}
  \textit{From the course description:}
  \\\medskip
  \textbf{Knowledge:}
  \begin{enumerate}
    \item Formally \textit{\textbf{state the definition}} of a game and \textit{\textbf{explain}} the key differences between games of different types.
    \item \textit{\textbf{In detail account for}} the equilibrium (solution) concepts that are relevant for these games (Nash Equilibrium, Subgame Perfect Nash Equilibrium, Bayes-Nash Equilibrium, Perfect Bayesian Equilibrium).
    \item \textit{\textbf{Identify}} a number of special games and particular \textit{\textbf{issues}} associated with them, such as repeated games (including infinitely repeated games), auctions and signaling games.
  \end{enumerate}
\columnbreak
\vfill\null
\end{multicols}
\end{frame}

\begin{frame}{Learning outcome}
\begin{multicols}{2}
  \color{lightgray}
  \textit{From the course description:}
  \\\medskip
  \textbf{Knowledge:}
  \begin{itemize}\color{lightgray}
    \item[\textcolor{lightgray}{1.}] Formally \textit{\textbf{state the definition}} of a game and \textit{\textbf{explain}} the key differences between games of different types.
    \item[\textcolor{lightgray}{2.}] \textit{\textbf{In detail account for}} the equilibrium (solution) concepts that are relevant for these games (Nash Equilibrium, Subgame Perfect Nash Equilibrium, Bayes-Nash Equilibrium, Perfect Bayesian Equilibrium).
    \item[\textcolor{lightgray}{3.}] \textit{\textbf{Identify}} a number of special games and particular \textit{\textbf{issues}} associated with them, such as repeated games (including infinitely repeated games), auctions and signaling games.
  \end{itemize}
\columnbreak\color{black}
%\vfill\null
\textbf{Skills:}
\begin{enumerate}
  \item Explicitly \textit{\textbf{solve}} for the equilibria of these games.
  \item \textit{\textbf{Explain}} the relevant steps in the reasoning of the solution.
  \item \textit{\textbf{Interpret}} the outcomes of the analysis.
  \item Apply equilibrium \textit{\textbf{refinements}} and \textit{\textbf{discuss}} the solution concepts
\end{enumerate}
\textbf{Competencies:}
\begin{enumerate}
  \item \textit{\textbf{Analyze}} strategic situations by \textit{\textbf{modeling}} them as formal games.
  \item \textit{\textbf{Set up, prove, analyze and apply}} the theories and methods used in the course in an \textit{\textbf{independent}} manner.
  \item \textit{\textbf{Evaluate and discuss}} the crucial assumptions underlying the theory.
\end{enumerate}
\end{multicols}
\end{frame}


\section{Exam example}

\begin{frame}{Example from the exam Autumn 2018}
\begin{multicols}{2}
  Take 5 min to discuss with your neighbor(s):
  \begin{itemize}
    \item[1.a] "The reason that players cannot achieve a good outcome in the prisoner’s dilemma is that they cannot communicate." True or false? Explain in 2-3 sentences.
    \item[1.c] "Iterated Elimination of Strictly Dominated Strategies never eliminates a Nash Equilibrium" True or false? Explain in 2-3 sentences.
    \item[1.d] You are writing your dating app profile and want to signal that you are adventurous. Give an example of a signal that is not credible and an example that is credible and explain the reasons why.
  \end{itemize}
\columnbreak
\vfill\null
\end{multicols}
\end{frame}

\begin{frame}{Example from the exam Autumn 2018}
\begin{multicols}{2}\color{lightgray}
  Take 5 min to discuss with your neighbor(s):
  \begin{itemize}
    \item[1.a] "The reason that players cannot achieve a good outcome in the prisoner’s dilemma is that they cannot communicate." True or false? Explain in 2-3 sentences.
    \color{lightgray}
    \item[\color{lightgray}1.c] "Iterated Elimination of Strictly Dominated Strategies never eliminates a Nash Equilibrium" True or false? Explain in 2-3 sentences.
    \item[\color{lightgray}1.d] You are writing your dating app profile and want to signal that you are adventurous. Give an example of a signal that is not credible and an example that is credible and explain the reasons why.
  \end{itemize}
\vfill\null
\columnbreak
\begin{itemize}
  \item[1.a] True or false? \\ Why?
\end{itemize}
\end{multicols}
\end{frame}

\begin{frame}{Example from the exam Autumn 2018 solution guide}
\begin{multicols}{2}\color{lightgray}
  Take 5 min to discuss with your neighbor(s):
  \begin{itemize}
    \item[1.a] "The reason that players cannot achieve a good outcome in the prisoner’s dilemma is that they cannot communicate." True or false? Explain in 2-3 sentences.
    \color{lightgray}
    \item[\color{lightgray}1.c] "Iterated Elimination of Strictly Dominated Strategies never eliminates a Nash Equilibrium" True or false? Explain in 2-3 sentences.
    \item[\color{lightgray}1.d] You are writing your dating app profile and want to signal that you are adventurous. Give an example of a signal that is not credible and an example that is credible and explain the reasons why.
  \end{itemize}
\vfill\null
\columnbreak
\begin{itemize}
  \item[1.a] False! Even if players could communicate their best response would still be to play Fink (confess).
\end{itemize}
\end{multicols}
\end{frame}

\begin{frame}{Example from the exam Autumn 2018}
\begin{multicols}{2}\color{lightgray}
  Take 10 min. to discuss with your neighbor(s):
  \begin{itemize}\color{lightgray}
    \item[\color{lightgray}1.a] "The reason that players cannot achieve a good outcome in the prisoner’s dilemma is that they cannot communicate." True or false? Explain in 2-3 sentences.
    \item[1.c] \color{black}"Iterated Elimination of Strictly Dominated Strategies never eliminates a Nash Equilibrium" True or false? Explain in 2-3 sentences.
    \item[\color{lightgray}1.d] \color{lightgray}You are writing your dating app profile and want to signal that you are adventurous. Give an example of a signal that is not credible and an example that is credible and explain the reasons why.
  \end{itemize}
\vfill\null
\columnbreak
\begin{itemize}\color{lightgray}
  \item[\color{lightgray}1.a] False! Even if players could communicate their best response would still be to play Fink (confess).
  \item[1.c] \color{black} True or false? \\ Why?
\end{itemize}
\end{multicols}
\end{frame}

\begin{frame}{Example from the exam Autumn 2018 solution guide}
\begin{multicols}{2}\color{lightgray}
  Take 5 min to discuss with your neighbor(s):
  \begin{itemize}\color{lightgray}
    \item[\color{lightgray}1.a] "The reason that players cannot achieve a good outcome in the prisoner’s dilemma is that they cannot communicate." True or false? Explain in 2-3 sentences.
    \item[1.c] \color{black}"Iterated Elimination of Strictly Dominated Strategies never eliminates a Nash Equilibrium" True or false? Explain in 2-3 sentences.
    \item[\color{lightgray}1.d] \color{lightgray}You are writing your dating app profile and want to signal that you are adventurous. Give an example of a signal that is not credible and an example that is credible and explain the reasons why.
  \end{itemize}
\vfill\null
\columnbreak
\begin{itemize}\color{lightgray}
  \item[\color{lightgray}1.a] False! Even if players could communicate their best response would still be to play Fink (confess).
  \item[1.c] \color{black}True! Nash equilibrium is a refinement. However, this does not hold for weakly dominated strategies.
\end{itemize}
\end{multicols}
\end{frame}

\begin{frame}{Example from the exam Autumn 2018}
\begin{multicols}{2}\color{lightgray}
  Take 5 min to discuss with your neighbor(s):
  \begin{itemize}\color{lightgray}
    \item[\color{lightgray}1.a] "The reason that players cannot achieve a good outcome in the prisoner’s dilemma is that they cannot communicate." True or false? Explain in 2-3 sentences.
    \item[\color{lightgray}1.c] "Iterated Elimination of Strictly Dominated Strategies never eliminates a Nash Equilibrium" True or false? Explain in 2-3 sentences.
    \item[1.d] \color{black}You are writing your dating app profile and want to signal that you are adventurous. Give an example of a signal that is not credible and an example that is credible and explain the reasons why.
  \end{itemize}
\vfill\null
\columnbreak
\begin{itemize}\color{lightgray}
  \item[\color{lightgray}1.a] False! Even if players could communicate their best response would still be to play Fink (confess).
  \item[\color{lightgray}1.c] True! Nash equilibrium is a refinement. However, this does not hold for
weakly dominated strategies.
  \item[1.d] \color{black} Credible? \\ Not credible? \\ Difference?
\end{itemize}
\end{multicols}
\end{frame}

\begin{frame}{Example from the exam Autumn 2018 solution guide}
\begin{multicols}{2}\color{lightgray}
  Take 5 min to discuss with your neighbor(s):
  \begin{itemize}\color{lightgray}
    \item[\color{lightgray}1.a] "The reason that players cannot achieve a good outcome in the prisoner’s dilemma is that they cannot communicate." True or false? Explain in 2-3 sentences.
    \item[\color{lightgray}1.c] "Iterated Elimination of Strictly Dominated Strategies never eliminates a Nash Equilibrium" True or false? Explain in 2-3 sentences.
    \item[1.d] \color{black}You are writing your dating app profile and want to signal that you are adventurous. Give an example of a signal that is not credible and an example that is credible and explain the reasons why.
  \end{itemize}
\vfill\null
\columnbreak
\begin{itemize}\color{lightgray}
  \item[\color{lightgray}1.a] False! Even if players could communicate their best response would still be to play Fink (confess).
  \item[\color{lightgray}1.c] True! Nash equilibrium is a refinement. However, this does not hold for
weakly dominated strategies.
  \item[1.d] \color{black}\textit{\textbf{Credible signals:}} Show a picture of you skydiving, swimming with
sharks etc. \textit{\textbf{Not credible:}} Write that you are adventurous or only claim that you
have been skydiving etc. There needs to be a differential cost that makes it affordable for those with a hidden desirable trait (being adventurous), not affordable for those without this trait.\\
\textit{\textbf{There is a cost to the signal:}} Those with the desirable trait are more likely to send the signal. Those who exhibit the signal are more desirable.
\end{itemize}
\end{multicols}
\end{frame}


\section{PS1, Ex. 1: Basics of game theory}

\begin{frame}{PS1: Exercise 1}
\begin{multicols}{2}
What is game theory and why do we do it? To answer this, briefly discuss the following questions:
\begin{itemize}
  \item[(a)] What are the ingredients of a (normal form) game?
  \item[(b)] How do we analyze games?
  \item[(c)] Why do you think it is practical to analyze problems as games?
\end{itemize}
\vfill\null
\columnbreak
\textit{Take 5 min. to discuss it with your neighbor(s)}
\vfill\null
\end{multicols}
\end{frame}

\begin{frame}{PS1: Exercise 1}
\begin{multicols}{2}
What is game theory and why do we do it? To answer this, briefly discuss the following questions:
\begin{itemize}
  \item[(a)] What are the ingredients of a (normal form) game?
  \item[(b)] How do we analyze games?
  \item[(c)] Why do you think it is practical to analyze problems as games?
\end{itemize}
\vfill\null
\columnbreak
\begin{itemize}
  \item[(a)] A normal form game consists of:
    \begin{enumerate}
      \item The set of players $i$
      \item The possible strategy sets $S_i\in \{s_1,s_2,...,s_n\}$ for each player $i$
      \item Each players utility (payoff) function $u_i(s_1,s_2,...,s_n)$
    \end{enumerate}
\end{itemize}
\vfill\null
\end{multicols}
\end{frame}

\begin{frame}{PS1: Exercise 1}
\begin{multicols}{2}
What is game theory and why do we do it? To answer this, briefly discuss the following questions:
\begin{itemize}
  \item[(a)] What are the ingredients of a (normal form) game?
  \item[(b)] How do we analyze games?
  \item[(c)] Why do you think it is practical to analyze problems as games?
\end{itemize}
\vfill\null
\columnbreak
\begin{itemize}
  \item[(a)] A normal form game consists of:
    \begin{enumerate}
      \item The set of players $i$
      \item The possible strategy sets $S_i\in \{s_1,s_2,...,s_n\}$ for each player $i$
      \item Each players utility (payoff) function $u_i(s_1,s_2,...,s_n)$
    \end{enumerate}
  \item[(b)] Define a solution concept, state the assumptions it relies on, and its possible limitations.\\
  E.g. Iterative Elimination of Strictly Dominated Strategies (IESDS) requires common knowledge of rationality but is not always sufficient to find the Nash Equilibrium.
\end{itemize}
\vfill\null
\end{multicols}
\end{frame}

\begin{frame}{PS1: Exercise 1}
\begin{multicols}{2}
What is game theory and why do we do it? To answer this, briefly discuss the following questions:
\begin{itemize}
  \item[(a)] What are the ingredients of a (normal form) game?
  \item[(b)] How do we analyze games?
  \item[(c)] Why do you think it is practical to analyze problems as games?
\end{itemize}
\vfill\null
\columnbreak
\begin{itemize}
  \item[(a)] A normal form game consists of:
    \begin{enumerate}
      \item The set of players $i$
      \item The possible strategy sets $S_i\in \{s_1;s_2;...;s_n\}$ for each player $i$
      \item Each players utility (payoff) function $u_i(s_1,s_2,...,s_n)$
    \end{enumerate}
  \item[(b)] Define a solution concept, state the assumptions it relies on, and its possible limitations.\\
  E.g. Iterative Elimination of Strictly Dominated Strategies (IESDS) requires common knowledge of rationality but is not always sufficient to find the Nash Equilibrium.
  \item[(c)] Complex situations can be analyzed in an unambiguous way through modelling them as games and applying logic.
\end{itemize}
\vfill\null
\end{multicols}
\end{frame}


\section{PS1, Ex. 2: IESDS}

\begin{frame}{PS1: Exercise 2}
\begin{multicols}{2}
Solve this game by Iterative Elimination of Strictly Dominated Strategies (IESDS):
\begin{table}
  \begin{tabular}{c|c|c|c}
          & $t_1$ & $t_2$ & $t_3$ \\
    \midrule
    $s_1$ & 5, 0  & 3, 3  & 1, 1 \\
    \midrule
    $s_2$ & 3, 4  & 2, 2  & 3, 1 \\
    \midrule
    $s_3$ & 2, 2  & 1, 1  & 0, 5
  \end{tabular}
\end{table}
\vfill\null
\columnbreak
\textit{Take 5 min. to solve it on your own or with your neighbor(s)}
\vfill\null
\end{multicols}
\end{frame}

\begin{frame}{PS1: Exercise 2}
\begin{multicols}{2}
Solve this game by Iterative Elimination of Strictly Dominated Strategies (IESDS):
\begin{table}
  \begin{tabular}{c|c|c|c}
          & $t_1$ & $t_2$ & $t_3$ \\
    \midrule
    \textcolor{blue}{$s_1$} & \textcolor{blue}{5}, 0  & \textcolor{blue}{3}, 3  & \textcolor{blue}{1}, 1 \\
    \midrule
    \textcolor{blue}{$s_2$} & \textcolor{blue}{3}, 4  & \textcolor{blue}{2}, 2  & \textcolor{blue}{3}, 1 \\
    \midrule
    \textcolor{red}{$s_3$} & \textcolor{red}{2}, 2  & \textcolor{red}{1}, 1  & \textcolor{red}{0}, 5
  \end{tabular}
\end{table}
Player 1: $s_3$ is strictly dominated by $s_1$ as well as $s_2$ and can be eliminated, giving us the reduced form game:
\begin{table}
  \begin{tabular}{c|c|c|c}
          & $t_1$ & $t_2$ & $t_3$ \\
    \midrule
    $s_1$ & 5, 0  & 3, 3  & 1, 1 \\
    \midrule
    $s_2$ & 3, 4  & 2, 2  & 3, 1
  \end{tabular}
\end{table}
\vfill\null
\columnbreak
\vfill\null
\end{multicols}
\end{frame}

\begin{frame}{PS1: Exercise 2}
\begin{multicols}{2}
Solve this game by Iterative Elimination of Strictly Dominated Strategies (IESDS):
\begin{table}
  \begin{tabular}{c|c|c|c}
          & $t_1$ & $t_2$ & $t_3$ \\
    \midrule
    \textcolor{blue}{$s_1$} & \textcolor{blue}{5}, 0  & \textcolor{blue}{3}, 3  & \textcolor{blue}{1}, 1 \\
    \midrule
    \textcolor{blue}{$s_2$} & \textcolor{blue}{3}, 4  & \textcolor{blue}{2}, 2  & \textcolor{blue}{3}, 1 \\
    \midrule
    \sout{\textcolor{red}{$s_3$}} & \sout{\textcolor{red}{2}, 2}  & \sout{\textcolor{red}{1}, 1}  & \sout{\textcolor{red}{0}, 5}
  \end{tabular}
\end{table}
Player 1: $s_3$ is strictly dominated by $s_1$ as well as $s_2$ and can be eliminated, giving us the reduced form game:
\begin{table}
  \begin{tabular}{c|c|c|c}
          & $t_1$ & \textcolor{blue}{$t_2$} & \textcolor{red}{$t_3$} \\
    \midrule
    $s_1$ & 5, 0  & 3, \textcolor{blue}{3}  & 1, \textcolor{red}{1} \\
    \midrule
    $s_2$ & 3, 4  & 2, \textcolor{blue}{2}  & 3, \textcolor{red}{1}
  \end{tabular}
\end{table}
Player 2: $t_3$ is strictly dominated by $t_2$ and can be eliminated.
\vfill\null
\columnbreak
Giving us a new reduced form game:
\begin{table}
  \begin{tabular}{c|c|c}
          & $t_1$ & $t_2$ \\
    \midrule
    $s_1$ & 5, 0  & 3, 3  \\
    \midrule
    $s_2$ & 3, 4  & 2, 2
  \end{tabular}
\end{table}
\vfill\null
\end{multicols}
\end{frame}

\begin{frame}{PS1: Exercise 2}
\begin{multicols}{2}
Solve this game by Iterative Elimination of Strictly Dominated Strategies (IESDS):
\begin{table}
  \begin{tabular}{c|c|c|c}
          & $t_1$ & $t_2$ & $t_3$ \\
    \midrule
    \textcolor{blue}{$s_1$} & \textcolor{blue}{5}, 0  & \textcolor{blue}{3}, 3  & \textcolor{blue}{1}, 1 \\
    \midrule
    \textcolor{blue}{$s_2$} & \textcolor{blue}{3}, 4  & \textcolor{blue}{2}, 2  & \textcolor{blue}{3}, 1 \\
    \midrule
    \sout{\textcolor{red}{$s_3$}} & \sout{\textcolor{red}{2}, 2}  & \sout{\textcolor{red}{1}, 1}  & \sout{\textcolor{red}{0}, 5}
  \end{tabular}
\end{table}
Player 1: $s_3$ is strictly dominated by $s_1$ as well as $s_2$ and can be eliminated, giving us the reduced form game:
\begin{table}
  \begin{tabular}{c|c|c|c}
          & $t_1$ & \textcolor{blue}{$t_2$} & \sout{\textcolor{red}{$t_3$}} \\
    \midrule
    $s_1$ & 5, 0  & 3, \textcolor{blue}{3}  & \sout{1, \textcolor{red}{1}} \\
    \midrule
    $s_2$ & 3, 4  & 2, \textcolor{blue}{2}  & \sout{3, \textcolor{red}{1}}
  \end{tabular}
\end{table}
Player 2: $t_3$ is strictly dominated by $t_2$ and can be eliminated.
\vfill\null
\columnbreak
Giving us a new reduced form game:
\begin{table}
  \begin{tabular}{c|c|c}
          & $t_1$ & $t_2$ \\
    \midrule
    \textcolor{blue}{$s_1$} & \textcolor{blue}{5}, 0  & \textcolor{blue}{3}, 3  \\
    \midrule
    \textcolor{red}{$s_2$} & \textcolor{red}{3}, 4  & \textcolor{red}{2}, 2
  \end{tabular}
\end{table}
Player 1: $s_2$ is strictly dominated by $s1$ and is eliminated. Reduced form game:
\begin{table}
  \begin{tabular}{c|c|c}
          & $t_1$ & $t_2$ \\
    \midrule
    $s_1$ & 5, 0  & 3, 3
  \end{tabular}
\end{table}
\vfill\null
\end{multicols}
\end{frame}

\begin{frame}{PS1: Exercise 2}
\begin{multicols}{2}
  Solve this game by Iterative Elimination of Strictly Dominated Strategies (IESDS):
  \begin{table}
    \begin{tabular}{c|c|c|c}
            & $t_1$ & $t_2$ & $t_3$ \\
      \midrule
      \textcolor{blue}{$s_1$} & \textcolor{blue}{5}, 0  & \textcolor{blue}{3}, 3  & \textcolor{blue}{1}, 1 \\
      \midrule
      \textcolor{blue}{$s_2$} & \textcolor{blue}{3}, 4  & \textcolor{blue}{2}, 2  & \textcolor{blue}{3}, 1 \\
      \midrule
      \sout{\textcolor{red}{$s_3$}} & \sout{\textcolor{red}{2}, 2}  & \sout{\textcolor{red}{1}, 1}  & \sout{\textcolor{red}{0}, 5}
    \end{tabular}
  \end{table}
  Player 1: $s_3$ is strictly dominated by $s_1$ as well as $s_2$ and can be eliminated, giving us the reduced form game:
  \begin{table}
    \begin{tabular}{c|c|c|c}
            & $t_1$ & \textcolor{blue}{$t_2$} & \sout{\textcolor{red}{$t_3$}} \\
      \midrule
      $s_1$ & 5, 0  & 3, \textcolor{blue}{3}  & \sout{1, \textcolor{red}{1}} \\
      \midrule
      $s_2$ & 3, 4  & 2, \textcolor{blue}{2}  & \sout{3, \textcolor{red}{1}}
    \end{tabular}
  \end{table}
  Player 2: $t_3$ is strictly dominated by $t_2$ and can be eliminated.
\vfill\null
\columnbreak
Giving us a new reduced form game:
\begin{table}
  \begin{tabular}{c|c|c}
          & $t_1$ & $t_2$ \\
    \midrule
    \textcolor{blue}{$s_1$} & \textcolor{blue}{5}, 0  & \textcolor{blue}{3}, 3  \\
    \midrule
    \sout{\textcolor{red}{$s_2$}} & \sout{\textcolor{red}{3}, 4}  & \sout{\textcolor{red}{2}, 2}
  \end{tabular}
\end{table}
Player 1: $s_2$ is strictly dominated by $s_1$ and is eliminated. Reduced form game:
\begin{table}
  \begin{tabular}{c|c|c}
          & \textcolor{red}{$t_1$} & \textcolor{blue}{$t_2$} \\
    \midrule
    $s_1$ & 5, \textcolor{red}{0}  & 3, \textcolor{blue}{3}
  \end{tabular}
\end{table}
Player 2: $t_1$ is strictly dominated by $t_2$ and is eliminated. I.e. IESDS provides the unique strategy profile $(s_1,t_2)$, implying that this is also the Nash Equilibrium:
\begin{table}
  \begin{tabular}{c|c}
          & $t_2$ \\
    \midrule
    $s_1$ & 3, 3
  \end{tabular}
\end{table}
\vfill\null
\end{multicols}
\end{frame}


\section{PS1, Ex. 3: The Travelers' Dilemma}

\begin{frame}{PS1: Exercise 3}
\begin{multicols}{2}
  The Travelers' Dilemma:\\\smallskip
  “An airline loses two suitcases belonging to two different travelers. Both suitcases look identical and contain identical items. An airline manager tasked to settle the claims of both travelers explains that the airline is liable for a maximum of \$100 per suitcase, and in order to determine an honest appraised value of the antiques the manager separates both travelers so they can’t confer, and asks them to write down the amount of their value no less than \$0 and no larger than \$100. He also tells them that if both write down the same number, he will treat that number as the true dollar value of both suitcases and reimburse both travelers that amount.
\vfill\null
\columnbreak
However, if one writes down a smaller number than the other, this smaller number will be taken as the true dollar value, and both travelers will receive that amount along with the following: \$1 extra will be paid to the traveler who wrote down the lower value and a \$1 fine imposed on the person who wrote down the higher amount.”
\begin{itemize}
  \item[(a)] Write down the normal form of this game: players, strategy sets, payoffs
\item[(b)] Can you solve this game by IESDS?
\item[(c)] What number do you think each traveler will write down? Why? An informal
discussion of the reasoning will suffice.
\end{itemize}
\textit{Take 10 min. to find the answers on your own or with your neighbor(s)}
\vfill\null
\end{multicols}
\end{frame}

\begin{frame}{PS1: Exercise 3}
\begin{multicols}{2}
  \begin{itemize}
    \item[(a)] Write down the normal form of this game: players, strategy sets, payoffs
  \item[(b)] Can you solve this game by IESDS?
  \item[(c)] What number do you think each traveler will write down? Why? An informal
  discussion of the reasoning will suffice.
  \end{itemize}
\vfill\null
\columnbreak
\begin{itemize}
  \item[(a)] A normal form game consists of:
    \begin{enumerate}
      \item The set of players: Traveler 1 and Traveler 2.
      \item Strategy sets: $S_i = \{0; 0.01; ... ; 99:99; 100\}$ for $i = 1; 2$
      \item Payoffs for player $i\neq j$:
    \end{enumerate}
\end{itemize}
\begin{align*}
  u_i(s_i,s_j)=
  \left\{ \begin{array}{ccl}
  s_i   & \mbox{if} & s_i=s_j \\
  s_i+1 & \mbox{if} & s_i<s_j \\
  s_j-1 & \mbox{if} & s_i>s_j
  \end{array}\right.
\end{align*}
\vfill\null
\end{multicols}
\end{frame}

\begin{frame}{PS1: Exercise 3}
\begin{multicols}{2}
  \begin{itemize}
    \item[(a)] Write down the normal form of this game: players, strategy sets, payoffs
  \item[(b)] Can you solve this game by IESDS?
  \item[(c)] What number do you think each traveler will write down? Why? An informal
  discussion of the reasoning will suffice.
  \end{itemize}
\vfill\null
\columnbreak
\begin{itemize}
  \item[(a)] A normal form game consists of:
    \begin{enumerate}
      \item The set of players: Traveler 1 and Traveler 2.
      \item Strategy sets: $S_i = \{0; 0.01; ... ; 99:99; 100\}$ for $i = 1; 2$
      \item Payoffs for player $i\neq j$:
    \end{enumerate}
\end{itemize}
\begin{align*}
  u_i(s_i,s_j)=
  \left\{ \begin{array}{ccl}
  s_i   & \mbox{if} & s_i=s_j \\
  s_i+1 & \mbox{if} & s_i<s_j \\
  s_j-1 & \mbox{if} & s_i>s_j
  \end{array}\right.
\end{align*}
\begin{itemize}
  \item[(b)] No, as no strategy is always dominated by one other strategy no matter what the other traveler plays.
\end{itemize}
\vfill\null
\end{multicols}
\end{frame}

\begin{frame}{PS1: Exercise 3}
\begin{multicols}{2}
  \begin{itemize}
    \item[(a)] Write down the normal form of this game: players, strategy sets, payoffs
  \item[(b)] Can you solve this game by IESDS?
  \item[(c)] What number do you think each traveler will write down? Why? An informal
  discussion of the reasoning will suffice.
  \end{itemize}
\vfill\null
\columnbreak
\begin{itemize}
  \item[(a)] A normal form game consists of:
    \begin{enumerate}
      \item The set of players: Traveler 1 and Traveler 2.
      \item Strategy sets: $S_i = \{0; 0.01; ... ; 99:99; 100\}$ for $i = 1; 2$
      \item Payoffs for player $i\neq j$:
    \end{enumerate}
\end{itemize}
\begin{align*}
  u_i(s_i,s_j)=
  \left\{ \begin{array}{ccl}
  s_i   & \mbox{if} & s_i=s_j \\
  s_i+1 & \mbox{if} & s_i<s_j \\
  s_j-1 & \mbox{if} & s_i>s_j
  \end{array}\right.
\end{align*}
\begin{itemize}
  \item[(b)] No, as no strategy is always dominated by one other strategy no matter what the other traveler plays.
  \item[(c)] Given common knowledge of rationality each traveler will avoid getting "underbid" by the other, i.e. the Nash Equilibrium is $s_i,s_j=(0,0)$ as there is no incentive to deviate.
\end{itemize}
\vfill\null
\end{multicols}
\end{frame}


\section{PS1, Ex. 4: IESDS}

\begin{frame}{PS1: Exercise 4}
Solve these games by iterative elimination of strictly dominated strategies:
\begin{multicols}{2}
\begin{table}
  \begin{tabular}{c|c|c|c}
          & $t_1$ & $t_2$ & $t_3$ \\
    \midrule
    $s_1$ & 5, 0  & 2, 3  & 1, 1 \\
    \midrule
    $s_2$ & 2, 4  & 2, 2  & 3, 1 \\
    \midrule
    $s_3$ & 2, 2  & 1, 1  & 0, 5
  \end{tabular}
\end{table}
\vfill\null
\columnbreak
\begin{table}
  \begin{tabular}{c|c|c|c}
          & $t_1$ & $t_2$ & $t_3$ \\
    \midrule
    $s_1$ & 5, 0  & 2, 3  & 1, 1 \\
    \midrule
    $s_2$ & 2, 4  & 2, 2  & 3, 1 \\
    \midrule
    $s_3$ & 2, 2  & 1, 1  & 1, 5
  \end{tabular}
\end{table}
\vfill\null
\end{multicols}
\textit{Take 5 min. to solve them on your own or with your neighbor(s)}
\end{frame}

\begin{frame}{PS1: Exercise 4}
Solve these games by iterative elimination of strictly dominated strategies:
\begin{multicols}{2}
\begin{table}
  \begin{tabular}{c|c|c|c}
          & $t_1$ & $t_2$ & $t_3$ \\
    \midrule
    $s_1$ & 5, 0  & 2, 3  & 1, 1 \\
    \midrule
    $s_2$ & 2, 4  & 2, 2  & 3, 1 \\
    \midrule
    \sout{$s_3$} & \sout{2, 2}  & \sout{1, 1}  & \sout{0, 5}
  \end{tabular}
\end{table}
\begin{enumerate}
  \item Player 1: $s_3$ is strictly dominated by $s_1$ and is eliminated
  \item Player 2: After eliminating $s_3$, $t_3$ is strictly dominated by $t_2$ and is eliminated, giving us the reduced form game:
\end{enumerate}
\begin{table}
  \begin{tabular}{c|c|c}
          & $t_1$ & $t_2$ \\
    \midrule
    $s_1$ & 5, 0  & 2, 3  \\
    \midrule
    $s_2$ & 2, 4  & 2, 2  \\
  \end{tabular}
\end{table}
\vfill\null
\columnbreak
\begin{table}
  \begin{tabular}{c|c|c|c}
          & $t_1$ & $t_2$ & $t_3$ \\
    \midrule
    $s_1$ & 5, 0  & 2, 3  & 1, 1 \\
    \midrule
    $s_2$ & 2, 4  & 2, 2  & 3, 1 \\
    \midrule
    $s_3$ & 2, 2  & 1, 1  & 1, 5
  \end{tabular}
\end{table}
\vfill\null
\end{multicols}
\end{frame}


\section{PS1, Ex. 5: The higher number wins}

\begin{frame}{PS1: Exercise 5}
\begin{multicols}{2}
  Mikael and Jonas are playing a game instead of working. The game has the following
  rules: Both secretly pick a (natural) number between 1 and 5. Then they reveal the
  numbers to each other. If both have picked the same number, nobody gets anything. If
  Jonas’s number is higher than Mikael’s number, Mikael has to pay Jonas 1 kr. If Mikael’s
  number is higher than Jonas’s, Jonas has to pay 10 kr. to Mikael
\vfill\null
\columnbreak
\begin{enumerate}
  \item[(a)] Does this game seem fair to you?
  \item[(b)] Write the game in bimatrix form.
  \item[(c)] Are there any strictly dominated strategies? Solve the game by iterated elimination of strictly dominated strategies.
  \item[(d)] What is the outcome of the game if both Mikael and Jonas are rational, know that the other is rational, know that the other knows that they are rational etc.?
\end{enumerate}
\vfill\null
\end{multicols}
\textit{Take 10 min. to answer the questions on your own or with your neighbor(s)}
\end{frame}

\begin{frame}{PS1: Exercise 5}
\begin{multicols}{2}
  \begin{enumerate}
    \item[(a)] Does this game seem fair to you?
    \item[(b)] Write the game in bimatrix form.
    \item[(c)] Are there any strictly dominated strategies? Solve the game by iterated elimination of strictly dominated strategies.
    \item[(d)] What is the outcome of the game if both Mikael and Jonas are rational, know that the other is rational, know that the other knows that they are rational etc.?
  \end{enumerate}
\vfill\null
\columnbreak
\vfill\null
\begin{table}
  \footnotesize
  \begin{tabular}{c|c|c|c|c|c}
        & "1"     & "2"     & "3"     & "4"     & "5"     \\
    \midrule
    "1" & 0, 0    & -1, 1   & -1, 1   & -1, 1   & -1, 1   \\
    \midrule
    "2" & 10, -10 & 0, 0    & -1, 1   & -1, 1   & -1, 1   \\
    \midrule
    "3" & 10, -10 & 10, -10 & 0, 0    & -1, 1   & -1, 1   \\
    \midrule
    "4" & 10, -10 & 10, -10 & 10, -10 & 0, 0    & -1, 1   \\
    \midrule
    "5" & 10, -10 & 10, -10 & 10, -10 & 10, -10 & 0, 0    \\
  \end{tabular}
\end{table}
\end{multicols}
\end{frame}


\section{PS1, Ex. 6: Three player game}

\begin{frame}{PS1: Exercise 6}
  We can also write games with more than two players. Consider the game below where
  player 1 chooses the bi-matrix (A or B), player 2 chooses the row (C or D), and player 3
  chooses the column (E or F). In each cell, the first number gives the payoff of Player 1,
  the second number the payoff of Player 2, and the third number the payoff of Player 3.
\begin{multicols}{2}
\begin{table}
  \begin{tabular}{c|c|c}
      & E       & F       \\
    \midrule
    C & 0, 2, 2 & 2, 1, 1 \\
    \midrule
    D & 0, 1, 1 & 3, 0, 0
  \end{tabular}
  \center A
\end{table}
\vfill\null
\columnbreak
\begin{table}
  \begin{tabular}{c|c|c}
      & E       & F       \\
    \midrule
    C & 1, 0, 1 & 3, 1, 2 \\
    \midrule
    D & 1, 1, 0 & 5, 2, 1
  \end{tabular}
  \center B
\end{table}
\vfill\null
\end{multicols}
Find the pure strategy profiles that survive iterated elimination of strictly dominated strategies. \textit{(10 min.)}
\end{frame}

\begin{frame}{PS1: Exercise 6}
\begin{multicols}{2}
\begin{table}
  \begin{tabular}{c|c|c}
      & E       & F       \\
    \midrule
    C & \textcolor{red}{0}, 2, 2 & \textcolor{red}{2}, 1, 1 \\
    \midrule
    D & \textcolor{red}{0}, 1, 1 & \textcolor{red}{3}, 0, 0
  \end{tabular}
  \center \textcolor{red}{A}
\end{table}
\columnbreak
\begin{table}
  \begin{tabular}{c|c|c}
      & E       & F       \\
    \midrule
    C & \textcolor{blue}{1}, 0, 1 & \textcolor{blue}{3}, 1, 2 \\
    \midrule
    D & \textcolor{blue}{1}, 1, 0 & \textcolor{blue}{5}, 2, 1
  \end{tabular}
  \center \textcolor{blue}{B}
\end{table}
\end{multicols}
\nth{1} step: Player 1: A is strictly dominated by B, thus, matrix A can be eliminated:
\begin{multicols}{2}
\vfill\null
\columnbreak
\begin{table}
  \begin{tabular}{c|c|c}
      & E       & F       \\
    \midrule
    C & 1, 0, 1 & 3, 1, 2 \\
    \midrule
    D & 1, 1, 0 & 5, 2, 1
  \end{tabular}
  \center B
\end{table}
\vfill\null
\end{multicols}
\end{frame}

\begin{frame}{PS1: Exercise 6}
\begin{multicols}{2}
\begin{table}
  \begin{tabular}{c|c|c}
      & E       & F       \\
    \midrule
    C & \textcolor{red}{0}, 2, 2 & \textcolor{red}{2}, 1, 1 \\
    \midrule
    D & \textcolor{red}{0}, 1, 1 & \textcolor{red}{3}, 0, 0
  \end{tabular}
  \center \textcolor{red}{A}
\end{table}
\columnbreak
\begin{table}
  \begin{tabular}{c|c|c}
      & E       & F       \\
    \midrule
    C & \textcolor{blue}{1}, 0, 1 & \textcolor{blue}{3}, 1, 2 \\
    \midrule
    D & \textcolor{blue}{1}, 1, 0 & \textcolor{blue}{5}, 2, 1
  \end{tabular}
  \center \textcolor{blue}{B}
\end{table}
\end{multicols}
\nth{1} step: Player 1: A is strictly dominated by B, thus, matrix A can be eliminated:
\begin{multicols}{2}
\vfill\null
\columnbreak
\begin{table}
  \begin{tabular}{c|c|c}
      & E       & F       \\
    \midrule
    \textcolor{red}{C} & 1, \textcolor{red}{0}, 1 & 3, \textcolor{red}{1}, 2 \\
    \midrule
    \textcolor{blue}{D} & 1, \textcolor{blue}{1}, 0 & 5, \textcolor{blue}{2}, 1
  \end{tabular}
  \center B
\end{table}
\end{multicols}
\nth{2} step: Player 2: After matrix A is eliminated, C is strictly dominated by D and we eliminate C.
\end{frame}

\begin{frame}{PS1: Exercise 6}
\begin{multicols}{2}
\begin{table}
  \begin{tabular}{c|c|c}
      & E       & F       \\
    \midrule
    C & \textcolor{red}{0}, 2, 2 & \textcolor{red}{2}, 1, 1 \\
    \midrule
    D & \textcolor{red}{0}, 1, 1 & \textcolor{red}{3}, 0, 0
  \end{tabular}
  \center \textcolor{red}{A}
\end{table}
\columnbreak
\begin{table}
  \begin{tabular}{c|c|c}
      & E       & F       \\
    \midrule
    C & \textcolor{blue}{1}, 0, 1 & \textcolor{blue}{3}, 1, 2 \\
    \midrule
    D & \textcolor{blue}{1}, 1, 0 & \textcolor{blue}{5}, 2, 1
  \end{tabular}
  \center \textcolor{blue}{B}
\end{table}
\end{multicols}
\nth{1} step: Player 1: A is strictly dominated by B, thus, matrix A can be eliminated:
\begin{multicols}{2}
\vfill\null
\columnbreak
\begin{table}
  \begin{tabular}{c|c|c}
      & E       & F       \\
    \midrule
    \sout{\textcolor{red}{C}} & \sout{1, \textcolor{red}{0}, 1} & \sout{3, \textcolor{red}{1}, 2} \\
    \midrule
    \textcolor{blue}{D} & 1, \textcolor{blue}{1}, 0 & 5, \textcolor{blue}{2}, 1
  \end{tabular}
  \center B
\end{table}
\end{multicols}
\nth{2} step: Player 2: After matrix A is eliminated, C is strictly dominated by D and we eliminate C.
\\\medskip
\nth{3} step: Player 3: After matrix A and row C is eliminated, E (payoff = 0) is strictly dominated by F (payoff = 1) and we eliminate E.
\\\medskip
The unique pure strategy profile that survives IESDS is $(B;D;F)$.
\end{frame}


\section{Preparation for exercise classes}

\begin{frame}{Preparation for all exercise classes}
\begin{multicols}{2}
To get through all problem sets you need to show up prepared:
\begin{itemize}
  \item \textit{\textbf{Ideally:}} Read in the curriculum and participate in the lecture. Print the problem set and try to solve it.
  \item \textit{\textbf{Bare minimum:}} Read through the lecture slides and the problem set.
\end{itemize}
%\vfill\null
\columnbreak
\begin{itemize}
  \item "In all problem sets, there will be two types of exercises, A and B. When you show up for class, you are expected to \textit{\textbf{have done}} all A-exercises and \textit{\textbf{have read and understood}} all B-exercises. A-exercises will not be solved on the whiteboard. Instead, you will get the final answer, and have the opportunity to discuss the solution with your TA."
\end{itemize}
\end{multicols}
\end{frame}


% \begin{frame}%{References}
%   \printbibliography
% \end{frame}

\end{document}
