\documentclass[8pt,apectratio=169]{beamer}

\usetheme[progressbar=frametitle]{metropolis}
\usepackage{appendixnumberbeamer}
\usepackage[style=authoryear, backend=bibtex8, natbib=true, maxcitenames=2]{biblatex}

\usepackage[utf8]{inputenc} % utf8x  defines more symbols, but may cause compatible problems
\usepackage{lmodern,textcomp} % Latin Modern fonts, contains €

\usepackage{graphicx}
\usepackage{import}

\usepackage{booktabs}
\usepackage[scale=2]{ccicons}

\usepackage{pgfplots}
\usepgfplotslibrary{dateplot}

\usepackage{xspace}
\newcommand{\themename}{\textbf{\textsc{metropolis}}\xspace}

% Math
\usepackage{amsmath}
\usepackage{bm} % bold symbol in math mode
\counterwithin*{equation}{section} % reset the equation number whenever section is stepped

% Optional packages
\usepackage{xcolor}
\usepackage{multicol}
\usepackage{multirow,array}
\usepackage{subcaption} % for subfigure and subtable
\usepackage{hyperref}
\usepackage{epigraph}
\usepackage[super,negative]{nth} % allows writing 1st, 2nd, 3rd with superscript
\usepackage{ulem} % use the "sout" tag to "strikethrough" text
\usepackage{cancel} % https://tex.stackexchange.com/questions/75525/how-to-write-crossed-out-math-in-latex
\usepackage{tcolorbox}

% Select what to do with command \comment:
  % \newcommand{\comment}[1]{}  %comments not shown
  % \newcommand{\comment}[1]{\par {\bfseries \color{blue} #1 \par}} %comments shown
% Select what to do with todonotes: i.e. \todo{}, \todo[inline]{}
  % \usepackage[disable]{todonotes} % notes not shown
  % \usepackage[draft]{todonotes}   % notes shown

%\numberwithin{equation}{section}

%\addbibresource{references}

\titlegraphic{\hfill \includegraphics[width=0.15 \textwidth]{figures/logo}}
\title{Microeconomics III: Problem Set 8\footnote{Slides created for exercise class 3 and 4, with reservation for possible errors.\\}}
\author{Thor Donsby Noe (\href{mailto:thor.noe@econ.ku.dk}{thor.noe@econ.ku.dk})
        \& Christopher Borberg (\href{mailto:christopher.borberg@econ.ku.dk}{christopher.borberg@econ.ku.dk})
        }
\date{November 13 2019} % \today
\institute{\normalsize Department of Economics, University of Copenhagen}

    % \definecolor{BlueTOL}{HTML}{222255}
    \definecolor{BrownTOL}{HTML}{666633}
    \definecolor{GreenTOL}{HTML}{225522}
    % \setbeamercolor{normal text}{fg=BlueTOL,bg=white}
    \setbeamercolor{alerted text}{fg=BrownTOL}
    \setbeamercolor{example text}{fg=GreenTOL}
    \setbeamercolor{background canvas}{bg=white}

    \setbeamercolor{block title alerted}{use=alerted text,
        fg=alerted text.fg,
        bg=alerted text.bg!80!alerted text.fg}
    \setbeamercolor{block body alerted}{use={block title alerted, alerted text},
        fg=alerted text.fg,
        bg=block title alerted.bg!50!alerted text.bg}
    \setbeamercolor{block title example}{use=example text,
        fg=example text.fg,
        bg=example text.bg!80!example text.fg}
    \setbeamercolor{block body example}{use={block title example, example text},
        fg=example text.fg,
        bg=block title example.bg!50!example text.bg}


\begin{document}
\maketitle


% ------------------------------------------------------------------------------
% ------ FRAME -----------------------------------------------------------------
% ------------------------------------------------------------------------------
\begin{frame}{Outline}
\tableofcontents
\end{frame}


\section{Motivation}

\begin{frame}{Motivation}
\begin{multicols}{2}
From the course description:
\begin{itemize}
  \item[1.] This course furthers the introduction of game theory and its applications in economic models.
  \item[2.] The student who successfully completes the course will learn the basics of game theory and will be enabled to work further with advanced game theory.
\end{itemize}
\columnbreak
\begin{itemize}
  \item[3.] The student will also learn how economic problems involving strategic situations can be modeled using game theory, as well as how these models are solved.
  \item[4.] The course intention is that the student becomes able to work with modern economic theory, for instance within the areas of industrial organization, macroeconomics, international economics, labor economics, public economics, political economics and financial economics.
\end{itemize}
\end{multicols}
\end{frame}


\section{Overview}

\begin{frame}{Overview of the course}
\begin{multicols}{2}
Course content
\begin{enumerate}
  \item Static games with complete information,
  \item Static games with incomplete information,
  \item Dynamic games with complete information,
  \item Dynamic games with incomplete information,
\end{enumerate}
%\vfill\null
\columnbreak
Form
\begin{itemize}
  \item 13 lectures
  \item 13 sessions in exercise classes w. 12 problem sets
  \item 3 assignments
\end{itemize}
Check out the course description:
\\
\href{https://kurser.ku.dk/course/aØka08005u}{https://kurser.ku.dk/course/aØka08005u}
%\includegraphics[width=0.5 \textwidth]{graphics/solar_panels}
\end{multicols}
\end{frame}


\begin{frame}{Learning outcome}
\begin{multicols}{2}
  \textbf{Knowledge:}
  \begin{enumerate}
    \item Formally state the definition of a game and explain the key differences between games of different types.
    \item In detail account for the equilibrium (solution) concepts that are relevant for these games (Nash Equilibrium, Subgame Perfect Nash Equilibrium, Bayes-Nash Equilibrium, Perfect Bayesian Equilibrium).
    \item Identify a number of special games and particular issues associated with them, such as repeated games (including infinitely repeated games), auctions and signaling games.
  \end{enumerate}
%\vfill\null
\columnbreak
\textbf{Skills:}
\begin{enumerate}
  \item Explicitly solve for the equilibria of these games.
  \item Explain the relevant steps in the reasoning of the solution.
  \item Interpret the outcomes of the analysis.
  \item Apply equilibrium refinements and discuss the solution concepts
\end{enumerate}
\textbf{Competencies:}
\begin{enumerate}
  \item Analyze strategic situations by modeling them as formal games.
  \item Set up, prove, analyze and apply the theories and methods used in the course in an independent manner.
  \item Evaluate and discuss the crucial assumptions underlying the theory.
\end{enumerate}
\end{multicols}
\end{frame}


\begin{frame}{Exam}
\begin{multicols}{2}
Form and content
\begin{itemize}
  \item Two hours without aids on Peter Bangs Vej.
  \item Cook-book approach as well as reflection
\end{itemize}
%\vfill\null
\columnbreak
Example from solution guide for the exam Autumn 2018:
\begin{itemize}
  \item \textit{Missing due to the KU site being down}.
\end{itemize}
%\includegraphics[width=0.5 \textwidth]{graphics/solar_panels}
\end{multicols}
\end{frame}

\section{Expectations}

\begin{frame}{Expectations to your preparation}
\begin{multicols}{2}
\underline{Preparation for all exercise classes}:
\begin{itemize}
  \item Read in the curriculum and participate in the lecture:
  \item Bare minimum: Read through the lecture slides.
  \item \textit{In all problem sets, there will be two types of exercises, A and B. When you show up for class, you are expected to have done all A-exercises and have read and understood all B-exercises. A-exercises will not be solved on the whiteboard. Instead, you will get the final answer, and have the opportunity to discuss the solution with your TA.}
\end{itemize}
%\vfill\null
\columnbreak
\underline{Your own expectations to participation}
\\ \medskip
\textit{2 minute session where you discuss with your neighbor:}
\\ \medskip
\textbf{\textit{How do I prefer to learn?}}
\\ \medskip
Consider what you can learn from
\begin{enumerate}
  \item Trying to solve the \textit{A} exercises the before ex. class.
  \item Reflecting over approach to the \textit{B} exercises before the ex. class.
  \item Discussing a question with your neighbor/in plenum before the teaching assistant gives the answer?
\end{enumerate}
%\includegraphics[width=0.5 \textwidth]{graphics/solar_panels}
\end{multicols}
\end{frame}


\section{Problem Set 1}


% \begin{frame}%{References}
%   \printbibliography
% \end{frame}

\end{document}
