\documentclass[8pt,apectratio=169]{beamer}

\usetheme[progressbar=frametitle]{metropolis}
\usepackage{appendixnumberbeamer}
\usepackage[style=authoryear, backend=bibtex8, natbib=true, maxcitenames=2]{biblatex}

\usepackage[utf8]{inputenc} % utf8x  defines more symbols, but may cause compatible problems
\usepackage{lmodern,textcomp} % Latin Modern fonts, contains €

\usepackage{graphicx}
\usepackage{import}

\usepackage{booktabs}
\usepackage[scale=2]{ccicons}

\usepackage{pgfplots}
\usepgfplotslibrary{dateplot}

\usepackage{xspace}
\newcommand{\themename}{\textbf{\textsc{metropolis}}\xspace}

% Math
\usepackage{amsmath}
\usepackage{bm} % bold symbol in math mode
\counterwithin*{equation}{section} % reset the equation number whenever section is stepped

% Optional packages
\usepackage{xcolor}
\usepackage{multicol}
\usepackage{multirow,array}
\usepackage{subcaption} % for subfigure and subtable
\usepackage{hyperref}
\usepackage{epigraph}
\usepackage[super,negative]{nth} % allows writing 1st, 2nd, 3rd with superscript
\usepackage{ulem} % use the "sout" tag to "strikethrough" text
\usepackage{cancel} % https://tex.stackexchange.com/questions/75525/how-to-write-crossed-out-math-in-latex
\usepackage{tcolorbox}

% Select what to do with command \comment:
  % \newcommand{\comment}[1]{}  %comments not shown
  % \newcommand{\comment}[1]{\par {\bfseries \color{blue} #1 \par}} %comments shown
% Select what to do with todonotes: i.e. \todo{}, \todo[inline]{}
  % \usepackage[disable]{todonotes} % notes not shown
  % \usepackage[draft]{todonotes}   % notes shown

%\numberwithin{equation}{section}

%\addbibresource{references}

\titlegraphic{\hfill \includegraphics[width=0.15 \textwidth]{figures/logo}}
\title{Microeconomics III: Problem Set 8\footnote{Slides created for exercise class 3 and 4, with reservation for possible errors.\\}}
\author{Thor Donsby Noe (\href{mailto:thor.noe@econ.ku.dk}{thor.noe@econ.ku.dk})
        \& Christopher Borberg (\href{mailto:christopher.borberg@econ.ku.dk}{christopher.borberg@econ.ku.dk})
        }
\date{November 13 2019} % \today
\institute{\normalsize Department of Economics, University of Copenhagen}

    % \definecolor{BlueTOL}{HTML}{222255}
    \definecolor{BrownTOL}{HTML}{666633}
    \definecolor{GreenTOL}{HTML}{225522}
    % \setbeamercolor{normal text}{fg=BlueTOL,bg=white}
    \setbeamercolor{alerted text}{fg=BrownTOL}
    \setbeamercolor{example text}{fg=GreenTOL}
    \setbeamercolor{background canvas}{bg=white}

    \setbeamercolor{block title alerted}{use=alerted text,
        fg=alerted text.fg,
        bg=alerted text.bg!80!alerted text.fg}
    \setbeamercolor{block body alerted}{use={block title alerted, alerted text},
        fg=alerted text.fg,
        bg=block title alerted.bg!50!alerted text.bg}
    \setbeamercolor{block title example}{use=example text,
        fg=example text.fg,
        bg=example text.bg!80!example text.fg}
    \setbeamercolor{block body example}{use={block title example, example text},
        fg=example text.fg,
        bg=block title example.bg!50!example text.bg}

\begin{document}
\maketitle

% ------------------------------------------------------------------------------
% ------ FRAME -----------------------------------------------------------------
% ------------------------------------------------------------------------------
\begin{frame}{Outline}
\tableofcontents
\end{frame}


\section{Kahoot!}

\begin{frame}{Kahoot: A exercises}
  In small groups:
  \begin{itemize}
    \item Get prepared to answer the A exercises (10 min).
  \end{itemize}
  \includegraphics[width=\textwidth]{figures/Kahoot}
\end{frame}


\section{PS3, Ex. 1 (A): Dominance and best response}

\begin{frame}{PS3, Ex. 1 (A): Dominance and best response}
  \begin{enumerate}
    \item (A) Show that for each of the following two games, the only Nash equilibrium is in pure strategies. Describe the intuition for this result. What do these two games have in common?
  \end{enumerate}
  \begin{multicols}{2}
    \begin{table}
      \begin{tabular}{cc|c|c|}
        & \multicolumn{1}{c}{} & \multicolumn{2}{c}{\color{blue}Player 2}\\
        \parbox[t]{1mm}{\multirow{3}{*}{\rotatebox[origin=r]{90}{\color{red}Player 1}}}
        & \multicolumn{1}{c}{} & \multicolumn{1}{c}{L}  & \multicolumn{1}{c}{\color{blue}R} \\\cline{3-4}
        & U & 5, 5 & 1, \textcolor{blue}{6}  \\\cline{3-4}
        & \color{red} & \textcolor{red}{6}, 1 & \textcolor{red}{2}, \textcolor{blue}{2} \\\cline{3-4}
      \end{tabular}
    \end{table}
    $(D,R)$ is a unique Pure Strategy Nash Equilibrium (PSNE). The game is a Prisoner's Dilemma as it fulfills:
    \begin{align*}
      T>R>P>S\Leftrightarrow6>5>2>1
    \end{align*}
    i.e. the \textbf{T}emptation to deviate (6) is greater than the \textbf{R}eward for cooperating on the socially optimal outcome (5) and the \textbf{P}unishment payoff (2) is greater than the "\textbf{S}ucker's" payoff (1).
  \vfill\null \columnbreak
  \vfill\null
  \end{multicols}
\end{frame}
\begin{frame}{PS3, Ex. 1 (A): Dominance and best response}
  \begin{enumerate}
    \item (A) Show that for each of the following two games, the only Nash equilibrium is in pure strategies. Describe the intuition for this result. What do these two games have in common?
  \end{enumerate}
  \begin{multicols}{2}
    \begin{table}
      \begin{tabular}{cc|c|c|}
        & \multicolumn{1}{c}{} & \multicolumn{2}{c}{\color{blue}Player 2}\\
        \parbox[t]{1mm}{\multirow{3}{*}{\rotatebox[origin=r]{90}{\color{red}Player 1}}}
        & \multicolumn{1}{c}{} & \multicolumn{1}{c}{L}  & \multicolumn{1}{c}{\color{blue}R} \\\cline{3-4}
        & U & 5, 5 & 1, \textcolor{blue}{6}  \\\cline{3-4}
        & \color{red}D & \textcolor{red}{6}, 1 & \textcolor{red}{2}, \textcolor{blue}{2} \\\cline{3-4}
      \end{tabular}
    \end{table}
    $(D,R)$ is a unique Pure Strategy Nash Equilibrium (PSNE). The game is a Prisoner's Dilemma as it fulfills:
    \begin{align*}
      T>R>P>S\Leftrightarrow6>5>2>1
    \end{align*}
    i.e. the \textbf{T}emptation to deviate (6) is greater than the \textbf{R}eward for cooperating on the socially optimal outcome (5) and the \textbf{P}unishment payoff (2) is greater than the "\textbf{S}ucker's" payoff (1).
  \vfill\null \columnbreak
    \begin{table}
      \begin{tabular}{cc|c|c|c|}
        & \multicolumn{1}{c}{} & \multicolumn{3}{c}{\color{blue}Player 2}\\
        \parbox[t]{1mm}{\multirow{3}{*}{\rotatebox[origin=r]{90}{\color{red}Player 1}}}
        & \multicolumn{1}{c}{} & \multicolumn{1}{c}{L} & \multicolumn{1}{c}{C} & \multicolumn{1}{c}{R} \\\cline{3-5}
        & U & \textcolor{red}{1}, 0 & \textcolor{red}{1}, \textcolor{blue}{2} & 0, 1 \\\cline{3-5}
        & D & 0, \textcolor{blue}{3} & 0, 1 & \textcolor{red}{2}, 0 \\\cline{3-5}
      \end{tabular}
    \end{table}
    $(U,C)$ is a unique Pure Strategy Nash Equilibrium (PSNE) as no other combination of (mixed or pure) strategies gives as high payoffs.\\\medskip
    Iterated Elimination of Strictly Dominated Strategies (IESDS) leads to the same outcome as the best responses (eliminate $R$ then $D$ and lastly $L$).
  \vfill\null
  \end{multicols}
  \textbf{As both games can be solved by IESDS they both have a unique PSNE.}
\end{frame}


\section{PS3, Ex. 2 (A): Equilibrium selection}

\begin{frame}{PS3, Ex. 2 (A): Equilibrium selection}
  \begin{itemize}
    \item[2.] (A) Solve for all pure strategy Nash equilibria. Which equilibrium do you find most reasonable?
  \end{itemize}
  \begin{multicols}{2}
    \begin{table}
      \begin{tabular}{cc|c|c|c|}
        & \multicolumn{1}{c}{} & \multicolumn{3}{c}{\color{blue}Player 2}\\
        & \multicolumn{1}{c}{} & \multicolumn{1}{c}{a} & \multicolumn{1}{c}{b} & \multicolumn{1}{c}{c} \\\cline{3-5}
        \parbox[t]{1mm}{\multirow{3}{*}{\rotatebox[origin=c]{90}{\color{red}Player 1}}}
        & A & \textcolor{red}{2}, \textcolor{blue}{2} & \textcolor{red}{0}, 0 & -1, \textcolor{blue}{2} \\\cline{3-5}
        & B & 0, \textcolor{blue}{0} & \textcolor{red}{0}, \textcolor{blue}{0} & 0, \textcolor{blue}{0} \\\cline{3-5}
        & C & \textcolor{red}{2}, -1 & \textcolor{red}{0}, 0 & \textcolor{red}{1}, \textcolor{blue}{1} \\\cline{3-5}
      \end{tabular}
    \end{table}
    $PSNE=\{(A,a),(B,b),(C,c)\}$.
  \vfill\null \columnbreak
  \vfill\null
  \end{multicols}
\end{frame}
\begin{frame}{PS3, Ex. 2 (A): Equilibrium selection}
  \begin{itemize}
    \item[2.] (A) Solve for all pure strategy Nash equilibria. Which equilibrium do you find most reasonable?
  \end{itemize}
  \begin{multicols}{2}
    \begin{table}
      \begin{tabular}{cc|c|c|c|}
        & \multicolumn{1}{c}{} & \multicolumn{3}{c}{\color{blue}Player 2}\\
        & \multicolumn{1}{c}{} & \multicolumn{1}{c}{a} & \multicolumn{1}{c}{b} & \multicolumn{1}{c}{c} \\\cline{3-5}
        \parbox[t]{1mm}{\multirow{3}{*}{\rotatebox[origin=c]{90}{\color{red}Player 1}}}
        & A & \textcolor{red}{2}, \textcolor{blue}{2} & \textcolor{red}{0}, 0 & -1, \textcolor{blue}{2} \\\cline{3-5}
        & B & 0, \textcolor{blue}{0} & \textcolor{red}{0}, \textcolor{blue}{0} & 0, \textcolor{blue}{0} \\\cline{3-5}
        & C & \textcolor{red}{2}, -1 & \textcolor{red}{0}, 0 & \textcolor{red}{1}, \textcolor{blue}{1} \\\cline{3-5}
      \end{tabular}
    \end{table}
    $PSNE=\{(A,a),(B,b),(C,c)\}$.
  \vfill\null \columnbreak
    For \textbf{risk neutral} players $(A,a)$ is the most reasonable as it maximizes payoff for both players.\\\medskip
    For \textbf{risk averse} players avoiding $A$ and $a$ eliminates the risk of a negative payoff. $(C,c)$ is more reasonable than $(B,b)$ as the payoffs are positive.
  \vfill\null
  \end{multicols}
\end{frame}


\section{PS3, Ex. 3 (A): NE proof using IEWDS}

\begin{frame}{PS3, Ex. 3 (A): NE proof using IEWDS}
  \begin{itemize}
    \item[3.] (A)  We have seen in the lectures that IESDS never eliminates a Nash Equilibrium. However, we saw in Problem Set 2 that this is not true if we do iterated elimination of weakly dominated strategies (IEWDS.) Go through the proof in the slides from lecture 2 and identify the step that is no longer true if we replace IESDS by IEWDS. That is, explain why the proof is no longer true when we replace ‘strict domination’ by ‘weak domination’.
  \end{itemize}
  \begin{multicols}{2}
  \vfill\null \columnbreak
  \vfill\null
  \end{multicols}
\end{frame}
\begin{frame}{PS3, Ex. 3 (A): NE proof using IEWDS}
    The proof that all NE survive IESDS holds by contradiction. We \underline{\textbf{highlight}} where the contradiction breaks down using IEWDS instead:
  \begin{multicols}{2}
    \begin{itemize}
      \item Let $(s_1^{*},s_2^{*})$ be a NE.
      \item Say we carry out \underline{\textbf{IEWDS}} and $s_1^{*}$ is the first NE strategy to be eliminated (in round $n$ of elimination).
      \item Then there must be a strategy $s_1^{'}\neq s_1^{*}$ that \underline{\textbf{weakly}} dominates $s_1^{*}$, i.e.
    \end{itemize}
      \begin{align*}
        \forall s_2\in S_2^n:\ u_1(s_1^{*},s_2)\underbrace{\bm{\leq}}_\text{\underline{\textbf{Weak}}}u_1(s_1^{'},s_2)
      \end{align*}
      \begin{itemize}
        \item[] \underline{\textbf{and the inequality holds strictly for}} \underline{\textbf{at least one strategy $\bm{s_2^{'}\in S_2^n}$}} where $S_2^n$ is the set of player-2 strategies that have not been eliminated in rounds $1,...,n-1$.
      \end{itemize}
  \vfill\null \columnbreak
  \vfill\null
  \end{multicols}
\end{frame}
\begin{frame}{PS3, Ex. 3 (A): NE proof using IEWDS}
    The proof that all NE survive IESDS holds by contradiction. We \underline{\textbf{highlight}} where the contradiction breaks down using IEWDS instead:
  \begin{multicols}{2}
    \begin{itemize}
      \item Let $(s_1^{*},s_2^{*})$ be a NE.
      \item Say we carry out \underline{\textbf{IEWDS}} and $s_1^{*}$ is the first NE strategy to be eliminated (in round $n$ of elimination).
      \item Then there must be a strategy $s_1^{'}\neq s_1^{*}$ that \underline{\textbf{weakly}} dominates $s_1^{*}$, i.e.
    \end{itemize}
      \begin{align}
        \forall s_2\in S_2^n:\ u_1(s_1^{*},s_2)\underbrace{\bm{\leq}}_\text{\underline{\textbf{Weak}}}u_1(s_1^{'},s_2)\label{proof}
      \end{align}
      \begin{itemize}
        \item[] \underline{\textbf{and the inequality holds strictly for}} \underline{\textbf{at least one strategy $\bm{s_2^{'}\in S_2^n}$}} where $S_2^n$ is the set of player-2 strategies that have not been eliminated in rounds $1,...,n-1$.
      \end{itemize}
  \vfill\null\columnbreak
    \begin{itemize}
      \item Since $s_2^{*}\in S_2^n$, inequality \eqref{proof} also means
      \begin{align*}
        u_1(s_1^{*},s_2^{*})\underbrace{\bm{\leq}}_\text{\underline{\textbf{Weak}}}u_1(s_1^{'},s_2^{*})
      \end{align*}
      \item \sout{But} $(s_1^{*},s_2^{*})$ is a NE, so by definition
      \begin{align*}
        \forall s_1\in S_1:\ u_1(s_1^{*},s_2^{*})\geq u_1(s_1,s_2^{*})
      \end{align*}
      \item \underline{\textbf{No}} contradiction!
    \end{itemize}
    \underline{\textbf{Conclusion}}: for a NE $(s_1^{*},s_2^{*})$ IEWDS can eliminate $s_1^{*}$ if $s_1^{'},s_2^{'}$ exist such that:
    \begin{align*}
      \text{for }s_1^{'}\in S_1^n:\ u_1(s_1^{*},s_2^{*})=u_1(s_1^{'},s_2^{*})
    \end{align*}
    and
    \begin{align*}
      \text{for }s_2^{'}\in S_2^n:\ u_1(s_1^{*},s_2^{'})<u_1(s_1^{'},s_2^{'})
    \end{align*}
  \vfill\null
  \end{multicols}
\end{frame}


\section{PS3, Ex. 4 (A): Mixed strategy price competition}

\begin{frame}{PS3, Ex. 4 (A): Mixed strategy price competition}
    \begin{itemize}
      \item[4.] (A). Consider price competition between two firms when some consumers are informed about prices and others are not. Firms have zero marginal cost and they set price simultaneously; for the sake of this example, assume each price can only take one of the following values: 80, 54, 38. The market consists of two consumers. The uninformed consumer will visit a firm at random (probabilities $\frac{1}{2},\frac{1}{2}$) and buy from it, regardless of the price. The informed consumer will visit the firm with the lowest price and buy from it. If both firms set the same price, assume that the informed consumer picks a firm at random (probabilities $\frac{1}{2},\frac{1}{2}$).
      \item[(a)] Argue that this game can be represented by the following bimatrix.
      \vspace{-4pt}
      \begin{table}
        \begin{tabular}{c|c|c|c|}
          \multicolumn{1}{c}{} & \multicolumn{1}{c}{$p_2$=80} & \multicolumn{1}{c}{$p_2$=54} & \multicolumn{1}{c}{$p_2$=38} \\\cline{2-4}
          $p_1$=80 & 80, 80 & 40, 81 & 40, 57 \\\cline{2-4}
          $p_1$=54 & 81, 40 & 54, 54 & 27, 57 \\\cline{2-4}
          $p_1$=38 & 57, 40 & 57, 27 & 38, 38 \\\cline{2-4}
        \end{tabular}
      \end{table}
      \item[(b)] Show that there is no Nash equilibrium in pure strategies.
      \item[(c)] Confirm that the following strategy profile is a Nash equilibrium: each firm plays price 80 with probability 0.232, price 54 with probability 0.361, and price 38 with probability 0.407.
      \item[(d)] Why do you think the equilibrium is so different from the standard Bertrand pricing game (i.e. where competition drives price down to marginal cost)?
    \end{itemize}
\end{frame}
\begin{frame}{PS3, Ex. 4 (A): Mixed strategy price competition}
    \begin{itemize}
      \item[(a)] The game in normal form and bimatrix:
    \end{itemize}
    Players: \textit{Firm 1, Firm 2}. Strategies: $p_i\in S_i = S = \{80, 54, 38\}$\\\medskip
    Payoffs consist of \textit{payoff from the informed consumer} + \textit{payoff from the uninformed}.
    I.e. payoffs for player $i\neq j$:
    \begin{align*}
      u_i(p_i,p_j)=
      \left\{ \begin{array}{rcl}
      p_i + \frac{1}{2}p_i & \mbox{if} & p_i < p_j \\
      \frac{1}{2}p_i + \frac{1}{2}p_i & \mbox{if} & p_i = p_j \\
      0 + \frac{1}{2}p_i & \mbox{if} & p_i > p_j
      \end{array}\right.
      =
      \left\{ \begin{array}{rcl}
      \frac{3}{2}p_i & \mbox{if} & p_i < p_j \\
                 p_i & \mbox{if} & p_i = p_j \\
      \frac{1}{2}p_i & \mbox{if} & p_i > p_j
      \end{array}\right.
    \end{align*}
  \vfill\null
\end{frame}
\begin{frame}{PS3, Ex. 4 (A): Mixed strategy price competition}
    \begin{itemize}
      \item[(a)] The game in normal form and bimatrix:
    \end{itemize}
    Players: \textit{Firm 1, Firm 2}. Strategies: $p_i\in S_i = S = \{80, 54, 38\}$\\\medskip
    Payoffs consist of \textit{payoff from the informed consumer} + \textit{payoff from the uninformed}.
    I.e. payoffs for player $i\neq j$:
    \begin{align*}
      u_i(p_i,p_j)=
      \left\{ \begin{array}{rcl}
      p_i + \frac{1}{2}p_i & \mbox{if} & p_i < p_j \\
      \frac{1}{2}p_i + \frac{1}{2}p_i & \mbox{if} & p_i = p_j \\
      0 + \frac{1}{2}p_i & \mbox{if} & p_i > p_j
      \end{array}\right.
      =
      \left\{ \begin{array}{rcl}
      \frac{3}{2}p_i & \mbox{if} & p_i < p_j \\
                 p_i & \mbox{if} & p_i = p_j \\
      \frac{1}{2}p_i & \mbox{if} & p_i > p_j
      \end{array}\right.
    \end{align*}
    Which can be represented as:
      \vspace{-6pt}
      \begin{table}
        \begin{tabular}{c|c|c|c|}
          \multicolumn{1}{c}{} & \multicolumn{1}{c}{$p_j$=80} & \multicolumn{1}{c}{$p_j$=54} & \multicolumn{1}{c}{$p_j$=38} \\\cline{2-4}
          $p_i$=80 & 80, - & $\frac{1}{2}$80=40, - & $\frac{1}{2}$80=40, - \\\cline{2-4}
          $p_i$=54 & $\frac{3}{2}$54=81, - & 54, - & $\frac{1}{2}$54=27, - \\\cline{2-4}
          $p_i$=38 & $\frac{3}{2}$80=57, - & $\frac{3}{2}$38=57, - & 38, - \\\cline{2-4}
        \end{tabular}
      \end{table}
  \vfill\null
\end{frame}
\begin{frame}{PS3, Ex. 4 (A): Mixed strategy price competition}
    \begin{itemize}
      \item[(a)] The game in normal form and bimatrix:
    \end{itemize}
    Players: \textit{Firm 1, Firm 2}. Strategies: $p_i\in S_i = S = \{80, 54, 38\}$\\\medskip
    Payoffs consist of \textit{payoff from the informed consumer} + \textit{payoff from the uninformed}.
    I.e. payoffs for player $i\neq j$:
    \begin{align*}
      u_i(p_i,p_j)=
      \left\{ \begin{array}{rcl}
      p_i + \frac{1}{2}p_i & \mbox{if} & p_i < p_j \\
      \frac{1}{2}p_i + \frac{1}{2}p_i & \mbox{if} & p_i = p_j \\
      0 + \frac{1}{2}p_i & \mbox{if} & p_i > p_j
      \end{array}\right.
      =
      \left\{ \begin{array}{rcl}
      \frac{3}{2}p_i & \mbox{if} & p_i < p_j \\
                 p_i & \mbox{if} & p_i = p_j \\
      \frac{1}{2}p_i & \mbox{if} & p_i > p_j
      \end{array}\right.
    \end{align*}
    Which can be represented as:
      \vspace{-6pt}
      \begin{table}
        \begin{tabular}{c|c|c|c|}
          \multicolumn{1}{c}{} & \multicolumn{1}{c}{$p_j$=80} & \multicolumn{1}{c}{$p_j$=54} & \multicolumn{1}{c}{$p_j$=38} \\\cline{2-4}
          $p_i$=80 & 80, - & $\frac{1}{2}$80=40, - & $\frac{1}{2}$80=40, - \\\cline{2-4}
          $p_i$=54 & $\frac{3}{2}$54=81, - & 54, - & $\frac{1}{2}$54=27, - \\\cline{2-4}
          $p_i$=38 & $\frac{3}{2}$80=57, - & $\frac{3}{2}$38=57, - & 38, - \\\cline{2-4}
        \end{tabular}
      \end{table}
    \begin{itemize}
      \item[(b)] Show that there is no Nash equilibrium in pure strategies:
    \end{itemize}
    \vspace{-4pt}
    \begin{table}
      \begin{tabular}{cc|c|c|c|}
        & \multicolumn{1}{c}{} & \multicolumn{3}{c}{\color{blue}Firm 2}\\
        & \multicolumn{1}{c}{} & \multicolumn{1}{c}{$p_2$=80} & \multicolumn{1}{c}{$p_2$=54} & \multicolumn{1}{c}{$p_2$=38} \\\cline{3-5}
        \parbox[t]{1mm}{\multirow{3}{*}{\rotatebox[origin=c]{90}{\color{red}Firm 1}}}
        & $p_1$=80 & 80, 80 & 40, \textcolor{blue}{81} & \textcolor{red}{40}, 57 \\\cline{3-5}
        & $p_1$=54 & \textcolor{red}{81}, 40 & 54, 54 & 27, \textcolor{blue}{57} \\\cline{3-5}
        & $p_1$=38 & 57, \textcolor{blue}{40} & \textcolor{red}{57}, 27 & 38, 38 \\\cline{3-5}
      \end{tabular}
    \end{table}
  \vfill\null
\end{frame}
\begin{frame}{PS3, Ex. 4 (A): Mixed strategy price competition}
    \begin{itemize}
      \item[(c)] Confirm that the following strategy profile is a Nash equilibrium: each firm plays price 80 with probability 0.232, price 54 with probability 0.361, and price 38 with probability 0.407.
    \end{itemize}
    \begin{tabular}{|l|}
      \cline{1-1}
      \textbf{Remember}: In an equilibrium in mixed strategies, a player is indifferent between all \\
      pure strategies that she is choosing with positive probability.\\\cline{1-1}
    \end{tabular}
    \begin{table}
      \begin{tabular}{c|c|c|c|}
        \multicolumn{1}{c}{} & \multicolumn{1}{c}{$p_2$=80} & \multicolumn{1}{c}{$p_2$=54} & \multicolumn{1}{c}{$p_2$=38} \\\cline{2-4}
        $p_1$=80 & 80, 80 & 40, 81 & 40, 57 \\\cline{2-4}
        $p_1$=54 & 81, 40 & 54, 54 & 27, 57 \\\cline{2-4}
        $p_1$=38 & 57, 40 & 57, 27 & 38, 38 \\\cline{2-4}
      \end{tabular}
    \end{table}
  \vfill\null
\end{frame}
\begin{frame}{PS3, Ex. 4 (A): Mixed strategy price competition}
    \begin{itemize}
      \item[(c)] Confirm that the following strategy profile is a Nash equilibrium: each firm plays price 80 with probability 0.232, price 54 with probability 0.361, and price 38 with probability 0.407.
    \end{itemize}
    \begin{tabular}{|l|}
      \cline{1-1}
      \textbf{Remember}: In an equilibrium in mixed strategies, a player is indifferent between all \\
      pure strategies that she is choosing with positive probability.\\\cline{1-1}
    \end{tabular}
    \begin{table}
      \begin{tabular}{c|c|c|c|}
        \multicolumn{1}{c}{} & \multicolumn{1}{c}{$p_2$=80} & \multicolumn{1}{c}{$p_2$=54} & \multicolumn{1}{c}{$p_2$=38} \\\cline{2-4}
        $p_1$=80 & 80, 80 & 40, 81 & 40, 57 \\\cline{2-4}
        $p_1$=54 & 81, 40 & 54, 54 & 27, 57 \\\cline{2-4}
        $p_1$=38 & 57, 40 & 57, 27 & 38, 38 \\\cline{2-4}
      \end{tabular}
    \end{table}
    Check that firm $i$ is indifferent between all pure strategies when the opposing firm's strategy is given by the probability distribution $\widehat{p_j}=(0.232,0.361)$:
    \begin{align*}
      u_i(p_i=80,\widehat{p_j})= 0.232\cdot80 + 0.361\cdot40 + (1-0.232-0.361)\cdot40 = 49.280 \approx 49.3\\
      u_i(p_i=54,\widehat{p_j})= 0.232\cdot81 + 0.361\cdot54 + (1-0.232-0.361)\cdot27 = 49.275 \approx 49.3\\
      u_i(p_i=38,\widehat{p_j})= 0.232\cdot57 + 0.361\cdot57 + (1-0.232-0.361)\cdot38 = 49.267 \approx 49.3
    \end{align*}
    There are rounding errors as the exact mixed strategy profile is $\widehat{p_j}=\left(\frac{193}{833},\frac{8127}{22491}\right)$.
  \vfill\null
\end{frame}
\begin{frame}{PS3, Ex. 4 (A): Mixed strategy price competition}
    \begin{itemize}
      \item[(d)] Why do you think the equilibrium is so different from the standard Bertrand pricing game (i.e. where competition drives price down to marginal cost)?
    \end{itemize}
  \vfill\null
\end{frame}
\begin{frame}{PS3, Ex. 4 (A): Mixed strategy price competition}
    \begin{itemize}
      \item[(d)] Why do you think the equilibrium is so different from the standard Bertrand pricing game (i.e. where competition drives price down to marginal cost)?
    \end{itemize}
    In the standard Bertrand Oligopoly price competition would lead to the perfectly competitive outcome (price = marginal cost), here:
    \begin{align*}
      p_1^{*}=p_2^{*}=c=0
    \end{align*}
    Introduction of an uninformed consumer dampens the effect of price competition as a firm $i$ can expect a revenue of at least $\frac{1}{2}p_i$ no matter what price $p_i$ it sets.\\\medskip
    Price competition could be increased by lowering the probability that an uninformed customer randomly picks the firm, i.e. through:
    \begin{enumerate}
      \item A higher share of informed customers.
      \item More competing firms (moreover, increasing the pure price competition).
    \end{enumerate}
  \vfill\null
\end{frame}


\section{PS3, Ex. 5: Luxembourg as a rogue state}

\begin{frame}{PS3, Ex. 5: Luxembourg as a rogue state}
  \begin{multicols}{2}
    Assume that Luxembourg has turned into a rogue state. It is close to acquiring nuclear weapons, which would threaten the stability in the whole region. The Vatican ($V$) and Denmark ($D$) are preparing an attack on Luxembourg’s nuclear research facilities to stop or slow down its nuclear program. The probability that the attack will be a success is
    \begin{align*}
      p(s_V,s_D)=s_V+s_D-s_vs_D,
    \end{align*}
    where $s_i\in[0,1]$ is the share of its military capacity that country $i\ (i\in\{V,D\})$ uses in the attack. If the attack is successful then each country receives a payoff of 1. The cost of participating in the attack for country $i$ is
    \begin{align*}
      c_i(s_i)=s_i^2
    \end{align*}
    The objective of each country is to maximize its expected payoff from the attack minus the cost.
  \vfill\null\columnbreak
    \begin{itemize}
      \item[(a)] Suppose that the Vatican and Denmark choose the shares of military capacity to use in the attack simultaneously and independently. Find the Nash equilibrium (NE) of this game.
      \item[(b)] Find the social optimum (SO) under the condition that the two countries use the same share of their military capacity. I.e., find the $\bar{s}_V=\bar{s}_D=\bar{s}$ that maximizes aggregate payoff from the attack minus costs. Compare with the equilibrium from question (a) and give an intuitive explanation of your findings.
    \end{itemize}
    \hfill \includegraphics[width=0.20 \textwidth]{figures/nuclear}
  \vfill\null
  \end{multicols}
\end{frame}
\begin{frame}{PS3, Ex. 5: Luxembourg as a rogue state}
  \begin{multicols}{2}
    \begin{itemize}
      \item[(a)] Find the NE in the static game:
    \end{itemize}
    Expected payoff for player $i\neq j$:
    \begin{align*}
      u_i(s_i,s_j)=\underbrace{s_i+s_j-s_is_j}_\text{Probability of success}-\underbrace{s_i^2}_\text{Cost}
    \end{align*}
  \vfill\null\columnbreak
  \vfill\null
  \end{multicols}
\end{frame}
\begin{frame}{PS3, Ex. 5: Luxembourg as a rogue state}
  \begin{multicols}{2}
    \begin{itemize}
      \item[(a)] Find the NE in the static game:
    \end{itemize}
    Expected payoff for player $i\neq j$:
    \begin{align*}
      u_i(s_i,s_j)=\underbrace{s_i+s_j-s_is_j}_\text{Probability of success}-\underbrace{s_i^2}_\text{Cost}
    \end{align*}
    Find the best-response function for $i$:
    \begin{align*}
      FOC:\ \frac{\delta u_i}{\delta s_i}=1+0-s_j-2s_i&=0\\
       s_i&=\frac{1-s_j}{2}
    \end{align*}
  \vfill\null\columnbreak
  \vfill\null
  \end{multicols}
\end{frame}
\begin{frame}{PS3, Ex. 5: Luxembourg as a rogue state}
  \begin{multicols}{2}
    \begin{itemize}
      \item[(a)] Find the NE in the static game:
    \end{itemize}
    Expected payoff for player $i\neq j$:
    \begin{align*}
      u_i(s_i,s_j)=\underbrace{s_i+s_j-s_is_j}_\text{Probability of success}-\underbrace{s_i^2}_\text{Cost}
    \end{align*}
    Find the best-response function for $i$:
    \begin{align*}
      FOC:\ \frac{\delta u_i}{\delta s_i}=1+0-s_j-2s_i&=0\\
       s_i&=\frac{1-s_j}{2}
    \end{align*}
    Taking advantage of symmetry $s_i^{*}=s_j^{*}$:
    \begin{align*}
       s_i^{*}&=\frac{1-s_i^{*}}{2}\\
      2s_i^{*}+s_i^{*}&=1\\
       s_i^{*}&=\frac{1}{3}\equiv s^{NE}
    \end{align*}
    i.e. $NE=\left\{(s_D^{*},s_V^{*})=(\frac{1}{3},\frac{1}{3})\right\}$
  \vfill\null\columnbreak
  \vfill\null
  \end{multicols}
\end{frame}
\begin{frame}{PS3, Ex. 5: Luxembourg as a rogue state}
  \begin{multicols}{2}
    \begin{itemize}
      \item[(a)] Find the NE in the static game:
    \end{itemize}
    Expected payoff for player $i\neq j$:
    \begin{align*}
      u_i(s_i,s_j)=\underbrace{s_i+s_j-s_is_j}_\text{Probability of success}-\underbrace{s_i^2}_\text{Cost}
    \end{align*}
    Find the best-response function for $i$:
    \begin{align*}
      FOC:\ \frac{\delta u_i}{\delta s_i}=1+0-s_j-2s_i&=0\\
       s_i&=\frac{1-s_j}{2}
    \end{align*}
    Taking advantage of symmetry $s_i^{*}=s_j^{*}$:
    \begin{align*}
       s_i^{*}&=\frac{1-s_i^{*}}{2}\\
      2s_i^{*}+s_i^{*}&=1\\
       s_i^{*}&=\frac{1}{3}\equiv s^{NE}
    \end{align*}
    i.e. $NE=\left\{(s_D^{*},s_V^{*})=(\frac{1}{3},\frac{1}{3})\right\}$
  \vfill\null\columnbreak
    \begin{itemize}
      \item[(b)] Find the SO given shares are equal:
    \end{itemize}
  \vfill\null
  \end{multicols}
\end{frame}
\begin{frame}{PS3, Ex. 5: Luxembourg as a rogue state}
  \begin{multicols}{2}
    \begin{itemize}
      \item[(a)] Find the NE in the static game:
    \end{itemize}
    Expected payoff for player $i\neq j$:
    \begin{align*}
      u_i(s_i,s_j)=\underbrace{s_i+s_j-s_is_j}_\text{Probability of success}-\underbrace{s_i^2}_\text{Cost}
    \end{align*}
    Find the best-response function for $i$:
    \begin{align*}
      FOC:\ \frac{\delta u_i}{\delta s_i}=1+0-s_j-2s_i&=0\\
       s_i&=\frac{1-s_j}{2}
    \end{align*}
    Taking advantage of symmetry $s_i^{*}=s_j^{*}$:
    \begin{align*}
       s_i^{*}&=\frac{1-s_i^{*}}{2}\\
      2s_i^{*}+s_i^{*}&=1\\
       s_i^{*}&=\frac{1}{3}\equiv s^{NE}
    \end{align*}
    i.e. $NE=\left\{(s_D^{*},s_V^{*})=(\frac{1}{3},\frac{1}{3})\right\}$
  \vfill\null\columnbreak
    \begin{itemize}
      \item[(b)] Find the SO given shares are equal:
    \end{itemize}
    Expected payoff for $i$, $\bar{s}_D=\bar{s}_V=\bar{s}$:
    \begin{align*}
      u_i(\bar{s})&=\underbrace{\bar{s}+\bar{s}-\bar{s}\bar{s}}_\text{Probability of success}-\underbrace{\bar{s}^2}_\text{Cost}\\
                  &=2\bar{s}-2\bar{s}^2
    \end{align*}
  \vfill\null
  \end{multicols}
\end{frame}
\begin{frame}{PS3, Ex. 5: Luxembourg as a rogue state}
  \begin{multicols}{2}
    \begin{itemize}
      \item[(a)] Find the NE in the static game:
    \end{itemize}
    Expected payoff for player $i\neq j$:
    \begin{align*}
      u_i(s_i,s_j)=\underbrace{s_i+s_j-s_is_j}_\text{Probability of success}-\underbrace{s_i^2}_\text{Cost}
    \end{align*}
    Find the best-response function for $i$:
    \begin{align*}
      FOC:\ \frac{\delta u_i}{\delta s_i}=1+0-s_j-2s_i&=0\\
       s_i&=\frac{1-s_j}{2}
    \end{align*}
    Taking advantage of symmetry $s_i^{*}=s_j^{*}$:
    \begin{align*}
       s_i^{*}&=\frac{1-s_i^{*}}{2}\\
      2s_i^{*}+s_i^{*}&=1\\
       s_i^{*}&=\frac{1}{3}\equiv s^{NE}
    \end{align*}
    i.e. $NE=\left\{(s_D^{*},s_V^{*})=(\frac{1}{3},\frac{1}{3})\right\}$
  \vfill\null\columnbreak
    \begin{itemize}
      \item[(b)] Find the SO given shares are equal:
    \end{itemize}
    Expected payoff for $i$, $\bar{s}_D=\bar{s}_V=\bar{s}$:
    \begin{align*}
      u_i(\bar{s})&=\underbrace{\bar{s}+\bar{s}-\bar{s}\bar{s}}_\text{Probability of success}-\underbrace{\bar{s}^2}_\text{Cost}\\
                  &=2\bar{s}-2\bar{s}^2
    \end{align*}
    Social planner target function:
    \begin{align*}
      \pi^S(\bar{s})&=\underbrace{2}_\text{Countries}(2\bar{s}-2\bar{s}^2)=4\bar{s}-4\bar{s}^2
    \end{align*}
  \vfill\null
  \end{multicols}
\end{frame}
\begin{frame}{PS3, Ex. 5: Luxembourg as a rogue state}
  \begin{multicols}{2}
    \begin{itemize}
      \item[(a)] Find the NE in the static game:
    \end{itemize}
    Expected payoff for player $i\neq j$:
    \begin{align*}
      u_i(s_i,s_j)=\underbrace{s_i+s_j-s_is_j}_\text{Probability of success}-\underbrace{s_i^2}_\text{Cost}
    \end{align*}
    Find the best-response function for $i$:
    \begin{align*}
      FOC:\ \frac{\delta u_i}{\delta s_i}=1+0-s_j-2s_i&=0\\
       s_i&=\frac{1-s_j}{2}
    \end{align*}
    Taking advantage of symmetry $s_i^{*}=s_j^{*}$:
    \begin{align*}
       s_i^{*}&=\frac{1-s_i^{*}}{2}\\
      2s_i^{*}+s_i^{*}&=1\\
       s_i^{*}&=\frac{1}{3}\equiv s^{NE}
    \end{align*}
    i.e. $NE=\left\{(s_D^{*},s_V^{*})=(\frac{1}{3},\frac{1}{3})\right\}$
  \vfill\null\columnbreak
    \begin{itemize}
      \item[(b)] Find the SO given shares are equal:
    \end{itemize}
    Expected payoff for $i$, $\bar{s}_D=\bar{s}_V=\bar{s}$:
    \begin{align*}
      u_i(\bar{s})&=\underbrace{\bar{s}+\bar{s}-\bar{s}\bar{s}}_\text{Probability of success}-\underbrace{\bar{s}^2}_\text{Cost}\\
                  &=2\bar{s}-2\bar{s}^2
    \end{align*}
    Social planner target function:
    \begin{align*}
      \pi^S(\bar{s})&=\underbrace{2}_\text{Countries}(2\bar{s}-2\bar{s}^2)=4\bar{s}-4\bar{s}^2
    \end{align*}
    Find the social optimum (SO):
    \begin{align*}
      FOC:\ \frac{\delta\pi^S}{\delta s_i}=4-8\bar{S}&=0\\
       \bar{S}&=\frac{4}{8}=\frac{1}{2}>\frac{1}{3}
    \end{align*}
    i.e. the SO is higher than the NE as the positive externality is not rewarded, leading to an incentive to free ride.
  \vfill\null
  \end{multicols}
\end{frame}




\section{PS3, Ex. 6: Cournot Oligopoly with three firms}

\begin{frame}{PS3, Ex. 6: Cournot Oligopoly with three firms}
  \begin{multicols}{2}
    There are three identical firms in an industry. Their production quantities are denoted $q_1$, $q_2$, and $q_3$. The inverse demand function is
    \begin{align*}
      p=1-Q,\text{     where }Q=q_1+q_2+q_3.
    \end{align*}
    The marginal cost is zero.
    \begin{itemize}
      \item[(a)] Compute the quantities in the Cournot equilibrium, i.e., the Nash Equilibrium of the game where the firms simultaneously choose quantities.
      \item[(b)] What is the price in the Cournot-equilibrium?
      \item[(c)] Show that if two of the three firms merge (transforming the industry into a duopoly), the profits of the merging firms decrease. Explain.
      \item[(d)] What happens if all three firms merge?
    \end{itemize}
  \vfill\null \columnbreak
    \includegraphics[width=.5 \textwidth]{figures/ford}
  \vfill\null
  \end{multicols}
\end{frame}


\begin{frame}{PS3, Ex. 6: Cournot Oligopoly with three firms}
  \begin{multicols}{2}
    \begin{itemize}
      \item[a)] Quantities in the Cournot equilibrium
    \end{itemize}
    The payoff function for firm $i\in\{1,2,3\}$:
    \begin{align*}
        \pi_i&=(1-q_i-q_j-q_k)q_i
    \end{align*}
  \vfill\null \columnbreak
  \vfill\null
  \end{multicols}
\end{frame}
\begin{frame}{PS3, Ex. 6: Cournot Oligopoly with three firms}
  \begin{multicols}{2}
    \begin{itemize}
      \item[a)] Quantities in the Cournot equilibrium
    \end{itemize}
    The payoff function for firm $i\in\{1,2,3\}$:
    \begin{align*}
        \pi_i&=(1-q_i-q_j-q_k)q_i
    \end{align*}
    Best-Response (BR) function for firm $i$:
    \begin{align*}
        \frac{\delta\pi_i}{\delta q_i}=1-2q_i-q_j-q_k&=0\\
                                                  q_i&=\frac{1-q_j-q_k}{2}
    \end{align*}
  \vfill\null \columnbreak
  \vfill\null
  \end{multicols}
\end{frame}
\begin{frame}{PS3, Ex. 6: Cournot Oligopoly with three firms}
  \begin{multicols}{2}
    \begin{itemize}
      \item[a)] Quantities in the Cournot equilibrium
    \end{itemize}
    The payoff function for firm $i\in\{1,2,3\}$:
    \begin{align*}
        \pi_i&=(1-q_i-q_j-q_k)q_i
    \end{align*}
    Best-Response (BR) function for firm $i$:
    \begin{align*}
        \frac{\delta\pi_i}{\delta q_i}=1-2q_i-q_j-q_k&=0\\
                                                  q_i&=\frac{1-q_j-q_k}{2}
    \end{align*}
    Due to symmetry $q_i^{*}=q_j^{*}=q_k^{*}=q^{NE}$:
    \begin{align*}
        q_i^{*} &= \frac{1-2q_i^{*}}{2}\\
        q_i^{*} &= \frac{1}{4}\equiv q^{NE}
    \end{align*}
  \vfill\null \columnbreak
  \vfill\null
  \end{multicols}
\end{frame}
\begin{frame}{PS3, Ex. 6: Cournot Oligopoly with three firms}
  \begin{multicols}{2}
    \begin{itemize}
      \item[a)] Quantities in the Cournot equilibrium
    \end{itemize}
    The payoff function for firm $i\in\{1,2,3\}$:
    \begin{align*}
        \pi_i&=(1-q_i-q_j-q_k)q_i
    \end{align*}
    Best-Response (BR) function for firm $i$:
    \begin{align*}
        \frac{\delta\pi_i}{\delta q_i}=1-2q_i-q_j-q_k&=0\\
                                                  q_i&=\frac{1-q_j-q_k}{2}
    \end{align*}
    Due to symmetry $q_i^{*}=q_j^{*}=q_k^{*}=q^{NE}$:
    \begin{align*}
        q_i^{*} &= \frac{1-2q_i^{*}}{2}\\
        q_i^{*} &= \frac{1}{4}\equiv q^{NE}
    \end{align*}
    \begin{itemize}
      \item[(b)] What is the price in the Cournot-equilibrium?
    \end{itemize}
  \vfill\null \columnbreak
  \vfill\null
  \end{multicols}
\end{frame}
\begin{frame}{PS3, Ex. 6: Cournot Oligopoly with three firms}
  \begin{multicols}{2}
    \begin{itemize}
      \item[a)] Quantities in the Cournot equilibrium
    \end{itemize}
    The payoff function for firm $i\in\{1,2,3\}$:
    \begin{align*}
        \pi_i&=(1-q_i-q_j-q_k)q_i
    \end{align*}
    Best-Response (BR) function for firm $i$:
    \begin{align*}
        \frac{\delta\pi_i}{\delta q_i}=1-2q_i-q_j-q_k&=0\\
                                                  q_i&=\frac{1-q_j-q_k}{2}
    \end{align*}
    Due to symmetry $q_i^{*}=q_j^{*}=q_k^{*}=q^{NE}$:
    \begin{align*}
        q_i^{*} &= \frac{1-2q_i^{*}}{2}\\
        q_i^{*} &= \frac{1}{4}\equiv q^{NE}
    \end{align*}
    \begin{itemize}
      \item[(b)] Price in the Cournot-equilibrium:
    \end{itemize}
    \begin{align*}
      p^{*}=1-q_i^{*}-q_j^{*}-q_k^{*}=\frac{1}{4}\Rightarrow\pi_i^{*}=\frac{1}{16}
    \end{align*}
  \vfill\null \columnbreak
  \vfill\null
  \end{multicols}
\end{frame}
\begin{frame}{PS3, Ex. 6: Cournot Oligopoly with three firms}
  \begin{multicols}{2}
    \begin{itemize}
      \item[a)] Quantities in the Cournot equilibrium
    \end{itemize}
    The payoff function for firm $i\in\{1,2,3\}$:
    \begin{align*}
        \pi_i&=(1-q_i-q_j-q_k)q_i
    \end{align*}
    Best-Response (BR) function for firm $i$:
    \begin{align*}
        \frac{\delta\pi_i}{\delta q_i}=1-2q_i-q_j-q_k&=0\\
                                                  q_i&=\frac{1-q_j-q_k}{2}
    \end{align*}
    Due to symmetry $q_i^{*}=q_j^{*}=q_k^{*}=q^{NE}$:
    \begin{align*}
        q_i^{*} &= \frac{1-2q_i^{*}}{2}\\
        q_i^{*} &= \frac{1}{4}\equiv q^{NE}
    \end{align*}
    \begin{itemize}
      \item[(b)] Price in the Cournot-equilibrium:
    \end{itemize}
    \begin{align*}
      p^{*}=1-q_i^{*}-q_j^{*}-q_k^{*}=\frac{1}{4}\Rightarrow\pi_i^{*}=\frac{1}{16}
    \end{align*}
  \vfill\null \columnbreak
    \begin{itemize}
      \item[(c)] Show that if two of the three firms merge (transforming the industry into a duopoly), the profits of the merging firms decrease. Explain.
    \end{itemize}
  \vfill\null
  \end{multicols}
\end{frame}
\begin{frame}{PS3, Ex. 6: Cournot Oligopoly with three firms}
  \begin{multicols}{2}
    \begin{itemize}
      \item[a)] Quantities in the Cournot equilibrium
    \end{itemize}
    The payoff function for firm $i\in\{1,2,3\}$:
    \begin{align*}
        \pi_i&=(1-q_i-q_j-q_k)q_i
    \end{align*}
    Best-Response (BR) function for firm $i$:
    \begin{align*}
        \frac{\delta\pi_i}{\delta q_i}=1-2q_i-q_j-q_k&=0\\
                                                  q_i&=\frac{1-q_j-q_k}{2}
    \end{align*}
    Due to symmetry $q_i^{*}=q_j^{*}=q_k^{*}=q^{NE}$:
    \begin{align*}
        q_i^{*} &= \frac{1-2q_i^{*}}{2}\\
        q_i^{*} &= \frac{1}{4}\equiv q^{NE}
    \end{align*}
    \begin{itemize}
      \item[(b)] Price in the Cournot-equilibrium:
    \end{itemize}
    \begin{align*}
      p^{*}=1-q_i^{*}-q_j^{*}-q_k^{*}=\frac{1}{4}\Rightarrow\pi_i^{*}=\frac{1}{16}
    \end{align*}
  \vfill\null \columnbreak
    \begin{itemize}
      \item[(c)] Firm \textit{1} and \textit{2} merge to firm $m$.
    \end{itemize}
    The payoff function for firm $i\in\{\bm{m},3\}$:
    \begin{align*}
        \pi_i&=(1-q_i-q_j)q_i
    \end{align*}
  \vfill\null
  \end{multicols}
\end{frame}
\begin{frame}{PS3, Ex. 6: Cournot Oligopoly with three firms}
  \begin{multicols}{2}
    \begin{itemize}
      \item[a)] Quantities in the Cournot equilibrium
    \end{itemize}
    The payoff function for firm $i\in\{1,2,3\}$:
    \begin{align*}
        \pi_i&=(1-q_i-q_j-q_k)q_i
    \end{align*}
    Best-Response (BR) function for firm $i$:
    \begin{align*}
        \frac{\delta\pi_i}{\delta q_i}=1-2q_i-q_j-q_k&=0\\
                                                  q_i&=\frac{1-q_j-q_k}{2}
    \end{align*}
    Due to symmetry $q_i^{*}=q_j^{*}=q_k^{*}=q^{NE}$:
    \begin{align*}
        q_i^{*} &= \frac{1-2q_i^{*}}{2}\\
        q_i^{*} &= \frac{1}{4}\equiv q^{NE}
    \end{align*}
    \begin{itemize}
      \item[(b)] Price in the Cournot-equilibrium:
    \end{itemize}
    \begin{align*}
      p^{*}=1-q_i^{*}-q_j^{*}-q_k^{*}=\frac{1}{4}\Rightarrow\pi_i^{*}=\frac{1}{16}
    \end{align*}
  \vfill\null \columnbreak
    \begin{itemize}
      \item[(c)] Firm \textit{1} and \textit{2} merge to firm $m$.
    \end{itemize}
    The payoff function for firm $i\in\{\bm{m},3\}$:
    \begin{align*}
        \pi_i&=(1-q_i-q_j)q_i
    \end{align*}
    BR function for firm $i$ in the duopoly:
    \begin{align*}
        q_i&=\frac{1-q_j}{2}\\
        q_i^{*} &= \frac{1-2q_i^{*}}{2},\ \ q_i^{*}=q_j^{*}\\
        q_i^{*} &= \frac{1}{3}\equiv q^{NE}
    \end{align*}
  \vfill\null
  \end{multicols}
\end{frame}
\begin{frame}{PS3, Ex. 6: Cournot Oligopoly with three firms}
  \begin{multicols}{2}
    \begin{itemize}
      \item[a)] Quantities in the Cournot equilibrium
    \end{itemize}
    The payoff function for firm $i\in\{1,2,3\}$:
    \begin{align*}
        \pi_i&=(1-q_i-q_j-q_k)q_i
    \end{align*}
    Best-Response (BR) function for firm $i$:
    \begin{align*}
        \frac{\delta\pi_i}{\delta q_i}=1-2q_i-q_j-q_k&=0\\
                                                  q_i&=\frac{1-q_j-q_k}{2}
    \end{align*}
    Due to symmetry $q_i^{*}=q_j^{*}=q_k^{*}=q^{NE}$:
    \begin{align*}
        q_i^{*} &= \frac{1-2q_i^{*}}{2}\\
        q_i^{*} &= \frac{1}{4}\equiv q^{NE}
    \end{align*}
    \begin{itemize}
      \item[(b)] Price in the Cournot-equilibrium:
    \end{itemize}
    \begin{align*}
      p^{*}=1-q_i^{*}-q_j^{*}-q_k^{*}=\frac{1}{4}\Rightarrow\pi_i^{*}=\frac{1}{16}
    \end{align*}
  \vfill\null \columnbreak
    \begin{itemize}
      \item[(c)] Firm \textit{1} and \textit{2} merge to firm $m$.
    \end{itemize}
    The payoff function for firm $i\in\{\bm{m},3\}$:
    \begin{align*}
        \pi_i&=(1-q_i-q_j)q_i
    \end{align*}
    BR function for firm $i$ in the duopoly:
    \begin{align*}
        q_i&=\frac{1-q_j}{2}\\
        q_i^{*} &= \frac{1-2q_i^{*}}{2},\ \ q_i^{*}=q_j^{*}\\
        q_i^{*} &= \frac{1}{3}\equiv q^{NE}
    \end{align*}
    By merging the rivalry is internalized by reducing joint output which increase market price and the profit margin:
    \begin{align*}
      p^{*}=1-q_m^{*}-q_3^{*}=\frac{1}{3}\Rightarrow\pi_m^{*}=\pi_3^{*}=\frac{1}{9}
    \end{align*}
    Are Firm \textit{1} and \textit{2} better or worse off? Why?
  \vfill\null
  \end{multicols}
\end{frame}
\begin{frame}{PS3, Ex. 6: Cournot Oligopoly with three firms}
  \begin{multicols}{2}
    \begin{itemize}
      \item[a)] Quantities in the Cournot equilibrium
    \end{itemize}
    The payoff function for firm $i\in\{1,2,3\}$:
    \begin{align*}
        \pi_i&=(1-q_i-q_j-q_k)q_i
    \end{align*}
    Best-Response (BR) function for firm $i$:
    \begin{align*}
        \frac{\delta\pi_i}{\delta q_i}=1-2q_i-q_j-q_k&=0\\
                                                  q_i&=\frac{1-q_j-q_k}{2}
    \end{align*}
    Due to symmetry $q_i^{*}=q_j^{*}=q_k^{*}=q^{NE}$:
    \begin{align*}
        q_i^{*} &= \frac{1-2q_i^{*}}{2}\\
        q_i^{*} &= \frac{1}{4}\equiv q^{NE}
    \end{align*}
    \begin{itemize}
      \item[(b)] Price in the Cournot-equilibrium:
    \end{itemize}
    \begin{align*}
      p^{*}=1-q_i^{*}-q_j^{*}-q_k^{*}=\frac{1}{4}\Rightarrow\pi_i^{*}=\frac{1}{16}
    \end{align*}
  \vfill\null \columnbreak
    \begin{itemize}
      \item[(c)] Firm \textit{1} and \textit{2} merge to firm $m$.
    \end{itemize}
    The payoff function for firm $i\in\{\bm{m},3\}$:
    \begin{align*}
        \pi_i&=(1-q_i-q_j)q_i
    \end{align*}
    BR function for firm $i$ in the duopoly:
    \begin{align*}
        q_i&=\frac{1-q_j}{2}\\
        q_i^{*} &= \frac{1-2q_i^{*}}{2},\ \ q_i^{*}=q_j^{*}\\
        q_i^{*} &= \frac{1}{3}\equiv q^{NE}
    \end{align*}
    By merging the rivalry is internalized by reducing joint output which increase market price and the profit margin:
    \begin{align*}
      p^{*}=1-q_m^{*}-q_3^{*}=\frac{1}{3}\Rightarrow\pi_m^{*}=\pi_3^{*}=\frac{1}{9}
    \end{align*}
    However, Firm \textit{1} and \textit{2} each get $\frac{1}{18}<\frac{1}{16}$ and are worse off as the third firm reacts to the higher price by increasing output.
  \vfill\null
  \end{multicols}
\end{frame}
\begin{frame}{PS3, Ex. 6: Cournot Oligopoly with three firms}
  \begin{multicols}{2}
    \begin{itemize}
      \item[a)] Quantities in the Cournot equilibrium
    \end{itemize}
    The payoff function for firm $i\in\{1,2,3\}$:
    \begin{align*}
        \pi_i&=(1-q_i-q_j-q_k)q_i
    \end{align*}
    Best-Response (BR) function for firm $i$:
    \begin{align*}
        \frac{\delta\pi_i}{\delta q_i}=1-2q_i-q_j-q_k&=0\\
                                                  q_i&=\frac{1-q_j-q_k}{2}
    \end{align*}
    Due to symmetry $q_i^{*}=q_j^{*}=q_k^{*}=q^{NE}$:
    \begin{align*}
        q_i^{*} &= \frac{1-2q_i^{*}}{2}\\
        q_i^{*} &= \frac{1}{4}\equiv q^{NE}
    \end{align*}
    \begin{itemize}
      \item[(b)] Price in the Cournot-equilibrium:
    \end{itemize}
    \begin{align*}
      p^{*}=1-q_i^{*}-q_j^{*}-q_k^{*}=\frac{1}{4}\Rightarrow\pi_i^{*}=\frac{1}{16}
    \end{align*}
  \vfill\null \columnbreak
    \begin{itemize}
      \item[(c)] Firm \textit{1} and \textit{2} merge to firm $m$.
    \end{itemize}
    \begin{align*}
        \pi_i&=(1-q_i-q_j)q_i
    \end{align*}
    \begin{align*}
        q_i^{*} &= \frac{1}{3}\equiv q^{NE}
    \end{align*}
    By merging the rivalry is internalized by reducing joint output which increase market price and the profit margin:
    \begin{align*}
      p^{*}=1-q_m^{*}-q_3^{*}=\frac{1}{3}\Rightarrow\pi_m^{*}=\pi_3^{*}=\frac{1}{9}
    \end{align*}
    However, Firm \textit{1} and \textit{2} each get $\frac{1}{18}<\frac{1}{16}$ and are worse off as the third firm reacts to the higher price by increasing output.
    \begin{itemize}
      \item[(d)] What happens if all three firms merge?
    \end{itemize}
  \vfill\null
  \end{multicols}
\end{frame}
\begin{frame}{PS3, Ex. 6: Cournot Oligopoly with three firms}
  \begin{multicols}{2}
    \begin{itemize}
      \item[a)] Quantities in the Cournot equilibrium
    \end{itemize}
    The payoff function for firm $i\in\{1,2,3\}$:
    \begin{align*}
        \pi_i&=(1-q_i-q_j-q_k)q_i
    \end{align*}
    Best-Response (BR) function for firm $i$:
    \begin{align*}
        \frac{\delta\pi_i}{\delta q_i}=1-2q_i-q_j-q_k&=0\\
                                                  q_i&=\frac{1-q_j-q_k}{2}
    \end{align*}
    Due to symmetry $q_i^{*}=q_j^{*}=q_k^{*}=q^{NE}$:
    \begin{align*}
        q_i^{*} &= \frac{1-2q_i^{*}}{2}\\
        q_i^{*} &= \frac{1}{4}\equiv q^{NE}
    \end{align*}
    \begin{itemize}
      \item[(b)] Price in the Cournot-equilibrium:
    \end{itemize}
    \begin{align*}
      p^{*}=1-q_i^{*}-q_j^{*}-q_k^{*}=\frac{1}{4}\Rightarrow\pi_i^{*}=\frac{1}{16}
    \end{align*}
  \vfill\null \columnbreak
    \begin{itemize}
      \item[(c)] Firm \textit{1} and \textit{2} merge to firm $m$.
    \end{itemize}
    \begin{align*}
        \pi_i&=(1-q_i-q_j)q_i
    \end{align*}
    \begin{align*}
        q_i^{*} &= \frac{1}{3}\equiv q^{NE}
    \end{align*}
    By merging the rivalry is internalized by reducing joint output which increase market price and the profit margin:
    \begin{align*}
      p^{*}=1-q_m^{*}-q_3^{*}=\frac{1}{3}\Rightarrow\pi_m^{*}=\pi_3^{*}=\frac{1}{9}
    \end{align*}
    However, Firm \textit{1} and \textit{2} each get $\frac{1}{18}<\frac{1}{16}$ and are worse off as the third firm reacts to the higher price by increasing output.
    \begin{itemize}
      \item[(d)] A full merger maximizes joint profits:
    \end{itemize}
    \begin{align*}
      q_\text{monopoly}^{*}=p_\text{monopoly}^{*}=\frac{1}{2}\Rightarrow \pi_\text{monopoly}^{*}=\frac{1}{4}>\frac{2}{9}
    \end{align*}
  \vfill\null
  \end{multicols}
\end{frame}



\section{PS3, Ex. 7: Mixed Strategy Nash Equilibria}

\begin{frame}{PS3, Ex. 7: Mixed Strategy Nash Equilibria}
  Plot the mixed best responses of each player (in a "(p,q)-diagram" - see the textbook). And find all Nash equilibria (pure and mixed) in the games below
  \begin{multicols}{2}
    \begin{itemize}
      \item[(a)]
    \end{itemize}
    \vspace{-16pt}
    \begin{table}
      \begin{tabular}{cl|c|c|}
          & \multicolumn{1}{c}{} & \multicolumn{2}{c}{Player 2}\\
          \parbox[t]{1mm}{\multirow{3}{*}{\rotatebox[origin=r]{90}{Player 1}}}
          & \multicolumn{1}{c}{} & \multicolumn{1}{c}{L ($q$)} & \multicolumn{1}{c}{R (1-$q$)} \\\cline{3-4}
          & T  ($p$)  & 0, 0 & 0, 0 \\\cline{3-4}
          & B  (1-$p$)& 0, 0 & 1, 1 \\\cline{3-4}
      \end{tabular}
    \end{table}
    \begin{itemize}
      \item[(b)]
    \end{itemize}
    \vspace{-16pt}
    \begin{table}
      \begin{tabular}{cl|c|c|}
          & \multicolumn{1}{c}{} & \multicolumn{2}{c}{Player 2}\\
          \parbox[t]{1mm}{\multirow{3}{*}{\rotatebox[origin=r]{90}{Player 1}}}
          & \multicolumn{1}{c}{} & \multicolumn{1}{c}{L ($q$)} & \multicolumn{1}{c}{R (1-$q$)} \\\cline{3-4}
          & T  ($p$)  & 1, 3 & 1, 0 \\\cline{3-4}
          & B  (1-$p$)& 1, 1 & 5, 5 \\\cline{3-4}
      \end{tabular}
    \end{table}
  \vfill\null \columnbreak
  \begin{itemize}
    \item[(c)]
  \end{itemize}
  \vspace{-16pt}
  \begin{table}
    \begin{tabular}{cl|c|c|}
        & \multicolumn{1}{c}{} & \multicolumn{2}{c}{Player 2}\\
        \parbox[t]{1mm}{\multirow{3}{*}{\rotatebox[origin=r]{90}{Player 1}}}
        & \multicolumn{1}{c}{} & \multicolumn{1}{c}{L ($q$)} & \multicolumn{1}{c}{R (1-$q$)} \\\cline{3-4}
        & T  ($p$)  & 3, 2 & 1, 2 \\\cline{3-4}
        & B  (1-$p$)& 0, 1 & 1, 2 \\\cline{3-4}
    \end{tabular}
  \end{table}
  \begin{itemize}
    \item[(d)]
  \end{itemize}
  \vspace{-16pt}
  \begin{table}
    \begin{tabular}{cl|c|c|}
        & \multicolumn{1}{c}{} & \multicolumn{2}{c}{Player 2}\\
        & \multicolumn{1}{c}{} & \multicolumn{1}{c}{$t_1$ ($q$)} & \multicolumn{1}{c}{$t_2$ (1-$q$)} \\\cline{3-4}
        \parbox[t]{1mm}{\multirow{3}{*}{\rotatebox[origin=c]{90}{Player 1}}}
        & $s_1$ ($p_1$)         & 2, 1 & 3, 0 \\\cline{3-4}
        & $s_2$ ($p_2$)         & 1, 2 & 4, 3 \\\cline{3-4}
        & $s_3$ (1-$p_1$-$p_2$) & 0, 1 & 0, 3 \\\cline{3-4}
    \end{tabular}
  \end{table}
  \vfill\null
  \end{multicols}
  \begin{tabular}{|l|}
      \cline{1-1}
      \textbf{Hint}: Find the probabilities $q$ for which Player 1 is indifferent, e.g. $u_1(T,q)=u_1(B,q)$.\\
                      and the probabilities $p$ for which Player 2 is indifferent, e.g. $u_2(L,p)=u_2(R,p)$.\\\cline{1-1}
  \end{tabular}
\end{frame}
\begin{frame}{PS3, Ex. 7.a: Mixed Strategy Nash Equilibria}
  \begin{multicols}{2}
    \begin{itemize}
      \item[(a)] Plot the mixed best responses and find all NE (pure and mixed):
    \end{itemize}
    \begin{table}
      \begin{tabular}{cl|c|c|}
        & \multicolumn{1}{c}{} & \multicolumn{2}{c}{\color{blue}Player 2}\\
        \parbox[t]{1mm}{\multirow{3}{*}{\rotatebox[origin=r]{90}{\color{red}Player 1}}}
          & \multicolumn{1}{c}{} & \multicolumn{1}{c}{L ($q$)} & \multicolumn{1}{c}{R (1-$q$)} \\\cline{3-4}
          & T  ($p$)  & \textcolor{red}{0}, \textcolor{blue}{0} & 0, \textcolor{blue}{0} \\\cline{3-4}
          & B  (1-$p$)& \textcolor{red}{0}, 0 & \textcolor{red}{1}, \textcolor{blue}{1} \\\cline{3-4}
      \end{tabular}
    \end{table}
    Player 1:
    \begin{itemize}
      \item Indifferent if $q=1\Rightarrow p\in[0,1]$
      \item Prefers $B$ if $q<1\Rightarrow p=0$.
    \end{itemize}
  \vfill\null \columnbreak
  \vfill\null
  \end{multicols}
\end{frame}
\begin{frame}{PS3, Ex. 7.a: Mixed Strategy Nash Equilibria}
  \begin{multicols}{2}
    \begin{itemize}
      \item[(a)] Plot the mixed best responses and find all NE (pure and mixed):
    \end{itemize}
    \begin{table}
      \begin{tabular}{cl|c|c|}
        & \multicolumn{1}{c}{} & \multicolumn{2}{c}{\color{blue}Player 2}\\
        \parbox[t]{1mm}{\multirow{3}{*}{\rotatebox[origin=r]{90}{\color{red}Player 1}}}
          & \multicolumn{1}{c}{} & \multicolumn{1}{c}{L ($q$)} & \multicolumn{1}{c}{R (1-$q$)} \\\cline{3-4}
          & T  ($p$)  & \textcolor{red}{0}, \textcolor{blue}{0} & 0, \textcolor{blue}{0} \\\cline{3-4}
          & B  (1-$p$)& \textcolor{red}{0}, 0 & \textcolor{red}{1}, \textcolor{blue}{1} \\\cline{3-4}
      \end{tabular}
    \end{table}
    Player 1:
    \begin{itemize}
      \item Indifferent if $q=1\Rightarrow p\in[0,1]$
      \item Prefers $B$ if $q<1\Rightarrow p=0$.
    \end{itemize}
    Player 2:
    \begin{itemize}
      \item Indifferent if $p=1\Rightarrow q\in[0,1]$
      \item Prefers $R$ if $p<1\Rightarrow q=0$.
    \end{itemize}
    i.e. two Pure Strategy Nash Equilibria:
    \begin{align*}
      PSNE=\{(T,L),(B,R)\}
    \end{align*}
  \vfill\null \columnbreak
  \vfill\null
  \end{multicols}
\end{frame}
\begin{frame}{PS3, Ex. 7.a: Mixed Strategy Nash Equilibria}
  \begin{multicols}{2}
    \begin{itemize}
      \item[(a)] Plot the mixed best responses and find all NE (pure and mixed):
    \end{itemize}
    \begin{table}
      \begin{tabular}{cl|c|c|}
        & \multicolumn{1}{c}{} & \multicolumn{2}{c}{\color{blue}Player 2}\\
        \parbox[t]{1mm}{\multirow{3}{*}{\rotatebox[origin=r]{90}{\color{red}Player 1}}}
          & \multicolumn{1}{c}{} & \multicolumn{1}{c}{L ($q$)} & \multicolumn{1}{c}{R (1-$q$)} \\\cline{3-4}
          & T  ($p$)  & \textcolor{red}{0}, \textcolor{blue}{0} & 0, \textcolor{blue}{0} \\\cline{3-4}
          & B  (1-$p$)& \textcolor{red}{0}, 0 & \textcolor{red}{1}, \textcolor{blue}{1} \\\cline{3-4}
      \end{tabular}
    \end{table}
    Player 1:
    \begin{itemize}
      \item Indifferent if $q=1\Rightarrow p\in[0,1]$
      \item Prefers $B$ if $q<1\Rightarrow p=0$.
    \end{itemize}
    Player 2:
    \begin{itemize}
      \item Indifferent if $p=1\Rightarrow q\in[0,1]$
      \item Prefers $R$ if $p<1\Rightarrow q=0$.
    \end{itemize}
    i.e. two Pure Strategy Nash Equilibria:
    \begin{align*}
      PSNE=\{(T,L),(B,R)\}
    \end{align*}
  \vfill\null \columnbreak
    \includegraphics[width=\columnwidth]{figures/5a1}
  \vfill\null
  \end{multicols}
\end{frame}
\begin{frame}{PS3, Ex. 7.a: Mixed Strategy Nash Equilibria}
  \begin{multicols}{2}
    \begin{itemize}
      \item[(a)] Plot the mixed best responses and find all NE (pure and mixed):
    \end{itemize}
    \begin{table}
      \begin{tabular}{cl|c|c|}
        & \multicolumn{1}{c}{} & \multicolumn{2}{c}{\color{blue}Player 2}\\
        \parbox[t]{1mm}{\multirow{3}{*}{\rotatebox[origin=r]{90}{\color{red}Player 1}}}
          & \multicolumn{1}{c}{} & \multicolumn{1}{c}{L ($q$)} & \multicolumn{1}{c}{R (1-$q$)} \\\cline{3-4}
          & T  ($p$)  & \textcolor{red}{0}, \textcolor{blue}{0} & 0, \textcolor{blue}{0} \\\cline{3-4}
          & B  (1-$p$)& \textcolor{red}{0}, 0 & \textcolor{red}{1}, \textcolor{blue}{1} \\\cline{3-4}
      \end{tabular}
    \end{table}
    Player 1:
    \begin{itemize}
      \item Indifferent if $q=1\Rightarrow p\in[0,1]$
      \item Prefers $B$ if $q<1\Rightarrow p=0$.
    \end{itemize}
    Player 2:
    \begin{itemize}
      \item Indifferent if $p=1\Rightarrow q\in[0,1]$
      \item Prefers $R$ if $p<1\Rightarrow q=0$.
    \end{itemize}
    i.e. two Pure Strategy Nash Equilibria:
    \begin{align*}
      PSNE=\{(T,L),(B,R)\}
    \end{align*}
  \vfill\null \columnbreak
    \includegraphics[width=\columnwidth]{figures/5a2}
    We only find the two Mixed Strategy NE (MSNE). Both coincide with the PSNE:
    \begin{align*}
      (p^{*},q^{*})=\left\{(1,1),(0,0)\right\}
    \end{align*}
  \vfill\null
  \end{multicols}
\end{frame}
\begin{frame}{PS3, Ex. 7.b: Mixed Strategy Nash Equilibria}
  \begin{multicols}{2}
    \begin{itemize}
      \item[(b)] Plot the mixed best responses and find all NE (pure and mixed):
    \end{itemize}
    \begin{table}
      \begin{tabular}{cl|c|c|}
        & \multicolumn{1}{c}{} & \multicolumn{2}{c}{\color{blue}Player 2}\\
        \parbox[t]{1mm}{\multirow{3}{*}{\rotatebox[origin=r]{90}{\color{red}Player 1}}}
        & \multicolumn{1}{c}{} & \multicolumn{1}{c}{L ($q$)} & \multicolumn{1}{c}{R (1-$q$)} \\\cline{3-4}
        & T  ($p$)  & \textcolor{red}{1}, \textcolor{blue}{3} & 1, 0 \\\cline{3-4}
        & B  (1-$p$)& \textcolor{red}{1}, 1 & \textcolor{red}{5}, \textcolor{blue}{5} \\\cline{3-4}
      \end{tabular}
    \end{table}
    Player 1 is indifferent if:
    \begin{align*}
      1 &= 1q + 5(1-q) \\
      5q&= 4          \\
      q &= 1
    \end{align*}
  \vfill\null \columnbreak
  \vfill\null
  \end{multicols}
\end{frame}
\begin{frame}{PS3, Ex. 7.b: Mixed Strategy Nash Equilibria}
  \begin{multicols}{2}
    \begin{itemize}
      \item[(b)] Plot the mixed best responses and find all NE (pure and mixed):
    \end{itemize}
    \begin{table}
      \begin{tabular}{cl|c|c|}
        & \multicolumn{1}{c}{} & \multicolumn{2}{c}{\color{blue}Player 2}\\
        \parbox[t]{1mm}{\multirow{3}{*}{\rotatebox[origin=r]{90}{\color{red}Player 1}}}
        & \multicolumn{1}{c}{} & \multicolumn{1}{c}{L ($q$)} & \multicolumn{1}{c}{R (1-$q$)} \\\cline{3-4}
        & T  ($p$)  & \textcolor{red}{1}, \textcolor{blue}{3} & 1, 0 \\\cline{3-4}
        & B  (1-$p$)& \textcolor{red}{1}, 1 & \textcolor{red}{5}, \textcolor{blue}{5} \\\cline{3-4}
      \end{tabular}
    \end{table}
    Player 1 is indifferent if:
    \begin{align*}
      1 &= 1q + 5(1-q) \\
      5q&= 4          \\
      q &= 1
    \end{align*}
    Player 2 is indifferent if:
    \begin{align*}
      3p + 1(1-p)&= 0p+5(1-p)\\
      7p         &= 4     \\
      p          &= \frac{4}{7}
    \end{align*}
    i.e. two Pure Strategy Nash Equilibria:
    \begin{align*}
      PSNE=\{(T,L),(B,R)\}
    \end{align*}
  \vfill\null \columnbreak
  \vfill\null
  \end{multicols}
\end{frame}
\begin{frame}{PS3, Ex. 7.b: Mixed Strategy Nash Equilibria}
  \begin{multicols}{2}
    \begin{itemize}
      \item[(b)] Plot the mixed best responses and find all NE (pure and mixed):
    \end{itemize}
    \begin{table}
      \begin{tabular}{cl|c|c|}
        & \multicolumn{1}{c}{} & \multicolumn{2}{c}{\color{blue}Player 2}\\
        \parbox[t]{1mm}{\multirow{3}{*}{\rotatebox[origin=r]{90}{\color{red}Player 1}}}
        & \multicolumn{1}{c}{} & \multicolumn{1}{c}{L ($q$)} & \multicolumn{1}{c}{R (1-$q$)} \\\cline{3-4}
        & T  ($p$)  & \textcolor{red}{1}, \textcolor{blue}{3} & 1, 0 \\\cline{3-4}
        & B  (1-$p$)& \textcolor{red}{1}, 1 & \textcolor{red}{5}, \textcolor{blue}{5} \\\cline{3-4}
      \end{tabular}
    \end{table}
    Player 1 is indifferent if:
    \begin{align*}
      1 &= 1q + 5(1-q) \\
      5q&= 4          \\
      q &= 1
    \end{align*}
    Player 2 is indifferent if:
    \begin{align*}
      3p + 1(1-p)&= 0p+5(1-p)\\
      7p         &= 4     \\
      p          &= \frac{4}{7}
    \end{align*}
    i.e. two Pure Strategy Nash Equilibria:
    \begin{align*}
      PSNE=\{(T,L),(B,R)\}
    \end{align*}
  \vfill\null \columnbreak
    \includegraphics[width=\columnwidth]{figures/5b1}
  \vfill\null
  \end{multicols}
\end{frame}
\begin{frame}{PS3, Ex. 7.b: Mixed Strategy Nash Equilibria}
  \begin{multicols}{2}
    \begin{itemize}
      \item[(b)] Plot the mixed best responses and find all NE (pure and mixed):
    \end{itemize}
    \begin{table}
      \begin{tabular}{cl|c|c|}
        & \multicolumn{1}{c}{} & \multicolumn{2}{c}{\color{blue}Player 2}\\
        \parbox[t]{1mm}{\multirow{3}{*}{\rotatebox[origin=r]{90}{\color{red}Player 1}}}
        & \multicolumn{1}{c}{} & \multicolumn{1}{c}{L ($q$)} & \multicolumn{1}{c}{R (1-$q$)} \\\cline{3-4}
        & T  ($p$)  & \textcolor{red}{1}, \textcolor{blue}{3} & 1, 0 \\\cline{3-4}
        & B  (1-$p$)& \textcolor{red}{1}, 1 & \textcolor{red}{5}, \textcolor{blue}{5} \\\cline{3-4}
      \end{tabular}
    \end{table}
    Player 1 is indifferent if:
    \begin{align*}
      1 &= 1q + 5(1-q) \\
      5q&= 4          \\
      q &= 1
    \end{align*}
    Player 2 is indifferent if:
    \begin{align*}
      3p + 1(1-p)&= 0p+5(1-p)\\
      7p         &= 4     \\
      p          &= \frac{4}{7}
    \end{align*}
    i.e. two Pure Strategy Nash Equilibria:
    \begin{align*}
      PSNE=\{(T,L),(B,R)\}
    \end{align*}
  \vfill\null \columnbreak
    \includegraphics[width=\columnwidth]{figures/5b2}
    From drawing, the two PSNE are contained in two Mixed Strategy Nash Equilibria (MSNE):
    \begin{align*}
      (p^{*},q^{*})=\left\{(0,0)\right\}\cup\left\{(p,1):p\in\left[\frac{4}{7},1\right]\right\}
    \end{align*}
  \vfill\null
  \end{multicols}
\end{frame}
\begin{frame}{PS3, Ex. 7.c: Mixed Strategy Nash Equilibria}
  \begin{multicols}{2}
    \begin{itemize}
      \item[(c)] Plot the mixed best responses and find all NE (pure and mixed):
    \end{itemize}
    \begin{table}
      \begin{tabular}{cl|c|c|}
        & \multicolumn{1}{c}{} & \multicolumn{2}{c}{\color{blue}Player 2}\\
        \parbox[t]{1mm}{\multirow{3}{*}{\rotatebox[origin=r]{90}{\color{red}Player 1}}}
        & \multicolumn{1}{c}{} & \multicolumn{1}{c}{L ($q$)} & \multicolumn{1}{c}{R (1-$q$)} \\\cline{3-4}
        & T  ($p$)  & \textcolor{red}{3}, \textcolor{blue}{2} & \textcolor{red}{1}, \textcolor{blue}{2} \\\cline{3-4}
        & B  (1-$p$)& 0, 1 & \textcolor{red}{1}, \textcolor{blue}{2} \\\cline{3-4}
      \end{tabular}
    \end{table}
    Player 1 is indifferent if:
    \begin{align*}
      3q+(1-q) &= (1-q) \\
      q &= 0
    \end{align*}
  \vfill\null \columnbreak
  \vfill\null
  \end{multicols}
\end{frame}
\begin{frame}{PS3, Ex. 7.c: Mixed Strategy Nash Equilibria}
  \begin{multicols}{2}
    \begin{itemize}
      \item[(c)] Plot the mixed best responses and find all NE (pure and mixed):
    \end{itemize}
    \begin{table}
      \begin{tabular}{cl|c|c|}
        & \multicolumn{1}{c}{} & \multicolumn{2}{c}{\color{blue}Player 2}\\
        \parbox[t]{1mm}{\multirow{3}{*}{\rotatebox[origin=r]{90}{\color{red}Player 1}}}
        & \multicolumn{1}{c}{} & \multicolumn{1}{c}{L ($q$)} & \multicolumn{1}{c}{R (1-$q$)} \\\cline{3-4}
        & T  ($p$)  & \textcolor{red}{3}, \textcolor{blue}{2} & \textcolor{red}{1}, \textcolor{blue}{2} \\\cline{3-4}
        & B  (1-$p$)& 0, 1 & \textcolor{red}{1}, \textcolor{blue}{2} \\\cline{3-4}
      \end{tabular}
    \end{table}
    Player 1 is indifferent if:
    \begin{align*}
      3q+(1-q) &= (1-q) \\
      q &= 0
    \end{align*}
    Player 2 is indifferent if:
    \begin{align*}
      2p + (1-p) &= 2 \\
      p + 1      &= 2 \\
      p          &= 1
    \end{align*}
    i.e. three PSNE exist:
    \begin{align*}
      PSNE=\{(T,L),(T,R),(B,R)\}
    \end{align*}
  \vfill\null \columnbreak
  \vfill\null
  \end{multicols}
\end{frame}
\begin{frame}{PS3, Ex. 7.c: Mixed Strategy Nash Equilibria}
  \begin{multicols}{2}
    \begin{itemize}
      \item[(c)] Plot the mixed best responses and find all NE (pure and mixed):
    \end{itemize}
    \begin{table}
      \begin{tabular}{cl|c|c|}
        & \multicolumn{1}{c}{} & \multicolumn{2}{c}{\color{blue}Player 2}\\
        \parbox[t]{1mm}{\multirow{3}{*}{\rotatebox[origin=r]{90}{\color{red}Player 1}}}
        & \multicolumn{1}{c}{} & \multicolumn{1}{c}{L ($q$)} & \multicolumn{1}{c}{R (1-$q$)} \\\cline{3-4}
        & T  ($p$)  & \textcolor{red}{3}, \textcolor{blue}{2} & \textcolor{red}{1}, \textcolor{blue}{2} \\\cline{3-4}
        & B  (1-$p$)& 0, 1 & \textcolor{red}{1}, \textcolor{blue}{2} \\\cline{3-4}
      \end{tabular}
    \end{table}
    Player 1 is indifferent if:
    \begin{align*}
      3q+(1-q) &= (1-q) \\
      q &= 0
    \end{align*}
    Player 2 is indifferent if:
    \begin{align*}
      2p + (1-p) &= 2 \\
      p + 1      &= 2 \\
      p          &= 1
    \end{align*}
    i.e. three PSNE exist:
    \begin{align*}
      PSNE=\{(T,L),(T,R),(B,R)\}
    \end{align*}
  \vfill\null \columnbreak
    \includegraphics[width=\columnwidth]{figures/5c1}
  \vfill\null
  \end{multicols}
\end{frame}
\begin{frame}{PS3, Ex. 7.c: Mixed Strategy Nash Equilibria}
  \begin{multicols}{2}
    \begin{itemize}
      \item[(c)] Plot the mixed best responses and find all NE (pure and mixed):
    \end{itemize}
    \begin{table}
      \begin{tabular}{cl|c|c|}
        & \multicolumn{1}{c}{} & \multicolumn{2}{c}{\color{blue}Player 2}\\
        \parbox[t]{1mm}{\multirow{3}{*}{\rotatebox[origin=r]{90}{\color{red}Player 1}}}
        & \multicolumn{1}{c}{} & \multicolumn{1}{c}{L ($q$)} & \multicolumn{1}{c}{R (1-$q$)} \\\cline{3-4}
        & T  ($p$)  & \textcolor{red}{3}, \textcolor{blue}{2} & \textcolor{red}{1}, \textcolor{blue}{2} \\\cline{3-4}
        & B  (1-$p$)& 0, 1 & \textcolor{red}{1}, \textcolor{blue}{2} \\\cline{3-4}
      \end{tabular}
    \end{table}
    Player 1 is indifferent if:
    \begin{align*}
      3q+(1-q) &= (1-q) \\
      q &= 0
    \end{align*}
    Player 2 is indifferent if:
    \begin{align*}
      2p + (1-p) &= 2 \\
      p + 1      &= 2 \\
      p          &= 1
    \end{align*}
    i.e. three PSNE exist:
    \begin{align*}
      PSNE=\{(T,L),(T,R),(B,R)\}
    \end{align*}
  \vfill\null \columnbreak
    \includegraphics[width=\columnwidth]{figures/5c2}
    From drawing, we find that the three PSNE are contained in just two MSNE:
    \begin{align*}
      (p^{*},q^{*})=&\left\{(1,q):q\in(0,1]\right\}\cup\\
                    &\left\{(p,0):p\in[0,1]\right\}
    \end{align*}
  \vfill\null
  \end{multicols}
\end{frame}
\begin{frame}{PS3, Ex. 7.d: Mixed Strategy Nash Equilibria}
  \begin{multicols}{2}
    \begin{itemize}
      \item[(d)] $PSNE=\{(s_1,t_1),(s_2,t_2)\}$
    \end{itemize}
    \begin{table}
      \begin{tabular}{l|c|c|}
          \multicolumn{1}{c}{}  & \multicolumn{1}{c}{$t_1$ ($q$)} & \multicolumn{1}{c}{$t_2$ (1-$q$)} \\\cline{2-3}
          $s_1$ ($p_1$)         & \textcolor{red}{2}, \textcolor{blue}{1} & 3, 0 \\\cline{2-3}
          $s_2$ ($p_2$)         & 1, 2 & \textcolor{red}{4}, \textcolor{blue}{3} \\\cline{2-3}
          $s_3$ (1-$p_1$-$p_2$) & 0, 1 & 0, \textcolor{blue}{3} \\\cline{2-3}
      \end{tabular}
    \end{table}
  \vfill\null \columnbreak
  \vfill\null
  \end{multicols}
\end{frame}
\begin{frame}{PS3, Ex. 7.d: Mixed Strategy Nash Equilibria}
  \begin{multicols}{2}
    \begin{itemize}
      \item[(d)] $PSNE=\{(s_1,t_1),(s_2,t_2)\}$
    \end{itemize}
    \begin{table}
      \begin{tabular}{l|c|c|}
          \multicolumn{1}{c}{}  & \multicolumn{1}{c}{$t_1$ ($q$)} & \multicolumn{1}{c}{$t_2$ (1-$q$)} \\\cline{2-3}
          $s_1$ ($p_1$)         & \textcolor{red}{2}, \textcolor{blue}{1} & 3, 0 \\\cline{2-3}
          $s_2$ ($p_2$)         & 1, 2 & \textcolor{red}{4}, \textcolor{blue}{3} \\\cline{2-3}
          $s_3$ (1-$p_1$-$p_2$) & 0, 1 & 0, \textcolor{blue}{3} \\\cline{2-3}
      \end{tabular}
    \end{table}
    \textbf{IESDS}: $s_2>s_3$, thus $s_3$ can be eliminated and 1-$p_1$-$p_2=0\Rightarrow p_2=1-p_1$
    \begin{table}
      \begin{tabular}{cl|c|c|}
        & \multicolumn{1}{c}{} & \multicolumn{2}{c}{\color{blue}Player 2}\\
        \parbox[t]{1mm}{\multirow{3}{*}{\rotatebox[origin=r]{90}{\color{red}Player 1}}}
        & \multicolumn{1}{c}{}  & \multicolumn{1}{c}{$t_1$ ($q$)} & \multicolumn{1}{c}{$t_2$ (1-$q$)} \\\cline{3-4}
        & $s_1$ ($p_1$)  & \textcolor{red}{2}, \textcolor{blue}{1} & 3, 0 \\\cline{3-4}
        & $s_2$ (1-$p_1$)& 1, 2 & \textcolor{red}{4}, \textcolor{blue}{3} \\\cline{3-4}
      \end{tabular}
    \end{table}
  \vfill\null \columnbreak
  \vfill\null
  \end{multicols}
\end{frame}
\begin{frame}{PS3, Ex. 7.d: Mixed Strategy Nash Equilibria}
  \begin{multicols}{2}
    \begin{itemize}
      \item[(d)] $PSNE=\{(s_1,t_1),(s_2,t_2)\}$
    \end{itemize}
    \vspace{-8pt}
    \begin{table}
      \begin{tabular}{l|c|c|}
          \multicolumn{1}{c}{}  & \multicolumn{1}{c}{$t_1$ ($q$)} & \multicolumn{1}{c}{$t_2$ (1-$q$)} \\\cline{2-3}
          $s_1$ ($p_1$)         & \textcolor{red}{2}, \textcolor{blue}{1} & 3, 0 \\\cline{2-3}
          $s_2$ ($p_2$)         & 1, 2 & \textcolor{red}{4}, \textcolor{blue}{3} \\\cline{2-3}
          $s_3$ (1-$p_1$-$p_2$) & 0, 1 & 0, \textcolor{blue}{3} \\\cline{2-3}
      \end{tabular}
    \end{table}
    \vspace{-2pt}
    \textbf{IESDS}: $s_2>s_3$, thus $s_3$ can be eliminated and 1-$p_1$-$p_2=0\Rightarrow p_2=1-p_1$
    \vspace{-6pt}
    \begin{table}
      \begin{tabular}{cl|c|c|}
        & \multicolumn{1}{c}{} & \multicolumn{2}{c}{\color{blue}Player 2}\\
        \parbox[t]{1mm}{\multirow{3}{*}{\rotatebox[origin=r]{90}{\color{red}Player 1}}}
        & \multicolumn{1}{c}{}  & \multicolumn{1}{c}{$t_1$ ($q$)} & \multicolumn{1}{c}{$t_2$ (1-$q$)} \\\cline{3-4}
        & $s_1$ ($p_1$)  & \textcolor{red}{2}, \textcolor{blue}{1} & 3, 0 \\\cline{3-4}
        & $s_2$ (1-$p_1$)& 1, 2 & \textcolor{red}{4}, \textcolor{blue}{3} \\\cline{3-4}
      \end{tabular}
    \end{table}
    Player 1 is indifferent if:
    \begin{align*}
      2q+3(1-q) &= q+4(1-q) \\
      q &= 1-q \Rightarrow q = \frac{1}{2}
    \end{align*}
  \vfill\null \columnbreak
  \vfill\null
  \end{multicols}
\end{frame}
\begin{frame}{PS3, Ex. 7.d: Mixed Strategy Nash Equilibria}
  \begin{multicols}{2}
    \begin{itemize}
      \item[(d)] $PSNE=\{(s_1,t_1),(s_2,t_2)\}$
    \end{itemize}
    \vspace{-8pt}
    \begin{table}
      \begin{tabular}{l|c|c|}
          \multicolumn{1}{c}{}  & \multicolumn{1}{c}{$t_1$ ($q$)} & \multicolumn{1}{c}{$t_2$ (1-$q$)} \\\cline{2-3}
          $s_1$ ($p_1$)         & \textcolor{red}{2}, \textcolor{blue}{1} & 3, 0 \\\cline{2-3}
          $s_2$ ($p_2$)         & 1, 2 & \textcolor{red}{4}, \textcolor{blue}{3} \\\cline{2-3}
          $s_3$ (1-$p_1$-$p_2$) & 0, 1 & 0, \textcolor{blue}{3} \\\cline{2-3}
      \end{tabular}
    \end{table}
    \vspace{-2pt}
    \textbf{IESDS}: $s_2>s_3$, thus $s_3$ can be eliminated and 1-$p_1$-$p_2=0\Rightarrow p_2=1-p_1$
    \vspace{-6pt}
    \begin{table}
      \begin{tabular}{cl|c|c|}
        & \multicolumn{1}{c}{} & \multicolumn{2}{c}{\color{blue}Player 2}\\
        \parbox[t]{1mm}{\multirow{3}{*}{\rotatebox[origin=r]{90}{\color{red}Player 1}}}
        & \multicolumn{1}{c}{}  & \multicolumn{1}{c}{$t_1$ ($q$)} & \multicolumn{1}{c}{$t_2$ (1-$q$)} \\\cline{3-4}
        & $s_1$ ($p_1$)  & \textcolor{red}{2}, \textcolor{blue}{1} & 3, 0 \\\cline{3-4}
        & $s_2$ (1-$p_1$)& 1, 2 & \textcolor{red}{4}, \textcolor{blue}{3} \\\cline{3-4}
      \end{tabular}
    \end{table}
    Player 1 is indifferent if:
    \begin{align*}
      2q+3(1-q) &= q+4(1-q) \\
      q &= 1-q \Rightarrow q = \frac{1}{2}
    \end{align*}
    Player 2 is indifferent if:
    \begin{align*}
      p_1 + 2(1-p_1)  &= 3(1-p_1) \\
      p_1             &= 1-p_1 \Rightarrow p_1 = \frac{1}{2}
    \end{align*}
  \vfill\null \columnbreak
  \vfill\null
  \end{multicols}
\end{frame}
\begin{frame}{PS3, Ex. 7.d: Mixed Strategy Nash Equilibria}
  \begin{multicols}{2}
    \begin{itemize}
      \item[(d)] $PSNE=\{(s_1,t_1),(s_2,t_2)\}$
    \end{itemize}
    \vspace{-8pt}
    \begin{table}
      \begin{tabular}{l|c|c|}
          \multicolumn{1}{c}{}  & \multicolumn{1}{c}{$t_1$ ($q$)} & \multicolumn{1}{c}{$t_2$ (1-$q$)} \\\cline{2-3}
          $s_1$ ($p_1$)         & \textcolor{red}{2}, \textcolor{blue}{1} & 3, 0 \\\cline{2-3}
          $s_2$ ($p_2$)         & 1, 2 & \textcolor{red}{4}, \textcolor{blue}{3} \\\cline{2-3}
          $s_3$ (1-$p_1$-$p_2$) & 0, 1 & 0, \textcolor{blue}{3} \\\cline{2-3}
      \end{tabular}
    \end{table}
    \vspace{-2pt}
    \textbf{IESDS}: $s_2>s_3$, thus $s_3$ can be eliminated and 1-$p_1$-$p_2=0\Rightarrow p_2=1-p_1$
    \vspace{-6pt}
    \begin{table}
      \begin{tabular}{cl|c|c|}
        & \multicolumn{1}{c}{} & \multicolumn{2}{c}{\color{blue}Player 2}\\
        \parbox[t]{1mm}{\multirow{3}{*}{\rotatebox[origin=r]{90}{\color{red}Player 1}}}
        & \multicolumn{1}{c}{}  & \multicolumn{1}{c}{$t_1$ ($q$)} & \multicolumn{1}{c}{$t_2$ (1-$q$)} \\\cline{3-4}
        & $s_1$ ($p_1$)  & \textcolor{red}{2}, \textcolor{blue}{1} & 3, 0 \\\cline{3-4}
        & $s_2$ (1-$p_1$)& 1, 2 & \textcolor{red}{4}, \textcolor{blue}{3} \\\cline{3-4}
      \end{tabular}
    \end{table}
    Player 1 is indifferent if:
    \begin{align*}
      2q+3(1-q) &= q+4(1-q) \\
      q &= 1-q \Rightarrow q = \frac{1}{2}
    \end{align*}
    Player 2 is indifferent if:
    \begin{align*}
      p_1 + 2(1-p_1)  &= 3(1-p_1) \\
      p_1             &= 1-p_1 \Rightarrow p_1 = \frac{1}{2}
    \end{align*}
  \vfill\null \columnbreak
    \includegraphics[width=\columnwidth]{figures/5d1}
  \vfill\null
  \end{multicols}
\end{frame}
\begin{frame}{PS3, Ex. 7.d: Mixed Strategy Nash Equilibria}
  \begin{multicols}{2}
    \begin{itemize}
      \item[(d)] $PSNE=\{(s_1,t_1),(s_2,t_2)\}$
    \end{itemize}
    \vspace{-10pt}
    \begin{table}
      \begin{tabular}{l|c|c|}
          \multicolumn{1}{c}{}  & \multicolumn{1}{c}{$t_1$ ($q$)} & \multicolumn{1}{c}{$t_2$ (1-$q$)} \\\cline{2-3}
          $s_1$ ($p_1$)         & \textcolor{red}{2}, \textcolor{blue}{1} & 3, 0 \\\cline{2-3}
          $s_2$ ($p_2$)         & 1, 2 & \textcolor{red}{4}, \textcolor{blue}{3} \\\cline{2-3}
          $s_3$ (1-$p_1$-$p_2$) & 0, 1 & 0, \textcolor{blue}{3} \\\cline{2-3}
      \end{tabular}
    \end{table}
    \vspace{-4pt}
    \textbf{IESDS}: $s_2>s_3$, thus $s_3$ can be eliminated and 1-$p_1$-$p_2=0\Rightarrow p_2=1-p_1$
    \vspace{-6pt}
    \begin{table}
      \begin{tabular}{cl|c|c|}
        & \multicolumn{1}{c}{} & \multicolumn{2}{c}{\color{blue}Player 2}\\
        \parbox[t]{1mm}{\multirow{3}{*}{\rotatebox[origin=r]{90}{\color{red}Player 1}}}
        & \multicolumn{1}{c}{}  & \multicolumn{1}{c}{$t_1$ ($q$)} & \multicolumn{1}{c}{$t_2$ (1-$q$)} \\\cline{3-4}
        & $s_1$ ($p_1$)  & \textcolor{red}{2}, \textcolor{blue}{1} & 3, 0 \\\cline{3-4}
        & $s_2$ (1-$p_1$)& 1, 2 & \textcolor{red}{4}, \textcolor{blue}{3} \\\cline{3-4}
      \end{tabular}
    \end{table}
    \vspace{-2pt}
    Player 1 is indifferent if:
    \vspace{-4pt}
    \begin{align*}
      2q+3(1-q) &= q+4(1-q) \\
      q &= 1-q \Rightarrow q = \frac{1}{2}
    \end{align*}
    Player 2 is indifferent if:
    \vspace{-4pt}
    \begin{align*}
      p_1 + 2(1-p_1)  &= 3(1-p_1) \\
      p_1             &= 1-p_1 \Rightarrow p_1 = \frac{1}{2}
    \end{align*}
  \vfill\null \columnbreak
    \includegraphics[width=\columnwidth]{figures/5d2}
    In the reduced game, three MSNE exist:
    \begin{align*}
      (p_1^{*},q^{*})=\left\{(0,0),(1/2,1/2),(1,1)\right\}
    \end{align*}
    And in the full game: $\left[(p_1^{*},p_2^{*}),(q^{*})\right]=$
    \begin{align*}
      \left\{\left[(0,1),(0)\right];\left[\left(\frac{1}{2},\frac{1}{2}\right),\left(\frac{1}{2}\right)\right];\left[(1,0),(1)\right]\right\}
    \end{align*}
  \vfill\null
  \end{multicols}
\end{frame}


\section{PS3, Ex. 8: Mixed Strategy Nash Equilibria}

\begin{frame}{PS3, Ex. 8: Mixed Strategy Nash Equilibria}
  Find all (pure and mixed) Nash equilibria in the following game:
    \begin{table}
      \begin{tabular}{l|c|c|c|}
          \multicolumn{1}{c}{}  & \multicolumn{1}{c}{L ($q_1$)} & \multicolumn{1}{c}{C ($q_2$)} & \multicolumn{1}{c}{R (1-$q_1$-$q_2$)} \\\cline{2-4}
          T ($p$)   & 4, 1 & 2, 3 & 0, 4 \\\cline{2-4}
          B (1-$p$) & 2, 3 & 1, 2 & 5, 0 \\\cline{2-4}
      \end{tabular}
    \end{table}
    \textbf{Hints:}
    \begin{enumerate}
      \item Highlight the best responses in the matrix.
      \item Find the relationship between $q_1$ and $q_2$ for which \textbf{Player 1 is indifferent}.
      \item Write up the \textbf{best responses for Player 1}: $p^{*}(q_1,q_2)$, i.e. $BR_1(q_1,q_2)$.
      \item Pairwise find the probabilities $p$ for which \textbf{Player 2 is indifferent}, e.g. between $L$ and $C$, then $L$ and $R$, and finally between $C$ and $R$.
      \item Write up the \textbf{best responses for Player 2}:
      \begin{align*}
        BR_2(p)=\left(q_1^{*}(p),q_2^{*}(p)\right)=\left\{ \begin{array}{ll}
            \vdots              & \vdots  \\
            \{(0,x):x\in[0,1]\} & p = 2/3 \\
            (0,0)               & p > 2/3 \\
        \end{array}\right.
      \end{align*}
      \item \textbf{Find the NE} (pure and mixed). In MSNE both players must be indifferent between their respective pure strategies..
    \end{enumerate}
  \vfill\null
\end{frame}
\begin{frame}{PS3, Ex. 8: Mixed Strategy Nash Equilibria}
  \begin{multicols}{2}
    \begin{itemize}
      \item[1.] Highlight the best responses in the matrix:
    \end{itemize}
    \begin{table}
      \begin{tabular}{l|c|c|c|}
          \multicolumn{1}{c}{}  & \multicolumn{1}{c}{L ($q_1$)} & \multicolumn{1}{c}{C ($q_2$)} & \multicolumn{1}{c}{R (1-$q_1$-$q_2$)} \\\cline{2-4}
          T ($p$)   & \textcolor{red}{4}, 1 & \textcolor{red}{2}, 3 & 0, \textcolor{blue}{4} \\\cline{2-4}
          B (1-$p$) & 2, \textcolor{blue}{3} & 1, 2 & \textcolor{red}{5}, 0 \\\cline{2-4}
      \end{tabular}
    \end{table}
    No Pure Strategy Nash Equilibrium (PSNE) exist.
  \vfill\null \columnbreak
  \vfill\null
  \end{multicols}
\end{frame}
\begin{frame}{PS3, Ex. 8: Mixed Strategy Nash Equilibria}
  \begin{multicols}{2}
    \begin{table}
      \begin{tabular}{l|c|c|c|}
          \multicolumn{1}{c}{}  & \multicolumn{1}{c}{L ($q_1$)} & \multicolumn{1}{c}{C ($q_2$)} & \multicolumn{1}{c}{R (1-$q_1$-$q_2$)} \\\cline{2-4}
          T ($p$)   & \textcolor{red}{4}, 1 & \textcolor{red}{2}, 3 & 0, \textcolor{blue}{4} \\\cline{2-4}
          B (1-$p$) & 2, \textcolor{blue}{3} & 1, 2 & \textcolor{red}{5}, 0 \\\cline{2-4}
      \end{tabular}
    \end{table}
    \begin{itemize}
      \item[2.] Find the relationship between $q_1$ and $q_2$ for which \textbf{Player 1 is indifferent}:
    \end{itemize}
    Player 1 is indifferent if:
    \begin{align*}
      4q_1 + 2q_2 &= 2q_1 + q_2 + 5(1-q_1-q_2)\\
      7q_1 + 6q_2 &= 5 \\
      q_1 + \frac{6}{7}q_2 &= \frac{5}{7}
    \end{align*}
  \vfill\null \columnbreak
  \vfill\null
  \end{multicols}
\end{frame}
\begin{frame}{PS3, Ex. 8: Mixed Strategy Nash Equilibria}
  \begin{multicols}{2}
    \begin{table}
      \begin{tabular}{l|c|c|c|}
          \multicolumn{1}{c}{}  & \multicolumn{1}{c}{L ($q_1$)} & \multicolumn{1}{c}{C ($q_2$)} & \multicolumn{1}{c}{R (1-$q_1$-$q_2$)} \\\cline{2-4}
          T ($p$)   & \textcolor{red}{4}, 1 & \textcolor{red}{2}, 3 & 0, \textcolor{blue}{4} \\\cline{2-4}
          B (1-$p$) & 2, \textcolor{blue}{3} & 1, 2 & \textcolor{red}{5}, 0 \\\cline{2-4}
      \end{tabular}
    \end{table}
    Player 1 is indifferent if:
    \begin{align*}
      4q_1 + 2q_2 &= 2q_1 + q_2 + 5(1-q_1-q_2)\\
      7q_1 + 6q_2 &= 5 \\
      q_1 + \frac{6}{7}q_2 &= \frac{5}{7}
    \end{align*}
    \begin{itemize}
      \item[3.] Write up the \textbf{best responses for Player 1}: $p^{*}(q_1,q_2)=$, i.e.
    \end{itemize}
    \begin{align*}
      BR_1(q_1,q_2)=\left\{ \begin{array}{ll}
          1                 & q_1 + \frac{6}{7}q_2 > \frac{5}{7} \\
          \left[0,1\right]  & q_1 + \frac{6}{7}q_2 = \frac{5}{7} \\
          0                 & q_1 + \frac{6}{7}q_2 < \frac{5}{7}
      \end{array}\right.
    \end{align*}
  \vfill\null \columnbreak
  \vfill\null
  \end{multicols}
\end{frame}
\begin{frame}{PS3, Ex. 8: Mixed Strategy Nash Equilibria}
  \begin{multicols}{2}
    \begin{table}
      \begin{tabular}{l|c|c|c|}
          \multicolumn{1}{c}{}  & \multicolumn{1}{c}{L ($q_1$)} & \multicolumn{1}{c}{C ($q_2$)} & \multicolumn{1}{c}{R (1-$q_1$-$q_2$)} \\\cline{2-4}
          T ($p$)   & \textcolor{red}{4}, 1 & \textcolor{red}{2}, 3 & 0, \textcolor{blue}{4} \\\cline{2-4}
          B (1-$p$) & 2, \textcolor{blue}{3} & 1, 2 & \textcolor{red}{5}, 0 \\\cline{2-4}
      \end{tabular}
    \end{table}
    Player 1's best responses: $p^{*}(q_1,q_2)$, i.e.
    \begin{align*}
      BR_1(q_1,q_2)=\left\{ \begin{array}{ll}
          1                 & q_1 + \frac{6}{7}q_2 > \frac{5}{7}\\
          \left[0,1\right]  & q_1 + \frac{6}{7}q_2 = \frac{5}{7}\\
          0                 & q_1 + \frac{6}{7}q_2 < \frac{5}{7}
      \end{array}\right.
    \end{align*}
    \begin{itemize}
      \item[4.] Pairwise find the probabilities $p$ for which \textbf{Player 2 is indifferent}, e.g. between $L$ and $C$, then $L$ and $R$, and finally between $C$ and $R$.
    \end{itemize}
  \vfill\null \columnbreak
    Player 2 is indifferent between $L$ and $C$ if:
    \begin{align*}
      p+3(1-p)&= 3p + 2(1-p) \\
      1-p     &= 2p \\
      p       &= \frac{1}{3}
    \end{align*}
    If $p<1/3$ prefer $L$; if $p>1/3$ prefer $C$.\\\medskip
  \vfill\null
  \end{multicols}
\end{frame}
\begin{frame}{PS3, Ex. 8: Mixed Strategy Nash Equilibria}
  \begin{multicols}{2}
    \begin{table}
      \begin{tabular}{l|c|c|c|}
          \multicolumn{1}{c}{}  & \multicolumn{1}{c}{L ($q_1$)} & \multicolumn{1}{c}{C ($q_2$)} & \multicolumn{1}{c}{R (1-$q_1$-$q_2$)} \\\cline{2-4}
          T ($p$)   & \textcolor{red}{4}, 1 & \textcolor{red}{2}, 3 & 0, \textcolor{blue}{4} \\\cline{2-4}
          B (1-$p$) & 2, \textcolor{blue}{3} & 1, 2 & \textcolor{red}{5}, 0 \\\cline{2-4}
      \end{tabular}
    \end{table}
    Player 1's best responses: $p^{*}(q_1,q_2)$, i.e.
    \begin{align*}
      BR_1(q_1,q_2)=
      \left\{ \begin{array}{ll}
          1                 & q_1 + \frac{6}{7}q_2 > \frac{5}{7}\\
          \left[0,1\right]  & q_1 + \frac{6}{7}q_2 = \frac{5}{7}\\
          0                 & q_1 + \frac{6}{7}q_2 < \frac{5}{7}
      \end{array}\right.
    \end{align*}
    \begin{itemize}
      \item[4.] Pairwise find the probabilities $p$ for which \textbf{Player 2 is indifferent}, e.g. between $L$ and $C$, then $L$ and $R$, and finally between $C$ and $R$.
    \end{itemize}
  \vfill\null \columnbreak
    Player 2 is indifferent between $L$ and $C$ if:
    \begin{align*}
      p+3(1-p)&= 3p + 2(1-p) \\
      1-p     &= 2p \\
      p       &= \frac{1}{3}
    \end{align*}
    If $p<1/3$ prefer $L$; if $p>1/3$ prefer $C$.\\\medskip
    Player 2 is indifferent between $L$ and $R$ if:
    \begin{align*}
      p+3(1-p)&= 4p \\
      3       &= 6p \\
      p       &= \frac{1}{2}
    \end{align*}
    If $p<1/2$ prefer $L$; if $p>1/2$ prefer $R$.\\\medskip
  \vfill\null
  \end{multicols}
\end{frame}
\begin{frame}{PS3, Ex. 8: Mixed Strategy Nash Equilibria}
  \begin{multicols}{2}
    \begin{table}
      \begin{tabular}{l|c|c|c|}
          \multicolumn{1}{c}{}  & \multicolumn{1}{c}{L ($q_1$)} & \multicolumn{1}{c}{C ($q_2$)} & \multicolumn{1}{c}{R (1-$q_1$-$q_2$)} \\\cline{2-4}
          T ($p$)   & \textcolor{red}{4}, 1 & \textcolor{red}{2}, 3 & 0, \textcolor{blue}{4} \\\cline{2-4}
          B (1-$p$) & 2, \textcolor{blue}{3} & 1, 2 & \textcolor{red}{5}, 0 \\\cline{2-4}
      \end{tabular}
    \end{table}
    Player 1's best responses: $p^{*}(q_1,q_2)$, i.e.
    \begin{align*}
      BR_1(q_1,q_2)=
      \left\{ \begin{array}{ll}
          1                 & q_1 + \frac{6}{7}q_2 > \frac{5}{7}\\
          \left[0,1\right]  & q_1 + \frac{6}{7}q_2 = \frac{5}{7}\\
          0                 & q_1 + \frac{6}{7}q_2 < \frac{5}{7}
      \end{array}\right.
    \end{align*}
    \begin{itemize}
      \item[4.] Pairwise find the probabilities $p$ for which \textbf{Player 2 is indifferent}, e.g. between $L$ and $C$, then $L$ and $R$, and finally between $C$ and $R$.
    \end{itemize}
  \vfill\null \columnbreak
    Player 2 is indifferent between $L$ and $C$ if:
    \begin{align*}
      p+3(1-p)&= 3p + 2(1-p) \\
      1-p     &= 2p \\
      p       &= \frac{1}{3}
    \end{align*}
    If $p<1/3$ prefer $L$; if $p>1/3$ prefer $C$.\\\medskip
    Player 2 is indifferent between $L$ and $R$ if:
    \begin{align*}
      p+3(1-p)&= 4p \\
      3       &= 6p \\
      p       &= \frac{1}{2}
    \end{align*}
    If $p<1/2$ prefer $L$; if $p>1/2$ prefer $R$.\\\medskip
    Player 2 is indifferent between $C$ and $R$ if:
    \begin{align*}
      3p+2(1-p) = 4p \Leftrightarrow 2 = 3p \Leftrightarrow p = \frac{2}{3}
    \end{align*}
    If $p<2/3$ prefer $C$; if $p>2/3$ prefer $R$.\\\medskip
  \vfill\null
  \end{multicols}
\end{frame}
\begin{frame}{PS3, Ex. 8: Mixed Strategy Nash Equilibria}
  \begin{multicols}{2}
    \begin{table}
      \begin{tabular}{l|c|c|c|}
          \multicolumn{1}{c}{}  & \multicolumn{1}{c}{L ($q_1$)} & \multicolumn{1}{c}{C ($q_2$)} & \multicolumn{1}{c}{R (1-$q_1$-$q_2$)} \\\cline{2-4}
          T ($p$)   & \textcolor{red}{4}, 1 & \textcolor{red}{2}, 3 & 0, \textcolor{blue}{4} \\\cline{2-4}
          B (1-$p$) & 2, \textcolor{blue}{3} & 1, 2 & \textcolor{red}{5}, 0 \\\cline{2-4}
      \end{tabular}
    \end{table}
    Player 1's best responses: $p^{*}(q_1,q_2)$, i.e.
    \begin{align*}
      BR_1(q_1,q_2)=
      \left\{ \begin{array}{ll}
          1                 & q_1 + \frac{6}{7}q_2 > \frac{5}{7}\\
          \left[0,1\right]  & q_1 + \frac{6}{7}q_2 = \frac{5}{7}\\
          0                 & q_1 + \frac{6}{7}q_2 < \frac{5}{7}
      \end{array}\right.
    \end{align*}
    \textbf{Player 2:} $\bm{BR_2(p)=\left(q_1^{*}(p),q_2^{*}(p)\right)}=$
    \begin{align*}
      \left\{ \begin{array}{ll}
          (1,0)                 & p < 1/3 \\
          \{(x,1-x):x\in[0,1]\} & p = 1/3 \\
          \vdots                & \vdots
      \end{array}\right.
    \end{align*}
  \vfill\null \columnbreak
    Player 2 is indifferent between $L$ and $C$ if:
    \begin{align*}
      p+3(1-p)&= 3p + 2(1-p) \\
      1-p     &= 2p \\
      p       &= \frac{1}{3}
    \end{align*}
    If $p<1/3$ prefer $L$; if $p>1/3$ prefer $C$.\\\medskip
    Player 2 is indifferent between $L$ and $R$ if:
    \begin{align*}
      p+3(1-p)&= 4p \\
      3       &= 6p \\
      p       &= \frac{1}{2}
    \end{align*}
    If $p<1/2$ prefer $L$; if $p>1/2$ prefer $R$.\\\medskip
    Player 2 is indifferent between $C$ and $R$ if:
    \begin{align*}
      3p+2(1-p) = 4p \Leftrightarrow 2 = 3p \Leftrightarrow p = \frac{2}{3}
    \end{align*}
    If $p<2/3$ prefer $C$; if $p>2/3$ prefer $R$.\\\medskip
  \vfill\null
  \end{multicols}
\end{frame}
\begin{frame}{PS3, Ex. 8: Mixed Strategy Nash Equilibria}
  \begin{multicols}{2}
    \begin{table}
      \begin{tabular}{l|c|c|c|}
          \multicolumn{1}{c}{}  & \multicolumn{1}{c}{L ($q_1$)} & \multicolumn{1}{c}{C ($q_2$)} & \multicolumn{1}{c}{R (1-$q_1$-$q_2$)} \\\cline{2-4}
          T ($p$)   & \textcolor{red}{4}, 1 & \textcolor{red}{2}, 3 & 0, \textcolor{blue}{4} \\\cline{2-4}
          B (1-$p$) & 2, \textcolor{blue}{3} & 1, 2 & \textcolor{red}{5}, 0 \\\cline{2-4}
      \end{tabular}
    \end{table}
    Player 1's best responses: $p^{*}(q_1,q_2)$, i.e.
    \begin{align*}
      BR_1(q_1,q_2)=
      \left\{ \begin{array}{ll}
          1                 & q_1 + \frac{6}{7}q_2 > \frac{5}{7}\\
          \left[0,1\right]  & q_1 + \frac{6}{7}q_2 = \frac{5}{7}\\
          0                 & q_1 + \frac{6}{7}q_2 < \frac{5}{7}
      \end{array}\right.
    \end{align*}
    \textbf{Player 2:} $\bm{BR_2(p)=\left(q_1^{*}(p),q_2^{*}(p)\right)}=$
    \begin{align*}
      \left\{ \begin{array}{ll}
          (1,0)                 & p < 1/3 \\
          \{(x,1-x):x\in[0,1]\} & p = 1/3 \\
          \vdots                & \vdots
      \end{array}\right.
    \end{align*}
    Note: if $p=\frac{1}{2}:u_2(C)>u_2(L)=u_2(R)$
    \begin{align*}
      \Rightarrow\text{For }p=\frac{1}{2}:&\frac{3+2}{2}>\frac{1+3}{2}=\frac{4+0}{2}\\
          \Rightarrow&\ \ \ \frac{5}{2}\ \ >\ \ \ \frac{4}{2}\ \ =\ \ \ \frac{4}{2}
    \end{align*}
  \vfill\null \columnbreak
    Player 2 is indifferent between $L$ and $C$ if:
    \begin{align*}
      p+3(1-p)&= 3p + 2(1-p) \\
      1-p     &= 2p \\
      p       &= \frac{1}{3}
    \end{align*}
    If $p<1/3$ prefer $L$; if $p>1/3$ prefer $C$.\\\medskip
    Player 2 is indifferent between $L$ and $R$ if:
    \begin{align*}
      p+3(1-p)&= 4p \\
      3       &= 6p \\
      p       &= \frac{1}{2}
    \end{align*}
    If $p<1/2$ prefer $L$; if $p>1/2$ prefer $R$.\\\medskip
    Player 2 is indifferent between $C$ and $R$ if:
    \begin{align*}
      3p+2(1-p) = 4p \Leftrightarrow 2 = 3p \Leftrightarrow p = \frac{2}{3}
    \end{align*}
    If $p<2/3$ prefer $C$; if $p>2/3$ prefer $R$.\\\medskip
  \vfill\null
  \end{multicols}
\end{frame}
\begin{frame}{PS3, Ex. 8: Mixed Strategy Nash Equilibria}
  \begin{multicols}{2}
    \begin{table}
      \begin{tabular}{l|c|c|c|}
          \multicolumn{1}{c}{}  & \multicolumn{1}{c}{L ($q_1$)} & \multicolumn{1}{c}{C ($q_2$)} & \multicolumn{1}{c}{R (1-$q_1$-$q_2$)} \\\cline{2-4}
          T ($p$)   & \textcolor{red}{4}, 1 & \textcolor{red}{2}, 3 & 0, \textcolor{blue}{4} \\\cline{2-4}
          B (1-$p$) & 2, \textcolor{blue}{3} & 1, 2 & \textcolor{red}{5}, 0 \\\cline{2-4}
      \end{tabular}
    \end{table}
    Player 1's best responses: $p^{*}(q_1,q_2)$, i.e.
    \begin{align*}
      BR_1(q_1,q_2)=
      \left\{ \begin{array}{ll}
          1                 & q_1 + \frac{6}{7}q_2 > \frac{5}{7}\\
          \left[0,1\right]  & q_1 + \frac{6}{7}q_2 = \frac{5}{7}\\
          0                 & q_1 + \frac{6}{7}q_2 < \frac{5}{7}
      \end{array}\right.
    \end{align*}
    \textbf{Player 2:} $\bm{BR_2(p)=\left(q_1^{*}(p),q_2^{*}(p)\right)}=$
    \begin{align*}
      \left\{ \begin{array}{ll}
          (1,0)                 & p < 1/3 \\
          \{(x,1-x):x\in[0,1]\} & p = 1/3 \\
          (0,1)                 & p\in\left(\frac{1}{3},\frac{2}{3}\right)\\
          \vdots                & \vdots
      \end{array}\right.
    \end{align*}
    Note: if $p=\frac{1}{2}:u_2(C)>u_2(L)=u_2(R)$
  \vfill\null \columnbreak
    Player 2 is indifferent between $L$ and $C$ if:
    \begin{align*}
      p+3(1-p)&= 3p + 2(1-p) \\
      1-p     &= 2p \\
      p       &= \frac{1}{3}
    \end{align*}
    If $p<1/3$ prefer $L$; if $p>1/3$ prefer $C$.\\\medskip
    \sout{Player 2 is indifferent between $L$ and $R$ if:}
    \begin{align*}
      \cancel{pp+3(1-p)}&= \cancel{4p } \\
      \cancel{p3}       &= \cancel{6p } \\
      \cancel{p}        &= \cancel{\frac{1}{2} }
    \end{align*}
    \sout{If $p<1/2$ prefer $L$; if $p>1/2$ prefer $R$.}\\\medskip
    Player 2 is indifferent between $C$ and $R$ if:
    \begin{align*}
      3p+2(1-p) = 4p \Leftrightarrow 2 = 3p \Leftrightarrow p = \frac{2}{3}
    \end{align*}
    If $p<2/3$ prefer $C$; if $p>2/3$ prefer $R$.\\\medskip
  \vfill\null
  \end{multicols}
\end{frame}
\begin{frame}{PS3, Ex. 8: Mixed Strategy Nash Equilibria}
  \begin{multicols}{2}
    \begin{table}
      \begin{tabular}{l|c|c|c|}
          \multicolumn{1}{c}{}  & \multicolumn{1}{c}{L ($q_1$)} & \multicolumn{1}{c}{C ($q_2$)} & \multicolumn{1}{c}{R (1-$q_1$-$q_2$)} \\\cline{2-4}
          T ($p$)   & \textcolor{red}{4}, 1 & \textcolor{red}{2}, 3 & 0, \textcolor{blue}{4} \\\cline{2-4}
          B (1-$p$) & 2, \textcolor{blue}{3} & 1, 2 & \textcolor{red}{5}, 0 \\\cline{2-4}
      \end{tabular}
    \end{table}
    Player 1's best responses: $p^{*}(q_1,q_2)$, i.e.
    \begin{align*}
      BR_1(q_1,q_2)=
      \left\{ \begin{array}{ll}
          1                 & q_1 + \frac{6}{7}q_2 > \frac{5}{7}\\
          \left[0,1\right]  & q_1 + \frac{6}{7}q_2 = \frac{5}{7}\\
          0                 & q_1 + \frac{6}{7}q_2 < \frac{5}{7}
      \end{array}\right.
    \end{align*}
    \textbf{Player 2:} $\bm{BR_2(p)=\left(q_1^{*}(p),q_2^{*}(p)\right)}=$
    \begin{align*}
      \left\{ \begin{array}{ll}
          (1,0)                 & p < 1/3 \\
          \{(x,1-x):x\in[0,1]\} & p = 1/3 \\
          (0,1)                 & p\in\left(\frac{1}{3},\frac{2}{3}\right)\\
          \{(0,x):x\in[0,1]\}   & p = 2/3 \\
          (0,0)                 & p > 2/3
      \end{array}\right.
    \end{align*}
    Note: if $p=\frac{1}{2}:u_2(C)>u_2(L)=u_2(R)$
  \vfill\null \columnbreak
    Player 2 is indifferent between $L$ and $C$ if:
    \begin{align*}
      p+3(1-p)&= 3p + 2(1-p) \\
      1-p     &= 2p \\
      p       &= \frac{1}{3}
    \end{align*}
    If $p<1/3$ prefer $L$; if $p>1/3$ prefer $C$.\\\medskip
    \sout{Player 2 is indifferent between $L$ and $R$ if:}
    \begin{align*}
      \cancel{pp+3(1-p)}&= \cancel{4p } \\
      \cancel{p3}       &= \cancel{6p } \\
      \cancel{p}        &= \cancel{\frac{1}{2} }
    \end{align*}
    \sout{If $p<1/2$ prefer $L$; if $p>1/2$ prefer $R$.}\\\medskip
    Player 2 is indifferent between $C$ and $R$ if:
    \begin{align*}
      3p+2(1-p) = 4p \Leftrightarrow 2 = 3p \Leftrightarrow p = \frac{2}{3}
    \end{align*}
    If $p<2/3$ prefer $C$; if $p>2/3$ prefer $R$.\\\medskip
  \vfill\null
  \end{multicols}
\end{frame}
\begin{frame}{PS3, Ex. 8: Mixed Strategy Nash Equilibria}
  \begin{multicols}{2}
    \begin{table}
      \begin{tabular}{l|c|c|c|}
          \multicolumn{1}{c}{}  & \multicolumn{1}{c}{L ($q_1$)} & \multicolumn{1}{c}{C ($q_2$)} & \multicolumn{1}{c}{R (1-$q_1$-$q_2$)} \\\cline{2-4}
          T ($p$)   & \textcolor{red}{4}, 1 & \textcolor{red}{2}, 3 & 0, \textcolor{blue}{4} \\\cline{2-4}
          B (1-$p$) & 2, \textcolor{blue}{3} & 1, 2 & \textcolor{red}{5}, 0 \\\cline{2-4}
      \end{tabular}
    \end{table}
    Player 1's best responses: $p^{*}(q_1,q_2)$, i.e.
    \begin{align*}
      BR_1(q_1,q_2)=
      \left\{ \begin{array}{ll}
          1                 & q_1 + \frac{6}{7}q_2 > \frac{5}{7}\\
          \left[0,1\right]  & q_1 + \frac{6}{7}q_2 = \frac{5}{7}\\
          0                 & q_1 + \frac{6}{7}q_2 < \frac{5}{7}
      \end{array}\right.
    \end{align*}
    Player 2: $BR_2(p)=\left(q_1^{*}(p),q_2^{*}(p)\right)=$
    \begin{align*}
      \left\{ \begin{array}{ll}
          (1,0)                 & p < 1/3 \\
          \textcolor{blue}{\{(x,1-x):x\in[0,1]\}} & \textcolor{blue}{p = 1/3} \\
          (0,1)                 & p\in\left(\frac{1}{3},\frac{2}{3}\right)\\
          \{(0,x):x\in[0,1]\}   & p = 2/3 \\
          (0,0)                 & p > 2/3
      \end{array}\right.
    \end{align*}
  \vfill\null \columnbreak
    \begin{itemize}
      \item[6.] \textbf{Find the NE} (pure and mixed). In MSNE both players must be indifferent between their respective pure strategies.
    \end{itemize}
    \vspace{-8pt}
    \begin{itemize}
      \item MSNE, Case 1: $p=1/3:$
    \end{itemize}
    \vspace{-10pt}
    \begin{align*}
      BR_2\left(\frac{1}{3}\right)=\{(x,1-x):x\in[0,1]\}\Rightarrow\\
      \underbrace{x}_{q_1} + \frac{6}{7}\underbrace{1-x}_{q_2} > \frac{5}{7} \Rightarrow BR_1\left(BR_2(\frac{1}{3})\right)=1\neq\frac{1}{3}
    \end{align*}
  \vfill\null
  \end{multicols}
\end{frame}
\begin{frame}{PS3, Ex. 8: Mixed Strategy Nash Equilibria}
  \begin{multicols}{2}
    \begin{table}
      \begin{tabular}{l|c|c|c|}
          \multicolumn{1}{c}{}  & \multicolumn{1}{c}{L ($q_1$)} & \multicolumn{1}{c}{C ($q_2$)} & \multicolumn{1}{c}{R (1-$q_1$-$q_2$)} \\\cline{2-4}
          T ($p$)   & \textcolor{red}{4}, 1 & \textcolor{red}{2}, 3 & 0, \textcolor{blue}{4} \\\cline{2-4}
          B (1-$p$) & 2, \textcolor{blue}{3} & 1, 2 & \textcolor{red}{5}, 0 \\\cline{2-4}
      \end{tabular}
    \end{table}
    Player 1's best responses: $p^{*}(q_1,q_2)$, i.e.
    \begin{align*}
      BR_1(q_1,q_2)=
      \left\{ \begin{array}{ll}
          1                 & q_1 + \frac{6}{7}q_2 > \frac{5}{7}\\
          \left[0,1\right]  & q_1 + \frac{6}{7}q_2 = \frac{5}{7}\\
          0                 & q_1 + \frac{6}{7}q_2 < \frac{5}{7}
      \end{array}\right.
    \end{align*}
    Player 2: $BR_2(p)=\left(q_1^{*}(p),q_2^{*}(p)\right)=$
    \begin{align*}
      \left\{ \begin{array}{ll}
          (1,0)                 & p < 1/3 \\
          \{(x,1-x):x\in[0,1]\} & p = 1/3 \\
          (0,1)                 & p\in\left(\frac{1}{3},\frac{2}{3}\right)\\
          \textcolor{blue}{\{(0,x):x\in[0,1]\}}   & \textcolor{blue}{p = 2/3} \\
          (0,0)                 & p > 2/3
      \end{array}\right.
    \end{align*}
  \vfill\null \columnbreak
    \begin{itemize}
      \item[6.] \textbf{Find the NE} (pure and mixed). In MSNE both players must be indifferent between their respective pure strategies.
    \end{itemize}
    \vspace{-8pt}
    \begin{itemize}
      \item MSNE, Case 1: $p=1/3:$
    \end{itemize}
    \vspace{-10pt}
    \begin{align*}
      \underbrace{x}_{q_1} + \frac{6}{7}\underbrace{(1-x)}_{q_2} > \frac{5}{7}\\
      \Rightarrow BR_1\left(BR_2\left(\frac{1}{3}\right)\right)=1\neq\frac{1}{3}
    \end{align*}
    \vspace{-12pt}
    \begin{itemize}
      \item MSNE, Case 2: $p=2/3:$
    \end{itemize}
    \vspace{-10pt}
    \begin{align*}
      \underbrace{0}_{q_1} + \frac{6}{7}\underbrace{x}_{q_2} = \frac{5}{7} \Leftrightarrow x=\frac{5}{6}\\
      \Rightarrow BR_1\left(0,\frac{5}{6}\right)=[0,1]\ni\frac{2}{3}
    \end{align*}
    How many NE are there in total?
  \vfill\null
  \end{multicols}
\end{frame}
\begin{frame}{PS3, Ex. 8: Mixed Strategy Nash Equilibria}
  \begin{multicols}{2}
    \begin{table}
      \begin{tabular}{l|c|c|c|}
          \multicolumn{1}{c}{}  & \multicolumn{1}{c}{L ($q_1$)} & \multicolumn{1}{c}{C ($q_2$)} & \multicolumn{1}{c}{R (1-$q_1$-$q_2$)} \\\cline{2-4}
          T ($p$)   & \textcolor{red}{4}, 1 & \textcolor{red}{2}, 3 & 0, \textcolor{blue}{4} \\\cline{2-4}
          B (1-$p$) & 2, \textcolor{blue}{3} & 1, 2 & \textcolor{red}{5}, 0 \\\cline{2-4}
      \end{tabular}
    \end{table}
    Player 1's best responses: $p^{*}(q_1,q_2)$, i.e.
    \begin{align*}
      BR_1(q_1,q_2)=
      \left\{ \begin{array}{ll}
          1                 & q_1 + \frac{6}{7}q_2 > \frac{5}{7}\\
          \textcolor{red}{\left[0,1\right]}  & \textcolor{red}{q_1 + \frac{6}{7}q_2 = \frac{5}{7}}\\
          0                 & q_1 + \frac{6}{7}q_2 < \frac{5}{7}
      \end{array}\right.
    \end{align*}
    Player 2: $BR_2(p)=\left(q_1^{*}(p),q_2^{*}(p)\right)=$
    \begin{align*}
      \left\{ \begin{array}{ll}
          (1,0)                 & p < 1/3 \\
          \{(x,1-x):x\in[0,1]\} & p = 1/3 \\
          (0,1)                 & p\in\left(\frac{1}{3},\frac{2}{3}\right)\\
          \textcolor{blue}{\{(0,x):x\in[0,1]\}}   & \textcolor{blue}{p = 2/3} \\
          (0,0)                 & p > 2/3
      \end{array}\right.
    \end{align*}
  \vfill\null \columnbreak
    \begin{itemize}
      \item[6.] \textbf{Find the NE} (pure and mixed). In MSNE both players must be indifferent between their respective pure strategies.
    \end{itemize}
    \vspace{-8pt}
    \begin{itemize}
      \item MSNE, Case 1: $p=1/3:$
    \end{itemize}
    \vspace{-10pt}
    \begin{align*}
      \underbrace{x}_{q_1} + \frac{6}{7}\underbrace{(1-x)}_{q_2} > \frac{5}{7}\\
      \Rightarrow BR_1\left(BR_2\left(\frac{1}{3}\right)\right)=1\neq\frac{1}{3}
    \end{align*}
    \vspace{-12pt}
    \begin{itemize}
      \item MSNE, Case 2: $p=2/3:$
    \end{itemize}
    \vspace{-10pt}
    \begin{align*}
      \underbrace{0}_{q_1} + \frac{6}{7}\underbrace{x}_{q_2} = \frac{5}{7} \Leftrightarrow x=\frac{5}{6}\\
      \Rightarrow BR_1\left(0,\frac{5}{6}\right)=[0,1]\ni\frac{2}{3}
    \end{align*}
    $\Rightarrow$ $BR_2\left(\frac{2}{3}\right)=\left(0,\frac{5}{6}\right)$ is a unique MSNE:
    \vspace{-4pt}
    \begin{align*}
      \left[\left(p^{*}\right),\left(q_1^{*},q_2^{*}\right)\right]
      =\left\{\left[\left(\frac{2}{3}\right),\left(0,\frac{5}{6}\right)\right]\right\}
    \end{align*}
  \vfill\null
  \end{multicols}
\end{frame}




% \section{PS3, Ex. }
%
% \begin{frame}{PS3, Ex. }
%   \begin{multicols}{2}
%     \begin{table}
%       \begin{tabular}{cc|c|c|}
%           & \multicolumn{1}{c}{} & \multicolumn{2}{c}{Player 2}\\
%           \parbox[t]{1mm}{\multirow{3}{*}{\rotatebox[origin=r]{90}{Player 1}}}
%           & \multicolumn{1}{c}{} & \multicolumn{1}{c}{A} & \multicolumn{1}{c}{B} \\\cline{3-4}
%           & A &  &  \\\cline{3-4}
%           & B &  &  \\\cline{3-4}
%       \end{tabular}
%     \end{table}
%   \vfill\null \columnbreak
%     \begin{table}
%       \begin{tabular}{cc|c|c|}
%         & \multicolumn{1}{c}{} & \multicolumn{2}{c}{\color{blue}Player 2}\\
%         \parbox[t]{1mm}{\multirow{3}{*}{\rotatebox[origin=r]{90}{\color{red}Player 1}}}
%         & \multicolumn{1}{c}{} & \multicolumn{1}{c}{A} & \multicolumn{1}{c}{B} \\\cline{3-4}
%         & A & \textcolor{red}{}, \textcolor{blue}{} &   \\\cline{3-4}
%         & B &  &  \\\cline{3-4}
%       \end{tabular}
%     \end{table}
%     \begin{table}
%       \begin{tabular}{c|c|c|}
%           \multicolumn{1}{c}{} & \multicolumn{1}{c}{A} & \multicolumn{1}{c}{B} \\\cline{2-3}
%           A &  &  \\\cline{2-3}
%           B &  &  \\\cline{2-3}
%       \end{tabular}
%     \end{table}
%   \vfill\null
%   \end{multicols}
% \end{frame}



% \begin{frame}%{References}
%   \printbibliography
%   \includegraphics[width=1.0 \textwidth]{figures/example}
% \end{frame}

\end{document}
