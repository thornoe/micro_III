\documentclass[8pt,apectratio=169]{beamer}

\usetheme[progressbar=frametitle]{metropolis}
\usepackage{appendixnumberbeamer}
\usepackage[style=authoryear, backend=bibtex8, natbib=true, maxcitenames=2]{biblatex}

\usepackage[utf8]{inputenc} % utf8x  defines more symbols, but may cause compatible problems
\usepackage{lmodern,textcomp} % Latin Modern fonts, contains €

\usepackage{graphicx}
\usepackage{import}

\usepackage{booktabs}
\usepackage[scale=2]{ccicons}

\usepackage{pgfplots}
\usepgfplotslibrary{dateplot}

\usepackage{xspace}
\newcommand{\themename}{\textbf{\textsc{metropolis}}\xspace}

% Math
\usepackage{amsmath}
\usepackage{bm} % bold symbol in math mode
\counterwithin*{equation}{section} % reset the equation number whenever section is stepped

% Optional packages
\usepackage{xcolor}
\usepackage{multicol}
\usepackage{multirow,array}
\usepackage{subcaption} % for subfigure and subtable
\usepackage{hyperref}
\usepackage{epigraph}
\usepackage[super,negative]{nth} % allows writing 1st, 2nd, 3rd with superscript
\usepackage{ulem} % use the "sout" tag to "strikethrough" text
\usepackage{cancel} % https://tex.stackexchange.com/questions/75525/how-to-write-crossed-out-math-in-latex
\usepackage{tcolorbox}

% Select what to do with command \comment:
  % \newcommand{\comment}[1]{}  %comments not shown
  % \newcommand{\comment}[1]{\par {\bfseries \color{blue} #1 \par}} %comments shown
% Select what to do with todonotes: i.e. \todo{}, \todo[inline]{}
  % \usepackage[disable]{todonotes} % notes not shown
  % \usepackage[draft]{todonotes}   % notes shown

%\numberwithin{equation}{section}

%\addbibresource{references}

\titlegraphic{\hfill \includegraphics[width=0.15 \textwidth]{figures/logo}}
\title{Microeconomics III: Problem Set 8\footnote{Slides created for exercise class 3 and 4, with reservation for possible errors.\\}}
\author{Thor Donsby Noe (\href{mailto:thor.noe@econ.ku.dk}{thor.noe@econ.ku.dk})
        \& Christopher Borberg (\href{mailto:christopher.borberg@econ.ku.dk}{christopher.borberg@econ.ku.dk})
        }
\date{November 13 2019} % \today
\institute{\normalsize Department of Economics, University of Copenhagen}

    % \definecolor{BlueTOL}{HTML}{222255}
    \definecolor{BrownTOL}{HTML}{666633}
    \definecolor{GreenTOL}{HTML}{225522}
    % \setbeamercolor{normal text}{fg=BlueTOL,bg=white}
    \setbeamercolor{alerted text}{fg=BrownTOL}
    \setbeamercolor{example text}{fg=GreenTOL}
    \setbeamercolor{background canvas}{bg=white}

    \setbeamercolor{block title alerted}{use=alerted text,
        fg=alerted text.fg,
        bg=alerted text.bg!80!alerted text.fg}
    \setbeamercolor{block body alerted}{use={block title alerted, alerted text},
        fg=alerted text.fg,
        bg=block title alerted.bg!50!alerted text.bg}
    \setbeamercolor{block title example}{use=example text,
        fg=example text.fg,
        bg=example text.bg!80!example text.fg}
    \setbeamercolor{block body example}{use={block title example, example text},
        fg=example text.fg,
        bg=block title example.bg!50!example text.bg}

\begin{document}
\maketitle

% Select what to do with command \intuition{}:
  \newcommand{\intuition}[1]{#1} % intuition shown
  %\newcommand{\intuition}[1]{[...]}  % intuition not shown


% ------------------------------------------------------------------------------
% ------ FRAME -----------------------------------------------------------------
% ------------------------------------------------------------------------------
\begin{frame}{Outline}
    \tableofcontents
\end{frame}

\section{PS6, Ex. 1 (A): Sequential bargaining }

\begin{frame}{PS6, Ex. 1 (A): Sequential bargaining  }
Consider the sequential bargaining game discussed in Lecture 6, but now with $K \geq 1$ stages (where $K$ is some arbitrary but fixed integer). Suppose $\delta = 1$ and $K=1,2,3$. Is there a first-mover advantage? Does your answer depend on the value of K?
    \vfill\null
\end{frame}

\begin{frame}{PS6, Ex. 1 (A): Sequential bargaining }
    \textbf{Explain $\delta$ mathematically\\}
    $delta$ is the discount factor which the payoff in the next game will be multiplied by, so if there player stand to gain 1 in the next round, and $\delta=0.5$, it is only worth $1*0.5=0.5$ to the player in the current round.\\
    \textbf{Explain $\delta$ intuitively}\\ Intuitively $\delta$ is the factor showing how patient the players are. The higher $\delta$, thee less the players will mind waiting for the next round.  \\
    \textbf{Explain the case $\delta=0$}\\
    In the case $\delta=0$, the players will have their payoff multiplied by 0 in the next round, so the game turns into an ultimatum game where the first mover can offer the other player anything and they will accept. There is a first mover advantage. \\
    \textbf{Explain the case $\delta=1$}\\
    In the case $\delta=1$, the players will have their payoff multiplied by 1 in the next round, so they won't care whether the game goes for another around. This will be the case for each round until the final round, which will then be an ultimatum game where the last mover can offer the other player anything and they will accept. There is no first mover advantage, but there is a last mover advantage. \\
    \textbf{Explain whether it depends on K}\\
    For $\delta$=1, the last mover will get the whole price pool, no matter how many rounds (K) the game is. The only case with a first mover advantage is for $K=1$, in which the first move is the same as the last.\\
    \vfill\null
\end{frame}

\section{PS6, Ex. 2 (A): Infinite-horizon bargaining}


\begin{frame}{PS6, Ex. 2 (A): Infinite-horizon bargaining}
Question 2.3 from Gibbons (p.131) looks at the infinite-horizon bargaining game where player 1 has discount factor $\delta_1$ and player 2 has discount factor $\delta_2$. It shows that the backward-induction outcome of this game is

\begin{align}
    \left( \frac{1-\delta_2}{1-\delta_1\delta_2},\frac{\delta_2(1-\delta_1)}{1-\delta_1\delta_2}\right)
\end{align}

Discuss how these payoffs change as each player becomes more or less patient, i.e. as we vary $\delta_1$ and $\delta_2$. What is the intuition? Show that these payoffs simplify to those derived in Lecture 6

\begin{align}
\left(\frac{1}{1+\delta},\frac{\delta}{1+\delta}\right)
\end{align}

for the case where $\delta_1=\delta_2$
\vfill\null
\end{frame}


\begin{frame}{PS6, Ex. 2 (A): Infinite-horizon bargaining}
    \begin{itemize}
    \item[Part one:] For the payoffs: $ \left( \frac{1-\delta_2}{1-\delta_1\delta_2},\frac{\delta_2(1-\delta_1)}{1-\delta_1\delta_2}\right)$ Discuss how the payoff change as each player becomes more or less patient.
    \end{itemize}
    \vfill\null
  \begin{multicols}{2}
    \begin{itemize}
      \item[(Step 1)] Write up partial derivatives for $\delta_2$'s and $\delta_1$'s effect on the outcome for player 1, are the partial derivatives positive or negative?
      \end{itemize}
    \vfill\null \columnbreak
    Information so far:
    \begin{itemize}
    \item[]
    \end{itemize}
    \vfill\null
  \end{multicols}
    \vfill\null
\end{frame}

\begin{frame}{PS6, Ex. 2 (A): Infinite-horizon bargaining}
    \begin{itemize}
    \item[Part one:] For the payoffs: $ \left( \frac{1-\delta_2}{1-\delta_1\delta_2},\frac{\delta_2(1-\delta_1)}{1-\delta_1\delta_2}\right)$ Discuss how the payoff change as each player becomes more or less patient.
    \end{itemize}
    \vfill\null
  \begin{multicols}{2}
    \begin{itemize}
      \item[(Step 1)] Write up partial derivatives for $\delta_2$'s and $\delta_1$'s effect on the outcome for player 1, are the partial derivatives positive or negative?
      \item[(Step 2)] Use the fact that it's a zero sum game to look at the change in outcome for player 2
      \end{itemize}
    \vfill\null \columnbreak
    Information so far:
    \begin{itemize}
    \item[1] $\frac{\partial s_1*}{\partial \delta_1} = \frac{(1-\delta_2)\delta_2}{(1-\delta_1\delta_2)^2}>0 $\\
    \item[2] $\frac{\partial s_1*}{\partial \delta_2} = -\frac{1-\delta_1}{(1-\delta_1\delta_2)^2}<0 $\\
    \end{itemize}
    \vfill\null
  \end{multicols}
 \vfill\null
\end{frame}

\begin{frame}{PS6, Ex. 2 (A): Infinite-horizon bargaining}
    \begin{itemize}
    \item[Part one:] For the payoffs: $ \left( \frac{1-\delta_2}{1-\delta_1\delta_2},\frac{\delta_2(1-\delta_1)}{1-\delta_1\delta_2}\right)$ Discuss how the payoff change as each player becomes more or less patient.
    \end{itemize}
    \vfill\null
  \begin{multicols}{2}
    \begin{itemize}
      \item[(Step 1)] Write up partial derivatives for $\delta_2$'s and $\delta_1$'s effect on the outcome for player 1, are the partial derivatives positive or negative?
      \item[(Step 2)] Use the fact that it's a zero sum game to look at the change in outcome for player 2
      \item[Answer] Player 1s payoff is increasing in $\delta_1$ and decreasing in $\delta_2$, vice versa for Player 2. This intuitively makes sense, because player i's bargaining power in later rounds will increase when his patience increase relative to player j.
      \end{itemize}
    \vfill\null \columnbreak
    Information so far:
    \begin{itemize}
    \item[1] $\frac{\partial s_1*}{\partial \delta_1} = \frac{(1-\delta_2)\delta_2}{(1-\delta_1\delta_2)^2}>0 $\\
    \item[2] $\frac{\partial s_1*}{\partial \delta_2} = -\frac{1-\delta_1}{(1-\delta_1\delta_2)^2}<0 $\\
    \end{itemize}
    \vfill\null
  \end{multicols}
\vfill\null
\end{frame}


\begin{frame}{PS6, Ex. 2 (A): Infinite-horizon bargaining}
    \begin{itemize}
    \item[Part two:] For the payoffs: $ \left( \frac{1-\delta_2}{1-\delta_1\delta_2},\frac{\delta_2(1-\delta_1)}{1-\delta_1\delta_2}\right)$ show that for $\delta_2=\delta_1$ the payoffs simplify to $\left(\frac{1}{1+\delta},\frac{\delta}{1+\delta}\right)$
    \end{itemize}
    \vfill\null
    \begin{itemize}
    \item[]Write up the payoffs with $\delta=\delta_1=\delta_2$ and use that: $1-x^2=(1+x)(1-x)$, to simplify \\
    \end{itemize}
    \vfill\null
    \begin{align*}
\left(\frac{1-\delta}{1-\delta^2},\frac{\delta(1-\delta)}{1-\delta^2}\right) \Rightarrow \left(\frac{1-\delta}{(1-\delta)(1+\delta)},\frac{\delta(1-\delta)}{(1-\delta)(1+\delta)}\right) \Rightarrow \left(\frac{1}{1+\delta},\frac{\delta}{1+\delta}\right)
    \end{align*}
    \vfill\null
\end{frame}

\section{PS6, Ex. 3: Dynamic games (imperfect information)}

\begin{frame}{PS6, Ex. 3: Dynamic games (imperfect information)}
    Find the SPNE in the four games.\\\bigskip
    Hints:
    \begin{enumerate}
      \item It becomes much easier to grasp dynamic games with imperfect information if you write the part with imperfect information in normal form (bi-matrix).
      \item Be careful to cover all of the strategy profile (in every subgame!) when writing up the subgame perfect Nash Equilibria (SPNE).
    \end{enumerate}
    \vfill\null
\end{frame}


\begin{frame}{PS6, Ex. 3.a: Dynamic games (imperfect information)}
    \begin{itemize}
      \item[(a)] Find the SPNE in the following game:
    \end{itemize}
    \begin{figure}[!h]
      \center
      \def\svgwidth{.8\columnwidth}
      \import{figures/}{6b_.pdf_tex}
    \end{figure}
    \vfill\null
\end{frame}
\begin{frame}{PS6, Ex. 3.a: Dynamic games (imperfect information)}
    \begin{itemize}
      \item[(a)] Find the SPNE in the following game:
    \end{itemize}
    \vspace{-4pt}
    \begin{figure}[!h]
      \center
      \def\svgwidth{.8\columnwidth}
      \import{figures/}{6a_.pdf_tex}
    \end{figure}
    \vspace{-4pt}
    \nth{2} and \nth{3} stage in normal form:
    \vspace{-4pt}
    \begin{table}
      \begin{tabular}{cl|c|c|}
        & \multicolumn{1}{c}{} & \multicolumn{2}{c}{Player 2}\\
        \parbox[t]{1mm}{\multirow{3}{*}{\rotatebox[origin=r]{90}{Player 1}}}
        & \multicolumn{1}{c}{} & \multicolumn{1}{c}{L} & \multicolumn{1}{c}{R} \\\cline{3-4}
        & $L_2$ & -3, -1 & 1, -2 \\\cline{3-4}
        & $R_2$ & -2, 1 & 3, 0 \\\cline{3-4}
      \end{tabular}
    \end{table}
    \vfill\null
\end{frame}
\begin{frame}{PS6, Ex. 3.a: Dynamic games (imperfect information)}
    \begin{itemize}
      \item[(a)] Find the SPNE in the following game:
    \end{itemize}
    \vspace{-4pt}
    \begin{figure}[!h]
      \center
      \def\svgwidth{.8\columnwidth}
      \import{figures/}{6a_.pdf_tex}
    \end{figure}
    \vspace{-4pt}
    \nth{2} and \nth{3} stage in normal form:
    \vspace{-4pt}
    \begin{table}
      \begin{tabular}{cl|c|c|}
        & \multicolumn{1}{c}{} & \multicolumn{2}{c}{\color{blue}Player 2}\\
        \parbox[t]{1mm}{\multirow{3}{*}{\rotatebox[origin=r]{90}{\color{red}Player 1}}}
        & \multicolumn{1}{c}{} & \multicolumn{1}{c}{\textcolor{blue}{L}} & \multicolumn{1}{c}{R} \\\cline{3-4}
        & $L_2$ & -3, \textcolor{blue}{-1} & 1, -2 \\\cline{3-4}
        & \textcolor{red}{$R_2$} & \textcolor{red}{-2}, \textcolor{blue}{1} & \textcolor{red}{3}, 0 \\\cline{3-4}
      \end{tabular}
    \end{table}
    \vfill\null
\end{frame}
\begin{frame}{PS6, Ex. 3.a: Dynamic games (imperfect information)}
    \begin{itemize}
      \item[(a)] Find the SPNE in the following game:
    \end{itemize}
    \vspace{-4pt}
    \begin{figure}[!h]
      \center
      \def\svgwidth{.8\columnwidth}
      \import{figures/}{6a.pdf_tex}
    \end{figure}
    \vspace{-4pt}
    \nth{2} and \nth{3} stage in normal form:
    \vspace{-4pt}
    \begin{table}
      \begin{tabular}{cl|c|c|}
        & \multicolumn{1}{c}{} & \multicolumn{2}{c}{\color{blue}Player 2}\\
        \parbox[t]{1mm}{\multirow{3}{*}{\rotatebox[origin=r]{90}{\color{red}Player 1}}}
        & \multicolumn{1}{c}{} & \multicolumn{1}{c}{\textcolor{blue}{L}} & \multicolumn{1}{c}{R} \\\cline{3-4}
        & $L_2$ & -3, \textcolor{blue}{-1} & 1, -2 \\\cline{3-4}
        & \textcolor{red}{$R_2$} & \textcolor{red}{-2}, \textcolor{blue}{1} & \textcolor{red}{3}, 0 \\\cline{3-4}
      \end{tabular}
    \end{table}
    \textbf{\textit{Write up the SPNE!}}
    \vfill\null
\end{frame}
\begin{frame}{PS6, Ex. 3.a: Dynamic games (imperfect information)}
    \begin{itemize}
      \item[(a)] Find the SPNE in the following game:
    \end{itemize}
    \vspace{-4pt}
    \begin{figure}[!h]
      \center
      \def\svgwidth{.8\columnwidth}
      \import{figures/}{6a.pdf_tex}
    \end{figure}
    \vspace{-4pt}
    \nth{2} and \nth{3} stage in normal form:
    \vspace{-4pt}
    \begin{table}
      \begin{tabular}{cl|c|c|}
        & \multicolumn{1}{c}{} & \multicolumn{2}{c}{\color{blue}Player 2}\\
        \parbox[t]{1mm}{\multirow{3}{*}{\rotatebox[origin=r]{90}{\color{red}Player 1}}}
        & \multicolumn{1}{c}{} & \multicolumn{1}{c}{\textcolor{blue}{L}} & \multicolumn{1}{c}{R} \\\cline{3-4}
        & $L_2$ & -3, \textcolor{blue}{-1} & 1, -2 \\\cline{3-4}
        & \textcolor{red}{$R_2$} & \textcolor{red}{-2}, \textcolor{blue}{1} & \textcolor{red}{3}, 0 \\\cline{3-4}
      \end{tabular}
    \end{table}
    $SPNE=\{s_1^{*},s_2^{*}\}=\{(L_1,R_2),L\}$ with outcome (0,2).
    \vfill\null
\end{frame}

\begin{frame}{PS6, Ex. 3.b: Dynamic games (imperfect information)}
    \begin{itemize}
      \item[(b)] Find the SPNE in the following game:
    \end{itemize}
    \begin{figure}[!h]
      \center
      \def\svgwidth{.8\columnwidth}
      \import{figures/}{6b_.pdf_tex}
    \end{figure}
    \vfill\null
\end{frame}
\begin{frame}{PS6, Ex. 3.b: Dynamic games (imperfect information)}
    \begin{itemize}
      \item[(b)] Find the SPNE in the following game:
    \end{itemize}
    \vspace{-4pt}
    \begin{figure}[!h]
      \center
      \def\svgwidth{.8\columnwidth}
      \import{figures/}{6b_.pdf_tex}
    \end{figure}
    \vspace{-4pt}
    \nth{2} and \nth{3} stage in normal form:
    \vspace{-4pt}
    \begin{table}
      \begin{tabular}{cl|c|c|}
        & \multicolumn{1}{c}{} & \multicolumn{2}{c}{Player 2}\\
        \parbox[t]{1mm}{\multirow{3}{*}{\rotatebox[origin=r]{90}{Player 1}}}
        & \multicolumn{1}{c}{} & \multicolumn{1}{c}{L} & \multicolumn{1}{c}{R} \\\cline{3-4}
        & $L_2$ & -6, -6 & -1, -1 \\\cline{3-4}
        & $R_2$ & -1, -1 & -3, -3 \\\cline{3-4}
      \end{tabular}
    \end{table}
    \vfill\null
\end{frame}
\begin{frame}{PS6, Ex. 3.b: Dynamic games (imperfect information)}
    \begin{itemize}
      \item[(b)] Find the SPNE in the following game:
    \end{itemize}
    \vspace{-4pt}
    \begin{figure}[!h]
      \center
      \def\svgwidth{.8\columnwidth}
      \import{figures/}{6b_.pdf_tex}
    \end{figure}
    \vspace{-4pt}
    \nth{2} and \nth{3} stage in normal form:
    \vspace{-4pt}
    \begin{table}
      \begin{tabular}{cl|c|c|}
        & \multicolumn{1}{c}{} & \multicolumn{2}{c}{\color{blue}Player 2}\\
        \parbox[t]{1mm}{\multirow{3}{*}{\rotatebox[origin=r]{90}{\color{red}Player 1}}}
        & \multicolumn{1}{c}{} & \multicolumn{1}{c}{L} & \multicolumn{1}{c}{R} \\\cline{3-4}
        & $L_2$ & -6, -6 & \textcolor{red}{-1}, \textcolor{blue}{-1} \\\cline{3-4}
        & $R_2$ & \textcolor{red}{-1}, \textcolor{blue}{-1} & -3, -3 \\\cline{3-4}
      \end{tabular}
    \end{table}
    \textbf{\textit{Two different pure strategy NE (PSNE) in the subgame. What now?}}
    \vfill\null
\end{frame}
\begin{frame}{PS6, Ex. 3.b: Dynamic games (imperfect information)}
    \begin{itemize}
      \item[(b)] Find the SPNE in the following game:
    \end{itemize}
    $R_1$ is strictly dominated by $L_1$ and we have two subgame perfect solutions:
    \begin{multicols}{2}
      \begin{figure}[!h]
        \center
        \def\svgwidth{\columnwidth}
        \import{figures/}{6b1.pdf_tex}
      \end{figure}
      \vfill\null\columnbreak
      \begin{figure}[!h]
        \center
        \def\svgwidth{\columnwidth}
        \import{figures/}{6b2.pdf_tex}
      \end{figure}
    \end{multicols}
    \vspace{-8pt}
    \textbf{\textit{Write up the SPNE!}}
    \vfill\null
\end{frame}
\begin{frame}{PS6, Ex. 3.b: Dynamic games (imperfect information)}
    \begin{itemize}
      \item[(b)] Find the SPNE in the following game:
    \end{itemize}
    $R_1$ is strictly dominated by $L_1$ and we have two subgame perfect solutions:
    \begin{multicols}{2}
      \begin{figure}[!h]
        \center
        \def\svgwidth{\columnwidth}
        \import{figures/}{6b1.pdf_tex}
      \end{figure}
      \vfill\null\columnbreak
      \begin{figure}[!h]
        \center
        \def\svgwidth{\columnwidth}
        \import{figures/}{6b2.pdf_tex}
      \end{figure}
    \end{multicols}
    \vspace{-8pt}
    $SPNE=\{s_1^{*},s_2^{*}\}=\{(L_1,L_2),R;(L_1,R_2),L\}$ both with outcome (0,2).
    \vfill\null
\end{frame}

\begin{frame}{PS6, Ex. 3.c: Dynamic games (imperfect information)}
    \begin{itemize}
      \item[(c)] Find the SPNE in the following game:
    \end{itemize}
    \begin{figure}[!h]
      \center
      \def\svgwidth{.8\columnwidth}
      \import{figures/}{6c_.pdf_tex}
    \end{figure}
    \vfill\null
\end{frame}
\begin{frame}{PS6, Ex. 3.c: Dynamic games (imperfect information)}
    \begin{itemize}
      \item[(c)] Find the SPNE in the following game:
    \end{itemize}
    \vspace{-4pt}
    \begin{figure}[!h]
      \center
      \def\svgwidth{.8\columnwidth}
      \import{figures/}{6c_3rd.pdf_tex}
    \end{figure}
    \vspace{-4pt}
    \nth{1} and \nth{2} stage in normal form (taking the \nth{3} stage as given):
    \vspace{-4pt}
    \begin{table}
      \begin{tabular}{cl|c|c|}
        & \multicolumn{1}{c}{} & \multicolumn{2}{c}{Player 2}\\
        \parbox[t]{1mm}{\multirow{3}{*}{\rotatebox[origin=r]{90}{Player 1}}}
        & \multicolumn{1}{c}{} & \multicolumn{1}{c}{L} & \multicolumn{1}{c}{R} \\\cline{3-4}
        & $L_1$ & 2, 1 & 3, 0 \\\cline{3-4}
        & $R_1$ & 3, 1 & 3, -1 \\\cline{3-4}
      \end{tabular}
    \end{table}
    \vfill\null
\end{frame}
\begin{frame}{PS6, Ex. 3.c: Dynamic games (imperfect information)}
    \begin{itemize}
      \item[(c)] Find the SPNE in the following game:
    \end{itemize}
    \vspace{-4pt}
    \begin{figure}[!h]
      \center
      \def\svgwidth{.8\columnwidth}
      \import{figures/}{6c.pdf_tex}
    \end{figure}
    \vspace{-4pt}
    \nth{1} and \nth{2} stage in normal form (taking the \nth{3} stage as given):
    \vspace{-4pt}
    \begin{table}
      \begin{tabular}{cl|c|c|}
        & \multicolumn{1}{c}{} & \multicolumn{2}{c}{\color{blue}Player 2}\\
        \parbox[t]{1mm}{\multirow{3}{*}{\rotatebox[origin=r]{90}{\color{red}Player 1}}}
        & \multicolumn{1}{c}{} & \multicolumn{1}{c}{\textcolor{blue}{L}} & \multicolumn{1}{c}{R} \\\cline{3-4}
        & $L_1$ & 2, \textcolor{blue}{1} & \textcolor{red}{3}, 0 \\\cline{3-4}
        & $R_1$ & \textcolor{red}{3}, \textcolor{blue}{1} & \textcolor{red}{3}, -1 \\\cline{3-4}
      \end{tabular}
    \end{table}
    \textbf{\textit{Consider how many subgames there are and write up the SPNE.}}
    \vfill\null
\end{frame}
\begin{frame}{PS6, Ex. 3.c: Dynamic games (imperfect information)}
    \begin{itemize}
      \item[(c)] Find the SPNE in the following game:
    \end{itemize}
    \vspace{-4pt}
    \begin{figure}[!h]
      \center
      \def\svgwidth{.8\columnwidth}
      \import{figures/}{6c.pdf_tex}
    \end{figure}
    \vspace{-4pt}
    \nth{1} and \nth{2} stage in normal form (taking the \nth{3} stage as given):
    \vspace{-4pt}
    \begin{table}
      \begin{tabular}{cl|c|c|}
        & \multicolumn{1}{c}{} & \multicolumn{2}{c}{\color{blue}Player 2}\\
        \parbox[t]{1mm}{\multirow{3}{*}{\rotatebox[origin=r]{90}{\color{red}Player 1}}}
        & \multicolumn{1}{c}{} & \multicolumn{1}{c}{\textcolor{blue}{L}} & \multicolumn{1}{c}{R} \\\cline{3-4}
        & $L_1$ & 2, \textcolor{blue}{1} & \textcolor{red}{3}, 0 \\\cline{3-4}
        & $R_1$ & \textcolor{red}{3}, \textcolor{blue}{1} & \textcolor{red}{3}, -1 \\\cline{3-4}
      \end{tabular}
    \end{table}
    $SPNE=\{s_1^{*},s_2^{*}\}=\{(R_1,L_2,R_2),L\}$ with outcome (3,1).
    \vfill\null
\end{frame}

\begin{frame}{PS6, Ex. 3.d: Dynamic games (imperfect information)}
    \begin{itemize}
      \item[(d)] Find the SPNE in the following game:
    \end{itemize}
    \begin{figure}[!h]
      \center
      \def\svgwidth{.8\columnwidth}
      \import{figures/}{6d_.pdf_tex}
    \end{figure}
    \vfill\null
\end{frame}
\begin{frame}{PS6, Ex. 3.d: Dynamic games (imperfect information)}
    \begin{itemize}
      \item[(d)] Find the SPNE in the following game:
    \end{itemize}
    \vspace{-4pt}
    \begin{figure}[!h]
      \center
      \def\svgwidth{.8\columnwidth}
      \import{figures/}{6d_.pdf_tex}
    \end{figure}
    \vspace{-4pt}
    \nth{2} and \nth{3} stage in normal form (Player 1 knows her own action in \nth{1} stage):
    \vspace{-4pt}
    \begin{figure}[!h]
      \center
      \def\svgwidth{.25\columnwidth}
      \import{figures/}{6d_reduced_.pdf_tex}
    \end{figure}
    \vspace{-9pt}
    \begin{table}
      \begin{tabular}{l|c|c|}
        \multicolumn{1}{c}{} & \multicolumn{1}{c}{$L$} & \multicolumn{1}{c}{$R$} \\\cline{2-3}
        $L_2$ & 2, 1 & 3, 0 \\\cline{2-3}
        $R_2$ & 2, 1 & 3, 0 \\\cline{2-3}
      \end{tabular}
      \enskip
      \begin{tabular}{l|c|c|}
        \multicolumn{1}{c}{} & \multicolumn{1}{c}{$L$} & \multicolumn{1}{c}{$R$} \\\cline{2-3}
        $L_2$ & 3, 1 & 0, 0 \\\cline{2-3}
        $R_2$ & 0, 0 & 3, -1 \\\cline{2-3}
      \end{tabular}
    \end{table}
    \vfill\null
\end{frame}
\begin{frame}{PS6, Ex. 3.d: Dynamic games (imperfect information)}
    \begin{itemize}
      \item[(d)] Find the SPNE in the following game:
    \end{itemize}
    \vspace{-4pt}
    \begin{figure}[!h]
      \center
      \def\svgwidth{.8\columnwidth}
      \import{figures/}{6d_.pdf_tex}
    \end{figure}
    \vspace{-4pt}
    \nth{2} and \nth{3} stage in normal form (Player 1 knows her own action in \nth{1} stage):
    \vspace{-4pt}
    \begin{figure}[!h]
      \center
      \def\svgwidth{.25\columnwidth}
      \import{figures/}{6d_reduced.pdf_tex}
    \end{figure}
    \vspace{-9pt}
    \begin{table}
      \begin{tabular}{l|c|c|}
        \multicolumn{1}{c}{} & \multicolumn{1}{c}{\color{blue}$L$} & \multicolumn{1}{c}{$R$} \\\cline{2-3}
        $L_2$ & \textcolor{red}{2}, \textcolor{blue}{1} & \textcolor{red}{3}, 0 \\\cline{2-3}
        $R_2$ & \textcolor{red}{2}, \textcolor{blue}{1} & \textcolor{red}{3}, 0 \\\cline{2-3}
      \end{tabular}
      \enskip
      \begin{tabular}{l|c|c|}
        \multicolumn{1}{c}{} & \multicolumn{1}{c}{\color{blue}$L$} & \multicolumn{1}{c}{$R$} \\\cline{2-3}
        $L_2$ & \textcolor{red}{3}, \textcolor{blue}{1} & 0, 0 \\\cline{2-3}
        $R_2$ & 0, \textcolor{blue}{0} & \textcolor{red}{3}, -1 \\\cline{2-3}
      \end{tabular}
    \end{table}
    \vfill\null
\end{frame}
\begin{frame}{PS6, Ex. 3.d: Dynamic games (imperfect information)}
    \begin{itemize}
      \item[(d)] Find the SPNE in the following game:
    \end{itemize}
    \vspace{-4pt}
    \begin{figure}[!h]
      \center
      \def\svgwidth{.8\columnwidth}
      \import{figures/}{6d.pdf_tex}
    \end{figure}
    \vspace{-4pt}
    \nth{2} and \nth{3} stage in normal form (Player 1 knows her own action in \nth{1} stage):
    \vspace{-4pt}
    \begin{figure}[!h]
      \center
      \def\svgwidth{.25\columnwidth}
      \import{figures/}{6d_reduced.pdf_tex}
    \end{figure}
    \vspace{-9pt}
    \begin{table}
      \begin{tabular}{l|c|c|}
        \multicolumn{1}{c}{} & \multicolumn{1}{c}{\color{blue}$L$} & \multicolumn{1}{c}{$R$} \\\cline{2-3}
        $L_2$ & \textcolor{red}{2}, \textcolor{blue}{1} & \textcolor{red}{3}, 0 \\\cline{2-3}
        $R_2$ & \textcolor{red}{2}, \textcolor{blue}{1} & \textcolor{red}{3}, 0 \\\cline{2-3}
      \end{tabular}
      \enskip
      \begin{tabular}{l|c|c|}
        \multicolumn{1}{c}{} & \multicolumn{1}{c}{\color{blue}$L$} & \multicolumn{1}{c}{$R$} \\\cline{2-3}
        $L_2$ & \textcolor{red}{3}, \textcolor{blue}{1} & 0, 0 \\\cline{2-3}
        $R_2$ & 0, \textcolor{blue}{0} & \textcolor{red}{3}, -1 \\\cline{2-3}
      \end{tabular}
    \end{table}
    \vspace{-4pt}
    \textbf{\textit{Write up the SPNE!}}
    \vfill\null
\end{frame}
\begin{frame}{PS6, Ex. 3.d: Dynamic games (imperfect information)}
    \begin{itemize}
      \item[(d)] Find the SPNE in the following game:
    \end{itemize}
    \vspace{-4pt}
    \begin{figure}[!h]
      \center
      \def\svgwidth{.8\columnwidth}
      \import{figures/}{6d.pdf_tex}
    \end{figure}
    \vspace{-4pt}
    \nth{2} and \nth{3} stage in normal form (Player 1 knows her own action in \nth{1} stage):
    \vspace{-4pt}
    \begin{figure}[!h]
      \center
      \def\svgwidth{.25\columnwidth}
      \import{figures/}{6d_reduced.pdf_tex}
    \end{figure}
    \vspace{-9pt}
    \begin{table}
      \begin{tabular}{l|c|c|}
        \multicolumn{1}{c}{} & \multicolumn{1}{c}{\color{blue}$L$} & \multicolumn{1}{c}{$R$} \\\cline{2-3}
        $L_2$ & \textcolor{red}{2}, \textcolor{blue}{1} & \textcolor{red}{3}, 0 \\\cline{2-3}
        $R_2$ & \textcolor{red}{2}, \textcolor{blue}{1} & \textcolor{red}{3}, 0 \\\cline{2-3}
      \end{tabular}
      \enskip
      \begin{tabular}{l|c|c|}
        \multicolumn{1}{c}{} & \multicolumn{1}{c}{\color{blue}$L$} & \multicolumn{1}{c}{$R$} \\\cline{2-3}
        $L_2$ & \textcolor{red}{3}, \textcolor{blue}{1} & 0, 0 \\\cline{2-3}
        $R_2$ & 0, \textcolor{blue}{0} & \textcolor{red}{3}, -1 \\\cline{2-3}
      \end{tabular}
    \end{table}
    \vspace{-4pt}
    $SPNE=\{s_1^{*},s_2^{*}\}=\{(R_1,L_2),L\}$ with outcome (3,1).
    \vfill\null
\end{frame}



\section{PS6, Ex. 4: The Mutated Seabass (imperfect information)}

\begin{frame}{PS6, Ex. 4: }
    Go back to exercise 4 in problem set 5. Write up the game tree for the situation in part (c), where the choice to acquire the weapon is not observed. Find the SPNE. What has changed?\\\bigskip
    \textbf{\textit{Last class we actually solved and discussed this part as an extension...}}
\end{frame}



\section{PS6, Ex. 5: Infinite-horizon bargaining with different discount factors}

\begin{frame}{PS6, Ex. 5: Infinite-horizon bargaining with different discount factors}
Consider Rubinstein's infinite-horizon bargaining game, but where each player has a different discount factor: $\delta_1,\delta_2$. Show that in the backwards induction outcome, player 1 offers the settlement

\begin{align}
    \left( \frac{1-\delta_2}{1-\delta_1\delta_2},\frac{\delta_2(1-\delta_1)}{1-\delta_1\delta_2}\right)
\end{align}
which player 2 accepts.
     \vfill\null
\end{frame}

\begin{frame}{PS6, Ex. 5: Infinite-horizon bargaining with different discount factors}
    \begin{itemize}
    \item[] Consider an infinite bargaining game. Show that in the BI outcome, when each player has their own discount factor $\delta_1,\delta_2 \epsilon [0;1]$, player 1 offers the settlement $ (s*,1-s*)= \left( \frac{1-\delta_2}{1-\delta_1\delta_2},\frac{\delta_2(1-\delta_1)}{1-\delta_1\delta_2}\right)$ which player 2 accepts.
    \item[(Step 1)] Start with a three stage game, but where player 1s payoff in turn 3 is denoted $s_3$. Write up the outcome for player 1 in round 3. Then use the potential outcome of round 3 to find the outcome in round 2. Do the same for round 1 with respects to round 2.
    \end{itemize}
    \vfill\null
    \begin{itemize}
        \item[Turn 3] P2 will choose to accept or decline $1-s_3\in [0;1]$: P1 proposes $s_3-1$ which P2 accepts and P1 gets $s_3$ for himself.
        \item[Turn 2] P1 will choose to accept or decline $s_2 \in [0;1]$: He will accept if $s_2 \geq s_3\delta_1$, P2 proposes $s_2 = s_3\delta_1$ which P1 accepts
        \item[Turn 1] P2 will choose to accept or decline $1-s_1 \in [0;1]$: She will accept if $1-s_1 \geq (1-s_3\delta_1)\delta_2$, P1 proposes $s_1 = 1- (1-s_3\delta_1)\delta_2$ which P2 accepts.
    \end{itemize}
     \vfill\null
\end{frame}

\begin{frame}{PS6, Ex. 5: Infinite-horizon bargaining with different discount factors}
    \begin{itemize}
    \item[] Consider an infinite bargaining game. Show that in the BI outcome, when each player has their own discount factor $\delta_1,\delta_2 \epsilon [0;1]$, player 1 offers the settlement $ (s*,1-s*)= \left( \frac{1-\delta_2}{1-\delta_1\delta_2},\frac{\delta_2(1-\delta_1)}{1-\delta_1\delta_2}\right)$ which player 2 accepts.
    \item[(Step 2)] Since the game is infinite, the players are playing the same game in turn 3 as in turn 1, ie. the outcome of turn 1 should be the same as in turn 3. Use this to find a stationary solution, where $s_{turn1}=s_{turn3}$
    \end{itemize}
    \vfill\null
  \begin{multicols}{2}
    \begin{align*}
        s^{*}&=s_1 = 1- (1-s^{*}\delta_1)\delta_2 \Rightarrow\\
        s^{*}&= 1-\delta_2+s^{*}\delta_1\delta_2 \Rightarrow\\
        s^{*}(1-\delta_1\delta_2)&= 1-\delta_2 \Rightarrow\\
        s^{*}&= \frac{1-\delta_2}{1-\delta_1\delta_2} \Rightarrow\\
        (s^{*},1-s^{*})&= \left( \frac{1-\delta_2}{1-\delta_1\delta_2},\frac{\delta_2(1-\delta_1)}{1-\delta_1\delta_2}\right)
    \end{align*}
    \vfill\null \columnbreak
    Information so far:
    \begin{itemize}
        \item[]\vspace{-8pt}
            \begin{itemize}\normalsize
            \item[Turn 1] $s_1 = 1- (1-s_3\delta_1)\delta_2$
            \item[Turn 2] $s_2 = s_3\delta_1$
            \item[Turn 3] $s_3$
            \end{itemize}
    \end{itemize}
    \vfill\null
  \end{multicols}
    \vfill\null
\end{frame}

\section{PS6, Ex. 6: Cornout, colluding to every-ones benefit?}

\begin{frame}{PS6, Ex. 6: Cornout, colluding to everyone's benefit?}
  \textbf{(A two stage game with simultaneous moves)} On July 12, 2001, the presidents of Toyota and PSA Group, Fujio Cho and Jean-Martin Folz, decided to jointly develop a small city car (...)
\end{frame}

\begin{frame}{PS6, Ex. 6.a: Cornout, colluding to everyone's benefit?}
    \begin{itemize}
    \item[(a)] Given the levels of research $x_1,x_2$, find the resulting levels of output $(q_1(x_1, x_2))$ and $(q_2(x_1, x_2))$ in the second stage.
    \end{itemize}
    \vfill\null
  \begin{multicols}{2}
    \vfill\null \columnbreak
    Information so far:
    \begin{itemize}
      \item[1] Price: $P(q_1,q_2)=2-q_1-q_2$
      \item[2] Cost production: $c_{1q} = c_{2q} = 1 - x_1 - x_2$
      \item[3] Cost research: $c_i=x_i^2$
    \end{itemize}
    \vfill\null
  \end{multicols}
\end{frame}
\begin{frame}{PS6, Ex. 6.a: Cornout, colluding to everyone's benefit?}
    \begin{itemize}
    \item[(a)] Given the levels of research $x_1,x_2$, find the resulting levels of output $(q_1(x_1, x_2))$ and $(q_2(x_1, x_2))$ in the second stage.
    \end{itemize}
    \vfill\null
  \begin{multicols}{2}
    \begin{itemize}
      \item[(Step 1)] Write up the payoff function, taking research as given.
    \end{itemize}
    \vfill\null \columnbreak
    Information so far:
    \begin{itemize}
      \item[1] Price: $P(q_1,q_2)=2-q_1-q_2$
      \item[2] Cost production: $c_{1q} = c_{2q} = 1 - x_1 - x_2$
      \item[3] Cost research: $c_i=x_i^2$
    \end{itemize}
    \vfill\null
  \end{multicols}
\end{frame}
\begin{frame}{PS6, Ex. 6.a: Cornout, colluding to everyone's benefit?}
    \begin{itemize}
    \item[(a)] Given the levels of research $x_1,x_2$, find the resulting levels of output $(q_1(x_1, x_2))$ and $(q_2(x_1, x_2))$ in the second stage.
    \end{itemize}
    \vfill\null
  \begin{multicols}{2}
    \begin{itemize}
      \item[(Step 1)] Write up the payoff function, taking research as given.
      \item[(Step 2)] Write up the FOC and find the best-response function for $q_i$.
    \end{itemize}
    \vfill\null \columnbreak
    Information so far:
    \begin{itemize}
      \item[1] Price: $P(q_1,q_2)=2-q_1-q_2$
      \item[2] Cost production: $c_{1q} = c_{2q} = 1 - x_1 - x_2$
      \item[3] Cost research: $c_i=x_i^2$
      \item[4] $Payoff_i$: $\pi_i(q_i,q_j,x_i,x_j) = (2-q_i-q_j)q_i-(1-x_i-x_j)q_i-x_i^2$
    \end{itemize}
    \vfill\null
  \end{multicols}
\end{frame}
\begin{frame}{PS6, Ex. 6.a: Cornout, colluding to everyone's benefit?}
    \begin{itemize}
    \item[(a)] Given the levels of research $x_1,x_2$, find the resulting levels of output $(q_1(x_1, x_2))$ and $(q_2(x_1, x_2))$ in the second stage.
    \end{itemize}
    \vfill\null
  \begin{multicols}{2}
    \begin{itemize}
      \item[(Step 1)] Write up the payoff function, taking research as given.
      \item[(Step 2)] Write up the FOC and find the best response function for $q_i$.
      \item[(Step 3)] Use symmetry to find the NE by setting $q_i=q_j$.
    \end{itemize}
    \vfill\null \columnbreak
    Information so far:
    \begin{itemize}
      \item[1] Price: $P(q_1,q_2)=2-q_1-q_2$
      \item[2] Cost production: $c_{1q} = c_{2q} = 1 - x_1 - x_2$
      \item[3] Cost research: $c_i=x_i^2$
      \item[4] $Payoff_i$: $\pi_i(q_i,q_j,x_i,x_j) = (2-q_i-q_j)q_i-(1-x_i-x_j)q_i-x_i^2$
      \item[6] $BR_i(q_j)$: $q_i = \frac{1}{2} - \frac{q_j}{2} + \frac{x_i+x_j}{2}$
    \end{itemize}
    \vfill\null
  \end{multicols}
\end{frame}
\begin{frame}{PS6, Ex. 6.a: Cornout, colluding to everyone's benefit?}
    \begin{itemize}
    \item[(a)] Given the levels of research $x_1,x_2$, find the resulting levels of output $(q_1(x_1, x_2))$ and $(q_2(x_1, x_2))$ in the second stage.
    \end{itemize}
    \vfill\null
  \begin{multicols}{2}
    \begin{itemize}
      \item[(Step 1)] Write up the payoff function, taking research as given.
      \item[(Step 2)] Write up the FOC and find the best response function for $q_i$.
      \item[(Step 3)] Use symmetry to find the NE by setting $q_i=q_j$:
      \begin{align*}
          q_i &= \frac{1}{2} - \frac{q_i}{2} + \frac{x_i+x_j}{2} \Rightarrow \\
          q_i &= \frac{1+x_i+x_j}{3}
      \end{align*}
      \item[NE:] \begin{math} \left(\frac{1+x_i+x_j}{3},\frac{1+x_i+x_j}{3}\right)\end{math}
    \end{itemize}
    \vfill\null \columnbreak
    Information so far:
    \begin{itemize}
      \item[1] Price: $P(q_1,q_2)=2-q_1-q_2$
      \item[2] Cost production: $c_{1q} = c_{2q} = 1 - x_1 - x_2$
      \item[3] Cost research: $c_i=x_i^2$
      \item[4] $Payoff_i$: $\pi_i(q_i,q_j,x_i,x_j) = (2-q_i-q_j)q_i-(1-x_i-x_j)q_i-x_i^2$
      \item[6] $BR_i(q_j)$: $q_i = \frac{1}{2} - \frac{q_j}{2} + \frac{x_i+x_j}{2}$
    \end{itemize}
    \vfill\null
  \end{multicols}
\end{frame}

\begin{frame}{PS6, Ex. 6.b: Cornout, colluding to everyone's benefit?}
    \begin{itemize}
    \item[(b)] Assume that the stage one decisions are made simultaneously and independently. That is, each firm i chooses $x_i$ in order to maximize its own profit (foreseeing the outcome of stage two). Using your results from (a), find the levels of research and output in the SPNE: $x_1^*,x_2^*,q_1(x_1^*,x_2^*),q_2(x_1^*,x_2^*)$.
    \end{itemize}
    \vfill\null
  \begin{multicols}{2}
    \begin{itemize}
      \item[(Step 1)] Write up the payoff function as a function of research
    \end{itemize}
    \vfill\null \columnbreak
    Information so far:
    \begin{itemize}
    \item[1] $BR_i(x_i,x_j): q_i(x_i,x_j)=\frac{1+x_i+x_j}{3}$
    \item[2] $Payoff_i$: $\pi_i(q_i,q_j,x_i,x_j) = [2-q_i-q_j-(1-x_i-x_j)]q_i-x_i^2$
    \end{itemize}
    \vfill\null
  \end{multicols}
\end{frame}

\begin{frame}{PS6, Ex. 6.b: Cornout, colluding to everyone's benefit?}
    \begin{itemize}
    \item[(b)] Assume that the stage one decisions are made simultaneously and independently. That is, each firm i chooses $x_i$ in order to maximize its own profit (foreseeing the outcome of stage two). Using your results from (a), find the levels of research and output in the SPNE: $x_1^*,x_2^*,q_1(x_1^*,x_2^*),q_2(x_1^*,x_2^*)$.
    \end{itemize}
    \vfill\null
  \begin{multicols}{2}
    \begin{itemize}
      \item[(Step 1)] Write up the payoff function as a function of research
      \item[(Step 2)] Write up the FOC for $x_i$
    \end{itemize}
    \vfill\null \columnbreak
    Information so far:
    \begin{itemize}
    \item[1] $BR_i(x_i,x_j): q_i(x_i,x_j)=\frac{1+x_i+x_j}{3}$
    \item[2] $Payoff_i$: $\pi_i(q_i,q_j,x_i,x_j) = [2-q_i-q_j-(1-x_i-x_j)]q_i-x_i^2$
    \item[3] $Payoff_i(x_1,x_2)$:
             \begin{align*}
               \pi_i =& [2-2\frac{1+x_i+x_j}{3}-(1-x_i-x_j)]\\
                      &\frac{1+x_i+x_j}{3}-x_i^2\Rightarrow\\
                              =& \frac{1+x_i+x_j}{3}\frac{1+x_i+x_j}{3}-x_i^2\Rightarrow\\
                              =& \frac{(1+x_i+x_j)^2}{9}-x_i^2
             \end{align*}
    \end{itemize}
    \vfill\null
  \end{multicols}
\end{frame}

\begin{frame}{PS6, Ex. 6.b: Cornout, colluding to everyone's benefit?}
    \begin{itemize}
    \item[(b)] Assume that the stage one decisions are made simultaneously and independently. That is, each firm i chooses $x_i$ in order to maximize its own profit (foreseeing the outcome of stage two). Using your results from (a), find the levels of research and output in the SPNE: $x_1^*,x_2^*,q_1(x_1^*,x_2^*),q_2(x_1^*,x_2^*)$.
    \end{itemize}
    \vfill\null
  \begin{multicols}{2}
    \begin{itemize}
      \item[(Step 1)] Write up the payoff function as a function of research
      \item[(Step 2)] Write up the FOC for $x_i$
      \item[(Step 3)] Use symmetry to find the SPNE by setting $x_i=x_j$ isolating $x_i$ and calculating $q_i$:
    \end{itemize}
    \vfill\null \columnbreak
    Information so far:
    \begin{itemize}
    \item[1] $BR_i(x_i,x_j): q_i(x_i,x_j)=\frac{1+x_i+x_j}{3}$
    \item[2] $Payoff_i$: $\pi_i(q_i,q_j,x_i,x_j) = [2-q_i-q_j-(1-x_i-x_j)]q_i-x_i^2$
    \item[3] $Payoff_i(x_1,x_2)$:
             \begin{align*}
               \pi_i =& [2-2\frac{1+x_i+x_j}{3}-(1-x_i-x_j)]\\
                      &\frac{1+x_i+x_j}{3}-x_i^2\Rightarrow\\
                              =& \frac{1+x_i+x_j}{3}\frac{1+x_i+x_j}{3}-x_i^2\Rightarrow\\
                              =& \frac{(1+x_i+x_j)^2}{9}-x_i^2
             \end{align*}
    \item[4] $FOC_i$: $\frac{2}{9}(1+x_i+x_j)-2x_i=0$
    \end{itemize}
    \vfill\null
  \end{multicols}
\end{frame}

\begin{frame}{PS6, Ex. 6.b: Cornout, colluding to everyone's benefit?}
    \begin{itemize}
    \item[(b)] Assume that the stage one decisions are made simultaneously and independently. That is, each firm i chooses $x_i$ in order to maximize its own profit (foreseeing the outcome of stage two). Using your results from (a), find the levels of research and output in the SPNE: $x_1^*,x_2^*,q_1(x_1^*,x_2^*),q_2(x_1^*,x_2^*)$.
    \end{itemize}
    \vfill\null
  \begin{multicols}{2}
    \begin{itemize}
      \item[(Step 1)] Write up the payoff function as a function of research
      \item[(Step 2)] Write up the FOC for $x_i$
      \item[(Step 3)] Use symmetry to find the SPNE by setting $x_i=x_j$ isolating $x_i$ and calculating $q_i$:
      \begin{align*}
          \frac{2}{9}(1+x_i+x_i)-2x_i=0 \Rightarrow x_i=\frac{1}{7}
      \end{align*}
      \item[SPNE:] \begin{math} \left(\frac{1}{7},\frac{1}{7},\frac{3}{7},\frac{3}{7}\right)\end{math}
    \end{itemize}
    \vfill\null \columnbreak
    Information so far:
    \begin{itemize}
    \item[1] $BR_i(x_i,x_j): q_i(x_i,x_j)=\frac{1+x_i+x_j}{3}$
    \item[2] $Payoff_i$: $\pi_i(q_i,q_j,x_i,x_j) = [2-q_i-q_j-(1-x_i-x_j)]q_i-x_i^2$
    \item[3] $Payoff_i(x_1,x_2)$:
             \begin{align*}
               \pi_i =& [2-2\frac{1+x_i+x_j}{3}-(1-x_i-x_j)]\\
                      &\frac{1+x_i+x_j}{3}-x_i^2\Rightarrow\\
                              =& \frac{1+x_i+x_j}{3}\frac{1+x_i+x_j}{3}-x_i^2\Rightarrow\\
                              =& \frac{(1+x_i+x_j)^2}{9}-x_i^2
             \end{align*}
    \item[4] $FOC_i$: $\frac{2}{9}(1+x_i+x_j)-2x_i=0$
    \end{itemize}
    \vfill\null
  \end{multicols}
\end{frame}

\begin{frame}{PS6, Ex. 6: Cornout, colluding to everyone's benefit?}
    \begin{itemize}
    \item[(c)] Assume now that the firms collude in the first stage. That is, they choose $x_1 and x_2$ to maximize their joint profit while taking into account that $q_1 and q_2$ will be chosen simultaneously and independently in stage two. Find the resulting levels of research and output: $x_1^**,x_2^**,q_1(x_1^**,x_2^**) and q_2(x_1^**,x_2^**)$.
    \end{itemize}
    \vfill\null
  \begin{multicols}{2}
    \begin{itemize}
      \item[(Step 1)] Write up the combined payoff function
    \end{itemize}
    \vfill\null \columnbreak
    Information so far:
    \begin{itemize}
    \item[1] $Payoff_i(x_1,x_2)$: $\pi_i(x_i,x_j) = \frac{(1+x_i+x_j)^2}{9}-x_i^2 $
    \end{itemize}
    \vfill\null
  \end{multicols}
\end{frame}

\begin{frame}{PS6, Ex. 6: Cornout, colluding to everyone's benefit?}
    \begin{itemize}
    \item[(c)] Assume now that the firms collude in the first stage. That is, they choose $x_1 and x_2$ to maximize their joint profit while taking into account that $q_1 and q_2$ will be chosen simultaneously and independently in stage two. Find the resulting levels of research and output: $x_1^**,x_2^**,q_1(x_1^**,x_2^**) and q_2(x_1^**,x_2^**)$.
    \end{itemize}
    \vfill\null
  \begin{multicols}{2}
    \begin{itemize}
      \item[(Step 1)] Write up the combined payoff function
      \item[(Step 2)] Write up the FOC for $x_i$
    \end{itemize}
    \vfill\null \columnbreak
    Information so far:
    \begin{itemize}
    \item[1] $Payoff_i(x_1,x_2)$: $\pi_i(x_i,x_j) = \frac{(1+x_i+x_j)^2}{9}-x_i^2 $
    \item[2] $\Pi(x_i,x_j= \pi_i + \pi_j = 2*\frac{(1+x_i+x_j)^2}{9}-\frac{x_i}{2}^2-\frac{x_j}{2}^2 $
    \end{itemize}
    \vfill\null
  \end{multicols}
\end{frame}

\begin{frame}{PS6, Ex. 6: Cornout, colluding to everyone's benefit?}
    \begin{itemize}
    \item[(c)] Assume now that the firms collude in the first stage. That is, they choose $x_1 and x_2$ to maximize their joint profit while taking into account that $q_1 and q_2$ will be chosen simultaneously and independently in stage two. Find the resulting levels of research and output: $x_1^**,x_2^**,q_1(x_1^**,x_2^**) and q_2(x_1^**,x_2^**)$.
    \end{itemize}
    \vfill\null
  \begin{multicols}{2}
    \begin{itemize}
      \item[(Step 1)] Write up the combined payoff function
      \item[(Step 2)] Write up the FOC for $x_i$
      \item[(Step 3)] Use symmetry to find the outcome by setting $x_i=x_j$ isolating $x_i$ and calculating $q_i$:
      \end{itemize}
    \vfill\null \columnbreak
    Information so far:
    \begin{itemize}
    \item[1] $Payoff_i(x_1,x_2)$: $\pi_i(x_i,x_j) = \frac{(1+x_i+x_j)^2}{9}-x_i^2 $
    \item[2] $\Pi(x_i,x_j= \pi_i + \pi_j = 2*\frac{(1+x_i+x_j)^2}{9}-\frac{x_i}{2}^2-\frac{x_j}{2}^2 $
    \item[3] $FOC_i$: $0=4*\frac{(1+x_i+x_j)}{9}-2x_i$
    \end{itemize}
    \vfill\null
  \end{multicols}
\end{frame}

\begin{frame}{PS6, Ex. 6: Cornout, colluding to everyone's benefit?}
    \begin{itemize}
    \item[(c)] Assume now that the firms collude in the first stage. That is, they choose $x_1 and x_2$ to maximize their joint profit while taking into account that $q_1 and q_2$ will be chosen simultaneously and independently in stage two. Find the resulting levels of research and output: $x_1^**,x_2^**,q_1(x_1^**,x_2^**) and q_2(x_1^**,x_2^**)$.
    \end{itemize}
    \vfill\null
  \begin{multicols}{2}
    \begin{itemize}
      \item[(Step 1)] Write up the combined payoff function
      \item[(Step 2)] Write up the FOC for $x_i$
      \item[(Step 3)] Use symmetry to find the outcome by setting $x_i=x_j$ isolating $x_i$ and calculating $q_i$:
      \begin{align*}
          0 = 0=4*\frac{(1+x_i+x_j)}{9}-2x_i \Rightarrow \\ x_i=\frac{2}{5}
      \end{align*}
      \item[Outcome:] \begin{math} \left(\frac{2}{5},\frac{2}{5},\frac{3}{5},\frac{3}{5}\right)\end{math}
    \end{itemize}
    \vfill\null \columnbreak
    Information so far:
    \begin{itemize}
    \item[1] $Payoff_i(x_1,x_2)$: $\pi_i(x_i,x_j) = \frac{(1+x_i+x_j)^2}{9}-x_i^2 $
    \item[2] $\Pi(x_i,x_j= \pi_i + \pi_j = 2*\frac{(1+x_i+x_j)^2}{9}-\frac{x_i}{2}^2-\frac{x_j}{2}^2 $
    \item[3] $FOC_i$: $0=4*\frac{(1+x_i+x_j)}{9}-2x_i$
    \end{itemize}
    \vfill\null
  \end{multicols}
\end{frame}

\begin{frame}{PS6, Ex. 6: Cornout, colluding to everyone's benefit?}
    \begin{itemize}
    \item[(d)] Based on your findings in (b) and (c), compare the outcomes in terms of consumer
welfare [hint: it is enough to look at total output] and firms’ profit [hint: no calculations are necessary]. Comment on the source of the difference
    \end{itemize}
    \vfill\null
Since the quantity in c is higher, this also means that the price is lower. Higher quantity and lower price means there is a higher consumer welfare.

By definition the profit in c is higher.

\vspace{-10pt}
  \begin{multicols}{2}
    \begin{itemize}
      \item Write up the combined payoff function
      \item Write up the FOC for $x_i$
      \item \intuition{The difference comes from the fact that the collusion in the first stage drives down the cost, the benefit of which is then distributed amongst companies and consumers}
    \end{itemize}
    \vfill\null \columnbreak
    Total outcome:
    \begin{itemize}
    \item[(a)] $(output_T,research_T)= \left(x_i+x_j, \frac{2}{3}(1+x_i+x_j)\right)$
    \item[(b)] $(output_T,research_T)= \left(\frac{2}{7},\frac{6}{7}\right)$
    \item[(c)] $(output_T,research_T)= \left(\frac{4}{5},\frac{6}{5}\right)$
    \end{itemize}
    \vfill\null
  \end{multicols}
\end{frame}


\section{PS6, Ex. 7: To keep or split (imperfect information)}

\begin{frame}{PS6, Ex. 7: To keep or split (imperfect information)}
  \begin{multicols}{2}
    Consider the following 2 × 2 game where payoffs are monetary:
    \begin{table}
      \begin{tabular}{l|c|c|}
          \multicolumn{1}{c}{} & \multicolumn{1}{c}{L} & \multicolumn{1}{c}{R} \\\cline{2-3}
          T & 3, 3 & 0, 4 \\\cline{2-3}
          B & 4, 0 & 1, 1 \\\cline{2-3}
      \end{tabular}
    \end{table}
    Before this game is played, Player 1 can choose whether, after the game is played, players should keep their own payoffs or split the aggregate payoff evenly between them. Player 2 observes this choice.
  \vfill\null \columnbreak
    \begin{itemize}
      \item[(a)] Write down the game tree of this two-stage game: be careful to represent the simultaneous-move game in the second stage using information sets.
      \item[(b)] Find the \underline{subgame perfect} Nash Equilibria (SPNE).
      \item[(c)] Now suppose that Player 2 cannot observe Player 1’s choice in the first stage. Draw the game tree (again using information sets) and find the \underline{subgame perfect} Nash Equilibria (SPNE).
    \end{itemize}
  \vfill\null
  \end{multicols}
\end{frame}


\begin{frame}{PS6, Ex. 7.a: To keep or split (imperfect information)}
  \begin{itemize}
    \item[(a)] Write down the game tree of this two-stage game: be careful to represent the simultaneous-move game in the second stage using information sets.
  \end{itemize}
  \vspace{-4pt}
  \nth{1} stage: Player 1 chooses Keep or Split. Player 2 observes the choice.\\\medskip
  \nth{2} stage: Player 2 chooses $L$ or $R$ ($L'$ or $R'$). The action is private information.\\\medskip
  \nth{3} stage: Player 1 chooses $T$ or $B$ ($T'$ or $B'$) without knowing what Player 2 did.
  \vspace{-4pt}
  \begin{figure}[!h]
    \center
    \def\svgwidth{.8\columnwidth}
    \import{figures/}{7a_extensive_form.pdf_tex}
  \end{figure}
  \vspace{-2pt}
  The order of stage 2 and 3 is arbitrary, but the \nth{2} stage must be private information.
  %(you can swap stage 2 and 3, then the new \nth{2} stage would be private information.)
  \vspace{-2pt}
  \begin{itemize}
    \item[(b)] \textbf{\textit{Find the subgame perfect Nash Equilibria (SPNE).}}
  \end{itemize}
\end{frame}


\begin{frame}{PS6, Ex. 7.b: To keep or split (imperfect information)}
  \begin{itemize}
    \item[(b)] Find the subgame perfect Nash Equilibria (SPNE).
  \end{itemize}
  \vspace{-4pt}
  \begin{figure}[!h]
    \center
    \def\svgwidth{.8\columnwidth}
    \import{figures/}{7a_extensive_form.pdf_tex}
  \end{figure}
  \vspace{-2pt}
  \begin{multicols}{2}
    \begin{figure}[!h]
      \center
      \def\svgwidth{.5\columnwidth}
      \import{figures/}{7b_.pdf_tex}
    \end{figure}
    \vspace{-8pt}
    \begin{table}
      \begin{tabular}{l|c|c|}
        \multicolumn{1}{c}{} & \multicolumn{1}{c}{L} & \multicolumn{1}{c}{R} \\\cline{2-3}
        T & 3, 3 & 0, 4 \\\cline{2-3}
        B & 4, 0 & 1, 1 \\\cline{2-3}
      \end{tabular}\
      \begin{tabular}{l|c|c|}
        \multicolumn{1}{c}{} & \multicolumn{1}{c}{L'} & \multicolumn{1}{c}{R'} \\\cline{2-3}
        T' & 3, 3 & 2, 2 \\\cline{2-3}
        B' & 2, 2 & 1, 1 \\\cline{2-3}
      \end{tabular}
    \end{table}
  \vfill\null \columnbreak
  \vfill\null
  \end{multicols}
\end{frame}
\begin{frame}{PS6, Ex. 7.b: To keep or split (imperfect information)}
  \begin{itemize}
    \item[(b)] Find the subgame perfect Nash Equilibria (SPNE).
  \end{itemize}
  \vspace{-4pt}
  \begin{figure}[!h]
    \center
    \def\svgwidth{.8\columnwidth}
    \import{figures/}{7a_extensive_form.pdf_tex}
  \end{figure}
  \vspace{-2pt}
  \begin{multicols}{2}
    \begin{figure}[!h]
      \center
      \def\svgwidth{.5\columnwidth}
      \import{figures/}{7b_.pdf_tex}
    \end{figure}
    \vspace{-8pt}
    \begin{table}
      \begin{tabular}{l|c|c|}
        \multicolumn{1}{c}{} & \multicolumn{1}{c}{L} & \multicolumn{1}{c}{\textcolor{blue}{R}} \\\cline{2-3}
        T & 3, 3 & 0, \textcolor{blue}{4} \\\cline{2-3}
        \textcolor{red}{B} & \textcolor{red}{4}, 0 & \textcolor{red}{1}, \textcolor{blue}{1} \\\cline{2-3}
        \end{tabular}\
        \begin{tabular}{l|c|c|}
          \multicolumn{1}{c}{} & \multicolumn{1}{c}{\textcolor{blue}{L'}} & \multicolumn{1}{c}{R'} \\\cline{2-3}
          \textcolor{red}{T'} & \textcolor{red}{3}, \textcolor{blue}{3} & \textcolor{red}{2}, 2 \\\cline{2-3}
          B' & 2, \textcolor{blue}{2} & 1, 1 \\\cline{2-3}
        \end{tabular}
    \end{table}
  \vfill\null \columnbreak
  \vfill\null
  \end{multicols}
\end{frame}
\begin{frame}{PS6, Ex. 7.b: To keep or split (imperfect information)}
  \begin{itemize}
    \item[(b)] Find the subgame perfect Nash Equilibria (SPNE).
  \end{itemize}
  \vspace{-4pt}
  \begin{figure}[!h]
    \center
    \def\svgwidth{.8\columnwidth}
    \import{figures/}{7b_extensive_form.pdf_tex}
  \end{figure}
  \vspace{-2pt}
  \begin{multicols}{2}
    \begin{figure}[!h]
      \center
      \def\svgwidth{.5\columnwidth}
      \import{figures/}{7b.pdf_tex}
    \end{figure}
    \vspace{-8pt}
    \begin{table}
      \begin{tabular}{l|c|c|}
        \multicolumn{1}{c}{} & \multicolumn{1}{c}{L} & \multicolumn{1}{c}{\textcolor{blue}{R}} \\\cline{2-3}
        T & 3, 3 & 0, \textcolor{blue}{4} \\\cline{2-3}
        \textcolor{red}{B} & \textcolor{red}{4}, 0 & \textcolor{red}{1}, \textcolor{blue}{1} \\\cline{2-3}
        \end{tabular}\
        \begin{tabular}{l|c|c|}
          \multicolumn{1}{c}{} & \multicolumn{1}{c}{\textcolor{blue}{L'}} & \multicolumn{1}{c}{R'} \\\cline{2-3}
          \textcolor{red}{T'} & \textcolor{red}{3}, \textcolor{blue}{3} & \textcolor{red}{2}, 2 \\\cline{2-3}
          B' & 2, \textcolor{blue}{2} & 1, 1 \\\cline{2-3}
        \end{tabular}
    \end{table}
  \vfill\null\columnbreak
  \vfill\null
  \textbf{\textit{Write up the full strategy profiles for the subgame perfect Nash Equilibria (SPNE).}}
  \end{multicols}
\end{frame}
\begin{frame}{PS6, Ex. 7.b: To keep or split (imperfect information)}
  \begin{itemize}
    \item[(b)] Find the subgame perfect Nash Equilibria (SPNE).
  \end{itemize}
  \vspace{-4pt}
  \begin{figure}[!h]
    \center
    \def\svgwidth{.8\columnwidth}
    \import{figures/}{7b_extensive_form.pdf_tex}
  \end{figure}
  \vspace{-2pt}
  \begin{multicols}{2}
    \begin{figure}[!h]
      \center
      \def\svgwidth{.5\columnwidth}
      \import{figures/}{7b.pdf_tex}
    \end{figure}
    \vspace{-8pt}
    \begin{table}
      \begin{tabular}{l|c|c|}
        \multicolumn{1}{c}{} & \multicolumn{1}{c}{L} & \multicolumn{1}{c}{\textcolor{blue}{R}} \\\cline{2-3}
        T & 3, 3 & 0, \textcolor{blue}{4} \\\cline{2-3}
        \textcolor{red}{B} & \textcolor{red}{4}, 0 & \textcolor{red}{1}, \textcolor{blue}{1} \\\cline{2-3}
        \end{tabular}\
        \begin{tabular}{l|c|c|}
          \multicolumn{1}{c}{} & \multicolumn{1}{c}{\textcolor{blue}{L'}} & \multicolumn{1}{c}{R'} \\\cline{2-3}
          \textcolor{red}{T'} & \textcolor{red}{3}, \textcolor{blue}{3} & \textcolor{red}{2}, 2 \\\cline{2-3}
          B' & 2, \textcolor{blue}{2} & 1, 1 \\\cline{2-3}
        \end{tabular}
    \end{table}
  \vfill\null\columnbreak
  \vspace{-8pt}
  $SPNE=\{(Split,B,T'),(R,L')\}$ with outcome (3,3).
  \vspace{-6pt}
  \begin{itemize}
    \item[(c)] Now suppose that Player 2 cannot observe Player 1’s choice in the first stage. \textbf{\textit{Draw the game tree (again using information sets)}} and find the subgame perfect Nash Equilibria (SPNE).
  \end{itemize}
  \vfill\null
  \end{multicols}
\end{frame}


\begin{frame}{PS6, Ex. 7.c: To keep or split (imperfect information)}
    \begin{itemize}
      \item[(c)] Now suppose that Player 2 cannot observe Player 1’s choice in the first stage. Draw the game tree (again using information sets) \textbf{\textit{and find the subgame perfect Nash Equilibria (SPNE).}}
    \end{itemize}
    \begin{figure}[!h]
      \center
      \def\svgwidth{.8\columnwidth}
      \import{figures/}{7c_extensive_form_.pdf_tex}
    \end{figure}
    \vfill\null
\end{frame}
\begin{frame}{PS6, Ex. 7.c: To keep or split (imperfect information)}
    \begin{itemize}
      \item[(c)] Find the subgame perfect Nash Equilibria (SPNE).
    \end{itemize}
    \vspace{-4pt}
    \begin{figure}[!h]
      \center
      \def\svgwidth{.8\columnwidth}
      \import{figures/}{7c_extensive_form_.pdf_tex}
    \end{figure}
    \vspace{-4pt}
    With \nth{2} and \nth{3} stage in normal form (Player 1 knows her own action in \nth{1} stage):
    \vspace{-4pt}
    \begin{multicols}{2}
      \begin{figure}[!h]
        \center
        \def\svgwidth{.5\columnwidth}
        \import{figures/}{7c_.pdf_tex}
      \end{figure}
      \vspace{-9pt}
      \begin{table}
        \begin{tabular}{l|c|c|}
          \multicolumn{1}{c}{} & \multicolumn{1}{c}{L} & \multicolumn{1}{c}{R} \\\cline{2-3}
          T & 3, 3 & 0, 4 \\\cline{2-3}
          B & 4, 0 & 1, 1 \\\cline{2-3}
        \end{tabular}\
        \begin{tabular}{l|c|c|}
          \multicolumn{1}{c}{} & \multicolumn{1}{c}{L} & \multicolumn{1}{c}{R} \\\cline{2-3}
          T' & 3, 3 & 2, 2 \\\cline{2-3}
          B' & 2, 2 & 1, 1 \\\cline{2-3}
        \end{tabular}
      \end{table}
    \vfill\null \columnbreak
    \vfill\null
    \textbf{\textit{How many subgames are there?}}
  \end{multicols}
\end{frame}
\begin{frame}{PS6, Ex. 7.c: To keep or split (imperfect information)}
    \begin{itemize}
      \item[(c)] Find the subgame perfect Nash Equilibria (SPNE).
    \end{itemize}
    \vspace{-4pt}
    \begin{figure}[!h]
      \center
      \def\svgwidth{.8\columnwidth}
      \import{figures/}{7c_extensive_form_.pdf_tex}
    \end{figure}
    \vspace{-8pt}
    \begin{multicols}{2}
      With \nth{2} and \nth{3} stage in normal form:
      \begin{figure}[!h]
        \center
        \def\svgwidth{.5\columnwidth}
        \import{figures/}{7c_.pdf_tex}
      \end{figure}
      \vspace{-9pt}
      \begin{table}
        \begin{tabular}{l|c|c|}
          \multicolumn{1}{c}{} & \multicolumn{1}{c}{L} & \multicolumn{1}{c}{R} \\\cline{2-3}
          T & 3, 3 & 0, 4 \\\cline{2-3}
          B & 4, 0 & 1, 1 \\\cline{2-3}
        \end{tabular}\
        \begin{tabular}{l|c|c|}
          \multicolumn{1}{c}{} & \multicolumn{1}{c}{L} & \multicolumn{1}{c}{R} \\\cline{2-3}
          T' & 3, 3 & 2, 2 \\\cline{2-3}
          B' & 2, 2 & 1, 1 \\\cline{2-3}
        \end{tabular}
      \end{table}
    \vfill\null \columnbreak
    There is only one subgame; the full game itself.\\\bigskip
    \textbf{\textit{Write up the game in normal form and solve it.}}
    \vfill\null
  \end{multicols}
\end{frame}
\begin{frame}{PS6, Ex. 7.c: To keep or split (imperfect information)}
    \begin{itemize}
      \item[(c)] Find the SPNE.
    \end{itemize}
    \vspace{-16pt}
    \begin{figure}[!h]
      \center
      \def\svgwidth{.8\columnwidth}
      \import{figures/}{7c_extensive_form_.pdf_tex}
    \end{figure}
    \vspace{-8pt}
    \begin{multicols}{2}
      With \nth{2} and \nth{3} stage in normal form:
      \vspace{-4pt}
      \begin{figure}[!h]
        \center
        \def\svgwidth{.5\columnwidth}
        \import{figures/}{7c_.pdf_tex}
      \end{figure}
      \vspace{-9pt}
      \begin{table}
        \begin{tabular}{l|c|c|}
          \multicolumn{1}{c}{} & \multicolumn{1}{c}{L} & \multicolumn{1}{c}{R} \\\cline{2-3}
          T & 3, 3 & 0, 4 \\\cline{2-3}
          B & 4, 0 & 1, 1 \\\cline{2-3}
        \end{tabular}\
        \begin{tabular}{l|c|c|}
          \multicolumn{1}{c}{} & \multicolumn{1}{c}{L} & \multicolumn{1}{c}{R} \\\cline{2-3}
          T' & 3, 3 & 2, 2 \\\cline{2-3}
          B' & 2, 2 & 1, 1 \\\cline{2-3}
        \end{tabular}
      \end{table}
    \vfill\null \columnbreak
    Full game:
    \vspace{-16pt}
    \begin{table}
      \begin{tabular}{l|c|c|}
        \multicolumn{1}{c}{} & \multicolumn{1}{c}{L} & \multicolumn{1}{c}{R} \\\cline{2-3}
        Keep, T, T' & 3, 3 & 0, 4 \\\cline{2-3}
        Keep, T, B' & 3, 3 & 0, 4 \\\cline{2-3}
        Keep, B, T' & 4, 0 & 1, 1 \\\cline{2-3}
        Keep, B, B' & 4, 0 & 1, 1 \\\cline{2-3}
        Split, T, T' & 3, 3 & 2, 2 \\\cline{2-3}
        Split, B, T' & 3, 3 & 2, 2 \\\cline{2-3}
        Split, T, B' & 2, 2 & 1, 1 \\\cline{2-3}
        Split, B, B' & 2, 2 & 1, 1 \\\cline{2-3}
      \end{tabular}
    \end{table}
    \vfill\null
  \end{multicols}
\end{frame}
\begin{frame}{PS6, Ex. 7.c: To keep or split (imperfect information)}
    \begin{itemize}
      \item[(c)] Find the SPNE.
    \end{itemize}
    \vspace{-16pt}
    \begin{figure}[!h]
      \center
      \def\svgwidth{.8\columnwidth}
      \import{figures/}{7c_extensive_form_.pdf_tex}
    \end{figure}
    \vspace{-8pt}
    \begin{multicols}{2}
      With \nth{2} and \nth{3} stage in normal form:
      \vspace{-4pt}
      \begin{figure}[!h]
        \center
        \def\svgwidth{.5\columnwidth}
        \import{figures/}{7c_.pdf_tex}
      \end{figure}
      \vspace{-9pt}
      \begin{table}
        \begin{tabular}{l|c|c|}
          \multicolumn{1}{c}{} & \multicolumn{1}{c}{L} & \multicolumn{1}{c}{R} \\\cline{2-3}
          T & 3, 3 & 0, 4 \\\cline{2-3}
          B & 4, 0 & 1, 1 \\\cline{2-3}
        \end{tabular}\
        \begin{tabular}{l|c|c|}
          \multicolumn{1}{c}{} & \multicolumn{1}{c}{L} & \multicolumn{1}{c}{R} \\\cline{2-3}
          T' & 3, 3 & 2, 2 \\\cline{2-3}
          B' & 2, 2 & 1, 1 \\\cline{2-3}
        \end{tabular}
      \end{table}
    \vfill\null \columnbreak
    Full game:
    \vspace{-16pt}
    \begin{table}
      \begin{tabular}{l|c|c|}
        \multicolumn{1}{c}{} & \multicolumn{1}{c}{L} & \multicolumn{1}{c}{R} \\\cline{2-3}
        Keep, T, T' & 3, 3 & 0, \textcolor{blue}{4} \\\cline{2-3}
        Keep, T, B' & 3, 3 & 0, \textcolor{blue}{4} \\\cline{2-3}
        Keep, B, T' & \textcolor{red}{4}, 0 & 1, \textcolor{blue}{1} \\\cline{2-3}
        Keep, B, B' & \textcolor{red}{4}, 0 & 1, \textcolor{blue}{1} \\\cline{2-3}
        Split, T, T' & 3, \textcolor{blue}{3} & \textcolor{red}{2}, 2 \\\cline{2-3}
        Split, B, T' & 3, \textcolor{blue}{3} & \textcolor{red}{2}, 2 \\\cline{2-3}
        Split, T, B' & 2, \textcolor{blue}{2} & 1, 1 \\\cline{2-3}
        Split, B, B' & 2, \textcolor{blue}{2} & 1, 1 \\\cline{2-3}
      \end{tabular}
    \end{table}
    \vfill\null
  \end{multicols}
\end{frame}
\begin{frame}{PS6, Ex. 7.c: To keep or split (imperfect information)}
    \begin{itemize}
      \item[(c)] \textit{\textbf{\underline{No SPNE exists}} (in pure strategies)}
    \end{itemize}
    \vspace{-16pt}
    \begin{figure}[!h]
      \center
      \def\svgwidth{.8\columnwidth}
      \import{figures/}{7c_extensive_form_.pdf_tex}
    \end{figure}
    \vspace{-8pt}
    \begin{multicols}{2}
      With \nth{2} and \nth{3} stage in normal form:
      \vspace{-4pt}
      \begin{figure}[!h]
        \center
        \def\svgwidth{.5\columnwidth}
        \import{figures/}{7c_.pdf_tex}
      \end{figure}
      \vspace{-9pt}
      \begin{table}
        \begin{tabular}{l|c|c|}
          \multicolumn{1}{c}{} & \multicolumn{1}{c}{L} & \multicolumn{1}{c}{R} \\\cline{2-3}
          T & 3, 3 & 0, 4 \\\cline{2-3}
          B & 4, 0 & 1, 1 \\\cline{2-3}
        \end{tabular}\
        \begin{tabular}{l|c|c|}
          \multicolumn{1}{c}{} & \multicolumn{1}{c}{L} & \multicolumn{1}{c}{R} \\\cline{2-3}
          T' & 3, 3 & 2, 2 \\\cline{2-3}
          B' & 2, 2 & 1, 1 \\\cline{2-3}
        \end{tabular}
      \end{table}
    \vfill\null \columnbreak
    Full game:
    \vspace{-16pt}
    \begin{table}
      \begin{tabular}{l|c|c|}
        \multicolumn{1}{c}{} & \multicolumn{1}{c}{L} & \multicolumn{1}{c}{R} \\\cline{2-3}
        Keep, T, T' & 3, 3 & 0, \textcolor{blue}{4} \\\cline{2-3}
        Keep, T, B' & 3, 3 & 0, \textcolor{blue}{4} \\\cline{2-3}
        Keep, B, T' & \textcolor{red}{4}, 0 & 1, \textcolor{blue}{1} \\\cline{2-3}
        Keep, B, B' & \textcolor{red}{4}, 0 & 1, \textcolor{blue}{1} \\\cline{2-3}
        Split, T, T' & 3, \textcolor{blue}{3} & \textcolor{red}{2}, 2 \\\cline{2-3}
        Split, B, T' & 3, \textcolor{blue}{3} & \textcolor{red}{2}, 2 \\\cline{2-3}
        Split, T, B' & 2, \textcolor{blue}{2} & 1, 1 \\\cline{2-3}
        Split, B, B' & 2, \textcolor{blue}{2} & 1, 1 \\\cline{2-3}
      \end{tabular}
    \end{table}
    \vfill\null
  \end{multicols}
\end{frame}




\section{Code examples} % out-comment: ctrl-shift-7 or ctrl-shift-* (use cmd for Mac)

\begin{frame}{Code examples}
  \begin{multicols}{2}
    % Game tree: % In general, I recommend drawing game trees in the hand as it is the fastest and resembles the exam situation. If you write your assignments on the computer, you can take a picture or leave space to draw the figures after printing. For the slides, I draw the game trees in Inkscape, which is a great piece of free software – when you have gotten used to it… Editing an existing game tree can be a quite straightforward start, but exporting the illustration to a LaTeX document can again be a bit cumbersome. If you’re persistent, you can find “7b_extensive_form.svg” in the figures folder of the zip-file and edit it with Inkscape. As you see, you can use LaTeX code such as $x_1$. Then you save it as type: “Portable Document Format (*.pdf)” and choose “Omit text in PDF and create LaTeX file” and “Use exported object’s size”, which creates two new files (*.pdf and *.pdf_tex). Both must be uploaded to Overleaf to even see how the figures looks, as the files make no sense on their own. To add them to your document, search for “svg” in the main.tex file and re-use my code.
    \begin{figure}[!h]
      \center
      \def\svgwidth{.8\columnwidth}
      \import{figures/}{long_.pdf_tex}
    \end{figure}
  \vfill\null \columnbreak
    Matrix, no player names:
    \vspace{-10pt}
    \begin{table} % as opposed to matrices with player names, each line does not start with "&" as there's no empty column for the name-box. Otherwise, see the explanations below.
      \begin{tabular}{l|c|c|}
        \multicolumn{1}{c}{} & \multicolumn{1}{c}{L (q)} & \multicolumn{1}{c}{R (1-q)} \\\cline{2-3}
        T (p)   &  &  \\\cline{2-3}
        B (1-p) &  &  \\\cline{2-3}
      \end{tabular}
    \end{table}
    Matrix, no colors:
    \vspace{-10pt}
    \begin{table}
      \begin{tabular}{cl|c|c|} % the number of total columns and which have vertical lines between them (left-align or center text).
        & \multicolumn{1}{c}{} & \multicolumn{2}{c}{Player 2}\\ % "2" is the number of columns in the matrix that the 2nd player name spans over
        \parbox[t]{1mm}{\multirow{3}{*}{\rotatebox[origin=r]{90}{Player 1}}} % "3" is the number of rows the 1st player name spans over (including the one with the column names)
        & \multicolumn{1}{c}{} & \multicolumn{1}{c}{L (q)} & \multicolumn{1}{c}{R (1-q)} \\\cline{3-4} % column names use the "\multicolumn" command to not draw vertical lines between them.
        & T (p)   &  &  \\\cline{3-4} % a horizontal line is drawn after the line break using "\cline{x-y}" where x and y are the column numbers of the cells to be underlined.
        & B (1-p) &  &  \\\cline{3-4}
      \end{tabular}
    \end{table}
    Matrix, with colors:
    \vspace{-10pt}
    \begin{table}
      \begin{tabular}{cl|c|c|}
        & \multicolumn{1}{c}{} & \multicolumn{2}{c}{\color{blue}Player 2}\\
        \parbox[t]{1mm}{\multirow{3}{*}{\rotatebox[origin=r]{90}{\color{red}Player 1}}}
        & \multicolumn{1}{c}{} & \multicolumn{1}{c}{L (q)} & \multicolumn{1}{c}{R (1-q)} \\\cline{3-4}
        & T (p)   & \textcolor{red}{1}, \textcolor{blue}{1} &   \\\cline{3-4}
        & B (1-p) &  &  \\\cline{3-4}
      \end{tabular}
    \end{table}
  \vfill\null
  \end{multicols}
\end{frame}


\end{document}
