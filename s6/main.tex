\documentclass[8pt,apectratio=169]{beamer}

\usetheme[progressbar=frametitle]{metropolis}
\usepackage{appendixnumberbeamer}
\usepackage[style=authoryear, backend=bibtex8, natbib=true, maxcitenames=2]{biblatex}

\usepackage[utf8]{inputenc} % utf8x  defines more symbols, but may cause compatible problems
\usepackage{lmodern,textcomp} % Latin Modern fonts, contains €

\usepackage{graphicx}
\usepackage{import}

\usepackage{booktabs}
\usepackage[scale=2]{ccicons}

\usepackage{pgfplots}
\usepgfplotslibrary{dateplot}

\usepackage{xspace}
\newcommand{\themename}{\textbf{\textsc{metropolis}}\xspace}

% Math
\usepackage{amsmath}
\usepackage{bm} % bold symbol in math mode
\counterwithin*{equation}{section} % reset the equation number whenever section is stepped

% Optional packages
\usepackage{xcolor}
\usepackage{multicol}
\usepackage{multirow,array}
\usepackage{subcaption} % for subfigure and subtable
\usepackage{hyperref}
\usepackage{epigraph}
\usepackage[super,negative]{nth} % allows writing 1st, 2nd, 3rd with superscript
\usepackage{ulem} % use the "sout" tag to "strikethrough" text
\usepackage{cancel} % https://tex.stackexchange.com/questions/75525/how-to-write-crossed-out-math-in-latex
\usepackage{tcolorbox}

% Select what to do with command \comment:
  % \newcommand{\comment}[1]{}  %comments not shown
  % \newcommand{\comment}[1]{\par {\bfseries \color{blue} #1 \par}} %comments shown
% Select what to do with todonotes: i.e. \todo{}, \todo[inline]{}
  % \usepackage[disable]{todonotes} % notes not shown
  % \usepackage[draft]{todonotes}   % notes shown

%\numberwithin{equation}{section}

%\addbibresource{references}

\titlegraphic{\hfill \includegraphics[width=0.15 \textwidth]{figures/logo}}
\title{Microeconomics III: Problem Set 8\footnote{Slides created for exercise class 3 and 4, with reservation for possible errors.\\}}
\author{Thor Donsby Noe (\href{mailto:thor.noe@econ.ku.dk}{thor.noe@econ.ku.dk})
        \& Christopher Borberg (\href{mailto:christopher.borberg@econ.ku.dk}{christopher.borberg@econ.ku.dk})
        }
\date{November 13 2019} % \today
\institute{\normalsize Department of Economics, University of Copenhagen}

    % \definecolor{BlueTOL}{HTML}{222255}
    \definecolor{BrownTOL}{HTML}{666633}
    \definecolor{GreenTOL}{HTML}{225522}
    % \setbeamercolor{normal text}{fg=BlueTOL,bg=white}
    \setbeamercolor{alerted text}{fg=BrownTOL}
    \setbeamercolor{example text}{fg=GreenTOL}
    \setbeamercolor{background canvas}{bg=white}

    \setbeamercolor{block title alerted}{use=alerted text,
        fg=alerted text.fg,
        bg=alerted text.bg!80!alerted text.fg}
    \setbeamercolor{block body alerted}{use={block title alerted, alerted text},
        fg=alerted text.fg,
        bg=block title alerted.bg!50!alerted text.bg}
    \setbeamercolor{block title example}{use=example text,
        fg=example text.fg,
        bg=example text.bg!80!example text.fg}
    \setbeamercolor{block body example}{use={block title example, example text},
        fg=example text.fg,
        bg=block title example.bg!50!example text.bg}

\begin{document}
\maketitle

% ------------------------------------------------------------------------------
% ------ FRAME -----------------------------------------------------------------
% ------------------------------------------------------------------------------
\begin{frame}{Outline}
    \tableofcontents
\end{frame}



\section{Kahoot!}

\begin{frame}{Kahoot: A exercises}
  Form a group for each table:
  \begin{itemize}
    \item Get prepared to answer the A exercises as a team (5 min).
  \end{itemize}
  \includegraphics[width=\textwidth]{figures/kahoot}
\end{frame}



\section{PS6, Ex. 1 (A): }

\begin{frame}{PS6, Ex. 1 (A): }
    \textbf{Explain $\delta$ mathematically\\}
    $delta$ is the discount factor which the payoff in the next game will be multiplied by, so if there player stand to gain 1 in the next round, and $\delta=0.5$, it is only worth $1*0.5=0.5$ to the player in the current round.\\
    \textbf{Explain $\delta$ intuitively}\\ Intuitively $\delta$ is the factor showing how patient the players are. The higher $\delta$, thee less the players will mind waiting for the next round.  \\
    \textbf{Explain the case $\delta=0$}\\
    In the case $\delta=0$, the players will have their payoff multiplied by 0 in the next round, so the game turns into an ultimatum game where the first mover can offer the other player anything and they will accept. There is a first mover advantage. \\
    \textbf{Explain the case $\delta=1$}\\
    In the case $\delta=1$, the players will have their payoff multiplied by 1 in the next round, so they won't care whether the game goes for another around. This will be the case for each round until the final round, which will then be an ultimatum game where the last mover can offer the other player anything and they will accept. There is no first mover advantage, but there is a last mover advantage. \\
    \textbf{Explain whether it depends on K}\\
    For $\delta$=1, the last mover will get the whole price pool, no matter how many rounds (K) the game is. The only case with a first mover advantage is for $K=1$, in which the first move is the same as the last.\\
    \vfill\null
\end{frame}

\section{PS6, Ex. 2 (A): }

\begin{frame}{PS6, Ex. 2 (A): }
    \begin{itemize}
    \item[part one:] For the payoffs: $ \left( \frac{1-\delta_2}{1-\delta_1\delta_2},\frac{\delta_2(1-\delta_1)}{1-\delta_1\delta_2}\right)$ Discuss how the payoff change as each player becomes more or less patient.
    \end{itemize}
    \vfill\null
  \begin{multicols}{2}
    \begin{itemize}
      \item[(Step 1)] Write up partial derivatives for $\delta_2$'s and $\delta_1$'s effect on the outcome for player 1, are the partial derivatives positive or negative?
      \item[(Step 2)] Use the fact that it's a zero sum game to look at the change in outcome for player 2
      \item[Answer] Player 1s payoff is increasing in $\delta_1$ and decreasing in $\delta_2$, vice versa for Player 2. This intuitively makes sense, because player i's bargaining power in later rounds will increase when his patience increase relative to player j.
      \end{itemize}
    \vfill\null \columnbreak
    Information so far:
    \begin{itemize}
    \item[1] $\frac{\partial s_1*}{\partial \delta_1} = \frac{(1-\delta_2)\delta_2}{(1-\delta_1\delta_2)^2}>0 $\\
    \item[2] $\frac{\partial s_1*}{\partial \delta_2} = -\frac{1-\delta_1}{(1-\delta_1\delta_2)^2}<0 $\\
    \end{itemize}
    \vfill\null
  \end{multicols}
\end{frame}

\begin{frame}{PS6, Ex. 2 (A): }
    \begin{itemize}
    \item[part two:] For the payoffs: $ \left( \frac{1-\delta_2}{1-\delta_1\delta_2},\frac{\delta_2(1-\delta_1)}{1-\delta_1\delta_2}\right)$ show that for $\delta_2=\delta_1$ the payoffs simplify to $\left(\frac{1}{1+\delta},\frac{\delta}{1+\delta}\right)$
    \end{itemize}
    Write up the payoffs with $\delta=\delta_1=\delta_2$ \\
    Use that: $1-x^2=(1+x)(1-x)$, to simplify \\
    Simplification: \\
    \begin{math} \left(\frac{1-\delta}{1-\delta^2},\frac{\delta(1-\delta)}{1-\delta^2}\right) \Rightarrow \left(\frac{1-\delta}{(1-\delta)(1+\delta)},\frac{\delta(1-\delta)}{(1-\delta)(1+\delta)}\right) \Rightarrow \left(\frac{1}{1+\delta},\frac{\delta}{1+\delta}\right)
    \end{math}
    \vfill\null
\end{frame}

\section{PS6, Ex. 3: Dynamic games (imperfect information)}

\begin{frame}{PS6, Ex. 3: Dynamic games (imperfect information)}
    Find the SPNE in the four games.\\\bigskip
    Hints:
    \begin{enumerate}
      \item It becomes much easier to grasp dynamic games with imperfect information if you write the part with imperfect information in normal form (bi-matrix).
      \item Be careful to cover all of the strategy profile (in every subgame!) when writing up the subgame perfect Nash Equilibria (SPNE).
    \end{enumerate}
    \vfill\null
\end{frame}


\begin{frame}{PS6, Ex. 3.a: Dynamic games (imperfect information)}
    \begin{itemize}
      \item[(a)] Find the SPNE in the following game:
    \end{itemize}
    \begin{figure}[!h]
      \center
      \def\svgwidth{.8\columnwidth}
      \import{figures/}{6b_.pdf_tex}
    \end{figure}
    \vfill\null
\end{frame}
\begin{frame}{PS6, Ex. 3.a: Dynamic games (imperfect information)}
    \begin{itemize}
      \item[(a)] Find the SPNE in the following game:
    \end{itemize}
    \vspace{-4pt}
    \begin{figure}[!h]
      \center
      \def\svgwidth{.8\columnwidth}
      \import{figures/}{6a_.pdf_tex}
    \end{figure}
    \vspace{-4pt}
    \nth{2} and \nth{3} stage in normal form:
    \vspace{-4pt}
    \begin{table}
      \begin{tabular}{cl|c|c|}
        & \multicolumn{1}{c}{} & \multicolumn{2}{c}{Player 2}\\
        \parbox[t]{1mm}{\multirow{3}{*}{\rotatebox[origin=r]{90}{Player 1}}}
        & \multicolumn{1}{c}{} & \multicolumn{1}{c}{L} & \multicolumn{1}{c}{R} \\\cline{3-4}
        & $L_2$ & -3, -1 & 1, -2 \\\cline{3-4}
        & $R_2$ & -2, 1 & 3, 0 \\\cline{3-4}
      \end{tabular}
    \end{table}
    \vfill\null
\end{frame}
\begin{frame}{PS6, Ex. 3.a: Dynamic games (imperfect information)}
    \begin{itemize}
      \item[(a)] Find the SPNE in the following game:
    \end{itemize}
    \vspace{-4pt}
    \begin{figure}[!h]
      \center
      \def\svgwidth{.8\columnwidth}
      \import{figures/}{6a_.pdf_tex}
    \end{figure}
    \vspace{-4pt}
    \nth{2} and \nth{3} stage in normal form:
    \vspace{-4pt}
    \begin{table}
      \begin{tabular}{cl|c|c|}
        & \multicolumn{1}{c}{} & \multicolumn{2}{c}{\color{blue}Player 2}\\
        \parbox[t]{1mm}{\multirow{3}{*}{\rotatebox[origin=r]{90}{\color{red}Player 1}}}
        & \multicolumn{1}{c}{} & \multicolumn{1}{c}{\textcolor{blue}{L}} & \multicolumn{1}{c}{R} \\\cline{3-4}
        & $L_2$ & -3, \textcolor{blue}{-1} & 1, -2 \\\cline{3-4}
        & \textcolor{red}{$R_2$} & \textcolor{red}{-2}, \textcolor{blue}{1} & \textcolor{red}{3}, 0 \\\cline{3-4}
      \end{tabular}
    \end{table}
    \vfill\null
\end{frame}
\begin{frame}{PS6, Ex. 3.a: Dynamic games (imperfect information)}
    \begin{itemize}
      \item[(a)] Find the SPNE in the following game:
    \end{itemize}
    \vspace{-4pt}
    \begin{figure}[!h]
      \center
      \def\svgwidth{.8\columnwidth}
      \import{figures/}{6a.pdf_tex}
    \end{figure}
    \vspace{-4pt}
    \nth{2} and \nth{3} stage in normal form:
    \vspace{-4pt}
    \begin{table}
      \begin{tabular}{cl|c|c|}
        & \multicolumn{1}{c}{} & \multicolumn{2}{c}{\color{blue}Player 2}\\
        \parbox[t]{1mm}{\multirow{3}{*}{\rotatebox[origin=r]{90}{\color{red}Player 1}}}
        & \multicolumn{1}{c}{} & \multicolumn{1}{c}{\textcolor{blue}{L}} & \multicolumn{1}{c}{R} \\\cline{3-4}
        & $L_2$ & -3, \textcolor{blue}{-1} & 1, -2 \\\cline{3-4}
        & \textcolor{red}{$R_2$} & \textcolor{red}{-2}, \textcolor{blue}{1} & \textcolor{red}{3}, 0 \\\cline{3-4}
      \end{tabular}
    \end{table}
    \textbf{\textit{Write up the SPNE!}}
    \vfill\null
\end{frame}
\begin{frame}{PS6, Ex. 3.a: Dynamic games (imperfect information)}
    \begin{itemize}
      \item[(a)] Find the SPNE in the following game:
    \end{itemize}
    \vspace{-4pt}
    \begin{figure}[!h]
      \center
      \def\svgwidth{.8\columnwidth}
      \import{figures/}{6a.pdf_tex}
    \end{figure}
    \vspace{-4pt}
    \nth{2} and \nth{3} stage in normal form:
    \vspace{-4pt}
    \begin{table}
      \begin{tabular}{cl|c|c|}
        & \multicolumn{1}{c}{} & \multicolumn{2}{c}{\color{blue}Player 2}\\
        \parbox[t]{1mm}{\multirow{3}{*}{\rotatebox[origin=r]{90}{\color{red}Player 1}}}
        & \multicolumn{1}{c}{} & \multicolumn{1}{c}{\textcolor{blue}{L}} & \multicolumn{1}{c}{R} \\\cline{3-4}
        & $L_2$ & -3, \textcolor{blue}{-1} & 1, -2 \\\cline{3-4}
        & \textcolor{red}{$R_2$} & \textcolor{red}{-2}, \textcolor{blue}{1} & \textcolor{red}{3}, 0 \\\cline{3-4}
      \end{tabular}
    \end{table}
    $SPNE=\{s_1^{*},s_2^{*}\}=\{(L_1,R_2),L\}$ with outcome (0,2).
    \vfill\null
\end{frame}

\begin{frame}{PS6, Ex. 3.b: Dynamic games (imperfect information)}
    \begin{itemize}
      \item[(b)] Find the SPNE in the following game:
    \end{itemize}
    \begin{figure}[!h]
      \center
      \def\svgwidth{.8\columnwidth}
      \import{figures/}{6b_.pdf_tex}
    \end{figure}
    \vfill\null
\end{frame}
\begin{frame}{PS6, Ex. 3.b: Dynamic games (imperfect information)}
    \begin{itemize}
      \item[(b)] Find the SPNE in the following game:
    \end{itemize}
    \vspace{-4pt}
    \begin{figure}[!h]
      \center
      \def\svgwidth{.8\columnwidth}
      \import{figures/}{6b_.pdf_tex}
    \end{figure}
    \vspace{-4pt}
    \nth{2} and \nth{3} stage in normal form:
    \vspace{-4pt}
    \begin{table}
      \begin{tabular}{cl|c|c|}
        & \multicolumn{1}{c}{} & \multicolumn{2}{c}{Player 2}\\
        \parbox[t]{1mm}{\multirow{3}{*}{\rotatebox[origin=r]{90}{Player 1}}}
        & \multicolumn{1}{c}{} & \multicolumn{1}{c}{L} & \multicolumn{1}{c}{R} \\\cline{3-4}
        & $L_2$ & -6, -6 & -1, -1 \\\cline{3-4}
        & $R_2$ & -1, -1 & -3, -3 \\\cline{3-4}
      \end{tabular}
    \end{table}
    \vfill\null
\end{frame}
\begin{frame}{PS6, Ex. 3.b: Dynamic games (imperfect information)}
    \begin{itemize}
      \item[(b)] Find the SPNE in the following game:
    \end{itemize}
    \vspace{-4pt}
    \begin{figure}[!h]
      \center
      \def\svgwidth{.8\columnwidth}
      \import{figures/}{6b_.pdf_tex}
    \end{figure}
    \vspace{-4pt}
    \nth{2} and \nth{3} stage in normal form:
    \vspace{-4pt}
    \begin{table}
      \begin{tabular}{cl|c|c|}
        & \multicolumn{1}{c}{} & \multicolumn{2}{c}{\color{blue}Player 2}\\
        \parbox[t]{1mm}{\multirow{3}{*}{\rotatebox[origin=r]{90}{\color{red}Player 1}}}
        & \multicolumn{1}{c}{} & \multicolumn{1}{c}{L} & \multicolumn{1}{c}{R} \\\cline{3-4}
        & $L_2$ & -6, -6 & \textcolor{red}{-1}, \textcolor{blue}{-1} \\\cline{3-4}
        & $R_2$ & \textcolor{red}{-1}, \textcolor{blue}{-1} & -3, -3 \\\cline{3-4}
      \end{tabular}
    \end{table}
    \textbf{\textit{Two different pure strategy NE (PSNE) in the subgame. What now?}}
    \vfill\null
\end{frame}
\begin{frame}{PS6, Ex. 3.b: Dynamic games (imperfect information)}
    \begin{itemize}
      \item[(b)] Find the SPNE in the following game:
    \end{itemize}
    $R_1$ is strictly dominated by $L_1$ and we have two subgame perfect solutions:
    \begin{multicols}{2}
      \begin{figure}[!h]
        \center
        \def\svgwidth{\columnwidth}
        \import{figures/}{6b1.pdf_tex}
      \end{figure}
      \vfill\null\columnbreak
      \begin{figure}[!h]
        \center
        \def\svgwidth{\columnwidth}
        \import{figures/}{6b2.pdf_tex}
      \end{figure}
    \end{multicols}
    \vspace{-8pt}
    \textbf{\textit{Write up the SPNE!}}
    \vfill\null
\end{frame}
\begin{frame}{PS6, Ex. 3.b: Dynamic games (imperfect information)}
    \begin{itemize}
      \item[(b)] Find the SPNE in the following game:
    \end{itemize}
    $R_1$ is strictly dominated by $L_1$ and we have two subgame perfect solutions:
    \begin{multicols}{2}
      \begin{figure}[!h]
        \center
        \def\svgwidth{\columnwidth}
        \import{figures/}{6b1.pdf_tex}
      \end{figure}
      \vfill\null\columnbreak
      \begin{figure}[!h]
        \center
        \def\svgwidth{\columnwidth}
        \import{figures/}{6b2.pdf_tex}
      \end{figure}
    \end{multicols}
    \vspace{-8pt}
    $SPNE=\{s_1^{*},s_2^{*}\}=\{(L_1,L_2),R;(L_1,R_2),L\}$ both with outcome (0,2).
    \vfill\null
\end{frame}

\begin{frame}{PS6, Ex. 3.c: Dynamic games (imperfect information)}
    \begin{itemize}
      \item[(c)] Find the SPNE in the following game:
    \end{itemize}
    \begin{figure}[!h]
      \center
      \def\svgwidth{.8\columnwidth}
      \import{figures/}{6c_.pdf_tex}
    \end{figure}
    \vfill\null
\end{frame}
\begin{frame}{PS6, Ex. 3.c: Dynamic games (imperfect information)}
    \begin{itemize}
      \item[(c)] Find the SPNE in the following game:
    \end{itemize}
    \vspace{-4pt}
    \begin{figure}[!h]
      \center
      \def\svgwidth{.8\columnwidth}
      \import{figures/}{6c_3rd.pdf_tex}
    \end{figure}
    \vspace{-4pt}
    \nth{1} and \nth{2} stage in normal form (taking the \nth{3} stage as given):
    \vspace{-4pt}
    \begin{table}
      \begin{tabular}{cl|c|c|}
        & \multicolumn{1}{c}{} & \multicolumn{2}{c}{Player 2}\\
        \parbox[t]{1mm}{\multirow{3}{*}{\rotatebox[origin=r]{90}{Player 1}}}
        & \multicolumn{1}{c}{} & \multicolumn{1}{c}{L} & \multicolumn{1}{c}{R} \\\cline{3-4}
        & $L_1$ & 2, 1 & 3, 0 \\\cline{3-4}
        & $R_1$ & 3, 1 & 3, -1 \\\cline{3-4}
      \end{tabular}
    \end{table}
    \vfill\null
\end{frame}
\begin{frame}{PS6, Ex. 3.c: Dynamic games (imperfect information)}
    \begin{itemize}
      \item[(c)] Find the SPNE in the following game:
    \end{itemize}
    \vspace{-4pt}
    \begin{figure}[!h]
      \center
      \def\svgwidth{.8\columnwidth}
      \import{figures/}{6c.pdf_tex}
    \end{figure}
    \vspace{-4pt}
    \nth{1} and \nth{2} stage in normal form (taking the \nth{3} stage as given):
    \vspace{-4pt}
    \begin{table}
      \begin{tabular}{cl|c|c|}
        & \multicolumn{1}{c}{} & \multicolumn{2}{c}{\color{blue}Player 2}\\
        \parbox[t]{1mm}{\multirow{3}{*}{\rotatebox[origin=r]{90}{\color{red}Player 1}}}
        & \multicolumn{1}{c}{} & \multicolumn{1}{c}{\textcolor{blue}{L}} & \multicolumn{1}{c}{R} \\\cline{3-4}
        & $L_1$ & 2, \textcolor{blue}{1} & \textcolor{red}{3}, 0 \\\cline{3-4}
        & $R_1$ & \textcolor{red}{3}, \textcolor{blue}{1} & \textcolor{red}{3}, -1 \\\cline{3-4}
      \end{tabular}
    \end{table}
    \textbf{\textit{Consider how many subgames there are and write up the SPNE.}}
    \vfill\null
\end{frame}
\begin{frame}{PS6, Ex. 3.c: Dynamic games (imperfect information)}
    \begin{itemize}
      \item[(c)] Find the SPNE in the following game:
    \end{itemize}
    \vspace{-4pt}
    \begin{figure}[!h]
      \center
      \def\svgwidth{.8\columnwidth}
      \import{figures/}{6c.pdf_tex}
    \end{figure}
    \vspace{-4pt}
    \nth{1} and \nth{2} stage in normal form (taking the \nth{3} stage as given):
    \vspace{-4pt}
    \begin{table}
      \begin{tabular}{cl|c|c|}
        & \multicolumn{1}{c}{} & \multicolumn{2}{c}{\color{blue}Player 2}\\
        \parbox[t]{1mm}{\multirow{3}{*}{\rotatebox[origin=r]{90}{\color{red}Player 1}}}
        & \multicolumn{1}{c}{} & \multicolumn{1}{c}{\textcolor{blue}{L}} & \multicolumn{1}{c}{R} \\\cline{3-4}
        & $L_1$ & 2, \textcolor{blue}{1} & \textcolor{red}{3}, 0 \\\cline{3-4}
        & $R_1$ & \textcolor{red}{3}, \textcolor{blue}{1} & \textcolor{red}{3}, -1 \\\cline{3-4}
      \end{tabular}
    \end{table}
    $SPNE=\{s_1^{*},s_2^{*}\}=\{(R_1,L_2,R_2),L\}$ with outcome (3,1).
    \vfill\null
\end{frame}

\begin{frame}{PS6, Ex. 3.d: Dynamic games (imperfect information)}
    \begin{itemize}
      \item[(d)] Find the SPNE in the following game:
    \end{itemize}
    \begin{figure}[!h]
      \center
      \def\svgwidth{.8\columnwidth}
      \import{figures/}{6d_.pdf_tex}
    \end{figure}
    \vfill\null
\end{frame}
\begin{frame}{PS6, Ex. 3.d: Dynamic games (imperfect information)}
    \begin{itemize}
      \item[(d)] Find the SPNE in the following game:
    \end{itemize}
    \vspace{-4pt}
    \begin{figure}[!h]
      \center
      \def\svgwidth{.8\columnwidth}
      \import{figures/}{6d_.pdf_tex}
    \end{figure}
    \vspace{-4pt}
    \nth{2} and \nth{3} stage in normal form (Player 1 knows her own action in \nth{1} stage):
    \vspace{-4pt}
    \begin{figure}[!h]
      \center
      \def\svgwidth{.25\columnwidth}
      \import{figures/}{6d_reduced_.pdf_tex}
    \end{figure}
    \vspace{-9pt}
    \begin{table}
      \begin{tabular}{l|c|c|}
        \multicolumn{1}{c}{} & \multicolumn{1}{c}{$L$} & \multicolumn{1}{c}{$R$} \\\cline{2-3}
        $L_2$ & 2, 1 & 3, 0 \\\cline{2-3}
        $R_2$ & 2, 1 & 3, 0 \\\cline{2-3}
      \end{tabular}
      \enskip
      \begin{tabular}{l|c|c|}
        \multicolumn{1}{c}{} & \multicolumn{1}{c}{$L$} & \multicolumn{1}{c}{$R$} \\\cline{2-3}
        $L_2$ & 3, 1 & 0, 0 \\\cline{2-3}
        $R_2$ & 0, 0 & 3, -1 \\\cline{2-3}
      \end{tabular}
    \end{table}
    \vfill\null
\end{frame}
\begin{frame}{PS6, Ex. 3.d: Dynamic games (imperfect information)}
    \begin{itemize}
      \item[(d)] Find the SPNE in the following game:
    \end{itemize}
    \vspace{-4pt}
    \begin{figure}[!h]
      \center
      \def\svgwidth{.8\columnwidth}
      \import{figures/}{6d_.pdf_tex}
    \end{figure}
    \vspace{-4pt}
    \nth{2} and \nth{3} stage in normal form (Player 1 knows her own action in \nth{1} stage):
    \vspace{-4pt}
    \begin{figure}[!h]
      \center
      \def\svgwidth{.25\columnwidth}
      \import{figures/}{6d_reduced.pdf_tex}
    \end{figure}
    \vspace{-9pt}
    \begin{table}
      \begin{tabular}{l|c|c|}
        \multicolumn{1}{c}{} & \multicolumn{1}{c}{\color{blue}$L$} & \multicolumn{1}{c}{$R$} \\\cline{2-3}
        $L_2$ & \textcolor{red}{2}, \textcolor{blue}{1} & 3, 0 \\\cline{2-3}
        $R_2$ & \textcolor{red}{2}, \textcolor{blue}{1} & 3, 0 \\\cline{2-3}
      \end{tabular}
      \enskip
      \begin{tabular}{l|c|c|}
        \multicolumn{1}{c}{} & \multicolumn{1}{c}{\color{blue}$L$} & \multicolumn{1}{c}{$R$} \\\cline{2-3}
        $L_2$ & \textcolor{red}{3}, \textcolor{blue}{1} & 0, 0 \\\cline{2-3}
        $R_2$ & 0, \textcolor{blue}{0} & 3, -1 \\\cline{2-3}
      \end{tabular}
    \end{table}
    \vfill\null
\end{frame}
\begin{frame}{PS6, Ex. 3.d: Dynamic games (imperfect information)}
    \begin{itemize}
      \item[(d)] Find the SPNE in the following game:
    \end{itemize}
    \vspace{-4pt}
    \begin{figure}[!h]
      \center
      \def\svgwidth{.8\columnwidth}
      \import{figures/}{6d.pdf_tex}
    \end{figure}
    \vspace{-4pt}
    \nth{2} and \nth{3} stage in normal form (Player 1 knows her own action in \nth{1} stage):
    \vspace{-4pt}
    \begin{figure}[!h]
      \center
      \def\svgwidth{.25\columnwidth}
      \import{figures/}{6d_reduced.pdf_tex}
    \end{figure}
    \vspace{-9pt}
    \begin{table}
      \begin{tabular}{l|c|c|}
        \multicolumn{1}{c}{} & \multicolumn{1}{c}{\color{blue}$L$} & \multicolumn{1}{c}{$R$} \\\cline{2-3}
        $L_2$ & \textcolor{red}{2}, \textcolor{blue}{1} & 3, 0 \\\cline{2-3}
        $R_2$ & \textcolor{red}{2}, \textcolor{blue}{1} & 3, 0 \\\cline{2-3}
      \end{tabular}
      \enskip
      \begin{tabular}{l|c|c|}
        \multicolumn{1}{c}{} & \multicolumn{1}{c}{\color{blue}$L$} & \multicolumn{1}{c}{$R$} \\\cline{2-3}
        $L_2$ & \textcolor{red}{3}, \textcolor{blue}{1} & 0, 0 \\\cline{2-3}
        $R_2$ & 0, \textcolor{blue}{0} & 3, -1 \\\cline{2-3}
      \end{tabular}
    \end{table}
    \vspace{-4pt}
    \textbf{\textit{Write up the SPNE!}}
    \vfill\null
\end{frame}
\begin{frame}{PS6, Ex. 3.d: Dynamic games (imperfect information)}
    \begin{itemize}
      \item[(d)] Find the SPNE in the following game:
    \end{itemize}
    \vspace{-4pt}
    \begin{figure}[!h]
      \center
      \def\svgwidth{.8\columnwidth}
      \import{figures/}{6d.pdf_tex}
    \end{figure}
    \vspace{-4pt}
    \nth{2} and \nth{3} stage in normal form (Player 1 knows her own action in \nth{1} stage):
    \vspace{-4pt}
    \begin{figure}[!h]
      \center
      \def\svgwidth{.25\columnwidth}
      \import{figures/}{6d_reduced.pdf_tex}
    \end{figure}
    \vspace{-9pt}
    \begin{table}
      \begin{tabular}{l|c|c|}
        \multicolumn{1}{c}{} & \multicolumn{1}{c}{\color{blue}$L$} & \multicolumn{1}{c}{$R$} \\\cline{2-3}
        $L_2$ & \textcolor{red}{2}, \textcolor{blue}{1} & 3, 0 \\\cline{2-3}
        $R_2$ & \textcolor{red}{2}, \textcolor{blue}{1} & 3, 0 \\\cline{2-3}
      \end{tabular}
      \enskip
      \begin{tabular}{l|c|c|}
        \multicolumn{1}{c}{} & \multicolumn{1}{c}{\color{blue}$L$} & \multicolumn{1}{c}{$R$} \\\cline{2-3}
        $L_2$ & \textcolor{red}{3}, \textcolor{blue}{1} & 0, 0 \\\cline{2-3}
        $R_2$ & 0, \textcolor{blue}{0} & 3, -1 \\\cline{2-3}
      \end{tabular}
    \end{table}
    \vspace{-4pt}
    $SPNE=\{s_1^{*},s_2^{*}\}=\{(R_1,L_2),L\}$ with outcome (3,1).
    \vfill\null
\end{frame}



\section{PS6, Ex. 4: The Mutated Seabass (imperfect information)}

\begin{frame}{PS6, Ex. 4: }
    Go back to exercise 4 in problem set 5. Write up the game tree for the situation in part (c), where the choice to acquire the weapon is not observed. Find the SPNE. What has changed?\\\bigskip
    \textit{Last class we actually solved this part and discussed it as an extension...}
\end{frame}




\section{PS6, Ex. 5: }

\begin{frame}{PS6, Ex. 5: }
    \begin{itemize}
    \item[] Consider an infinite bargaining game. Show that in the BI outcome, when each player has their own discount factor $\delta_1,\delta_2 \epsilon [0;1]$, player 1 offers the settlement $ (s*,1-s*)= \left( \frac{1-\delta_2}{1-\delta_1\delta_2},\frac{\delta_2(1-\delta_1)}{1-\delta_1\delta_2}\right)$ which player 2 accepts.
    \item[(Step 1)] Start with a three stage game, but where player 1s payoff in turn 3 is denoted s. Write up the outcome for player 1 in round 3. Then use the potential outcome of round 3 to find the outcome in round 2. Do the same for round 1 with respects to round 2.
    \end{itemize}
    \vfill\null
    \begin{itemize}
        \item[Turn 3] s
        \item[Turn 2] P1 will choose to accept or decline $s_2 \epsilon [0;1]$. He will accept if $s_2 \geq s\delta_1]$, p2 proposes $s_2 = s\delta_1$ which p1 accepts
        \item[Turn 1] P2 will choose to accept or decline $1-s_1 \epsilon [0;1]$. He will accept if $1-s_1 \geq (1-s\delta_1)\delta_2$, p1 proposes $s_1 = 1- (1-s\delta_1)\delta_2$ which p2 accepts
    \end{itemize}
     \vfill\null
\end{frame}

\begin{frame}{PS6, Ex. 5: }
    \begin{itemize}
    \item[] Consider an infinite bargaining game. Show that in the BI outcome, when each player has their own discount factor $\delta_1,\delta_2 \epsilon [0;1]$, player 1 offers the settlement $ (s*,1-s*)= \left( \frac{1-\delta_2}{1-\delta_1\delta_2},\frac{\delta_2(1-\delta_1)}{1-\delta_1\delta_2}\right)$ which player 2 accepts.
    \item[(Step 2)] Since the game is infinite, the players are playing the same game in turn 3 as in turn 1, ie. the outcome of turn 1 should be the same as in turn 3. Use this to find a stationary solution, where $s_turn1=s_turn3$
    \end{itemize}
    \vfill\null
  \begin{multicols}{2}
    \begin{itemize}
        \item[] $s*=s_1 = 1- (1-s*\delta_1)\delta_2 \Rightarrow$
        \item[] $s*= 1-\delta_2-s*\delta_1\delta_2 \Rightarrow$
        \item[] $s*(1+\delta_1\delta_2)= 1-\delta_2 \Rightarrow$
        \item[] $s*= \frac{1-\delta_2}{1+\delta_1\delta_2} \Rightarrow$
        \item[] $ (s*,1-s*)= \left( \frac{1-\delta_2}{1-\delta_1\delta_2},\frac{\delta_2(1-\delta_1)}{1-\delta_1\delta_2}\right)$
    \end{itemize}
    \vfill\null \columnbreak
    Information so far:
    \begin{itemize}
    \item[Turn 1] $s_1 = 1- (1-s\delta_1)\delta_2$\\
    \item[Turn 3] s\\
    \end{itemize}
    \vfill\null 
  \end{multicols}
    \vfill\null 
\end{frame}

\section{PS6, Ex. 6: }

\begin{frame}{PS6, Ex. 6: }
    \begin{itemize}
    \item[(a)] Consider the game where both of them choose their effort levels simultaneously and independently. Derive the best response functions. Find the (pure strategy) Nash equilibrium $(y_1^{NE}, y_2^{NE})$ with $y_1^{NE}, y_2^{NE} > 0$.
    \end{itemize}
    \vfill\null
  \begin{multicols}{2}
    \begin{itemize}
      \item[(Step 1)] Write up the payoff functions
      \item[(Step 2)] Write up the FOC and find the best response functions
      \item[(Step 3)] This is not a symmetric game, so you have to substitute $BR_i$ into $BR_j$ and then isolate $y_i$ and insert it into $BR_j$:
      \begin{align*}
          y_1&=\frac{y_1}{2}^{\frac{1}{2}} \Rightarrow y_1-2y_1^2=0 \Rightarrow y_1=\frac{1}{2} \\
          y_2&=\frac{\frac{1}{2}}{2}=\frac{1}{4}
      \end{align*}
      \item[NE:] \begin{math} \left(\frac{1}{2},\frac{1}{4}\right)\end{math}
    \end{itemize}
    \vfill\null \columnbreak
    Information so far:
    \begin{itemize}
    \item[1] Quality: $q(y_1, y_2) = y_1y_2.$\\
    \item[2] Costs: $C_1(y_1) = \frac{1}{3}(y_1)^3,\ \ C_2(y_2) = (y_2)^2.$\\
    \item[3] $Payoff_i$: $U_i(y_i,y_j) = q(y_1,y_2)-C_i(y_i).$ \\
    \item[4] $Payoff_1(y_1,y_2)$: $U_1(y_1,y_2) = y_1*y_2-\frac{1}{3}y_1^3$ \\
    \item[5] $Payoff_2(y_1,y_2)$: $U_1(y_1,y_2) = y_1*y_2-y_2^2$ \\
    \item[6] $BR_1(y_2)$: $y_1 = y_2^{1/2}$ \\
    \item[7] $BR_2(y_1)$: $y_2 = \frac{y_1}{2}$ \\
    \end{itemize}
    \vfill\null
  \end{multicols}
\end{frame}


\section{PS6, Ex. 7: To keep or split (imperfect information)}

\begin{frame}{PS6, Ex. 7: To keep or split (imperfect information)}
  \begin{multicols}{2}
    Consider the following 2 × 2 game where payoffs are monetary:
    \begin{table}
      \begin{tabular}{l|c|c|}
          \multicolumn{1}{c}{} & \multicolumn{1}{c}{L} & \multicolumn{1}{c}{R} \\\cline{2-3}
          T & 3, 3 & 0, 4 \\\cline{2-3}
          B & 4, 0 & 1, 1 \\\cline{2-3}
      \end{tabular}
    \end{table}
    Before this game is played, Player 1 can choose whether, after the game is played, players should keep their own payoffs or split the aggregate payoff evenly between them. Player 2 observes this choice.
  \vfill\null \columnbreak
    \begin{itemize}
      \item[(a)] Write down the game tree of this two-stage game: be careful to represent the simultaneous-move game in the second stage using information sets.
      \item[(b)] Find the \underline{subgame perfect} Nash Equilibria (SPNE).
      \item[(c)] Now suppose that Player 2 cannot observe Player 1’s choice in the first stage. Draw
      the game tree (again using information sets) and find the \underline{pure strategy} Nash Equilibria (PSNE).
    \end{itemize}
  \vfill\null
  \end{multicols}
\end{frame}


\begin{frame}{PS6, Ex. 7.a: To keep or split (imperfect information)}
  \begin{itemize}
    \item[(a)] Write down the game tree of this two-stage game: be careful to represent the simultaneous-move game in the second stage using information sets.
  \end{itemize}
  \vspace{-4pt}
  \nth{1} stage: Player 1 chooses Keep or Split. Player 2 observes the choice.\\\medskip
  \nth{2} stage: Player 2 chooses $L$ or $R$ ($L'$ or $R'$). The action is private information.\\\medskip
  \nth{3} stage: Player 1 chooses $T$ or $B$ ($T'$ or $B'$) without knowing what Player 2 did.
  \vspace{-4pt}
  \begin{figure}[!h]
    \center
    \def\svgwidth{.8\columnwidth}
    \import{figures/}{7a_extensive_form.pdf_tex}
  \end{figure}
  \vspace{-2pt}
  The order of stage 2 and 3 is arbitrary, but the \nth{2} stage must be private information.
  %(you can swap stage 2 and 3, then the new \nth{2} stage would be private information.)
  \vspace{-2pt}
  \begin{itemize}
    \item[(b)] \textbf{\textit{Find the subgame perfect Nash Equilibria (SPNE).}}
  \end{itemize}
\end{frame}


\begin{frame}{PS6, Ex. 7.b: To keep or split (imperfect information)}
  \begin{itemize}
    \item[(b)] Find the subgame perfect Nash Equilibria (SPNE).
  \end{itemize}
  \vspace{-4pt}
  \begin{figure}[!h]
    \center
    \def\svgwidth{.8\columnwidth}
    \import{figures/}{7a_extensive_form.pdf_tex}
  \end{figure}
  \vspace{-2pt}
  \begin{multicols}{2}
    \begin{figure}[!h]
      \center
      \def\svgwidth{.5\columnwidth}
      \import{figures/}{7b_.pdf_tex}
    \end{figure}
    \vspace{-8pt}
    \begin{table}
      \begin{tabular}{l|c|c|}
        \multicolumn{1}{c}{} & \multicolumn{1}{c}{L} & \multicolumn{1}{c}{R} \\\cline{2-3}
        T & 3, 3 & 0, 4 \\\cline{2-3}
        B & 4, 0 & 1, 1 \\\cline{2-3}
      \end{tabular}\
      \begin{tabular}{l|c|c|}
        \multicolumn{1}{c}{} & \multicolumn{1}{c}{L'} & \multicolumn{1}{c}{R'} \\\cline{2-3}
        T' & 3, 3 & 2, 2 \\\cline{2-3}
        B' & 2, 2 & 1, 1 \\\cline{2-3}
      \end{tabular}
    \end{table}
  \vfill\null \columnbreak
  \vfill\null
  \end{multicols}
\end{frame}
\begin{frame}{PS6, Ex. 7.b: To keep or split (imperfect information)}
  \begin{itemize}
    \item[(b)] Find the subgame perfect Nash Equilibria (SPNE).
  \end{itemize}
  \vspace{-4pt}
  \begin{figure}[!h]
    \center
    \def\svgwidth{.8\columnwidth}
    \import{figures/}{7a_extensive_form.pdf_tex}
  \end{figure}
  \vspace{-2pt}
  \begin{multicols}{2}
    \begin{figure}[!h]
      \center
      \def\svgwidth{.5\columnwidth}
      \import{figures/}{7b_.pdf_tex}
    \end{figure}
    \vspace{-8pt}
    \begin{table}
      \begin{tabular}{l|c|c|}
        \multicolumn{1}{c}{} & \multicolumn{1}{c}{L} & \multicolumn{1}{c}{\textcolor{blue}{R}} \\\cline{2-3}
        T & 3, 3 & 0, \textcolor{blue}{4} \\\cline{2-3}
        \textcolor{red}{B} & \textcolor{red}{4}, 0 & \textcolor{red}{1}, \textcolor{blue}{1} \\\cline{2-3}
        \end{tabular}\
        \begin{tabular}{l|c|c|}
          \multicolumn{1}{c}{} & \multicolumn{1}{c}{\textcolor{blue}{L'}} & \multicolumn{1}{c}{R'} \\\cline{2-3}
          \textcolor{red}{T'} & \textcolor{red}{3}, \textcolor{blue}{3} & \textcolor{red}{2}, 2 \\\cline{2-3}
          B' & 2, \textcolor{blue}{2} & 1, 1 \\\cline{2-3}
        \end{tabular}
    \end{table}
  \vfill\null \columnbreak
  \vfill\null
  \end{multicols}
\end{frame}
\begin{frame}{PS6, Ex. 7.b: To keep or split (imperfect information)}
  \begin{itemize}
    \item[(b)] Find the subgame perfect Nash Equilibria (SPNE).
  \end{itemize}
  \vspace{-4pt}
  \begin{figure}[!h]
    \center
    \def\svgwidth{.8\columnwidth}
    \import{figures/}{7b_extensive_form.pdf_tex}
  \end{figure}
  \vspace{-2pt}
  \begin{multicols}{2}
    \begin{figure}[!h]
      \center
      \def\svgwidth{.5\columnwidth}
      \import{figures/}{7b.pdf_tex}
    \end{figure}
    \vspace{-8pt}
    \begin{table}
      \begin{tabular}{l|c|c|}
        \multicolumn{1}{c}{} & \multicolumn{1}{c}{L} & \multicolumn{1}{c}{\textcolor{blue}{R}} \\\cline{2-3}
        T & 3, 3 & 0, \textcolor{blue}{4} \\\cline{2-3}
        \textcolor{red}{B} & \textcolor{red}{4}, 0 & \textcolor{red}{1}, \textcolor{blue}{1} \\\cline{2-3}
        \end{tabular}\
        \begin{tabular}{l|c|c|}
          \multicolumn{1}{c}{} & \multicolumn{1}{c}{\textcolor{blue}{L'}} & \multicolumn{1}{c}{R'} \\\cline{2-3}
          \textcolor{red}{T'} & \textcolor{red}{3}, \textcolor{blue}{3} & \textcolor{red}{2}, 2 \\\cline{2-3}
          B' & 2, \textcolor{blue}{2} & 1, 1 \\\cline{2-3}
        \end{tabular}
    \end{table}
  \vfill\null\columnbreak
  \vfill\null
  \textbf{\textit{Write up the full strategy profiles for the subgame perfect Nash Equilibria (SPNE).}}
  \end{multicols}
\end{frame}
\begin{frame}{PS6, Ex. 7.b: To keep or split (imperfect information)}
  \begin{itemize}
    \item[(b)] Find the subgame perfect Nash Equilibria (SPNE).
  \end{itemize}
  \vspace{-4pt}
  \begin{figure}[!h]
    \center
    \def\svgwidth{.8\columnwidth}
    \import{figures/}{7b_extensive_form.pdf_tex}
  \end{figure}
  \vspace{-2pt}
  \begin{multicols}{2}
    \begin{figure}[!h]
      \center
      \def\svgwidth{.5\columnwidth}
      \import{figures/}{7b.pdf_tex}
    \end{figure}
    \vspace{-8pt}
    \begin{table}
      \begin{tabular}{l|c|c|}
        \multicolumn{1}{c}{} & \multicolumn{1}{c}{L} & \multicolumn{1}{c}{\textcolor{blue}{R}} \\\cline{2-3}
        T & 3, 3 & 0, \textcolor{blue}{4} \\\cline{2-3}
        \textcolor{red}{B} & \textcolor{red}{4}, 0 & \textcolor{red}{1}, \textcolor{blue}{1} \\\cline{2-3}
        \end{tabular}\
        \begin{tabular}{l|c|c|}
          \multicolumn{1}{c}{} & \multicolumn{1}{c}{\textcolor{blue}{L'}} & \multicolumn{1}{c}{R'} \\\cline{2-3}
          \textcolor{red}{T'} & \textcolor{red}{3}, \textcolor{blue}{3} & \textcolor{red}{2}, 2 \\\cline{2-3}
          B' & 2, \textcolor{blue}{2} & 1, 1 \\\cline{2-3}
        \end{tabular}
    \end{table}
  \vfill\null\columnbreak
  \vspace{-8pt}
  $SPNE=\{(Split,B,T'),(R,L')\}$ with outcome (3,3).
  \vspace{-6pt}
  \begin{itemize}
    \item[(c)] Now suppose that Player 2 cannot observe Player 1’s choice in the first stage. \textbf{\textit{Draw the game tree (again using information sets)}} and find the pure strategy Nash Equilibria (PSNE).
  \end{itemize}
  \vfill\null
  \end{multicols}
\end{frame}


\begin{frame}{PS6, Ex. 7.c: To keep or split (imperfect information)}
    \begin{itemize}
      \item[(c)] Now suppose that Player 2 cannot observe Player 1’s choice in the first stage. Draw the game tree (again using information sets) \textbf{\textit{and find the pure strategy Nash Equilibria (PSNE).}}
    \end{itemize}
    \begin{figure}[!h]
      \center
      \def\svgwidth{.8\columnwidth}
      \import{figures/}{7c_extensive_form_.pdf_tex}
    \end{figure}
    \vfill\null
\end{frame}
\begin{frame}{PS6, Ex. 7.c: To keep or split (imperfect information)}
    \begin{itemize}
      \item[(c)] Find the pure strategy Nash Equilibria (PSNE).
    \end{itemize}
    \vspace{-4pt}
    \begin{figure}[!h]
      \center
      \def\svgwidth{.8\columnwidth}
      \import{figures/}{7c_extensive_form_.pdf_tex}
    \end{figure}
    \vspace{-4pt}
    \nth{2} and \nth{3} stage in normal form (Player 1 knows her own action in \nth{1} stage):
    \vspace{-4pt}
    \begin{multicols}{2}
      \begin{figure}[!h]
        \center
        \def\svgwidth{.5\columnwidth}
        \import{figures/}{7c_.pdf_tex}
      \end{figure}
      \vspace{-9pt}
      \begin{table}
        \begin{tabular}{l|c|c|}
          \multicolumn{1}{c}{} & \multicolumn{1}{c}{L} & \multicolumn{1}{c}{R} \\\cline{2-3}
          T & 3, 3 & 0, 4 \\\cline{2-3}
          B & 4, 0 & 1, 1 \\\cline{2-3}
        \end{tabular}\
        \begin{tabular}{l|c|c|}
          \multicolumn{1}{c}{} & \multicolumn{1}{c}{L} & \multicolumn{1}{c}{R} \\\cline{2-3}
          T' & 3, 3 & 2, 2 \\\cline{2-3}
          B' & 2, 2 & 1, 1 \\\cline{2-3}
        \end{tabular}
      \end{table}
    \vfill\null \columnbreak
    \vfill\null
  \end{multicols}
\end{frame}
\begin{frame}{PS6, Ex. 7.c: To keep or split (imperfect information)}
    \begin{itemize}
      \item[(c)] Find the PSNE.
    \end{itemize}
    \vspace{-16pt}
    \begin{figure}[!h]
      \center
      \def\svgwidth{.8\columnwidth}
      \import{figures/}{7c_extensive_form_.pdf_tex}
    \end{figure}
    \vspace{-8pt}
    \begin{multicols}{2}
      \nth{2} and \nth{3} stage in normal form:
      \vspace{-4pt}
      \begin{figure}[!h]
        \center
        \def\svgwidth{.5\columnwidth}
        \import{figures/}{7c_.pdf_tex}
      \end{figure}
      \vspace{-9pt}
      \begin{table}
        \begin{tabular}{l|c|c|}
          \multicolumn{1}{c}{} & \multicolumn{1}{c}{L} & \multicolumn{1}{c}{R} \\\cline{2-3}
          T & 3, 3 & 0, 4 \\\cline{2-3}
          B & 4, 0 & 1, 1 \\\cline{2-3}
        \end{tabular}\
        \begin{tabular}{l|c|c|}
          \multicolumn{1}{c}{} & \multicolumn{1}{c}{L} & \multicolumn{1}{c}{R} \\\cline{2-3}
          T' & 3, 3 & 2, 2 \\\cline{2-3}
          B' & 2, 2 & 1, 1 \\\cline{2-3}
        \end{tabular}
      \end{table}
    \vfill\null \columnbreak
    Full game:
    \vspace{-16pt}
    \begin{table}
      \begin{tabular}{l|c|c|}
        \multicolumn{1}{c}{} & \multicolumn{1}{c}{L} & \multicolumn{1}{c}{R} \\\cline{2-3}
        Keep, T, T' & 3, 3 & 0, 4 \\\cline{2-3}
        Keep, T, B' & 3, 3 & 0, 4 \\\cline{2-3}
        Keep, B, T' & 4, 0 & 1, 1 \\\cline{2-3}
        Keep, B, B' & 4, 0 & 1, 1 \\\cline{2-3}
        Split, T, T' & 3, 3 & 2, 2 \\\cline{2-3}
        Split, B, T' & 3, 3 & 2, 2 \\\cline{2-3}
        Split, T, B' & 2, 2 & 1, 1 \\\cline{2-3}
        Split, B, B' & 2, 2 & 1, 1 \\\cline{2-3}
      \end{tabular}
    \end{table}
    \vfill\null
  \end{multicols}
\end{frame}
\begin{frame}{PS6, Ex. 7.c: To keep or split (imperfect information)}
    \begin{itemize}
      \item[(c)] Find the PSNE.
    \end{itemize}
    \vspace{-16pt}
    \begin{figure}[!h]
      \center
      \def\svgwidth{.8\columnwidth}
      \import{figures/}{7c_extensive_form_.pdf_tex}
    \end{figure}
    \vspace{-8pt}
    \begin{multicols}{2}
      \nth{2} and \nth{3} stage in normal form:
      \vspace{-4pt}
      \begin{figure}[!h]
        \center
        \def\svgwidth{.5\columnwidth}
        \import{figures/}{7c_.pdf_tex}
      \end{figure}
      \vspace{-9pt}
      \begin{table}
        \begin{tabular}{l|c|c|}
          \multicolumn{1}{c}{} & \multicolumn{1}{c}{L} & \multicolumn{1}{c}{R} \\\cline{2-3}
          T & 3, 3 & 0, 4 \\\cline{2-3}
          B & 4, 0 & 1, 1 \\\cline{2-3}
        \end{tabular}\
        \begin{tabular}{l|c|c|}
          \multicolumn{1}{c}{} & \multicolumn{1}{c}{L} & \multicolumn{1}{c}{R} \\\cline{2-3}
          T' & 3, 3 & 2, 2 \\\cline{2-3}
          B' & 2, 2 & 1, 1 \\\cline{2-3}
        \end{tabular}
      \end{table}
    \vfill\null \columnbreak
    Full game:
    \vspace{-16pt}
    \begin{table}
      \begin{tabular}{l|c|c|}
        \multicolumn{1}{c}{} & \multicolumn{1}{c}{L} & \multicolumn{1}{c}{R} \\\cline{2-3}
        Keep, T, T' & 3, 3 & 0, \textcolor{blue}{4} \\\cline{2-3}
        Keep, T, B' & 3, 3 & 0, \textcolor{blue}{4} \\\cline{2-3}
        Keep, B, T' & \textcolor{red}{4}, 0 & 1, \textcolor{blue}{1} \\\cline{2-3}
        Keep, B, B' & \textcolor{red}{4}, 0 & 1, \textcolor{blue}{1} \\\cline{2-3}
        Split, T, T' & 3, \textcolor{blue}{3} & \textcolor{red}{2}, 2 \\\cline{2-3}
        Split, B, T' & 3, \textcolor{blue}{3} & \textcolor{red}{2}, 2 \\\cline{2-3}
        Split, T, B' & 2, \textcolor{blue}{2} & 1, 1 \\\cline{2-3}
        Split, B, B' & 2, \textcolor{blue}{2} & 1, 1 \\\cline{2-3}
      \end{tabular}
    \end{table}
    \vfill\null
  \end{multicols}
\end{frame}
\begin{frame}{PS6, Ex. 7.c: To keep or split (imperfect information)}
    \begin{itemize}
      \item[(c)] \textbf{\textit{NO PSNE EXISTS!}}
    \end{itemize}
    \vspace{-16pt}
    \begin{figure}[!h]
      \center
      \def\svgwidth{.8\columnwidth}
      \import{figures/}{7c_extensive_form_.pdf_tex}
    \end{figure}
    \vspace{-8pt}
    \begin{multicols}{2}
      \nth{2} and \nth{3} stage in normal form:
      \vspace{-4pt}
      \begin{figure}[!h]
        \center
        \def\svgwidth{.5\columnwidth}
        \import{figures/}{7c_.pdf_tex}
      \end{figure}
      \vspace{-9pt}
      \begin{table}
        \begin{tabular}{l|c|c|}
          \multicolumn{1}{c}{} & \multicolumn{1}{c}{L} & \multicolumn{1}{c}{R} \\\cline{2-3}
          T & 3, 3 & 0, 4 \\\cline{2-3}
          B & 4, 0 & 1, 1 \\\cline{2-3}
        \end{tabular}\
        \begin{tabular}{l|c|c|}
          \multicolumn{1}{c}{} & \multicolumn{1}{c}{L} & \multicolumn{1}{c}{R} \\\cline{2-3}
          T' & 3, 3 & 2, 2 \\\cline{2-3}
          B' & 2, 2 & 1, 1 \\\cline{2-3}
        \end{tabular}
      \end{table}
    \vfill\null \columnbreak
    Full game:
    \vspace{-16pt}
    \begin{table}
      \begin{tabular}{l|c|c|}
        \multicolumn{1}{c}{} & \multicolumn{1}{c}{L} & \multicolumn{1}{c}{R} \\\cline{2-3}
        Keep, T, T' & 3, 3 & 0, \textcolor{blue}{4} \\\cline{2-3}
        Keep, T, B' & 3, 3 & 0, \textcolor{blue}{4} \\\cline{2-3}
        Keep, B, T' & \textcolor{red}{4}, 0 & 1, \textcolor{blue}{1} \\\cline{2-3}
        Keep, B, B' & \textcolor{red}{4}, 0 & 1, \textcolor{blue}{1} \\\cline{2-3}
        Split, T, T' & 3, \textcolor{blue}{3} & \textcolor{red}{2}, 2 \\\cline{2-3}
        Split, B, T' & 3, \textcolor{blue}{3} & \textcolor{red}{2}, 2 \\\cline{2-3}
        Split, T, B' & 2, \textcolor{blue}{2} & 1, 1 \\\cline{2-3}
        Split, B, B' & 2, \textcolor{blue}{2} & 1, 1 \\\cline{2-3}
      \end{tabular}
    \end{table}
    \vfill\null
  \end{multicols}
\end{frame}




\section{Code examples} % out-comment: ctrl-shift-7 or ctrl-shift-* (use cmd for Mac)

\begin{frame}{Code examples}
  \begin{multicols}{2}
    % Game tree: % In general, I recommend drawing game trees in the hand as it is the fastest and resembles the exam situation. If you write your assignments on the computer, you can take a picture or leave space to draw the figures after printing. For the slides, I draw the game trees in Inkscape, which is a great piece of free software – when you have gotten used to it… Editing an existing game tree can be a quite straightforward start, but exporting the illustration to a LaTeX document can again be a bit cumbersome. If you’re persistent, you can find “7b_extensive_form.svg” in the figures folder of the zip-file and edit it with Inkscape. As you see, you can use LaTeX code such as $x_1$. Then you save it as type: “Portable Document Format (*.pdf)” and choose “Omit text in PDF and create LaTeX file” and “Use exported object’s size”, which creates two new files (*.pdf and *.pdf_tex). Both must be uploaded to Overleaf to even see how the figures looks, as the files make no sense on their own. To add them to your document, search for “svg” in the main.tex file and re-use my code.
    \begin{figure}[!h]
      \center
      \def\svgwidth{.8\columnwidth}
      \import{figures/}{long_.pdf_tex}
    \end{figure}
  \vfill\null \columnbreak
    Matrix, no player names:
    \vspace{-10pt}
    \begin{table} % as opposed to matrices with player names, each line does not start with "&" as there's no empty column for the name-box. Otherwise, see the explanations below.
      \begin{tabular}{l|c|c|}
        \multicolumn{1}{c}{} & \multicolumn{1}{c}{L (q)} & \multicolumn{1}{c}{R (1-q)} \\\cline{2-3}
        T (p)   &  &  \\\cline{2-3}
        B (1-p) &  &  \\\cline{2-3}
      \end{tabular}
    \end{table}
    Matrix, no colors:
    \vspace{-10pt}
    \begin{table}
      \begin{tabular}{cl|c|c|} % the number of total columns and which have vertical lines between them (left-align or center text).
        & \multicolumn{1}{c}{} & \multicolumn{2}{c}{Player 2}\\ % "2" is the number of columns in the matrix that the 2nd player name spans over
        \parbox[t]{1mm}{\multirow{3}{*}{\rotatebox[origin=r]{90}{Player 1}}} % "3" is the number of rows the 1st player name spans over (including the one with the column names)
        & \multicolumn{1}{c}{} & \multicolumn{1}{c}{L (q)} & \multicolumn{1}{c}{R (1-q)} \\\cline{3-4} % column names use the "\multicolumn" command to not draw vertical lines between them.
        & T (p)   &  &  \\\cline{3-4} % a horizontal line is drawn after the line break using "\cline{x-y}" where x and y are the column numbers of the cells to be underlined.
        & B (1-p) &  &  \\\cline{3-4}
      \end{tabular}
    \end{table}
    Matrix, with colors:
    \vspace{-10pt}
    \begin{table}
      \begin{tabular}{cl|c|c|}
        & \multicolumn{1}{c}{} & \multicolumn{2}{c}{\color{blue}Player 2}\\
        \parbox[t]{1mm}{\multirow{3}{*}{\rotatebox[origin=r]{90}{\color{red}Player 1}}}
        & \multicolumn{1}{c}{} & \multicolumn{1}{c}{L (q)} & \multicolumn{1}{c}{R (1-q)} \\\cline{3-4}
        & T (p)   & \textcolor{red}{1}, \textcolor{blue}{1} &   \\\cline{3-4}
        & B (1-p) &  &  \\\cline{3-4}
      \end{tabular}
    \end{table}
  \vfill\null
  \end{multicols}
\end{frame}


\end{document}
