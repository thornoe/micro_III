\section{PS7, Ex. 8: Trigger strategy (infinitely repeated game)}

\begin{frame}{PS7, Ex. 8: Trigger strategy (infinitely repeated game)}
  The next exercises use the following game $G$:
  \begin{table}
    \begin{tabular}{l|c|c|c|}
      \multicolumn{1}{c}{} & \multicolumn{1}{c}{L} & \multicolumn{1}{c}{M} & \multicolumn{1}{c}{H} \\\cline{2-4}
      L & 10, 10 & 3, 15 & 0, 7 \\\cline{2-4}
      M & 15, 3 & 7, 7 & -4, 5 \\\cline{2-4}
      H & 7, 0 & 5, -4 & -15, -15 \\\cline{2-4}
    \end{tabular}
  \end{table}
  Suppose that the Players play the infinitely repeated game $G(\infty)$ and that they would like to support as a SPNE the 'collusive' outcome in which $(L, L)$ is played every round.
  \begin{itemize}
    \item[(a)] Define a trigger strategy which delivers the collusive outcome in every period where no deviation has been made, and gives $(x_1, x_2)$ forever after a deviation.
    \item[(b)] A necessary (but not sufficient) condition for a SPNE is $x_1 = x_2 = M$. Explain why.
    \vspace{-4pt} \item[(c)] Suppose $\delta = 4/7$. Show by finding a profitable deviation that the above trigger strategy is not a SPNE. \vspace{-6pt}
  \end{itemize}
  \vfill\null
\end{frame}

\begin{frame}{PS7, Ex. 8.a: Trigger strategy (infinitely repeated game)}
    Consider $G(\infty)$, i.e. the infinitely repeated game with stage game $G$: \vspace{-6pt}
    \begin{table}
      \begin{tabular}{l|c|c|c|}
        \multicolumn{1}{c}{} & \multicolumn{1}{c}{L} & \multicolumn{1}{c}{M} & \multicolumn{1}{c}{H} \\\cline{2-4}
        L & 10, 10 & 3, 15 & 0, 7 \\\cline{2-4}
        M & 15, 3 & 7, 7 & -4, 5 \\\cline{2-4}
        H & 7, 0 & 5, -4 & -15, -15 \\\cline{2-4}
      \end{tabular}
    \end{table}
    \begin{itemize}
      \item[(a)] Define a trigger strategy which delivers the collusive outcome in every period where no deviation has been made, and gives $(x_1, x_2)$ forever after a deviation.
    \end{itemize}
    \vfill\null
\end{frame}
\begin{frame}{PS7, Ex. 8.a: Trigger strategy (infinitely repeated game)}
    Consider $G(\infty)$, i.e. the infinitely repeated game with stage game $G$: \vspace{-6pt}
    \begin{table}
      \begin{tabular}{l|c|c|c|}
        \multicolumn{1}{c}{} & \multicolumn{1}{c}{L} & \multicolumn{1}{c}{M} & \multicolumn{1}{c}{H} \\\cline{2-4}
        L & 10, 10 & 3, 15 & 0, 7 \\\cline{2-4}
        M & 15, 3 & 7, 7 & -4, 5 \\\cline{2-4}
        H & 7, 0 & 5, -4 & -15, -15 \\\cline{2-4}
      \end{tabular}
    \end{table}
    \begin{itemize}
      \item[(a)] Define a trigger strategy which delivers the collusive outcome in every period where no deviation has been made, and gives $(x_1, x_2)$ forever after a deviation.
    \end{itemize}
  \begin{multicols}{2}
    \begin{itemize}
      \item[(Step a)] Write up the trigger strategy.
    \end{itemize}
    \vfill\null\columnbreak
    \vfill\null
  \end{multicols}
\end{frame}
\begin{frame}{PS7, Ex. 8.a: Trigger strategy (infinitely repeated game)}
    Consider $G(\infty)$, i.e. the infinitely repeated game with stage game $G$: \vspace{-6pt}
    \begin{table}
      \begin{tabular}{l|c|c|c|}
        \multicolumn{1}{c}{} & \multicolumn{1}{c}{L} & \multicolumn{1}{c}{M} & \multicolumn{1}{c}{H} \\\cline{2-4}
        L & 10, 10 & 3, 15 & 0, 7 \\\cline{2-4}
        M & 15, 3 & 7, 7 & -4, 5 \\\cline{2-4}
        H & 7, 0 & 5, -4 & -15, -15 \\\cline{2-4}
      \end{tabular}
    \end{table}
    \begin{itemize}
      \item[(a)] Define a trigger strategy which delivers the collusive outcome in every period where no deviation has been made, and gives $(x_1, x_2)$ forever after a deviation.
    \end{itemize}
  \begin{multicols}{2}
    \begin{itemize}
      \item[(Step a)] Write up the trigger strategy.
    \end{itemize}
    \vfill\null\columnbreak
    Information so far:
    \begin{enumerate}
      \item Trigger strategy for Player $i\in1,2$: "If $t=1$ or if the outcome in all previous stages was $(L,L)$, play $L$. Otherwise, play $x_i$."
    \end{enumerate}
    \vfill\null
  \end{multicols}
\end{frame}

\begin{frame}{PS7, Ex. 8.b: Trigger strategy (infinitely repeated game)}
    Consider $G(\infty)$, i.e. the infinitely repeated game with stage game $G$: \vspace{-6pt}
    \begin{table}
      \begin{tabular}{l|c|c|c|}
        \multicolumn{1}{c}{} & \multicolumn{1}{c}{L} & \multicolumn{1}{c}{M} & \multicolumn{1}{c}{H} \\\cline{2-4}
        L & 10, 10 & 3, 15 & 0, 7 \\\cline{2-4}
        M & 15, 3 & 7, 7 & -4, 5 \\\cline{2-4}
        H & 7, 0 & 5, -4 & -15, -15 \\\cline{2-4}
      \end{tabular}
    \end{table}
    \begin{itemize}
      \item[(b)] A necessary (but not sufficient) condition for a SPNE is $x_1 = x_2 = M$. Explain why.
    \end{itemize}
  \begin{multicols}{2}
    \vfill\null\columnbreak
    Information so far:
    \begin{enumerate}
      \item Trigger strategy for Player $i\in1,2$: "If $t=1$ or if the outcome in all previous stages was $(L,L)$, play $L$. Otherwise, play $x_i$."
    \end{enumerate}
    \vfill\null
  \end{multicols}
\end{frame}
\begin{frame}{PS7, Ex. 8.b: Trigger strategy (infinitely repeated game)}
    Consider $G(\infty)$, i.e. the infinitely repeated game with stage game $G$: \vspace{-6pt}
    \begin{table}
      \begin{tabular}{l|c|c|c|}
        \multicolumn{1}{c}{} & \multicolumn{1}{c}{L} & \multicolumn{1}{c}{M} & \multicolumn{1}{c}{H} \\\cline{2-4}
        L & 10, 10 & 3, 15 & 0, 7 \\\cline{2-4}
        M & 15, 3 & 7, 7 & -4, 5 \\\cline{2-4}
        H & 7, 0 & 5, -4 & -15, -15 \\\cline{2-4}
      \end{tabular}
    \end{table}
    \begin{itemize}
      \item[(b)] A necessary (but not sufficient) condition for a SPNE is $x_1 = x_2 = M$. Explain why.
    \end{itemize}
  \begin{multicols}{2}
    \begin{itemize}
      \item[(Step a)] Find the PSNE in the stage game $G$.
    \end{itemize}
    \vfill\null\columnbreak
    Information so far:
    \begin{enumerate}
      \item Trigger strategy for Player $i\in1,2$: "If $t=1$ or if the outcome in all previous stages was $(L,L)$, play $L$. Otherwise, play $x_i$."
    \end{enumerate}
    \vfill\null
  \end{multicols}
\end{frame}
\begin{frame}{PS7, Ex. 8.b: Trigger strategy (infinitely repeated game)}
    Consider $G(\infty)$, i.e. the infinitely repeated game with stage game $G$: \vspace{-6pt}
    \begin{table}
      \begin{tabular}{l|c|c|c|}
        \multicolumn{1}{c}{} & \multicolumn{1}{c}{L} & \multicolumn{1}{c}{M} & \multicolumn{1}{c}{H} \\\cline{2-4}
        L & 10, 10 & 3, \textcolor{blue}{15} & \textcolor{red}{0}, 7 \\\cline{2-4}
        M & \textcolor{red}{15}, 3 & \textcolor{red}{7}, \textcolor{blue}{7} & -4, 5 \\\cline{2-4}
        H & 7, \textcolor{blue}{0} & 5, -4 & -15, -15 \\\cline{2-4}
      \end{tabular}
    \end{table}
    \begin{itemize}
      \item[(b)] A necessary (but not sufficient) condition for a SPNE is $x_1 = x_2 = M$. Explain why.
    \end{itemize}
  \begin{multicols}{2}
    \begin{itemize}
      \item[(Step a)] Find the PSNE in the stage game $G$.
    \end{itemize}
    \vfill\null\columnbreak
    Information so far:
    \begin{enumerate}
      \item Trigger strategy for Player $i\in1,2$: "If $t=1$ or if the outcome in all previous stages was $(L,L)$, play $L$. Otherwise, play $x_i$."
      \item Stage game NE: $(M,M)$.
    \end{enumerate}
    \vfill\null
  \end{multicols}
\end{frame}
\begin{frame}{PS7, Ex. 8.b: Trigger strategy (infinitely repeated game)}
    Consider $G(\infty)$, i.e. the infinitely repeated game with stage game $G$: \vspace{-6pt}
    \begin{table}
      \begin{tabular}{l|c|c|c|}
        \multicolumn{1}{c}{} & \multicolumn{1}{c}{L} & \multicolumn{1}{c}{M} & \multicolumn{1}{c}{H} \\\cline{2-4}
        L & 10, 10 & 3, \textcolor{blue}{15} & \textcolor{red}{0}, 7 \\\cline{2-4}
        M & \textcolor{red}{15}, 3 & \textcolor{red}{7}, \textcolor{blue}{7} & -4, 5 \\\cline{2-4}
        H & 7, \textcolor{blue}{0} & 5, -4 & -15, -15 \\\cline{2-4}
      \end{tabular}
    \end{table}
    \begin{itemize}
      \item[(b)] A necessary (but not sufficient) condition for a SPNE is $x_1 = x_2 = M$. Explain why.
    \end{itemize}
  \begin{multicols}{2}
    \begin{itemize}
      \item[(Step a)] Find the PSNE in the stage game $G$.
      \item[(Step b)] Explain.
    \end{itemize}
    \vfill\null\columnbreak
    Information so far:
    \begin{enumerate}
      \item Trigger strategy for Player $i\in1,2$: "If $t=1$ or if the outcome in all previous stages was $(L,L)$, play $L$. Otherwise, play $x_i$."
      \item Stage game NE: $(M,M)$.
    \end{enumerate}
    \vfill\null
  \end{multicols}
\end{frame}
\begin{frame}{PS7, Ex. 8.b: Trigger strategy (infinitely repeated game)}
    Consider $G(\infty)$, i.e. the infinitely repeated game with stage game $G$: \vspace{-6pt}
    \begin{table}
      \begin{tabular}{l|c|c|c|}
        \multicolumn{1}{c}{} & \multicolumn{1}{c}{L} & \multicolumn{1}{c}{M} & \multicolumn{1}{c}{H} \\\cline{2-4}
        L & 10, 10 & 3, \textcolor{blue}{15} & \textcolor{red}{0}, 7 \\\cline{2-4}
        M & \textcolor{red}{15}, 3 & \textcolor{red}{7}, \textcolor{blue}{7} & -4, 5 \\\cline{2-4}
        H & 7, \textcolor{blue}{0} & 5, -4 & -15, -15 \\\cline{2-4}
      \end{tabular}
    \end{table}
    \begin{itemize}
      \item[(b)] A necessary (but not sufficient) condition for a SPNE is $x_1 = x_2 = M$. Explain why.
    \end{itemize}
  \begin{multicols}{2}
    \begin{itemize}
      \item[(Step a)] Find the PSNE in the stage game $G$.
      \item[(Step b)] Explain.
    \end{itemize}
    For a trigger strategy to constitute a SPNE, the threat of (eternal and unchangeable) punishment must be credible, i.e. must be a stage game NE.\\\medskip
    Thus, $x_1 = x_2 = M$ is a necessary (but not sufficient) condition for the trigger strategies to constitute a SPNE.
    \vfill\null\columnbreak
    Information so far:
    \begin{enumerate}
      \item Trigger strategy for Player $i\in1,2$: "If $t=1$ or if the outcome in all previous stages was $(L,L)$, play $L$. Otherwise, play $x_i$."
      \item Unique stage game NE: $(M,M)$.
    \end{enumerate}
    \vfill\null
  \end{multicols}
\end{frame}

\begin{frame}{PS7, Ex. 8.c: Trigger strategy (infinitely repeated game)}
    Consider $G(\infty)$, i.e. the infinitely repeated game with stage game $G$: \vspace{-6pt}
    \begin{table}
      \begin{tabular}{l|c|c|c|}
        \multicolumn{1}{c}{} & \multicolumn{1}{c}{L} & \multicolumn{1}{c}{M} & \multicolumn{1}{c}{H} \\\cline{2-4}
        L & 10, 10 & 3, \textcolor{blue}{15} & \textcolor{red}{0}, 7 \\\cline{2-4}
        M & \textcolor{red}{15}, 3 & \textcolor{red}{7}, \textcolor{blue}{7} & -4, 5 \\\cline{2-4}
        H & 7, \textcolor{blue}{0} & 5, -4 & -15, -15 \\\cline{2-4}
      \end{tabular}
    \end{table}
    \begin{itemize}
      \vspace{-4pt} \item[(c)] Suppose $\delta = 4/7$. Show by finding a profitable deviation that the above trigger strategy is not a SPNE. \vspace{-6pt}
    \end{itemize}
  \begin{multicols}{2}
    \vfill\null\columnbreak
    Information so far:
    \begin{enumerate}
      \item Trigger Strategy (TS): "If $t=1$ or if the outcome in all previous stages was $(L,L)$, play $L$. If not, play $M$."
    \end{enumerate}
    \vfill\null
  \end{multicols}
\end{frame}
\begin{frame}{PS7, Ex. 8.c: Trigger strategy (infinitely repeated game)}
    Consider $G(\infty)$, i.e. the infinitely repeated game with stage game $G$: \vspace{-6pt}
    \begin{table}
      \begin{tabular}{l|c|c|c|}
        \multicolumn{1}{c}{} & \multicolumn{1}{c}{L} & \multicolumn{1}{c}{M} & \multicolumn{1}{c}{H} \\\cline{2-4}
        L & 10, 10 & 3, \textcolor{blue}{15} & \textcolor{red}{0}, 7 \\\cline{2-4}
        M & \textcolor{red}{15}, 3 & \textcolor{red}{7}, \textcolor{blue}{7} & -4, 5 \\\cline{2-4}
        H & 7, \textcolor{blue}{0} & 5, -4 & -15, -15 \\\cline{2-4}
      \end{tabular}
    \end{table}
    \begin{itemize}
      \vspace{-4pt} \item[(c)] Suppose $\delta = 4/7$. Show by finding a profitable deviation that the above trigger strategy is not a SPNE. \vspace{-6pt}
    \end{itemize}
  \begin{multicols}{2}
    \begin{itemize}
      \item[(Step a)] Given Player $j$ plays the Trigger Strategy (TS), write up Player $i$'s Optimal Deviation Strategy (ODS).
    \end{itemize}
    \vfill\null\columnbreak
    Information so far:
    \begin{enumerate}
      \item Trigger Strategy (TS): "If $t=1$ or if the outcome in all previous stages was $(L,L)$, play $L$. If not, play $M$."
    \end{enumerate}
    \vfill\null
  \end{multicols}
\end{frame}
\begin{frame}{PS7, Ex. 8.c: Trigger strategy (infinitely repeated game)}
    Consider $G(\infty)$, i.e. the infinitely repeated game with stage game $G$: \vspace{-6pt}
    \begin{table}
      \begin{tabular}{l|c|c|c|}
        \multicolumn{1}{c}{} & \multicolumn{1}{c}{L} & \multicolumn{1}{c}{M} & \multicolumn{1}{c}{H} \\\cline{2-4}
        L & 10, 10 & 3, \textcolor{blue}{15} & \textcolor{red}{0}, 7 \\\cline{2-4}
        M & \textcolor{red}{15}, 3 & \textcolor{red}{7}, \textcolor{blue}{7} & -4, 5 \\\cline{2-4}
        H & 7, \textcolor{blue}{0} & 5, -4 & -15, -15 \\\cline{2-4}
      \end{tabular}
    \end{table}
    \begin{itemize}
      \vspace{-4pt} \item[(c)] Suppose $\delta = 4/7$. Show by finding a profitable deviation that the above trigger strategy is not a SPNE. \vspace{-6pt}
    \end{itemize}
  \begin{multicols}{2}
    \begin{itemize}
      \item[(Step a)] Given Player $j$ plays the Trigger Strategy (TS), write up Player $i$'s Optimal Deviation Strategy (ODS).
    \end{itemize}
    \vfill\null\columnbreak
    Information so far:
    \begin{enumerate}
      \item Trigger Strategy (TS): "If $t=1$ or if the outcome in all previous stages was $(L,L)$, play $L$. If not, play $M$."
      \item Optimal Deviation Strategy (ODS): "Always play $M$."
    \end{enumerate}
    \vfill\null
  \end{multicols}
\end{frame}
\begin{frame}{PS7, Ex. 8.c: Trigger strategy (infinitely repeated game)}
    Consider $G(\infty)$, i.e. the infinitely repeated game with stage game $G$: \vspace{-6pt}
    \begin{table}
      \begin{tabular}{l|c|c|c|}
        \multicolumn{1}{c}{} & \multicolumn{1}{c}{L} & \multicolumn{1}{c}{M} & \multicolumn{1}{c}{H} \\\cline{2-4}
        L & 10, 10 & 3, \textcolor{blue}{15} & \textcolor{red}{0}, 7 \\\cline{2-4}
        M & \textcolor{red}{15}, 3 & \textcolor{red}{7}, \textcolor{blue}{7} & -4, 5 \\\cline{2-4}
        H & 7, \textcolor{blue}{0} & 5, -4 & -15, -15 \\\cline{2-4}
      \end{tabular}
    \end{table}
    \begin{itemize}
      \vspace{-4pt} \item[(c)] Suppose $\delta = 4/7$. Show by finding a profitable deviation that the above trigger strategy is not a SPNE. \vspace{-6pt}
    \end{itemize}
  \begin{multicols}{2}
    \begin{itemize}
      \item[(Step a)] Given Player $j$ plays the Trigger Strategy (TS), write up Player $i$'s Optimal Deviation Strategy (ODS).
      \item[(Step b)] Given Player $j$ plays TS, write up Player $i$'s respective payoffs from playing TS and ODS.
    \end{itemize}
    \vfill\null\columnbreak
    Information so far:
    \begin{enumerate}
      \item Trigger Strategy (TS): "If $t=1$ or if the outcome in all previous stages was $(L,L)$, play $L$. If not, play $M$."
      \item Optimal Deviation Strategy (ODS): "Always play $M$."
    \end{enumerate}
    \vfill\null
  \end{multicols}
\end{frame}
\begin{frame}{PS7, Ex. 8.c: Trigger strategy (infinitely repeated game)}
    Consider $G(\infty)$, i.e. the infinitely repeated game with stage game $G$: \vspace{-6pt}
    \begin{table}
      \begin{tabular}{l|c|c|c|}
        \multicolumn{1}{c}{} & \multicolumn{1}{c}{L} & \multicolumn{1}{c}{M} & \multicolumn{1}{c}{H} \\\cline{2-4}
        L & 10, 10 & 3, \textcolor{blue}{15} & \textcolor{red}{0}, 7 \\\cline{2-4}
        M & \textcolor{red}{15}, 3 & \textcolor{red}{7}, \textcolor{blue}{7} & -4, 5 \\\cline{2-4}
        H & 7, \textcolor{blue}{0} & 5, -4 & -15, -15 \\\cline{2-4}
      \end{tabular}
    \end{table}
    \begin{itemize}
      \vspace{-4pt} \item[(c)] Suppose $\delta = 4/7$. Show by finding a profitable deviation that the above trigger strategy is not a SPNE. \vspace{-6pt}
    \end{itemize}
  \begin{multicols}{2}
    \begin{itemize}
      \item[(Step a)] Given Player $j$ plays the Trigger Strategy (TS), write up Player $i$'s Optimal Deviation Strategy (ODS).
      \item[(Step b)] Given Player $j$ plays TS, write up Player $i$'s respective payoffs from playing TS and ODS.
    \end{itemize}
    Player $i$'s payoff from playing TS:
    \begin{align*}
      10+10\delta+10\delta^2+...=\sum_{t=1}^\infty10\delta^{t-1}=\frac{10}{1-\delta}
    \end{align*}
    \vfill\null\columnbreak
    Information so far:
    \begin{enumerate}
      \item Trigger Strategy (TS): "If $t=1$ or if the outcome in all previous stages was $(L,L)$, play $L$. If not, play $M$."
      \item Optimal Deviation Strategy (ODS): "Always play $M$."
      \item $U_i(TS,TS)=\frac{10}{1-\delta}$.
    \end{enumerate}
    \vfill\null
  \end{multicols}
\end{frame}
\begin{frame}{PS7, Ex. 8.c: Trigger strategy (infinitely repeated game)}
    Consider $G(\infty)$, i.e. the infinitely repeated game with stage game $G$: \vspace{-6pt}
    \begin{table}
      \begin{tabular}{l|c|c|c|}
        \multicolumn{1}{c}{} & \multicolumn{1}{c}{L} & \multicolumn{1}{c}{M} & \multicolumn{1}{c}{H} \\\cline{2-4}
        L & 10, 10 & 3, \textcolor{blue}{15} & \textcolor{red}{0}, 7 \\\cline{2-4}
        M & \textcolor{red}{15}, 3 & \textcolor{red}{7}, \textcolor{blue}{7} & -4, 5 \\\cline{2-4}
        H & 7, \textcolor{blue}{0} & 5, -4 & -15, -15 \\\cline{2-4}
      \end{tabular}
    \end{table}
    \begin{itemize}
      \vspace{-4pt} \item[(c)] Suppose $\delta = 4/7$. Show by finding a profitable deviation that the above trigger strategy is not a SPNE. \vspace{-6pt}
    \end{itemize}
  \begin{multicols}{2}
    \begin{itemize}
      \item[(Step a)] Given Player $j$ plays the Trigger Strategy (TS), write up Player $i$'s Optimal Deviation Strategy (ODS).
      \item[(Step b)] Given Player $j$ plays TS, write up Player $i$'s respective payoffs from playing TS and ODS.
    \end{itemize}
    Player $i$'s payoff from playing TS:
    \vspace{-4pt}
    \begin{align*}
      10+10\delta+10\delta^2+...=\sum_{t=1}^\infty10\delta^{t-1}=\frac{10}{1-\delta}
    \end{align*}
    Player $i$'s payoff from playing ODS:
    \vspace{-4pt}
    \begin{align*}
      15+7\delta+7\delta^2+...=15+\sum_{\bm{t=2}}^\infty7\delta^{t-1}=15+\frac{7\delta}{1-\delta}
    \end{align*}
    \vfill\null\columnbreak
    Information so far:
    \begin{enumerate}
      \item Trigger Strategy (TS): "If $t=1$ or if the outcome in all previous stages was $(L,L)$, play $L$. If not, play $M$."
      \item Optimal Deviation Strategy (ODS): "Always play $M$."
      \item $U_i(TS,TS)=\frac{10}{1-\delta}$
      \item $U_i(ODS,TS)=15+\frac{7\delta}{1-\delta}$
    \end{enumerate}
    \vfill\null
  \end{multicols}
\end{frame}
\begin{frame}{PS7, Ex. 8.c: Trigger strategy (infinitely repeated game)}
    Consider $G(\infty)$, i.e. the infinitely repeated game with stage game $G$: \vspace{-6pt}
    \begin{table}
      \begin{tabular}{l|c|c|c|}
        \multicolumn{1}{c}{} & \multicolumn{1}{c}{L} & \multicolumn{1}{c}{M} & \multicolumn{1}{c}{H} \\\cline{2-4}
        L & 10, 10 & 3, \textcolor{blue}{15} & \textcolor{red}{0}, 7 \\\cline{2-4}
        M & \textcolor{red}{15}, 3 & \textcolor{red}{7}, \textcolor{blue}{7} & -4, 5 \\\cline{2-4}
        H & 7, \textcolor{blue}{0} & 5, -4 & -15, -15 \\\cline{2-4}
      \end{tabular}
    \end{table}
    \begin{itemize}
      \vspace{-4pt} \item[(c)] Suppose $\delta = 4/7$. Show by finding a profitable deviation that the above trigger strategy is not a SPNE. \vspace{-6pt}
    \end{itemize}
  \begin{multicols}{2}
    \begin{itemize}
      \item[(Step a)] Given Player $j$ plays the Trigger Strategy (TS), write up Player $i$'s Optimal Deviation Strategy (ODS).
      \item[(Step b)] Given Player $j$ plays TS, write up Player $i$'s respective payoffs from playing TS and ODS.
      \item[(Step c)] Show that the deviation is preferred for $\delta=4/7$.
    \end{itemize}
    \vfill\null\columnbreak
    Information so far:
    \begin{enumerate}
      \item Trigger Strategy (TS): "If $t=1$ or if the outcome in all previous stages was $(L,L)$, play $L$. If not, play $M$."
      \item Optimal Deviation Strategy (ODS): "Always play $M$."
      \item $U_i(TS,TS)=\frac{10}{1-\delta}$
      \item $U_i(ODS,TS)=15+\frac{7\delta}{1-\delta}$
    \end{enumerate}
    \vfill\null
  \end{multicols}
\end{frame}
\begin{frame}{PS7, Ex. 8.c: Trigger strategy (infinitely repeated game)}
    Consider $G(\infty)$, i.e. the infinitely repeated game with stage game $G$: \vspace{-6pt}
    \begin{table}
      \begin{tabular}{l|c|c|c|}
        \multicolumn{1}{c}{} & \multicolumn{1}{c}{L} & \multicolumn{1}{c}{M} & \multicolumn{1}{c}{H} \\\cline{2-4}
        L & 10, 10 & 3, \textcolor{blue}{15} & \textcolor{red}{0}, 7 \\\cline{2-4}
        M & \textcolor{red}{15}, 3 & \textcolor{red}{7}, \textcolor{blue}{7} & -4, 5 \\\cline{2-4}
        H & 7, \textcolor{blue}{0} & 5, -4 & -15, -15 \\\cline{2-4}
      \end{tabular}
    \end{table}
    \begin{itemize}
      \vspace{-4pt} \item[(c)] Suppose $\delta = 4/7$. Show by finding a profitable deviation that the above trigger strategy is not a SPNE. \vspace{-6pt}
    \end{itemize}
  \begin{multicols}{2}
    \begin{itemize}
      \item[(Step a)] Given Player $j$ plays the Trigger Strategy (TS), write up Player $i$'s Optimal Deviation Strategy (ODS).
      \item[(Step b)] Given Player $j$ plays TS, write up Player $i$'s respective payoffs from playing TS and ODS.
      \item[(Step c)] Show that the deviation is preferred for $\delta=4/7$:
    \end{itemize}
    \vspace{-8pt}
    \begin{align*}
      U_i\left(ODS,TS\right)&>U_i\left(TS,TS\right)\\
      \Rightarrow15+\frac{7\frac{4}{7}}{1-\frac{4}{7}}&>\frac{10}{1-\frac{4}{7}},&&\text{for }\delta=\frac{4}{7}\\
      \Rightarrow\frac{73}{3}&>\frac{70}{3}&&\textit{Q.E.D.}
    \end{align*}
    \vfill\null\columnbreak
    Information so far:
    \begin{enumerate}
      \item Trigger Strategy (TS): "If $t=1$ or if the outcome in all previous stages was $(L,L)$, play $L$. If not, play $M$."
      \item Optimal Deviation Strategy (ODS): "Always play $M$."
      \item $U_i(TS,TS)=\frac{10}{1-\delta}$
      \item $U_i(ODS,TS)=15+\frac{7\delta}{1-\delta}$
    \end{enumerate}
    \vfill\null
  \end{multicols}
\end{frame}



\section{PS7, Ex. 9: Optimal Punishment Strategy (infinitely repeated game)}

\begin{frame}{PS7, Ex. 9: Optimal Punishment Strategy (infinitely repeated game)}
    We continue analyzing $G(\infty)$. As in \sout{Lecture 8} (Lecture 6, slides 50-68), consider the strategy profile $(OP,OP)$, where $OP$ stands for optimal punishment...\\\medskip
    \textit{[See the lecture slides and the full description of the exercise in the problem set.]}\\\medskip
    Stage game $G$:\vspace{-6pt}
    \begin{table}
      \begin{tabular}{l|c|c|c|}
        \multicolumn{1}{c}{} & \multicolumn{1}{c}{L} & \multicolumn{1}{c}{M} & \multicolumn{1}{c}{H} \\\cline{2-4}
        L & 10, 10 & 3, \textcolor{blue}{15} & \textcolor{red}{0}, 7 \\\cline{2-4}
        M & \textcolor{red}{15}, 3 & \textcolor{red}{7}, \textcolor{blue}{7} & -4, 5 \\\cline{2-4}
        H & 7, \textcolor{blue}{0} & 5, -4 & -15, -15 \\\cline{2-4}
      \end{tabular}
    \end{table}
\end{frame}

\begin{frame}{PS7, Ex. 9.a: Optimal Punishment Strategy (infinitely repeated game)}
    Consider $G(\infty)$, i.e. the infinitely repeated game with stage game $G$: \vspace{-6pt}
    \begin{table}
      \begin{tabular}{l|c|c|c|}
        \multicolumn{1}{c}{} & \multicolumn{1}{c}{L} & \multicolumn{1}{c}{M} & \multicolumn{1}{c}{H} \\\cline{2-4}
        L & 10, 10 & 3, \textcolor{blue}{15} & \textcolor{red}{0}, 7 \\\cline{2-4}
        M & \textcolor{red}{15}, 3 & \textcolor{red}{7}, \textcolor{blue}{7} & -4, 5 \\\cline{2-4}
        H & 7, \textcolor{blue}{0} & 5, -4 & -15, -15 \\\cline{2-4}
      \end{tabular}
    \end{table}
    \vspace{-4pt}
    \begin{itemize}
      \item[(a)] Discuss how this increased leniency over time may give player 1 an incentive to accept his punishment (and actually play according to $Q_D$, rather than deviate again).
    \end{itemize}
    \vfill\null
\end{frame}
\begin{frame}{PS7, Ex. 9.a: Optimal Punishment Strategy (infinitely repeated game)}
    Consider $G(\infty)$ with stage game $G$, underlining $(Q^D,Q^P)$ in the 'tough' stage: \vspace{-6pt}
    \begin{table}
      \begin{tabular}{l|c|c|c|}
        \multicolumn{1}{c}{} & \multicolumn{1}{c}{L} & \multicolumn{1}{c}{M} & \multicolumn{1}{c}{\textbf{\underline{H}}} \\\cline{2-4}
        L & 10, 10 & 3, \textcolor{blue}{15} & \textcolor{red}{0}, 7 \\\cline{2-4}
        \textbf{\underline{M}} & \textcolor{red}{15}, 3 & \textcolor{red}{7}, \textcolor{blue}{7} & \textbf{\underline{-4, 5}} \\\cline{2-4}
        H & 7, \textcolor{blue}{0} & 5, -4 & -15, -15 \\\cline{2-4}
      \end{tabular}
    \end{table}
    \vspace{-4pt}
    \begin{itemize}
      \item[(a)] Discuss how this increased leniency over time may give player 1 an incentive to accept his punishment (and actually play according to $Q^D$, rather than deviate again).
    \end{itemize}
    \vspace{-8pt}
  \begin{multicols}{2}
    The 'tough' stage (the \nth{1} round of punishment):\vspace{-4pt}
    \begin{itemize}
      \item P1 earns -4 from $(M, H)$ and 3 from $(L, M)$ in all later rounds.
      \item He can deviate by playing $L$ and earn 0, but then the punishment will start over again and P1 will therefore stay in the 'tough' stage.
      \item Both -4 and 0 is less than 3, which is why this punishment structure leaves P1 without an incentive to deviate from the 'punishment path' $(Q^D,Q^P)$.
    \end{itemize}
    \vfill\null\columnbreak
    \vfill\null
  \end{multicols}
\end{frame}
\begin{frame}{PS7, Ex. 9.a: Optimal Punishment Strategy (infinitely repeated game)}
    Consider $G(\infty)$ with stage game $G$, underlining $(Q^D,Q^P)$ in the 'mild' stage: \vspace{-6pt}
    \begin{table}
      \begin{tabular}{l|c|c|c|}
        \multicolumn{1}{c}{} & \multicolumn{1}{c}{L} & \multicolumn{1}{c}{\textbf{\underline{M}}} & \multicolumn{1}{c}{H} \\\cline{2-4}
        \textbf{\underline{L}} & 10, 10 & \textbf{\underline{3, \textcolor{blue}{15}}} & \textcolor{red}{0}, 7 \\\cline{2-4}
        M & \textcolor{red}{15}, 3 & \textcolor{red}{7}, \textcolor{blue}{7} & -4, 5 \\\cline{2-4}
        H & 7, \textcolor{blue}{0} & 5, -4 & -15, -15 \\\cline{2-4}
      \end{tabular}
    \end{table}
    \vspace{-6pt}
    \begin{itemize}
      \item[(a)] Discuss how this increased leniency over time may give player 1 an incentive to accept his punishment (and actually play according to $Q^D$, rather than deviate again).
    \end{itemize}
    \vspace{-8pt}
  \begin{multicols}{2}
    The 'tough' stage (the \nth{1} round of punishment):\vspace{-6pt}
    \begin{itemize}
      \item P1 earns -4 from $(M, H)$ and 3 from $(L, M)$ in all later rounds.
      \item He can deviate by playing $L$ and earn 0, but then the punishment will start over again and P1 will therefore stay in the 'tough' stage.
      \item Both -4 and 0 is less than 3, which is why this punishment structure leaves P1 without an incentive to deviate from the 'punishment path' $(Q^D,Q^P)$.
    \end{itemize}
    \vfill\null\columnbreak
    The 'mild' stage (from the \nth{2} round of punishment):\vspace{-6pt}
    \begin{itemize}
      \item P1 earns 3 from $(L, M)$ in this and all subsequent rounds.
      \item He can deviate by playing $M$ and earn 7, but then the punishment will start over again and P1 will earn -4 in the 'tough' stage (or alternatively 0 by deviating again).
      \item Both -4 and 0 is less than 3, which is why this punishment structure provides P1 with an incentive to stay in the 'mild stage'.
    \end{itemize}
    \vfill\null
  \end{multicols}
\end{frame}

\begin{frame}{PS7, Ex. 9.b: Optimal Punishment Strategy (infinitely repeated game)}
  \begin{multicols}{2}
    The 'tough' stage (the \nth{1} round of punishment):\vspace{-4pt}
    \begin{itemize}
      \item P1 earns -4 from $(M, H)$ and 3 from $(L, M)$ in all later rounds.
      \item He can deviate by playing $L$ and earn 0, but then the punishment will start over again and P1 will therefore stay in the 'tough' stage.
      \item Both -4 and 0 is less than 3, which is why this punishment structure leaves P1 without an incentive to deviate from the punishment path.
    \end{itemize}
    \vfill\null\columnbreak
    The 'mild' stage (from the \nth{2} round of punishment):\vspace{-4pt}
    \begin{itemize}
      \item P1 earns 3 from $(L, M)$ in this and all subsequent rounds.
      \item He can deviate by playing $M$ and earn 7, but then the punishment will start over again and P1 will earn -4 in the 'tough' stage (or alternatively 0 by deviating again).
      \item Both -4 and 0 is less than 3, which is why this punishment structure provides P1 with an incentive to stay in the 'mild stage'.
    \end{itemize}
    \vfill\null
  \end{multicols}
    \vspace{-24pt}
    \begin{itemize}
      \item[(b)] How does your answer relate to the following quote from Wikipedia?
      \item[] \textit{The “carrot and stick” approach is an idiom that refers to a policy of offering a combination of rewards and punishment to induce behavior. It is named in reference to a cart driver dangling a carrot in front of a mule and holding a stick behind it. The mule would move towards the carrot because it wants the reward of food, while also moving away from the stick behind it, since it does not want the punishment of pain, thus drawing the cart.}
    \end{itemize}
\end{frame}
\begin{frame}{PS7, Ex. 9.b: Optimal Punishment Strategy (infinitely repeated game)}
  \begin{itemize}
    \item[(b)] How does your answer relate to the following quote from Wikipedia?
    \item[] \textit{The “carrot and stick” approach is an idiom that refers to a policy of offering a combination of rewards and punishment to induce behavior. It is named in reference to a cart driver dangling a carrot in front of a mule and holding a stick behind it. The mule would move towards the carrot because it wants the reward of food, while also moving away from the stick behind it, since it does not want the punishment of pain, thus drawing the cart.}
  \end{itemize}
  \vspace{-4pt}
  \begin{multicols}{2}
    The 'tough' stage (the \nth{1} round of punishment):
    \begin{itemize}
      \item The promise of a future 'mild punishment' should be regarded as a reward for accepting the punishment without deviating from the 'punishment path' and is therefore \textcolor{orange}{the carrot}.
    \end{itemize}
    \vfill\null\columnbreak
    \vfill\null
  \end{multicols}
\end{frame}
\begin{frame}{PS7, Ex. 9.b: Optimal Punishment Strategy (infinitely repeated game)}
  \begin{itemize}
    \item[(b)] How does your answer relate to the following quote from Wikipedia?
    \item[] \textit{The “carrot and stick” approach is an idiom that refers to a policy of offering a combination of rewards and punishment to induce behavior. It is named in reference to a cart driver dangling a carrot in front of a mule and holding a stick behind it. The mule would move towards the carrot because it wants the reward of food, while also moving away from the stick behind it, since it does not want the punishment of pain, thus drawing the cart.}
  \end{itemize}
  \vspace{-4pt}
  \begin{multicols}{2}
    The 'tough' stage (the \nth{1} round of punishment):
    \begin{itemize}
      \item The promise of a future 'mild punishment' should be regarded as a reward for accepting the punishment without deviating from the 'punishment path' and is therefore \textcolor{orange}{the carrot}.
    \end{itemize}
    \vfill\null\columnbreak
    The 'mild' stage (from the \nth{2} round of punishment):
    \begin{itemize}
      \item The threat of a future 'tough punishment' should be regarded as further punishment for deviating in the first place and is therefore \textcolor{brown}{the stick}.
    \end{itemize}
    \vfill\null
  \end{multicols}
\end{frame}



\section{PS7, Ex. 10: Is the punishment credible? (infinitely repeated game)}

\begin{frame}{PS7, Ex. 10: Is the punishment credible? (infinitely repeated game)}
    We continue analyzing $G(\infty)$. Complete the proof that $(OP,OP)$ is a SPNE when $\delta=4/7$. We checked the first three points of the road map in Lecture 6. The last two points consist of checking that it is optimal for Player 2 to punish Player 1 after a deviation. In particular, you need to check that Player 2 has no profitable deviation when he is in the first and in the second round of punishing Player 1.
    \vfill\null
\end{frame}
\begin{frame}{PS7, Ex. 10: Is the punishment credible? (infinitely repeated game)}
    We continue analyzing $G(\infty)$. Complete the proof that $(OP,OP)$ is a SPNE when $\delta=4/7$. We checked the first three points of the road map in Lecture 6. The last two points consist of checking that it is optimal for Player 2 to punish Player 1 after a deviation. In particular, you need to check that Player 2 has no profitable deviation when he is in the first and in the second round of punishing Player 1.
  \begin{multicols}{2}
    In the lecture it was checked that Player 1 will not deviate from $(OP,OP)$ in:
    \begin{enumerate}
      \item Round 1, or if $(L,L)$ was played in all previous rounds.
      \item The \nth{1} round of being punished.
      \item Subsequent rounds of being punished.
    \end{enumerate}
    \vfill\null\columnbreak
    Underlining $(Q^D,Q^P)$ in the \nth{1} round of punishment (the 'tough' stage):
    \vspace{-6pt}
    \begin{table}
      \begin{tabular}{l|c|c|c|}
        \multicolumn{1}{c}{} & \multicolumn{1}{c}{L} & \multicolumn{1}{c}{M} & \multicolumn{1}{c}{\textbf{\underline{H}}} \\\cline{2-4}
        L & 10, 10 & 3, \textcolor{blue}{15} & \textcolor{red}{0}, 7 \\\cline{2-4}
        \textbf{\underline{M}} & \textcolor{red}{15}, 3 & \textcolor{red}{7}, \textcolor{blue}{7} & \textbf{\underline{-4, 5}} \\\cline{2-4}
        H & 7, \textcolor{blue}{0} & 5, -4 & -15, -15 \\\cline{2-4}
      \end{tabular}
    \end{table}
    Underlining $(Q^D,Q^P)$ in subsequent rounds of punishment (the 'mild' stage):
    \vspace{-6pt}
    \begin{table}
      \begin{tabular}{l|c|c|c|}
        \multicolumn{1}{c}{} & \multicolumn{1}{c}{L} & \multicolumn{1}{c}{\textbf{\underline{M}}} & \multicolumn{1}{c}{H} \\\cline{2-4}
        \textbf{\underline{L}} & 10, 10 & \textbf{\underline{3, \textcolor{blue}{15}}} & \textcolor{red}{0}, 7 \\\cline{2-4}
        M & \textcolor{red}{15}, 3 & \textcolor{red}{7}, \textcolor{blue}{7} & -4, 5 \\\cline{2-4}
        H & 7, \textcolor{blue}{0} & 5, -4 & -15, -15 \\\cline{2-4}
      \end{tabular}
    \end{table}
    \vfill\null
  \end{multicols}
  \vfill\null
\end{frame}
\begin{frame}{PS7, Ex. 10: Is the punishment credible? (infinitely repeated game)}
    We continue analyzing $G(\infty)$. Complete the proof that $(OP,OP)$ is a SPNE when $\delta=4/7$. We checked the first three points of the road map in Lecture 6. The last two points consist of checking that it is optimal for Player 2 to punish Player 1 after a deviation. In particular, you need to check that Player 2 has no profitable deviation when he is in the first and in the second round of punishing Player 1.
  \begin{multicols}{2}
    In the lecture it was checked that Player 1 will not deviate from $(OP,OP)$ in:
    \begin{enumerate}
      \item Round 1, or if $(L,L)$ was played in all previous rounds.
      \item The \nth{1} round of being punished.
      \item Subsequent rounds of being punished.
    \end{enumerate}
    Check that Player 2 will not deviate:
    \begin{itemize}
      \item[4.] When he is in the \nth{1} round of punishing Player 1.
      \item[5.] When he is in subsequent rounds of punishing Player 1.
    \end{itemize}
    \vfill\null\columnbreak
    Underlining $(Q^D,Q^P)$ in the \nth{1} round of punishment (the 'tough' stage):
    \vspace{-6pt}
    \begin{table}
      \begin{tabular}{l|c|c|c|}
        \multicolumn{1}{c}{} & \multicolumn{1}{c}{L} & \multicolumn{1}{c}{M} & \multicolumn{1}{c}{\textbf{\underline{H}}} \\\cline{2-4}
        L & 10, 10 & 3, \textcolor{blue}{15} & \textcolor{red}{0}, 7 \\\cline{2-4}
        \textbf{\underline{M}} & \textcolor{red}{15}, 3 & \textcolor{red}{7}, \textcolor{blue}{7} & \textbf{\underline{-4, 5}} \\\cline{2-4}
        H & 7, \textcolor{blue}{0} & 5, -4 & -15, -15 \\\cline{2-4}
      \end{tabular}
    \end{table}
    Underlining $(Q^D,Q^P)$ in subsequent rounds of punishment (the 'mild' stage):
    \vspace{-6pt}
    \begin{table}
      \begin{tabular}{l|c|c|c|}
        \multicolumn{1}{c}{} & \multicolumn{1}{c}{L} & \multicolumn{1}{c}{\textbf{\underline{M}}} & \multicolumn{1}{c}{H} \\\cline{2-4}
        \textbf{\underline{L}} & 10, 10 & \textbf{\underline{3, \textcolor{blue}{15}}} & \textcolor{red}{0}, 7 \\\cline{2-4}
        M & \textcolor{red}{15}, 3 & \textcolor{red}{7}, \textcolor{blue}{7} & -4, 5 \\\cline{2-4}
        H & 7, \textcolor{blue}{0} & 5, -4 & -15, -15 \\\cline{2-4}
      \end{tabular}
    \end{table}
    \vfill\null
  \end{multicols}
  \vfill\null
\end{frame}

\begin{frame}{PS7, Ex. 10: Is the punishment credible? (infinitely repeated game)}
  Use the roadmap to complete the proof that $(OP,OP)$ is a SPNE when $\delta=4/7$.\vspace{-4pt}
  \begin{multicols}{2}
    Check that Player 2 will not deviate:
    \begin{itemize}
      \item[4.] When he is in the \nth{1} round of punishing Player 1:
    \end{itemize}
    \vfill\null\columnbreak
    \vspace{-6pt}
    \begin{table}
      \begin{tabular}{l|c|c|c|}
        \multicolumn{1}{c}{} & \multicolumn{1}{c}{L} & \multicolumn{1}{c}{M} & \multicolumn{1}{c}{\textbf{\underline{H}}} \\\cline{2-4}
        L & 10, 10 & 3, \textcolor{blue}{15} & \textcolor{red}{0}, 7 \\\cline{2-4}
        \textbf{\underline{M}} & \textcolor{red}{15}, 3 & \textcolor{red}{7}, \textcolor{blue}{7} & \textbf{\underline{-4, 5}} \\\cline{2-4}
        H & 7, \textcolor{blue}{0} & 5, -4 & -15, -15 \\\cline{2-4}
      \end{tabular}
    \end{table}
    \vfill\null
  \end{multicols}
    \vfill\null
\end{frame}
\begin{frame}{PS7, Ex. 10: Is the punishment credible? (infinitely repeated game)}
  Use the roadmap to complete the proof that $(OP,OP)$ is a SPNE when $\delta=4/7$.\vspace{-4pt}
  \begin{multicols}{2}
    Check that Player 2 will not deviate:
    \begin{itemize}
      \item[4.] When he is in the \nth{1} round of punishing Player 1:
    \end{itemize}
    \vfill\null\columnbreak
    \vspace{-6pt}
    \begin{table}
      \begin{tabular}{l|c|c|c|}
        \multicolumn{1}{c}{} & \multicolumn{1}{c}{L} & \multicolumn{1}{c}{M} & \multicolumn{1}{c}{\textbf{\underline{H}}} \\\cline{2-4}
        L & 10, 10 & 3, \textcolor{blue}{15} & \textcolor{red}{0}, 7 \\\cline{2-4}
        \textbf{\underline{M}} & \textcolor{red}{15}, 3 & \textcolor{red}{7}, \textcolor{blue}{7} & \textbf{\underline{-4, 5}} \\\cline{2-4}
        H & 7, \textcolor{blue}{0} & 5, -4 & -15, -15 \\\cline{2-4}
      \end{tabular}
    \end{table}
    \vfill\null
  \end{multicols}
    \vspace{-24pt}
    If Player 2 does not deviate from $Q^P$, his expected utility is:
    \vspace{-6pt}
    \begin{align*}
      U_2(Q^P;Q^D)=5+15\delta+15\delta^2+...
                  =5+\sum_{\bm{t=2}}^\infty15\delta^{t-1}
                  =5+\frac{15\delta}{1-\delta}
                  =25,\text{ for }\delta=\frac{4}{7}
    \end{align*}
    If Player 2 deviates from $Q^P$ to play $M$, from the next round Player 1 will instead force Player 2 to take the punishment and play according to $Q^D$ forever:
    \begin{align*}
      U_2(M,Q^D;M,Q^P)&=7+\delta U_2(Q^D;Q^P)
                      =7+\delta\left(-4+\sum_{\bm{t=3}}^\infty3\delta^{t-1}\right)
                      =7+\delta\left(-4+\frac{3\delta}{1-\delta}\right)\\
                      &=7,\text{ for }\delta=4/7
    \end{align*}
    \vfill\null
\end{frame}
\begin{frame}{PS7, Ex. 10: Is the punishment credible? (infinitely repeated game)}
  Use the roadmap to complete the proof that $(OP,OP)$ is a SPNE when $\delta=4/7$.\vspace{-4pt}
  \begin{multicols}{2}
    Check that Player 2 will not deviate:
    \begin{itemize}
      \item[4.] When he is in the \nth{1} round of punishing Player 1:
    \end{itemize}
    \vfill\null\columnbreak
    \vspace{-6pt}
    \begin{table}
      \begin{tabular}{l|c|c|c|}
        \multicolumn{1}{c}{} & \multicolumn{1}{c}{L} & \multicolumn{1}{c}{M} & \multicolumn{1}{c}{\textbf{\underline{H}}} \\\cline{2-4}
        L & 10, 10 & 3, \textcolor{blue}{15} & \textcolor{red}{0}, 7 \\\cline{2-4}
        \textbf{\underline{M}} & \textcolor{red}{15}, 3 & \textcolor{red}{7}, \textcolor{blue}{7} & \textbf{\underline{-4, 5}} \\\cline{2-4}
        H & 7, \textcolor{blue}{0} & 5, -4 & -15, -15 \\\cline{2-4}
      \end{tabular}
    \end{table}
    \vfill\null
  \end{multicols}
    \vspace{-24pt}
    If Player 2 does not deviate from $Q^P$, his expected utility is:
    \vspace{-6pt}
    \begin{align*}
      U_2(Q^P;Q^D)=5+15\delta+15\delta^2+...
                  =5+\sum_{\bm{t=2}}^\infty15\delta^{t-1}
                  =5+\frac{15\delta}{1-\delta}
                  =25,\text{ for }\delta=\frac{4}{7}
    \end{align*}
    If Player 2 deviates from $Q^P$ to play $M$, from the next round Player 1 will instead force Player 2 to take the punishment and play according to $Q^D$ forever:
    \begin{align*}
      U_2(M,Q^D;M,Q^P)&=7+\delta U_2(Q^D;Q^P)
                      =7+\delta\left(-4+\sum_{\bm{t=3}}^\infty3\delta^{t-1}\right)
                      =7+\delta\left(-4+\frac{3\delta}{1-\delta}\right)\\
                      &=7,\text{ for }\delta=4/7
    \end{align*}
    \intuition{In the \nth{1} round of punishing Player 1, Player 2 expects higher utility from playing according to $Q^P$ (25) than from deviating (7), i.e. Player 2 has no incentive to deviate.}
    \vfill\null
\end{frame}

\begin{frame}{PS7, Ex. 10: Is the punishment credible? (infinitely repeated game)}
  Use the roadmap to complete the proof that $(OP,OP)$ is a SPNE when $\delta=4/7$.\vspace{-4pt}
  \begin{multicols}{2}
    Check that Player 2 will not deviate:
    \begin{itemize}
      \item[5.] When he is in subsequent rounds of punishing Player 1:
    \end{itemize}
    \vfill\null\columnbreak
    \vspace{-6pt}
    \begin{table}
      \begin{tabular}{l|c|c|c|}
        \multicolumn{1}{c}{} & \multicolumn{1}{c}{L} & \multicolumn{1}{c}{\textbf{\underline{M}}} & \multicolumn{1}{c}{H} \\\cline{2-4}
        \textbf{\underline{L}} & 10, 10 & \textbf{\underline{3, \textcolor{blue}{15}}} & \textcolor{red}{0}, 7 \\\cline{2-4}
        M & \textcolor{red}{15}, 3 & \textcolor{red}{7}, \textcolor{blue}{7} & -4, 5 \\\cline{2-4}
        H & 7, \textcolor{blue}{0} & 5, -4 & -15, -15 \\\cline{2-4}
      \end{tabular}
    \end{table}
    \vfill\null
  \end{multicols}
    \vfill\null
\end{frame}
\begin{frame}{PS7, Ex. 10: Is the punishment credible? (infinitely repeated game)}
  Use the roadmap to complete the proof that $(OP,OP)$ is a SPNE when $\delta=4/7$.\vspace{-4pt}
  \begin{multicols}{2}
    Check that Player 2 will not deviate:
    \begin{itemize}
      \item[5.] When he is in subsequent rounds of punishing Player 1:
    \end{itemize}
    \vfill\null\columnbreak
    \vspace{-6pt}
    \begin{table}
      \begin{tabular}{l|c|c|c|}
        \multicolumn{1}{c}{} & \multicolumn{1}{c}{L} & \multicolumn{1}{c}{\textbf{\underline{M}}} & \multicolumn{1}{c}{H} \\\cline{2-4}
        \textbf{\underline{L}} & 10, 10 & \textbf{\underline{3, \textcolor{blue}{15}}} & \textcolor{red}{0}, 7 \\\cline{2-4}
        M & \textcolor{red}{15}, 3 & \textcolor{red}{7}, \textcolor{blue}{7} & -4, 5 \\\cline{2-4}
        H & 7, \textcolor{blue}{0} & 5, -4 & -15, -15 \\\cline{2-4}
      \end{tabular}
    \end{table}
    \vspace{-8pt}
  \end{multicols}
    In each subsequent round, the outcome from $(Q^D,Q^P)$ is $(L,M)$ with payoffs $(3,15)$.
    \vfill\null
\end{frame}
\begin{frame}{PS7, Ex. 10: Is the punishment credible? (infinitely repeated game)}
  Use the roadmap to complete the proof that $(OP,OP)$ is a SPNE when $\delta=4/7$.\vspace{-4pt}
  \begin{multicols}{2}
    Check that Player 2 will not deviate:
    \begin{itemize}
      \item[5.] When he is in subsequent rounds of punishing Player 1:
    \end{itemize}
    \vfill\null\columnbreak
    \vspace{-6pt}
    \begin{table}
      \begin{tabular}{l|c|c|c|}
        \multicolumn{1}{c}{} & \multicolumn{1}{c}{L} & \multicolumn{1}{c}{\textbf{\underline{M}}} & \multicolumn{1}{c}{H} \\\cline{2-4}
        \textbf{\underline{L}} & 10, 10 & \textbf{\underline{3, \textcolor{blue}{15}}} & \textcolor{red}{0}, 7 \\\cline{2-4}
        M & \textcolor{red}{15}, 3 & \textcolor{red}{7}, \textcolor{blue}{7} & -4, 5 \\\cline{2-4}
        H & 7, \textcolor{blue}{0} & 5, -4 & -15, -15 \\\cline{2-4}
      \end{tabular}
    \end{table}
    \vspace{-8pt}
  \end{multicols}
    In each subsequent round, the outcome from $(Q^D,Q^P)$ is $(L,M)$ with payoffs $(3,15)$.\\\medskip
    \intuition{From the \nth{2} round of punishing Player 1, Player 2 expects to earn 15 in every round, i.e. he has no incentive to deviate.}
    \vfill\null
\end{frame}

\begin{frame}{PS7, Ex. 10: Is the punishment credible? (infinitely repeated game)}
  Use the roadmap to complete the proof that $(OP,OP)$ is a SPNE when $\delta=4/7$.\vspace{-4pt}
  \begin{multicols}{2}
    In the lecture it was checked that Player 1 will not deviate from $(OP,OP)$ in:
    \begin{enumerate}
      \item Round 1, or if $(L,L)$ was played in all previous rounds.
      \item The \nth{1} round of being punished.
      \item Subsequent rounds of being punished.
    \end{enumerate}
    In this exercise we have checked that Player 2 will not deviate:
    \begin{itemize}
      \item[4.] When he is in the \nth{1} round of punishing Player 1.
      \item[5.] When he is in subsequent rounds of punishing Player 1.
    \end{itemize}
    \vfill\null\columnbreak
    \vfill\null
  \end{multicols}
  \vfill\null
\end{frame}
\begin{frame}{PS7, Ex. 10: Is the punishment credible? (infinitely repeated game)}
  Use the roadmap to complete the proof that $(OP,OP)$ is a SPNE when $\delta=4/7$.\vspace{-4pt}
  \begin{multicols}{2}
    In the lecture it was checked that Player 1 will not deviate from $(OP,OP)$ in:
    \begin{enumerate}
      \item Round 1, or if $(L,L)$ was played in all previous rounds.
      \item The \nth{1} round of being punished.
      \item Subsequent rounds of being punished.
    \end{enumerate}
    In this exercise we have checked that Player 2 will not deviate:
    \begin{itemize}
      \item[4.] When he is in the \nth{1} round of punishing Player 1.
      \item[5.] When he is in subsequent rounds of punishing Player 1.
    \end{itemize}
    \vfill\null\columnbreak
    Condition 1 secures that $(OP,OP)$ is optimal \textbf{\textit{on}} the equilibrium path.\\\medskip
    Condition 2-5 secure that $(OP,OP)$ is optimal \textbf{\textit{off}} the equilibrium path.
    \vfill\null
  \end{multicols}
  \vfill\null
\end{frame}
\begin{frame}{PS7, Ex. 10: Is the punishment credible? (infinitely repeated game)}
  Use the roadmap to complete the proof that $(OP,OP)$ is a SPNE when $\delta=4/7$.\vspace{-4pt}
  \begin{multicols}{2}
    In the lecture it was checked that Player 1 will not deviate from $(OP,OP)$ in:
    \begin{enumerate}
      \item Round 1, or if $(L,L)$ was played in all previous rounds.
      \item The \nth{1} round of being punished.
      \item Subsequent rounds of being punished.
    \end{enumerate}
    In this exercise we have checked that Player 2 will not deviate:
    \begin{itemize}
      \item[4.] When he is in the \nth{1} round of punishing Player 1.
      \item[5.] When he is in subsequent rounds of punishing Player 1.
    \end{itemize}
    \vfill\null\columnbreak
    Condition 1 secures that $(OP,OP)$ is optimal \textbf{\textit{on}} the equilibrium path.\\\medskip
    Condition 2-5 secure that $(OP,OP)$ is optimal \textbf{\textit{off}} the equilibrium path.\\\medskip
    Therefore, we can conclude that $(OP,OP)$ is a SPNE for $\delta = 4/7$.
    \vfill\null
  \end{multicols}
  \vfill\null
\end{frame}
