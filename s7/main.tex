\documentclass[8pt,apectratio=169]{beamer}

\usetheme[progressbar=frametitle]{metropolis}
\usepackage{appendixnumberbeamer}
\usepackage[style=authoryear, backend=bibtex8, natbib=true, maxcitenames=2]{biblatex}

\usepackage[utf8]{inputenc} % utf8x  defines more symbols, but may cause compatible problems
\usepackage{lmodern,textcomp} % Latin Modern fonts, contains €

\usepackage{graphicx}
\usepackage{import}

\usepackage{booktabs}
\usepackage[scale=2]{ccicons}

\usepackage{pgfplots}
\usepgfplotslibrary{dateplot}

\usepackage{xspace}
\newcommand{\themename}{\textbf{\textsc{metropolis}}\xspace}

% Math
\usepackage{amsmath}
\usepackage{bm} % bold symbol in math mode
\counterwithin*{equation}{section} % reset the equation number whenever section is stepped

% Optional packages
\usepackage{xcolor}
\usepackage{multicol}
\usepackage{multirow,array}
\usepackage{subcaption} % for subfigure and subtable
\usepackage{hyperref}
\usepackage{epigraph}
\usepackage[super,negative]{nth} % allows writing 1st, 2nd, 3rd with superscript
\usepackage{ulem} % use the "sout" tag to "strikethrough" text
\usepackage{cancel} % https://tex.stackexchange.com/questions/75525/how-to-write-crossed-out-math-in-latex
\usepackage{tcolorbox}

% Select what to do with command \comment:
  % \newcommand{\comment}[1]{}  %comments not shown
  % \newcommand{\comment}[1]{\par {\bfseries \color{blue} #1 \par}} %comments shown
% Select what to do with todonotes: i.e. \todo{}, \todo[inline]{}
  % \usepackage[disable]{todonotes} % notes not shown
  % \usepackage[draft]{todonotes}   % notes shown

%\numberwithin{equation}{section}

%\addbibresource{references}

\titlegraphic{\hfill \includegraphics[width=0.15 \textwidth]{figures/logo}}
\title{Microeconomics III: Problem Set 8\footnote{Slides created for exercise class 3 and 4, with reservation for possible errors.\\}}
\author{Thor Donsby Noe (\href{mailto:thor.noe@econ.ku.dk}{thor.noe@econ.ku.dk})
        \& Christopher Borberg (\href{mailto:christopher.borberg@econ.ku.dk}{christopher.borberg@econ.ku.dk})
        }
\date{November 13 2019} % \today
\institute{\normalsize Department of Economics, University of Copenhagen}

    % \definecolor{BlueTOL}{HTML}{222255}
    \definecolor{BrownTOL}{HTML}{666633}
    \definecolor{GreenTOL}{HTML}{225522}
    % \setbeamercolor{normal text}{fg=BlueTOL,bg=white}
    \setbeamercolor{alerted text}{fg=BrownTOL}
    \setbeamercolor{example text}{fg=GreenTOL}
    \setbeamercolor{background canvas}{bg=white}

    \setbeamercolor{block title alerted}{use=alerted text,
        fg=alerted text.fg,
        bg=alerted text.bg!80!alerted text.fg}
    \setbeamercolor{block body alerted}{use={block title alerted, alerted text},
        fg=alerted text.fg,
        bg=block title alerted.bg!50!alerted text.bg}
    \setbeamercolor{block title example}{use=example text,
        fg=example text.fg,
        bg=example text.bg!80!example text.fg}
    \setbeamercolor{block body example}{use={block title example, example text},
        fg=example text.fg,
        bg=block title example.bg!50!example text.bg}

\begin{document}
\maketitle

% ------------------------------------------------------------------------------
% ------ FRAME -----------------------------------------------------------------
% ------------------------------------------------------------------------------
\begin{frame}{Outline}
    \tableofcontents
\end{frame}



\section{PS6, Ex. 1 (A): }

\begin{frame}{PS6, Ex. 1 (A): }
  \begin{multicols}{2}
    \vfill\null \columnbreak
    \vfill\null
  \end{multicols}
\end{frame}

\begin{frame}{PS6, Ex. 1 (A): }
  \begin{multicols}{2}
    \vfill\null \columnbreak
    \vfill\null
  \end{multicols}
\end{frame}



\section{PS6, Ex. 2 (A): }

\begin{frame}{PS6, Ex. 2 (A): }
  \begin{multicols}{2}
    \vfill\null \columnbreak
    \vfill\null
  \end{multicols}
\end{frame}

\begin{frame}{PS6, Ex. 2.a (A): }
  \begin{multicols}{2}
    \vfill\null \columnbreak
    \vfill\null
  \end{multicols}
\end{frame}



\section{PS6, Ex. 3 (A): }

\begin{frame}{PS6, Ex. 3 (A): }
  \begin{multicols}{2}
    \vfill\null \columnbreak
    \vfill\null
  \end{multicols}
\end{frame}

\begin{frame}{PS6, Ex. 3.a (A): }
  \begin{multicols}{2}
    \vfill\null \columnbreak
    \vfill\null
  \end{multicols}
\end{frame}



\section{PS6, Ex. 4: }

\begin{frame}{PS6, Ex. 4: }
  \begin{multicols}{2}
    \vfill\null \columnbreak
    \vfill\null
  \end{multicols}
\end{frame}

\begin{frame}{PS6, Ex. 4.a: }
  \begin{multicols}{2}
    \vfill\null \columnbreak
    \vfill\null
  \end{multicols}
\end{frame}



\section{PS6, Ex. 5: }

\begin{frame}{PS6, Ex. 5: }
  \begin{multicols}{2}
    \vfill\null \columnbreak
    \vfill\null
  \end{multicols}
\end{frame}

\begin{frame}{PS6, Ex. 5.a: }
  \begin{multicols}{2}
    \vfill\null \columnbreak
    \vfill\null
  \end{multicols}
\end{frame}



\section{PS6, Ex. 6: }

\begin{frame}{PS6, Ex. 6: }
  \begin{multicols}{2}
    \vfill\null \columnbreak
    \vfill\null
  \end{multicols}
\end{frame}

\begin{frame}{PS6, Ex. 6.a: }
  \begin{multicols}{2}
    \vfill\null \columnbreak
    \vfill\null
  \end{multicols}
\end{frame}



\section{PS6, Ex. 7: }

\begin{frame}{PS6, Ex. 7: }
  \begin{multicols}{2}
    \vfill\null \columnbreak
    \vfill\null
  \end{multicols}
\end{frame}

\begin{frame}{PS6, Ex. 7.a: }
  \begin{multicols}{2}
    \vfill\null \columnbreak
    \vfill\null
  \end{multicols}
\end{frame}



\section{PS6, Ex. 8: }

\begin{frame}{PS6, Ex. 8: }
  \begin{multicols}{2}
    \vfill\null \columnbreak
    \vfill\null
  \end{multicols}
\end{frame}

\begin{frame}{PS6, Ex. 8.a: }
  \begin{multicols}{2}
    \vfill\null \columnbreak
    \vfill\null
  \end{multicols}
\end{frame}



\section{PS6, Ex. 9: }

\begin{frame}{PS6, Ex. 9: }
  \begin{multicols}{2}
    \vfill\null \columnbreak
    \vfill\null
  \end{multicols}
\end{frame}

\begin{frame}{PS6, Ex. 9.a: }
  \begin{multicols}{2}
    \vfill\null \columnbreak
    \vfill\null
  \end{multicols}
\end{frame}



\section{Code examples} % out-comment: ctrl-shift-7 or ctrl-shift-* (use cmd for Mac)

\begin{frame}{Code examples}
  \begin{multicols}{2}
    % Game tree:
    \begin{figure}[!h]
      \center
      \def\svgwidth{.8\columnwidth}
      \import{figures/}{long_.pdf_tex}
    \end{figure}
  \vfill\null \columnbreak
    Matrix, no player names:
    \vspace{-10pt}
    \begin{table} % as opposed to matrices with player names, each line does not start with "&" as there's no empty column for the name-box. Otherwise, see the explanations below.
      \begin{tabular}{l|c|c|}
        \multicolumn{1}{c}{} & \multicolumn{1}{c}{L (q)} & \multicolumn{1}{c}{R (1-q)} \\\cline{2-3}
        T (p)   &  &  \\\cline{2-3}
        B (1-p) &  &  \\\cline{2-3}
      \end{tabular}
    \end{table}
    Matrix, no colors:
    \vspace{-10pt}
    \begin{table}
      \begin{tabular}{cl|c|c|} % the number of total columns and which have vertical lines between them (left-align or center text).
        & \multicolumn{1}{c}{} & \multicolumn{2}{c}{Player 2}\\ % "2" is the number of columns in the matrix that the 2nd player name spans over
        \parbox[t]{1mm}{\multirow{3}{*}{\rotatebox[origin=r]{90}{Player 1}}} % "3" is the number of rows the 1st player name spans over (including the one with the column names)
        & \multicolumn{1}{c}{} & \multicolumn{1}{c}{L (q)} & \multicolumn{1}{c}{R (1-q)} \\\cline{3-4} % column names use the "\multicolumn" command to not draw vertical lines between them.
        & T (p)   &  &  \\\cline{3-4} % a horizontal line is drawn after the line break using "\cline{x-y}" where x and y are the column numbers of the cells to be underlined.
        & B (1-p) &  &  \\\cline{3-4}
      \end{tabular}
    \end{table}
    Matrix, with colors:
    \vspace{-10pt}
    \begin{table}
      \begin{tabular}{cl|c|c|}
        & \multicolumn{1}{c}{} & \multicolumn{2}{c}{\color{blue}Player 2}\\
        \parbox[t]{1mm}{\multirow{3}{*}{\rotatebox[origin=r]{90}{\color{red}Player 1}}}
        & \multicolumn{1}{c}{} & \multicolumn{1}{c}{L (q)} & \multicolumn{1}{c}{R (1-q)} \\\cline{3-4}
        & T (p)   & \textcolor{red}{1}, \textcolor{blue}{1} &   \\\cline{3-4}
        & B (1-p) &  &  \\\cline{3-4}
      \end{tabular}
    \end{table}
  \vfill\null
  \end{multicols}
\end{frame}



\end{document}
